\section{Getting started}
\subsection{Tecnologies}
The tecnologies used during the development of Etherless, which are further described in this document appendix \textbf{A}, are presented in this section.
\subsubsection{Languages}
\begin{itemize}
	\item Javascript;
	\item Typescript;
	\item Solidity.
\end{itemize}
\subsubsection{Utilities}
\begin{itemize}
	\item Visual Studio Code;
	\item Node.js;
	\item Node package manager;
	\item AWS Lambda;
	\item AWS Elastic Beanstalk;
	\item IPFS;
	\item ESLint;
	\item Serverless Framework;
	\item Jest;
	\item Open Zeppelin;
	\item Infura;
	\item Ropsten testnet;
	\item Ganache;
	\item Truffle;
	\item Mocha;
	\item Chai;
	\item Acorn.
\end{itemize}
\pagebreak
\subsubsection{Libraries and dependencies}
\rowcolors{2}{lightRowColor}{darkRowColor}
\begin{longtable}[h!]{
		>{\centering\arraybackslash}p{0.3\textwidth}
		>{\centering\arraybackslash}p{0.3\textwidth} }
	\caption{Dependencies required for software usage} \\

	\coloredTableHead
	\textbf{\color{white}Dependency} &
	\textbf{\color{white}Version}
	\tabularnewline
	\endhead

	configstore & $\geq$5.0.1 \tabularnewline
	inquirer & $\geq$7.1.0 \tabularnewline
	node & $\geq$13.13.4 \tabularnewline
	acorn & $\geq$7.2.0 \tabularnewline
	ethers & $\geq$4.0.47 \tabularnewline
	ipfs-mini & $\geq$1.1.5 \tabularnewline
	ts-node & $\geq$8.9.1 \tabularnewline
	typescript & $\geq$3.8.3 \tabularnewline
	yargs & $\geq$15.3.1 \tabularnewline
	nyc & $\geq$15.0.1 \tabularnewline
	aws-sdk & $\geq$2.663.0 \tabularnewline
	solc & $\geq$0.5.17 \tabularnewline

\end{longtable}
\rowcolors{2}{lightRowColor}{darkRowColor}
\begin{longtable}{
		>{\centering\arraybackslash}p{0.3\textwidth}
		>{\centering\arraybackslash}p{0.3\textwidth} }
	\caption{Dependencies required for software testing and development} \\

	\coloredTableHead
	\textbf{\color{white}Dependency} &
	\textbf{\color{white}Version}
	\tabularnewline
	\endhead

	chai & $\geq$4.2.0 \tabularnewline
	mocha & $\geq$7.0.2 \tabularnewline
	jest & $\geq$26.0.1 \tabularnewline
	eslint & $\geq$6.8.0 \tabularnewline
	eslint-config-airbnb-base & $\geq$14.1.0 \tabularnewline
	eslint-config-google & $\geq$0.14.0 \tabularnewline
	eslint-plugin-import & $\geq$2.20.2 \tabularnewline
	truffle & $\geq$5.1.10 \tabularnewline
	truffle-hdwallet-provider & $\geq$1.0.17 \tabularnewline
	dotenv & $\geq$8.2.0 \tabularnewline
	openzeppelin & $\geq$2.8.0 \tabularnewline
	websocket-extensions & $\geq$0.1.4 \tabularnewline
	minimist & $\geq$1.2.5 \tabularnewline
	ganache & $\geq$6.4.4 \tabularnewline
	ganache-cli & $\geq$6.9.1 \tabularnewline

\end{longtable}
\subsection{Requirements}
In this section all of the basic requirements needed to run and test the product are described.
\subsubsection{Hardware}
\begin{itemize}
	\item \textbf{Internet connection}: in order to interact with the Ethereum blockchain and the AWS services.
\end{itemize}
\subsubsection{Software}
\begin{itemize}
	\item \textbf{AWS credentials}: in order to interact the Amazon Web Services, an AWS account is required;
	\item \textbf{ETH wallet}: in order to perform transactions on the blockchain Ethereum, a \textit{non empty} ETH wallet is required. Etherless allows the creation of a wallet on the run, but it does not contain any Ethers.
\end{itemize}
\subsection{Setup}
This section describes all the steps needed to install the Etherless product.
\subsubsection{Git}
\begin{itemize}
	\item \textbf{Windows}: download the .exe file from \href{https://git-scm.com/download/win}{Git official site}, and follow the given instruction;
	\item \textbf{Linux}: open a shell and type: \texttt{sudo apt install git}.
\end{itemize}
	For information on how to install Git on different OS, follow the official guide on: \href{https://git-scm.com/book/en/v2/Getting-Started-Installing-Git}{How to install Git}.
\subsubsection{Node}
	Download the executable file of the latest version, from \href{https://nodejs.org/it/download/}{Node official site} and follow the given instruction to install. This will also install Node package manager.
\subsubsection{Ganache}
	Ganache is available in two variants CLI or GUI, refer the \href{https://www.trufflesuite.com/docs/ganache/quickstart}{Ganache official web page} for the installation and usage.
\subsubsection{Truffle}
 In order to install Truffle you need to run \texttt{npm install truffle -g}, the minimum version required is the 4.0.0.
\subsection{Configuration}
This section shows how to configure the work environment, to be able to execute, test and develop on Etherless. This configuration steps should be followed in the given order.
\subsubsection{Cloning the repository}
To download the software product on your machine, from the GitHub Etherless repository, you should open a shell and position on the folder you want the software to be located. Then you should use the command \texttt{git clone https://github.com/RoundaboutTeam/etherless.git} and position on the \texttt{master} branch.
\subsubsection{Installing the dependencies}
To install all the packages and libreries needed to work with Etherless you should open a shell and position on the main directory of the software. Then you should run the \texttt{npm install} command, which will take care of installing all the dependencies specified in the \texttt{package.json} file.
\subsubsection{AWS and Ethereum configurations}
In the cloned repository two configuration files are not present for security reasons, since they contain the Ethereum and AWS credentials used to interact with these two services. They are present on the Etherless-server version already uploaded and active on AWS Elastic Beanstalk.
Any developer who might want to work with a local version of Etherless-server should create an \texttt{AWSconfig.json} file structured as:
\begin{quote}
\texttt{ \\
	\{\\
		"awsKey": <<insert AWS key>>, \\
		"awsSecretKey": <<insert AWS secret key>>, \\
		"awsRegion": <<insert AWS region>> \\
	\}\\
}
\end{quote}
While, to interact with the contract on the blockchain, a \texttt{smartConfig.json} file should be created by the developer, with this structure:
\begin{quote}
\texttt{ \\
	\{\\
	"walletAddress": <<insert wallet address>>, \\
	"privateKey": <<insert private key>>, \\
	"contractAddress": <<insert contract address>>, \\
	"networkName": <<insert network name>> \\
	\}\\
}
\end{quote}
The quoted parameters should be replaced with the actual credentials, and both files moved to \texttt{Config} directory inside Etherless-server.
\subsubsection{Enable \texttt{etherless} commands}
To enable using the keyword \texttt{etherless} to execute commands you should open a shell and position on the main directory of the software.
Then you should run the command \texttt{npm link} as Administrator.
\subsubsection{Truffle suite configuration}
	In order to interact with Ropsten testnet, before beginning the installation process you need two keys: an Infura project id (from your own Infura account) and the mnemonic phrase of your Ethereum account. Having these credentials you need to edit the \textit{truffle-config.js} file, replacing the matching fields at the top with your own. Now you are ready to interact with the Ropsten test network directly from Truffle.\\
	To interact only with a Ganache-cli local blockchain no further configuration is needed, as the information is already stored in the \textit{truffle-config.js} file.
\subsection{Functionalities}
	In this section will be listed the available functionalities and the related CLI commands that triggers them.
	\subsubsection{Signup}
	Creates a new account with an empty Ethereum wallet.
	\subsubsection*{Command}
	\texttt{etherless signup}
	\subsubsection{Login}
	Creates a login session with the given private key or mnemonic phrase, allowing to make requests on the Etherless platform.
	\subsubsection*{Command}
	\texttt{etherless login <<private key>>} \\
	\texttt{etherless login m <<mnemonic phrase>>}
	\subsubsection{Logout}
	Closes a previously created login session.
	\subsubsection*{Command}
	\texttt{etherless logout}
	\subsubsection{Who am I}
	Shows the address of the logged user.
	\subsubsection*{Command}
	\texttt{etherless whoami}
	\subsubsection{Init}
	Shows a brief description of the login and signup functionalities.
	\subsubsection*{Command}
	\texttt{etherless init}
	\subsubsection{Info}
	Returns information related to a given function, specifically: developer, name, signature, cost and description.
	\subsubsection*{Command}
	\texttt{etherless info <<function name>>}
	\subsubsection{History}
	List all the past executions of the user, can be limited to a certain amount of executions.
	\subsubsection*{Command}
	\texttt{etherless history [limit]}
	\subsubsection{Search}
	Returns a list of functions that contain the given keyword inside the name.
	\subsubsection*{Command}
	\texttt{etherless search <<keyword>>}
	\subsubsection{List}
	Displays a list of all the available functions, inside the Etherless platform.
	\subsubsection*{Command}
	\texttt{etherless list [m]}
	\subsubsection{Run}
	Requests the execution of a given function, with the given spaced parameters.
	\subsubsection*{Command}
	\texttt{etherless run <<function name>> [parameters list]}
	\subsubsection{Deploy}
	Requests the deployment of a new function, using the given .js file and the given function name.
	\subsubsection*{Command}
	\texttt{etherless deploy <<file .js>> <<function name>>}
	\subsubsection{Edit}
	Requests the editing of an existing function, using the given .js file and the given function name.
	\subsubsection*{Command}
	\texttt{etherless edit <<file .js>> <<function name>>}
	\subsubsection{Delete}
	Requests the delete of an existing function, using the given function name.
	\subsubsection*{Command}
	\texttt{etherless delete <<function name>>}
\subsection{Testing}
\subsubsection{Etherless}
Launching the \texttt{npm test} command inside the Etherless directory will automatically trigger all the developed tests for the whole software product.
\subsubsection{Etherless-cli and Etherless-server}
It is possible to test the Etherless-cli and Etherless-server modules indipendently, by launching the \texttt{npm test} command inside their directories.
\subsubsection{Etherless-smart}
It is possible to test Etherless-smart module, by launching the \texttt{truffle test} command inside the module's directory.
