\section{Getting started}
\subsection{Tecnologies}
The aim of this section is to present a list of all the technologies used during the development of \textit{Etherless}, which are further described in the appendix at the end of this document.
\subsubsection{Languages}
\begin{itemize}
	\item Javascript$_{G}$;
	\item Typescript$_{G}$;
	\item Solidity$_{G}$.
\end{itemize}
\subsubsection{Utilities}
\begin{itemize}
	\item Visual Studio Code$_{G}$;
	\item Node.js$_{G}$;
	\item Node Package Manager$_{G}$;
	\item AWS Lambda$_{G}$;
	\item AWS Elastic Beanstalk$_{G}$;
	\item IPFS$_{G}$;
	\item ESLint$_{G}$;
	\item Serverless Framework$_{G}$;
	\item Jest$_{G}$;
	\item Open Zeppelin$_{G}$;
	\item Infura$_{G}$;
	\item Ropsten$_{G}$;
	\item Ganache$_{G}$;
	\item Truffle$_{G}$;
	\item Mocha$_{G}$;
	\item Chai$_{G}$;
	\item Acorn$_{G}$.
\end{itemize}
\pagebreak
\subsubsection{Libraries and dependencies}
\rowcolors{2}{lightRowColor}{darkRowColor}
\begin{longtable}[h!]{
		>{\centering\arraybackslash}p{0.3\textwidth}
		>{\centering\arraybackslash}p{0.3\textwidth} }
	\caption{Dependencies required for software usage} \\

	\coloredTableHead
	\textbf{\color{white}Dependency} &
	\textbf{\color{white}Version}
	\tabularnewline
	\endhead

	configstore & $\geq$5.0.1 \tabularnewline
	inquirer & $\geq$7.1.0 \tabularnewline
	node & $\geq$12.18.0 \tabularnewline
	acorn & $\geq$7.2.0 \tabularnewline
	ethers & $\geq$4.0.47 \tabularnewline
	ipfs-mini & $\geq$1.1.5 \tabularnewline
	ts-node & $\geq$8.9.1 \tabularnewline
	typescript & $\geq$3.8.3 \tabularnewline
	yargs & $\geq$15.3.1 \tabularnewline
	nyc & $\geq$15.0.1 \tabularnewline
	aws-sdk & $\geq$2.663.0 \tabularnewline
	solc & $\geq$0.5.17 \tabularnewline

\end{longtable}
\rowcolors{2}{lightRowColor}{darkRowColor}
\begin{longtable}{
		>{\centering\arraybackslash}p{0.3\textwidth}
		>{\centering\arraybackslash}p{0.3\textwidth} }
	\caption{Dependencies required for software testing and development} \\

	\coloredTableHead
	\textbf{\color{white}Dependency} &
	\textbf{\color{white}Version}
	\tabularnewline
	\endhead

	chai & $\geq$4.2.0 \tabularnewline
	mocha & $\geq$7.0.2 \tabularnewline
	jest & $\geq$26.0.1 \tabularnewline
	eslint & $\geq$6.8.0 \tabularnewline
	eslint-config-airbnb-base & $\geq$14.1.0 \tabularnewline
	eslint-config-google & $\geq$0.14.0 \tabularnewline
	eslint-plugin-import & $\geq$2.20.2 \tabularnewline
	truffle & $\geq$5.1.10 \tabularnewline
	truffle-hdwallet-provider & $\geq$1.0.17 \tabularnewline
	dotenv & $\geq$8.2.0 \tabularnewline
	openzeppelin & $\geq$2.8.0 \tabularnewline
	websocket-extensions & $\geq$0.1.4 \tabularnewline
	minimist & $\geq$1.2.5 \tabularnewline
	ganache & $\geq$6.4.4 \tabularnewline
	ganache-cli & $\geq$6.9.1 \tabularnewline

\end{longtable}
\subsection{Requirements}
This section describes all of the basic requirements needed to run and test the product.
\subsubsection{Hardware}
\begin{itemize}
	\item \textbf{Internet connection}: needed in order to interact with the Ethereum blockchain and AWS$_{G}$.
\end{itemize}
\subsubsection{Software}
\begin{itemize}
	\item \textbf{AWS$_{G}$ credentials}: in order to interact with Amazon Web Services$_{G}$, an AWS$_{G}$ account is required;
	\item \textbf{ETH wallet}: in order to perform transactions on the Ethereum blockchain, a \textit{non empty} ETH wallet is required. With \textit{Etherless} the user can create a wallet on the run, but it does not contain any ETH.
\end{itemize}
\subsection{Setup}
This section describes all the steps needed to install \textit{Etherless}.
\subsubsection{Git}
\begin{itemize}
	\item \textbf{Windows}: download the .exe file from the \href{https://git-scm.com/download/win}{Git official website} and follow the given instructions;
	\item \textbf{Linux}: open a shell and type: \texttt{sudo apt install git}.
\end{itemize}
	For information on how to install Git on different OS, follow the official guide on: \href{https://git-scm.com/book/en/v2/Getting-Started-Installing-Git}{How to install Git}.
\subsubsection{Node}
	Download the the latest version of the executable file from the \href{https://nodejs.org/it/download/}{Node official website} and follow the given instructions. This will also install Node package manager.
\subsubsection{Ganache}
	Ganache$_{G}$ is available in two variants, CLI and GUI. Refer to the \href{https://www.trufflesuite.com/docs/ganache/quickstart}{Ganache$_{G}$ official website} for installation and usage.
\subsubsection{Truffle}
 In order to install Truffle$_{G}$ you need to run \texttt{npm install truffle -g} from your terminal window. The minimum version required is 4.0.0.
\subsection{Configuration}
This section shows how to configure the work environment to be able to execute, test and develop on \textit{Etherless}. These configuration steps should be followed in the given order.
\subsubsection{Cloning the repository}
To download the software product on your machine you should open a shell and change to the location you want the software to be in. Then you should use the command \texttt{git clone https://github.com/RoundaboutTeam/etherless.git} and switch to the \texttt{master} branch.
\subsubsection{Installing the dependencies}
To install all the packages and libraries needed to work with \textit{Etherless} you should open a shell and change to the main directory. Then you should run the command \texttt{npm install}, which will take care of installing all the dependencies specified in the \texttt{package.json} file.
\subsubsection{AWS and Ethereum configurations}
The files containing Ethereum and AWS$_{G}$ credentials used to interact with the corresponding services are not included in the cloned repository for security reasons. These are present on the Etherless-server version already uploaded and are active on AWS Elastic Beanstalk$_{G}$.
Any developer who might want to work with a local version of Etherless-server should create an \texttt{AWSconfig.json} file structured as follows:
\begin{quote}
\texttt{ \\
	\{\\
		"awsKey": <<insert AWS key>>, \\
		"awsSecretKey": <<insert AWS secret key>>, \\
		"awsRegion": <<insert AWS region>> \\
	\}\\
}
\end{quote}
To interact with the contracts on the Ethereum blockchain, a \texttt{smartConfig.json} file should be created with the following structure:
\begin{quote}
\texttt{ \\
	\{\\
	"walletAddress": <<insert wallet address>>, \\
	"privateKey": <<insert private key>>, \\
	"contractAddress": <<insert contract address>>, \\
	"networkName": <<insert network name>> \\
	\}\\
}
\end{quote}
The parameters within angular brackets should be replaced with the actual credentials and both files be moved to the \texttt{Config} directory inside Etherless-server.
\subsubsection{Enable \texttt{etherless} commands}
To enable using commands with the keyword \texttt{etherless} you should open a shell and change to the main directory. Then you should run the command \texttt{npm link} as Administrator.
\subsubsection{Truffle Suite Configuration}
	In order to interact with the Ropsten$_{G}$ testnet, before beginning the installation process you need two keys: an Infura$_{G}$ project id (from your own Infura$_{G}$ account) and the mnemonic phrase of your Ethereum account. Having these credentials you need to edit the \textit{truffle-config.js} file, replacing the matching fields at the top with your own. Now you are ready to interact with the Ropsten$_{G}$ test network directly from Truffle$_{G}$.\\
	To interact only with a Ganache-cli local blockchain no further configuration is needed, as the information is already stored in the \textit{truffle-config.js} file.
\subsection{Functionalities}
	This section lists all the available functionalities of \textit{Etherless} and the corresponding CLI commands that trigger them.
	\subsubsection{Signup}
	Creates a new account with an empty Ethereum wallet.
	\subsubsection*{Command}
	\texttt{etherless signup}
	\subsubsection{Login}
	Creates a login session with the given private key or mnemonic phrase, allowing to make requests on the \textit{Etherless} platform.
	\subsubsection*{Command}
	\texttt{etherless login <private\_key>} \\
	\texttt{etherless login -m <mnemonic\_phrase>}
	\subsubsection{Logout}
	Closes a previously created login session.
	\subsubsection*{Command}
	\texttt{etherless logout}
	\subsubsection{Who am I}
	Shows the address of the logged user.
	\subsubsection*{Command}
	\texttt{etherless whoami}
	\subsubsection{Init}
	Shows a brief description of the \texttt{login} and \texttt{signup} functionalities.
	\subsubsection*{Command}
	\texttt{etherless init}
	\subsubsection{Info}
	Returns informations related to a given function, specifically the developer address, the function name, the function signature, the function price and a brief description.
	\subsubsection*{Command}
	\texttt{etherless info <function\_name>}
	\subsubsection{History}
	Lists all the past executions of the user, can be limited to a certain amount of executions.
	\subsubsection*{Command}
	\texttt{etherless history [limit]}
	\subsubsection{Search}
	Returns a list of functions that contain the given keyword inside the name.
	\subsubsection*{Command}
	\texttt{etherless search <keyword>}
	\subsubsection{List}
	Displays a list of all available functions inside the \textit{Etherless} platform.
	\subsubsection*{Command}
	\texttt{etherless list [m]}
	\subsubsection{Run}
	Requests the execution of a given function with the given spaced parameters.
	\subsubsection*{Command}
	\texttt{etherless run <function\_name> [parameter\_list]}
	\subsubsection{Deploy}
	Requests the deployment of a new function, using the given source file path, function name and description.
	\subsubsection*{Command}
	\texttt{etherless deploy <path> <function\_name> <function\_desc>}
	\subsubsection{Edit}
	Requests the editing of an existing function, using the given function name and the field to modify (a new description or the path of the new source file).
	\subsubsection*{Command}
	\texttt{etherless edit <function\_name> [path] [newDesc]}
	\subsubsection{Delete}
	Requests the delete of an existing function, using the given function name.
	\subsubsection*{Command}
	\texttt{etherless delete <function\_name>}
\subsection{Testing}
\subsubsection{Etherless}
Launching the \texttt{npm test} command inside the \textit{Etherless} directory will automatically trigger all the developed tests for the software product.
\subsubsection{Etherless-cli and Etherless-server}
It is possible to test the Etherless-cli and Etherless-server modules indipendently by launching the \texttt{npm test} command inside their directories.
\subsubsection{Etherless-smart}
It is possible to test the Etherless-smart module by launching the \texttt{truffle test} command inside the module's directory.
