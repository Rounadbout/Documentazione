\section{Glossary}
	\subsection*{\fbox{A}}
		\subsubsection*{Acorn}
			A tiny and fast Javascript parser, written completely in Javascript. 
		\subsubsection*{AWS}
			Amazon Web Services is a cloud platform provided by the Amazon company, which offers numerous cloud computing services.
		\subsubsection*{AWS Lambda}
			Serverless cloud computing platform, event driven and provided by Amazon as part of Amazon Web Services.
			
		\subsubsection*{AWS Elastic Beanstalk}
			Service that allows you to deploy applications written in various languages in the cloud. It automatically manages the resources necessary for their execution.
			
	\subsection*{\fbox{C}}
		\subsubsection*{Chai}
			Assertion library that provides functions and methods that help you compare the output of a certain test with its expected value.
			
	\subsection*{\fbox{E}}
		\subsubsection*{ESLint}
			Static code analysis tool that allows the identification of problematic models into Javascript code. It has configurable and customizable rules, which guide the developer in writing the code.
		\subsubsection*{Ethers.js}
			Library that provides functionalities which simplify the communication with the Ethereum blockchain.
	
	\subsection*{\fbox{G}}
		\subsubsection*{Ganache}
			Personal blockchain for the development of Ethereum \DJ{}Apps.
	
	\subsection*{\fbox{I}}
		\subsubsection*{Infura}
			Hosted Ethereum node cluster that lets users run application without requiring any setup of an Ethereum node or wallet.
		\subsubsection*{IPFS}
			Protocol and peer-to-peer network for storing and sharing data in a distributed file system. IPFS uses content-addressing to uniquely identify each file in a global namespace connecting all computing devices. 
		\subsubsection*{ipfs-mini API}
			A tiny module for querying IPFS that works in the browser and node.
	
	\subsection*{\fbox{J}}
		\subsubsection*{Javascript}
			Object-oriented and event-driven programming language, commonly used in client-side Web programming and interpreted by a browser.
		\subsubsection*{Jest}
	
	\subsection*{\fbox{M}}
		\subsubsection*{Mocha}
			JavaScript test framework for Node.js programs, featuring browser support, asynchronous testing, test coverage reports and use of any assertion library.
	
	\subsection*{\fbox{N}}
		\subsubsection*{Node Package Manager}
			It is the main software used to manage Node.js modules and packages, allowing to share code for recurring problems between Javascript developers.
		\subsubsection*{Node.js}
			A runtime environment used to run Javascript applications. It is based on an asynchronous I/O model that operates on events.
	
	\subsection*{\fbox{O}}
		\subsubsection*{Open Zeppelin}
			Framework developed for safe blockchain applications. Supplies useful tools to write, deploy and manage \DJ{}Apps.
      
	\subsection*{\fbox{R}}
		\subsubsection*{Ropsten testnet}
			Official and public Ethereum testnet. It's the testnet with most similar behaviour to the main net, but without any cost for writing operations. Typically used to test application intended to be later deployed on the main net.
      
	\subsection*{\fbox{S}}
		\subsubsection*{Serverless Framework}
			Free and open-source web framework written using Node.js. Serverless allows developers to build web, mobile and IoT applications with serverless architectures using AWS Lambda and similar serverless services.
		\subsubsection*{Solidity}
			High level, object-oriented programming language used to develop smart contract on blockchain-based platforms.
		
	\subsection*{\fbox{T}}
		\subsubsection*{Truffle}
			Framework used to test and develop code on a blockchain.
		\subsubsection*{Typescript}
			Open-source programming language. It is a strict syntactical superset of Javascript and adds optional static typing to the language. Typescript is designed for development of large applications and trans-compiles to Javascript. As Typescript is a superset of Javascript, existing Javascript programs are also valid Typescript programs.
			
	\subsection*{\fbox{V}}
		\subsubsection*{Visual Studio Code}
			Free source-code editor, which supports debugging, code writing helping tools, code refactoring and embedded Git.
	
	\subsection*{\fbox{Y}}
		\subsubsection*{Yargs}
			Library that provided an easy way to build interactive command line interfaces, by parsing arguments and managing commands. 
	