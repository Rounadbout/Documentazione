  \section[A]{\fbox{A}}
  	\subsection*{Acorn}
  	Libreria che permette di eseguire il parsing di codice sorgente Javascript\ped{\textit{G}}.
  	\subsection*{AirBnb (Javascript)}
  	Estensione del linguaggio Javascript\ped{\textit{G}}, che fornisce svariate funzionalità aggiuntive al costo di una maggiore rigidità nella scrittura e compilazione del codice.
	\subsection*{Ambiente di esecuzione}
	Insieme di risorse necessarie per l'esecuzione di un'istanza di prodotto\ped{\textit{G}} software. Ambienti diversi sono completamente indipendenti tra loro.
	\subsection*{Apache Kafka}
	È una piattaforma open source\ped{\textit{G}} di stream processing, scritta in Java\ped{\textit{G}} e Scala. Per stream processing si intende una forma limitata di parallelizzazione dei processi. Fornisce buone performance nella gestione di un elevato numero di operazioni in tempo reale, da parte di migliaia di clienti.
	\subsection*{API}
	Per application programming interface, si intende un insieme di procedure fornite ai programmatori per supportare lo sviluppo di prodotti\ped{\textit{G}} software. Queste interfacce di elaborazione sono esposte da particolari programmi, per consentire l’utilizzo delle loro funzionalità.
	\subsection*{API Rest}
	È un sistema di trasmissione dei dati che sfrutta HTTP ed i suoi protocolli.
	\subsection*{Apprendimento approfondito}
	Sottocategoria del machine learning\ped{\textit{G}}, che crea modelli\ped{\textit{G}} di apprendimento su più livelli, basandosi sull’analisi dei dati forniti da algoritmi di calcolo statistico.
	\subsection*{Apprendimento automatico}
	È un insieme di metodi di calcolo che sfruttano l’esperienza, ovvero un’analisi di casi già risolti, per migliorare le prestazioni e l’accuratezza dei risultati.
	\subsection*{Async-await}
	Caratteristica sintattica di molti linguaggi di programmazione (in particolare Javascript\ped{\textit{G}}), che consente di strutturare una funzione asincrona in modo simile ad una normale funzione sincrona.
	\subsection*{Attore}
	Nel contesto del diagramma dei casi d’uso\ped{\textit{G}}, definito nel linguaggio Unified Modeling Language (UML\ped{\textit{G}}), si riferisce ad un utente (persona o un altro sistema esterno) che svolge un ruolo nell’interazione con il sistema principale.
	\subsection*{AWS}
	Amazon Web Services è una piattaforma cloud\ped{\textit{G}} fornita dall’azienda Amazon, che offre numerosi servizi di cloud\ped{\textit{G}} computing.
\pagebreak
\section[B]{\fbox{B}}
	\subsection*{Back-end}
	Viene definito di back-end\ped{\textit{G}}, un programma con il quale l’utente interagisce indirettamente, e che permette l’effettivo funzionamento di un’eventuale interfaccia front-end\ped{\textit{G}}.
	\subsection*{Behavior driven development (BDD)}
	Si tratta di un’evoluzione del modello di sviluppo software: test-driven development, ovvero la scrittura di software basata su test automatici già definiti; tramite l’introduzione di un design orientato agli oggetti e ai domini\ped{\textit{G}}.
	\subsection*{Blockchain}
	È una struttura dati condivisa ed immutabile, gestita come un registro digitale di blocchi concatenati, ordinati cronologicamente e crittografati. Si tratta di fatto di un database distribuito, dal contenuto non modificabile e condiviso tra tutti i nodi di una rete. Il suo principale impiego è la gestione di criptovaluta ed eventuali transazioni di quest’ultima tra soggetti della rete.
	\subsection*{Bootstrap}
	Framework\ped{\textit{G}} front-end\ped{\textit{G}}, che fornisce una serie di strumenti per la creazione di siti e applicazioni Web. Principalmente funzionalità HTML, CSS e Javascript\ped{\textit{G}}.
	\subsection*{Branch}
	All’interno del contesto Git\ped{\textit{G}}, si definisce un branch\ped{\textit{G}} come un puntatore ad un commit\ped{\textit{G}}.
	\subsection*{Business Domain Language (BDL)}
	È l'output del processo di analisi dei documenti legati al Business Domain\ped{\textit{G}}. Ovvero un linguaggio i cui verbi, nomi e predicati sono stati estratti da suddetti documenti in base alla loro frequenza e rilevanza.
	\subsection*{Business Application Language (BAL)}
	È l’insieme dei suggerimenti BAL approvati dai designer dell’API\ped{\textit{G}}, generati dalla valutazione combinata di feature BDD\ped{\textit{G}}, Business Domain Language\ped{\textit{G}} ed eventualmente ontologia aziendale (opzionale).
\pagebreak
\section[C]{\fbox{C}}
	\subsection*{CaaS}
	Containers-as-a-Service, è un modello di servizio cloud\ped{\textit{G}} che permette agli utenti di distribuire e gestire le applicazioni tramite tecniche di astrazione basate su container, utilizzando i datacenter on premise o il cloud\ped{\textit{G}}.
	\subsection*{Capitolato}
	È un documento tecnico, generalmente allegato ad un contratto di appalto e parte integrante di quest’ultimo. Questi vi fa riferimento per definire le specifiche tecniche delle opere da realizzare e che andranno ad estinguersi per effetto del contratto stesso.
	\subsection*{Caso d'uso}
	Utilizzati all’interno degli Use Case Diagram di UML\ped{\textit{G}}, i casi d’uso sono un insieme di scenari (sequenze di azioni), che hanno in comune uno scopo finale (obiettivo) per un utente (attore\ped{\textit{G}}). Vengono impiegati per descrivere situazioni nelle quali il sistema viene utilizzato per soddisfare uno o più bisogni dell’utente.
	\subsection*{Chai}
	Libreria di asserzioni, che fornisce numerose asserzioni da poter utilizzare per testare il proprio codice.
	\subsection*{Ciclo di Deming}
	Metodo di gestione iterativo in quattro fasi, per il controllo e il miglioramento continuo dei processi e dei prodotti\ped{\textit{G}}.
	\subsection*{CLI}
	Command line program, è un programma che accetta specifiche linee di testo come input, per eseguire funzioni del sistema operativo o dell’applicativo a cui è riferito.
	\subsection*{Cloud (computing)}
	È un termine con cui ci si riferisce alla tecnologia che permette di elaborare, archiviare e trasmettere dati in rete, sfruttando servizi on demand. In particolare con “Cloud” generalmente ci si riferisce all’hardware remoto dove tali dati vengono elaborati.
	\subsection*{Cluster}
	Raggruppamento di più elaboratori o terminali che formano un insieme integrato, in grado di svolgere funzioni.
	\subsection*{Commit}
	Nel contesto dei Version Control Systems\ped{\textit{G}} (VCS\ped{\textit{G}}), un commit\ped{\textit{G}} è inteso come una conferma (registrata) delle ultime modifiche fatte ad un codice sorgente e l’aggiunta di queste ultime alla repository\ped{\textit{G}} di progetto.
	\subsection*{Committente}
	Figura fisica o giuridica, che ordina ad altri l’esecuzione di un lavoro o di una prestazione, per suo conto. Ha il potere decisionale e di spesa, relativo alla gestione del lavoro commissionato.
	\subsection*{Configstore}
	Libreria di Node.js\ped{\textit{G}} che consente di caricare e mantenere configurazioni software relative al progetto che si sta sviluppando.
	\subsection*{Continuous integration}
	Pratica impiegata nello sviluppo di prodotti\ped{\textit{G}} software. Consiste nell’allineamento frequente degli ambienti di lavoro degli sviluppatori verso l’ambiente\ped{\textit{G}} condiviso, al fine di verificare costantemente il corretto funzionamento del prodotto\ped{\textit{G}}.
	\subsection*{COVID-19}
	Malattia infettiva respiratoria causata dal virus SARS-CoV-2. Identificata in Italia durante il febbraio 2020, ha costretto il governo ad attuare misure di quarantena nazionale a partire da marzo 2020.
	\subsection*{Cucumber}
	È uno strumento software che supporta Behaviour-driven development\ped{\textit{G}}, attraverso l’analizzatore di linguaggio Gherkin\ped{\textit{G}}. È in grado di tradurre la gestione del software in un linguaggio logico accessibile, per eventuali clienti. 
\pagebreak
\section[D]{\fbox{D}}
	\subsection*{Dashboard}
	Interfaccia che permette di monitorare in tempo reale l’andamento dei report e delle metriche aziendali.
	\subsection*{\DJ{}App}
	Una decentralized application, è un’applicazione con il proprio codice di back-end\ped{\textit{G}} in esecuzione su una rete peer-to-peer decentralizzata, come quella utilizzata per le blockchain\ped{\textit{G}}.
	\subsection*{Data-set}
	Una collezione di dati, strutturati in forma relazionale.
	\subsection*{Definition of Done (DoD)}
	La Definition of Done indica quando tutti i requisiti, o criteri di accettazione che un prodotto\ped{\textit{G}} software deve soddisfare, sono verificati. Quando tale definizione è confermata, il software è pronto per essere rilasciato e consegnato al cliente.
	\subsection*{Demo}
	Versione di prova di un prodotto software.
	\subsection*{Deploy}
	È la consegna o il rilascio al cliente, con opportuna installazione e messa in funzione, di un’applicazione o prodotto\ped{\textit{G}} software. Si presume che un prodotto\ped{\textit{G}} rilasciato soddisfi tutti i requisiti principali previsti.
	\subsection*{Design pattern}
	Nel contesto UML\ped{\textit{G}}, sono soluzioni progettuali a problemi ricorrenti. Hanno lo scopo di affrontare problematiche che riguardano la composizione di classi ed oggetti, consentire il ri-utilizzo degli oggetti esistenti, sfruttare l’ereditarietà e l’aggregazione.
	\subsection*{Domain}
	Contesto in cui un’applicazione software, un problema o un’attività, opera. Con particolare riferimento alla natura ed al significato delle informazioni che devono essere manipolate.
	\subsection*{Dominio}
	\emph{Sinonimo}: vedi Domain\ped{\textit{G}}.
	\subsection*{Driver (test)}
	Sono moduli che agiscono come temporaneo sostituto per un modulo\ped{\textit{G}} chiamante, dando gli stessi output del prodotto\ped{\textit{G}} vero e proprio, oppure interagendo con sistemi esterni.
\pagebreak
\section[E]{\fbox{E}}
	\subsection*{Elastic Container Service}
	È un servizio di cloud\ped{\textit{G}} computing, incluso nella AWS\ped{\textit{G}}, che gestisce container e permette agli sviluppatori di eseguire applicazioni in cloud\ped{\textit{G}} senza dover configurare l’ambiente di esecuzione\ped{\textit{G}}.
	\subsection*{Elastic Beanstalk}
	Servizio AWS\ped{\textit{G}} che consente di effettuare il deploy\ped{\textit{G}} in cloud\ped{\textit{G}} di applicazioni scritte in vari linguaggi, e la gestione delle risorse necessarie alla loro esecuzione.
	\subsection*{Environment}
	\emph{Sinonimo}: vedi ambiente di esecuzione\ped{\textit{G}}.
	\subsection*{ESLint}
	È uno strumento di analisi del codice statico, utilizzato per gestire modelli problematici identificati nel codice Javascript\ped{\textit{G}}.
	\subsection*{Ether (ETH)}
	Criptovaluta utilizzata nella piattaforma Ethereum\ped{\textit{G}}.
	\subsection*{Ethereum}
	È una piattaforma open source\ped{\textit{G}} globale per \DJ{}App\ped{\textit{G}}, definita inoltre come “blockchain\ped{\textit{G}}”. Un’applicazione viene eseguita tramite uno smart contract\ped{\textit{G}} che, sotto pagamento in Ether\ped{\textit{G}}, sfrutta la potenza computazionale della rete per raggiungere il suo scopo. La piattaforma, come le altre blockchain\ped{\textit{G}}, consente quindi lo scambio di denaro.
	\subsection*{Ethers.js}
	Libreria che fornisce funzionalità per semplificare la codifica di wallet\ped{\textit{G}} Ethereum\ped{\textit{G}}, la comunicazione con la blockchain\ped{\textit{G}} e con i relativi smart contract\ped{\textit{G}}.
	\subsection*{EVM}
	Ethereum\ped{\textit{G}} Virtual Machine, nel contesto di blockchain\ped{\textit{G}}, è il centro di calcolo virtuale che permette l’esecuzione degli smart contract\ped{\textit{G}}.
	\subsection*{Eventi (Ethereum)}
	Sono dei trigger asincroni con all’interno dei dati. Questi eventi vengono emessi da smart contract\ped{\textit{G}} per comunicare all’interno della blockchain\ped{\textit{G}} e con l’interfaccia utente.
	\subsection*{Excel}
	Programma per creare e gestire fogli elettronici, distribuito da Microsoft, in grado di fornire funzioni di calcolo, impaginazione, grafica e testi.
\pagebreak
\section[F]{\fbox{F}}
	\subsection*{FaaS}
	Function-as-a-Service, è una tipologia di servizio di cloud\ped{\textit{G}} computing, che permette un alto livello di astrazione nella creazione di applicazioni, scalabilità e costi proporzionati all’uso effettivo dei servizi.
	\subsection*{Framework}
	Piattaforma che funge da strato intermedio tra un sistema operativo ed il software che lo utilizza.
	\subsection*{Front-end}
	Parte di un programma visibile all’utente, con cui egli può interagire direttamente. È responsabile dell’acquisizione dei dati in input e della loro elaborazione, al fine di renderli utilizzabili dal back-end\ped{\textit{G}}.
\pagebreak
\section[G]{\fbox{G}}
	\subsection*{Ganache}
	Blockchain\ped{\textit{G}} personale per lo sviluppo di \DJ{}App\ped{\textit{G}} in Ethereum\ped{\textit{G}}.  
	\subsection*{GanttProject}
	Software di gestione dei progetti, basato su Java, la cui funzione principale è la possibilità di tracciare diagrammi di Gantt.
	\subsection*{Gas limit}
	Nel contesto Ethereum\ped{\textit{G}}, è la quantità massima di gas (ovvero mole di lavoro tradotta in Ether\ped{\textit{G}}) che si è disposti a pagare per una transazione.
	\subsection*{Gas price}
	Nel contesto Ethereum\ped{\textit{G}}, si riferisce alla quantità di ETH\ped{\textit{G}} che si è disposti a spendere per ogni unità di gas. Misurato in “Gwei” (sottomultiplo di Ether\ped{\textit{G}}).
	\subsection*{Gherkin}
	Formato per le specifiche di Cucumber\ped{\textit{G}}.
	\subsection*{Git}
	È un software di controllo versione\ped{\textit{G}} distribuito ed utilizzabile da interfaccia a riga di comando.
	\subsection*{GitHub}
	È un servizio di hosting per progetti software, che implementa lo strumento di controllo versione\ped{\textit{G}} (VCS\ped{\textit{G}}) distribuito Git\ped{\textit{G}}.
	\subsection*{Gmail}
	Servizio di posta elettronica fornito gratuitamente da Google.
	\subsection*{Grafana}
	È un software open source\ped{\textit{G}} che consente di generare grafici e dashboard\ped{\textit{G}} per il monitoraggio di ambienti e di sistemi.
\pagebreak
\section[H]{\fbox{H}}
	\subsection*{Highlights}
	Momenti salienti di un evento specifico. 
\pagebreak
\section[I]{\fbox{I}}
	\subsection*{IDE}
	È un software progettato per la realizzazione di applicazioni, che aggrega strumenti di sviluppo comuni in un’unica interfaccia utente grafica.
	\subsection*{Infura}
	Cluster\ped{\textit{G}} di nodi Ethereum\ped{\textit{G}}, che permette agli utenti di eseguire un’applicazione fornita, senza richiedere loro la configurazione di un nodo Ethereum\ped{\textit{G}} o di un wallet\ped{\textit{G}}.  
	\subsection*{Integrazione continua}
	\emph{Sinonimo}: vedi Continuous integration\ped{\textit{G}}.
	\subsection*{Inquirer.js}
	Modulo\ped{\textit{G}} Node.js\ped{\textit{G}}, che fornisce un'interfaccia a riga di comando interattiva.
	\subsection*{IPFS}
	Protocollo e rete peer-to-peer per l’archiviazione e condivisione di dati in un file system distribuito.
	\subsection*{IPFS-mini API}
	Modulo di dimensioni ridotte, utilizzato per comunicare con nodi IPFS{\textit{G}}. 
	\subsection*{IoT}
	Internet of Things, è un termine riferito all’estensione di Internet al mondo degli oggetti e dei luoghi fisici. Eseguita tramite chip e sensori inseriti al loro interno, che li consentono di interagire tra loro e con la realtà circostante.
	\subsection*{Issue}
	All'interno del contesto GitHub\ped{\textit{G}}, le issue\ped{\textit{G}} sono uno strumento virtuale utilizzato per tenere traccia di compiti da portare a termine, aggiornamenti da implementare o bug da risolvere di uno specifico progetto.
\pagebreak
\section[J]{\fbox{J}}
	\subsection*{Java}
	È un linguaggio di programmazione messo a punto per il Web e poi impiegato per lo sviluppo di programmi utilizzabili su qualsiasi tipo di computer.
	\subsection*{Javascript}
	Linguaggio di programmazione orientato agli oggetti ed eventi. Viene comunemente utilizzato nella programmazione Web lato client ed interpretato da un browser.
	\subsection*{Jest}
	Framework\ped{\textit{G}} di test, utilizzato per test di codice Javascript\ped{\textit{G}}, Typescript\ped{\textit{G}} e Node.js\ped{\textit{G}}. 
	\subsection*{JSON (Javascript Object Notation)}
	Si tratta di un formato adatto all’interscambio di dati fra applicazioni client/server. Basato sul linguaggio Javascript\ped{\textit{G}}. 
\pagebreak
\section[L]{\fbox{L}}
	\subsection*{Lambda}
	Piattaforma di cloud computing\ped{\textit{G}} serverless\ped{\textit{G}}, guidata dagli eventi\ped{\textit{G}} e fornita da Amazon come parte degli Amazon Web Services\ped{\textit{G}}.
	\subsection*{\LaTeX{}}
	È un linguaggio di markup (orientato alla rappresentazione/al layout del testo), che consente la composizione tipografica, la preparazione e l'elaborazione dei documenti.
	\subsection*{Linguaggio naturale}
	Linguaggio solitamente usato nella comunicazione fra individui di un gruppo sociale che lo condivide.
	\subsection*{Log}
	File costituito da un elenco cronologico delle attività svolte da un sistema operativo, database, o da altri programmi. Utilizzato per eventuali successive operazioni di verifica.
\pagebreak
\section[M]{\fbox{M}}
	\subsection*{Machine learning}
	\emph{Sinonimo}: vedi apprendimento automatico\ped{\textit{G}}.
	\subsection*{Microsoft Teams}
	È uno strumento di collaborazione basato su messaggistica istantanea, che permette di creare gruppi in grado di lavorare insieme e condividere le informazioni attraverso uno spazio comune.
	\subsection*{MIT}
	Licenza permissiva di software libero, creata dal Massachusetts Institute of Technology da cui deriva il nome. Si tratta di una licenza riutilizzabile nel software proprietario, qualora sia distribuita con tale software.
	\subsection*{Mnemonic phrase}
	Gruppo di parole segreto, che rappresenta un Wallet\ped{\textit{G}}. Quando utilizzate in sequenza corretta, queste permettono l’accesso alla criptovaluta contenuta in tale Wallet\ped{\textit{G}}.
	\subsection*{Mocha}
	Framework per il test di codice Javascript\ped{\textit{G}}, orientato verso la programmazione asincrona.
	\subsection*{Modulo}
	Elemento di programma che costituisce un’unità di sistema modulare.
	\subsection*{Modello (machine learning)}
	Definito come modello probabilistico generale dello spazio delle occorrenze, questo viene utilizzato da una macchina per produrre previsioni quanto più accurate possibile di risultato, quando sottoposta a nuovi casi.
\pagebreak
\section[N]{\fbox{N}}
	\subsection*{NoSQL}
	Sistema software (generalmente database) che archivia dati senza utilizzare il modello relazionale.
	\subsection*{Node.js}
	È un framework\ped{\textit{G}} utilizzato per scrivere applicazioni in Javascript\ped{\textit{G}}, lato server, basato su un modello di I/O asincrono che opera su eventi\ped{\textit{G}}.
	\subsection*{Node Package Manager (NPM)}
	È il principale software utilizzato per gestire moduli e package di Node.js\ped{\textit{G}}, consentendo di condividere codice per problemi ricorrenti tra gli sviluppatori Javascript\ped{\textit{G}}.
\pagebreak
\section[O]{\fbox{O}}
	\subsection*{Open API}
	È una specifica per i file di interfaccia leggibili da macchine per descrivere, produrre, consumare e visualizzare servizi web RESTful.
	\subsection*{Open source}
	Si definisce open source, un software non protetto da copyright e liberamente modificabile dagli utenti.
	\subsection*{Open Zeppelin}
	Framework\ped{\textit{G}} utilizzato per applicazioni blockchain\ped{\textit{G}} sicure. Fornisce strumenti per scrivere, rilasciare e gestire applicazioni decentralizzate (\DJ{}App\ped{\textit{G}}).
	\subsection*{Orange Canvas}
	Si tratta di un toolkit\ped{\textit{G}} utilizzabile per visualizzazione di dati, machine learning\ped{\textit{G}} e data mining. Consente una programmazione visuale, in grado di generare programmi attraverso la manipolazione di elementi grafici (widgets), invece del classico codice testuale.
\pagebreak
\section[P]{\fbox{P}}
	\subsection*{Plug-in}
	Modulo\ped{\textit{G}} aggiuntivo (estensione) di un programma, utilizzato per aumentarne le funzioni.
	\subsection*{Python}
	È un linguaggio di programmazione open source\ped{\textit{G}} e ad alto livello, in grado di implementare diversi paradigmi, tra cui l’orientamento ad oggetti. Ha un forte controllo dei tipi, che viene eseguito a runtime.
	\subsection*{Private key}
	Numero casuale a 256 Bit e 64 caratteri esadecimali, conosciuto solo all’utente che l’ha creato, attraverso una libreria o una funzione di hash. Viene utilizzato per firmare transazioni sulla blockchain\ped{\textit{G}} Ethereum\ped{\textit{G}}.
	\subsection*{Prodotto (software)}
	È l’unione ordinata di parti distinte, eseguita secondo regole rigorose. Può essere rappresentato da qualsiasi cosa, in grado di soddisfare un bisogno o un’esigenza.
	\subsection*{Product baseline}
	Si tratta di documentazione tecnica, utilizzata per descrivere la configurazione di un prodotto\ped{\textit{G}} software in un determinato momento del suo ciclo di vita. Illustra la baseline architetturale del prodotto\ped{\textit{G}}, in coerenza con la Technology Baseline\ped{\textit{G}}.
	\subsection*{Proponente}
	Figura fisica o giuridica, che presenta una proposta di capitolato\ped{\textit{G}}, legata ad un progetto da realizzare.
	\subsection*{Pull}
	Nel contesto Git\ped{\textit{G}}, il comando di pull\ped{\textit{G}} recupera dalla repository\ped{\textit{G}} remota tutte le modifiche effettuate da altre persone e le combina con le proprie modifiche effettuate, nella repository\ped{\textit{G}} locale.
	\subsection*{Push}
	Nel contesto Git\ped{\textit{G}}, il comando push\ped{\textit{G}} trasferisce tutte le modifiche confermate (commit\ped{\textit{G}}) della repository\ped{\textit{G}} locale, nella repository\ped{\textit{G}} remota.
\pagebreak
\section[R]{\fbox{R}}
	\subsection*{Redattore}
	Colui il quale si occupa della stesura di un documento.
	\subsection*{Regressione lineare (RL)}
	L'analisi di regressione lineare è una tecnica che permette di analizzare la relazione lineare tra una variabile dipendente (variabile di risposta) e una o più variabili indipendenti (predittori).
	\subsection*{Repository}
	Archivio o sito Web nel quale sono raccolti e conservati dati ed informazioni in formato digitale. Questi sono corredati da descrizioni (metadati), strutturate sotto forma di tabelle relazionali.
	\subsection*{Ropsten}
	Rete di test ufficiale e pubblica utilizzata per Ethereum\ped{\textit{G}}. Ha un funzionamento del tutto simile a quello della rete principale (main net), ma senza alcun costo associato alle operazioni di scrittura. Utilizzata generalmente per test di applicazioni che si intende poi rilasciare sulla rete principale.
\pagebreak
\section[S]{\fbox{S}}
	\subsection*{Sage Maker}
	È una piattaforma di machine learning\ped{\textit{G}} cloud\ped{\textit{G}}, che consente agli sviluppatori di creare, formare e distribuire modelli di machine learning\ped{\textit{G}} nel cloud\ped{\textit{G}}.
	\subsection*{Serverless}
	Framework\ped{\textit{G}} web open source\ped{\textit{G}}, sviluppato per la creazione di applicazioni su AWS Lambda\ped{\textit{G}}.
	\subsection*{Serverless (caratteristica)}
	Caratteristica di un network la cui gestione non viene incentrata su dei server, ma è invece dislocata tra i vari utenti che utilizzano il network stesso.
	\subsection*{Skype}
	Software proprietario di Microsoft, che fornisce servizi di messaggistica istantanea, VoIP e trasferimento di file.
	\subsection*{Slack}
	È uno strumento collaborativo che permette la creazione di aree di lavoro, mediante canali che trattano un particolare argomento.
	\subsection*{Smart contract}
	Sono protocolli informatici per facilitare, attuare o verificare, la negoziazione o l’esecuzione di un contratto digitale. Possiedono lo stesso valore di un contratto legale e fisico, ma senza la necessità di un garante esterno. Si tratta di codice software che viene eseguito sulla EVM\ped{\textit{G}}.
	\subsection*{Solidity}
	È un linguaggio di programmazione ad alto livello orientato agli oggetti, utilizzato nello sviluppo di smart contract\ped{\textit{G}} in piattaforme basate su blockchain\ped{\textit{G}}.
	\subsection*{Staging}
	Fase dello sviluppo software dove il prodotto\ped{\textit{G}} viene assemblato e viene effettuata una simulazione di un server live, in cui testare il corretto funzionamento dell'intero sistema.
	\subsection*{Strumento di controllo versione}
	È un sistema software che registra tutte le modifiche avvenute ad un insieme di file. Permettono la condivisione di file e modifiche, offrendo anche funzionalità di merging, backup e tracciamento di tali modifiche.
	\subsection*{Stub (test)}
	I test stub sono programmi in grado di simulare il comportamento di componenti o moduli software del prodotto\ped{\textit{G}} che si vuole testare.
	\subsection*{Support Vector Machine (SVM)}
	Modello di apprendimento supervisionato, associato ad algoritmi di apprendimento per la regressione\ped{\textit{G}} e la classificazione.
	\subsection*{Swagger}
	È un framework\ped{\textit{G}} open source\ped{\textit{G}}, dotato di un gran numero di strumenti per aiutare sviluppatori a progettare, costruire e documentare servizi web.
	\subsection*{Swift}
	Linguaggio di programmazione orientato agli oggetti, utilizzato per lo sviluppo di sistemi e applicativi Apple.
\pagebreak
\section[T]{\fbox{T}}
	\subsection*{Technology baseline}
	Definisce e motiva le tecnologie, i framework\ped{\textit{G}} e le librerie selezionate per lo sviluppo del prodotto\ped{\textit{G}}; Inoltre ne dimostra l’adeguatezza e il grado di integrazione, tramite un Proof of Concept coerente con gli obiettivi.
	\subsection*{Telegram}
	È un sevizio di messaggistica istantanea e chiamate VoIP.
	\subsection*{Template}
	In informatica, modello predefinito che consente di creare o inserire contenuti di diverso tipo in un documento o in una pagina Web.
	\subsection*{TestRPC}
	Emulatore di blockchain\ped{\textit{G}}, che permette di effettuare chiamate alla blockchain\ped{\textit{G}} senza l'onere di dover utilizzare un vero e proprio nodo Ethereum\ped{\textit{G}} (terminale sulla rete).
	\subsection*{Toolkit}
	Insieme di strumenti software, generalmente librerie, utilizzati per facilitare ed uniformare lo sviluppo di applicazioni complesse.
	\subsection*{Training}
	Nel contesto del machine learning\ped{\textit{G}}, effettuare il training\ped{\textit{G}} di un modello\ped{\textit{G}} significa imparare (a determinare) valori adeguati per tutti i pesi e lo sfasamento derivante da esempi già classificati.
	\subsection*{Travis-CI}
	Servizio di integrazione continua\ped{\textit{G}} utilizzato per creare e testare progetti software ospitati su GitHub\ped{\textit{G}}.
	\subsection*{Truffle}
	Framework\ped{\textit{G}} per lo sviluppo ed il testing di codice di una blockchain\ped{\textit{G}}. 
	\subsection*{Typescript}
	Linguaggio di programmazione open source\ped{\textit{G}} che punta ad estendere Javascript\ped{\textit{G}}, aggiungendo tipi al linguaggio.
\pagebreak
\section[U]{\fbox{U}}
	\subsection*{UML}
	Unified Modelling Language, è un linguaggio  standard per la documentazione e descrizione di un sistema o parti di esso attraverso dei modelli.
	\subsection*{Upfront}
	Il pagamento upfront si basa sulla consegna in anticipo di una somma, da parte del soggetto avvantaggiato nella stipula di uno scambio, verso il soggetto svantaggiato. In modo da riequilibrare finanziariamente lo scambio.
\pagebreak
\section[V]{\fbox{V}}
	\subsection*{Verbale}
	Documento redatto da un segretario, allo scopo di attestare discorsi, dichiarazioni o fatti, emersi durante un colloquio interno o esterno.
	\subsection*{Version Control System (VCS)}
	\textit{Sinonimo}: vedi Strumento di controllo versione\ped{\textit{G}}.
\pagebreak
\section[W]{\fbox{W}}
	\subsection*{Wallet}
	Portafoglio virtuale con il quale è possibile effettuare pagamenti online.
	\subsection*{Wireframe}
	Tipologia di rappresentazione in computer grafica di oggetti tridimensionali. Nel Web design rappresenta il modello iniziale di rappresentazione di un sito Web.
\pagebreak
\section[Y]{\fbox{Y}}
	\subsection*{Yargs}
	Libreria che fornisce supporto per l’interpretazione di stringhe JSON\ped{\textit{G}} denominate “optstring”.
\pagebreak
\section[Z]{\fbox{Z}}
	\subsection*{Zoom}
	Una piattaforma Web che consente di creare videoconferenze, o in generale videochiamate, tra soggetti remoti.
