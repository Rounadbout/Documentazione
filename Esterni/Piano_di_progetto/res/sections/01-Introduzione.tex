\section{Introduzione}
\subsection{Scopo del documento}
Questo documento viene redatto con lo scopo di presentare la pianificazione del gruppo \Gruppo{} per lo sviluppo del progetto \NomeProgetto{}. Verranno inoltre presentate un'analisi dei rischi e un'analisi dei costi riguardanti lo sviluppo del progetto.\\
Nel dettaglio i punti definiti nel documento sono:
\begin{itemize}
	\item analisi dei rischi relativi allo sviluppo del progetto;
	\item breve analisi del modello di sviluppo del progetto;
	\item pianificazione dettagliata dei tempi e delle attività;
	\item stima preventiva dell'utilizzo delle risorse disponibili.
\end{itemize}
\subsection{Scopo del prodotto}
Si svuole sviluppare una piattaforma cloud\ped{\textit{G}} che consenta agli sviluppatori di fare il deploy\ped{\textit{G}} di funzioni Javascript\ped{\textit{G}} e gestisca il pagamento per la loro esecuzione tramite la piattaforma Ethereum\ped{\textit{G}}.\\
Il prodotto\ped{\textit{G}} finale prevede quindi l'integrazione di due tecnologie, Serverless\ped{\textit{G}} e Ethereum\ped{\textit{G}}.\\
Il lato Serverless\ped{\textit{G}} si occupa dell'esecuzione delle funzioni fornite dagli sviluppatori. Tali funzioni vengono salvate ed eseguite in un servizio cloud\ped{\textit{G}} esterno, quale Amazon Web Services\ped{\textit{G}}.\\
La richiesta di utilizzo di una funzione e il successivo pagamento vengono invece gestiti tramite la piattaforma Ethereum\ped{\textit{G}} sfruttando gli smart contract\ped{\textit{G}}. Il pagamento viene effettuato in ETH\ped{\textit{G}}.\\
Una percetuale significativa di ogni pagamento viene riservata agli amministratori del servizio.\\
Lo sviluppatore e l'utente finale interagiscono con il prodotto\ped{\textit{G}} tramite una CLI\ped{\textit{G}} che prevede alcuni comandi intuitivi che permettono di usufruire di tutte le funzionalità fornite dalla piattaforma.
\subsection{Glossario}
Al fine di evitare possibili ambiguità, i termini tecnici utilizzati nei documenti formali vengono chiariti ed approfonditi nel \textit{Glossario}. Per facilitare la lettura, i termini presenti in tale documento sono contrassegnati in tutto il resto della documentazione da una 'G' a pedice.
\subsection{Riferimenti}
\subsubsection{Riferimenti Normativi}
\begin{itemize}
	\item \textbf{Norme di progetto}: \textit{Norme di progetto v1.0.0};
	\item \textbf{Capitolato\ped{\textit{G}} d'appalto Etherless}:\\\url{https://www.math.unipd.it/~tullio/IS-1/2019/Progetto/C2.pdf};
	\item \textbf{Regolamento organigramma di gruppo e specifica tecnico-economica}:\\\url{https://www.math.unipd.it/~tullio/IS-1/2019/Progetto/RO.html}.
\end{itemize}
\subsubsection{Riferimenti Informativi}
\begin{itemize}
	\item \textbf{Dispense del corso "Ingegneria del Software" sulla gestione di progetto}:\\\url{https://www.math.unipd.it/~tullio/IS-1/2019/Dispense/L06.pdf};
	\item \textbf{Software Engineering (10th edition) - Ian Sommerville};
	\item \textbf{Guide to the Software Engineering Body of Knowledge (v3) - IEEE Computer Society}.
\end{itemize}
\subsection{Scadenze}
Il gruppo \Gruppo{} ha deciso di rispettare le seguenti scadenze:
\begin{itemize}
	\item \textbf{Revisione dei Requisiti}: 2020-04-13;
	\item \textbf{Revisione di Progettazione}: 2020-05-11;
	\item \textbf{Revisione di Qualifica}: 2020-06-11;
	\item \textbf{Revisione di Accettazione}: 2020-07-13.
\end{itemize}
La pianificazione per lo svolgimento del progetto si basa su queste scadenze.
