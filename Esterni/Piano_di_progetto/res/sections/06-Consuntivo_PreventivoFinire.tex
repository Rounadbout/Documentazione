\section{Consuntivo di periodo e preventivo a finire}
In questa sezione verranno indicate le spese, totali e per ruolo, sostenute al termine di ciascuna fase.
Il bilancio presentato potrà essere:
\begin{itemize}
	\item \textbf{positivo} se il totale preventivato è superiore ai valori del consuntivo;
	\item \textbf{pari} se il totale preventivato è pari ai valori del consuntivo;
	\item \textbf{negativo} se il totale preventivato è inferiore ai valori del consuntivo.
\end{itemize}
Verrà inoltre presentato un preventivo a finire che terrà conto dei soli periodi rendicontati.
\subsection{Periodo di Analisi}
	\subsubsection{Consuntivo di periodo}
	\rowcolors{2}{lightRowColor}{darkRowColor}
	\begin{longtable}{
		>{\centering}p{0.25\textwidth}
		>{\centering}p{0.08\textwidth}
		>{\centering}p{0.08\textwidth}
		>{\centering}p{0.15\textwidth}
		>{\centering\arraybackslash}p{0.15\textwidth} }

		\coloredTableHead
		\textbf{\color{white}Ruolo} &
		\textbf{\color{white}Ore} &
		\textbf{\color{white}Delta ore} &
		\textbf{\color{white}Costo in \euro{}} &
		\textbf{\color{white}Delta costo}
		\tabularnewline
		\endhead

		% Contenuto della tabella
		% Ruolo & Ore & Costo\\
		Responsabile    & 28 & -2 & 840,00 & -60 \\
		Amministratore  & 70 & +0 & 1.400,00 & +0,00 \\
		Analista        & 63 & +3 & 1.575,00 & +75,00 \\
		Progettista     & 20 & +0 & 440,00 & +0,00 \\
		Programmatore   & - & - & - & - \\
		Verificatore    & 74 & +4 & 1.110,00 & +60,00 \\
		\textbf{Totale Effettivo} & \multicolumn{2}{c}{\textbf{255}} & \multicolumn{2}{c}{\textbf{5365,00}} \\
		\textbf{Delta} & \multicolumn{2}{c}{\textbf{+4}} & \multicolumn{2}{c}{\textbf{+75,00}} \\

		\rowcolor{white}\caption{Consuntivo per il periodo di Analisi}	\\

	\end{longtable}
	\subsubsection{Conclusione}
		Come riportato nella tabella del Consuntivo per il periodo di Analisi, il preventivo orario per i ruoli di Amministratore e Progettista si è rivelato sufficiente per svolgere il lavoro previsto; invece si è rivelato necessario dedicare più ore lavorative rispetto a quanto preventivato per i ruoli di Analista e Verificatore, mentre è stato impiegato un monte ore ridotto per il ruolo di Responsabile di Progetto. Di seguito sono riportate le motivazioni:
		\begin{itemize}
			\item \textbf{Responsabile di Progetto:} sono state impiegate meno ore rispetto a quelle previste data la minore difficoltà di stesura del piano di progetto e di pianificazione del lavoro, rispetto a quanto previsto;
			\item \textbf{Analista:} alcuni requisiti hanno presentato delle difficoltà di comprensione, con un conseguente aumento del monte ore necessario per la loro comprensione e stesura all'interno dell'\textit{AdR{}};
			\item \textbf{Verificatore:} l'aggiornamento delle \textit{NdP{}} e l'applicazione scorretta di alcune norme causata dall'inesperienza dei membri del gruppo, ha portato ad un aumento delle ore spese per questo ruolo.
		\end{itemize}

\subsection{Preventivo a finire}
Il bilancio finale è negativo.
Non si rendono necessarie misure di mitigazione, nonostante l'aumento del monte ore rispetto al preventivo, in quanto la fase di Analisi non è rendicontata.
