\section{Consuntivo di periodo e preventivo a finire}
In questa sezione verranno indicate le spese, totali e per ruolo, sostenute al termine di ciascuna fase.
Il bilancio presentato potrà essere:
\begin{itemize}
	\item \textbf{positivo:} se il totale preventivato è superiore ai valori del consuntivo;
	\item \textbf{pari:} se il totale preventivato è pari ai valori del consuntivo;
	\item \textbf{negativo:} se il totale preventivato è inferiore ai valori del consuntivo.
\end{itemize}
Verrà inoltre presentato un preventivo a finire che terrà conto dei soli periodi rendicontati.

\subsection{Periodo di Analisi}
	\subsubsection{Consuntivo di periodo}
	\rowcolors{2}{lightRowColor}{darkRowColor}
	\begin{longtable}{
		>{\centering}p{0.25\textwidth}
		>{\centering}p{0.08\textwidth}
		>{\centering}p{0.08\textwidth}
		>{\centering}p{0.15\textwidth}
		>{\centering\arraybackslash}p{0.15\textwidth} }

		\coloredTableHead
		\textbf{\color{white}Ruolo} &
		\textbf{\color{white}Ore} &
		\textbf{\color{white}Delta ore} &
		\textbf{\color{white}Costo in \euro{}} &
		\textbf{\color{white}Delta costo}
		\tabularnewline
		\endhead

		% Contenuto della tabella
		% Ruolo & OreEffettive & DeltaOre & Costo & DeltaCosto \\
		Responsabile    & 28 & -2 &   840,00 & -60 \\
		Amministratore  & 70 & +0 & 1.400,00 & +0,00 \\
		Analista        & 63 & +3 & 1.575,00 & +75,00 \\
		Progettista     & 20 & +0 &   440,00 & +0,00 \\
		Programmatore   & -  & -  & -        & - \\
		Verificatore    & 74 & +4 & 1.110,00 & +60,00 \\
		\textbf{Totale Effettivo} & \multicolumn{2}{c}{\textbf{255}} & \multicolumn{2}{c}{\textbf{5365,00}} \\
		\textbf{Delta} & \multicolumn{2}{c}{\textbf{+4}} & \multicolumn{2}{c}{\textbf{+75,00}} \\

		\rowcolor{white}\caption{Consuntivo per il periodo di Analisi}	\\

	\end{longtable}
	\subsubsection{Conclusione}
		Come riportato nella tabella del Consuntivo per il periodo di Analisi, il preventivo orario per i ruoli di Amministratore e Progettista si è rivelato sufficiente per svolgere il lavoro previsto; invece si è rivelato necessario dedicare più ore lavorative rispetto a quanto preventivato per i ruoli di Analista e Verificatore, mentre è stato impiegato un monte ore ridotto per il ruolo di Responsabile di Progetto. Di seguito sono riportate le motivazioni:
		\begin{itemize}
			\item \textbf{Responsabile di Progetto:} sono state impiegate meno ore rispetto a quelle previste data la minore difficoltà di stesura del piano di progetto e di pianificazione del lavoro, rispetto a quanto previsto;
			\item \textbf{Analista:} alcuni requisiti hanno presentato delle difficoltà di comprensione, con un conseguente aumento del monte ore necessario per la loro comprensione e stesura all'interno dell'\textit{AdR{} 1.0.0};
			\item \textbf{Verificatore:} l'aggiornamento delle \textit{NdP{}} e l'applicazione scorretta di alcune norme causata dall'inesperienza dei membri del gruppo, ha portato ad un aumento delle ore spese per questo ruolo.
		\end{itemize}

	\subsubsection{Preventivo a finire}
	Il bilancio finale è negativo. Nonostante ciò, non riteniamo necessario attuare misure di mitigazione o modifiche alla pianificazione dei periodi successivi, in quanto, oltre al fatto che il periodo di Analisi non è rendicontato, siamo riusciti ad individuare le situazioni e le motivazioni che hanno portato a richiedere ore di lavoro aggiuntive rispetto a quelle preventivate. Ciascun componente del gruppo si impegnerà per cercare di evitare che situazioni di questo tipo si possano ripetere nei periodi successivi.

\subsection{Periodo di Consolidamento dei Requisiti}
	\subsubsection{Consuntivo di periodo}
	\rowcolors{2}{lightRowColor}{darkRowColor}
	\begin{longtable}{
		>{\centering}p{0.25\textwidth}
		>{\centering}p{0.08\textwidth}
		>{\centering}p{0.08\textwidth}
		>{\centering}p{0.15\textwidth}
		>{\centering\arraybackslash}p{0.15\textwidth} }

		\coloredTableHead
		\textbf{\color{white}Ruolo} &
		\textbf{\color{white}Ore} &
		\textbf{\color{white}Delta ore} &
		\textbf{\color{white}Costo in \euro{}} &
		\textbf{\color{white}Delta costo}
		\tabularnewline
		\endhead

		% Contenuto della tabella
		% Ruolo & OreEffettive & DeltaOre & Costo & DeltaCosto \\
		Responsabile    & 5  & +0 & 150,00 & +0,00 \\
		Amministratore  & 6  & +0 & 120,00 & +0,00 \\
		Analista        & 10 & +0 & 250,00 & +0,00 \\
		Progettista     & -  & -  & -       & +0,00 \\
		Programmatore   & -  & -  & -       & +0,00 \\
		Verificatore    & 15 & +0 & 225,00 & +0,00 \\
		\textbf{Totale Effettivo} & \multicolumn{2}{c}{\textbf{36}} & \multicolumn{2}{c}{\textbf{745,00}} \\
		\textbf{Delta} & \multicolumn{2}{c}{\textbf{+0}} & \multicolumn{2}{c}{\textbf{+0,00}} \\

		\rowcolor{white}\caption{Consuntivo per il periodo di Consolidamento dei Requisiti}	\\

	\end{longtable}
	\subsubsection{Conclusione}
		Il totale delle ore preventivate per il periodo di Consolidamento dei Requisiti si è rivelato corretto per svolgere il lavoro pianificato.
	\subsubsection{Preventivo a finire}
	Il bilancio finale è pari. Non si rendono quindi necessarie misure di mitigazione o modifiche alla pianificazione dei periodi successivi.
	
\subsection{Periodo di Progettazione Architetturale}
	\subsubsection{Consuntivo di periodo}
	\rowcolors{2}{lightRowColor}{darkRowColor}
	\begin{longtable}{
		>{\centering}p{0.25\textwidth}
		>{\centering}p{0.08\textwidth}
		>{\centering}p{0.08\textwidth}
		>{\centering}p{0.15\textwidth}
		>{\centering\arraybackslash}p{0.15\textwidth} }

		\coloredTableHead
		\textbf{\color{white}Ruolo} &
		\textbf{\color{white}Ore} &
		\textbf{\color{white}Delta ore} &
		\textbf{\color{white}Costo in \euro{}} &
		\textbf{\color{white}Delta costo}
		\tabularnewline
		\endhead

		% Contenuto della tabella
		% Ruolo & OreEffettive & DeltaOre & Costo & DeltaCosto \\
		Responsabile    & 12 & +0 &   360,00 & +  0,00 \\
		Amministratore  & 24 & +0 &   480,00 & +  0,00 \\
		Analista        & 35 & +2 &   875,00 & + 50,00 \\
		Progettista     & 65 & -5 & 1.474,00 & -110,00 \\
		Programmatore   & 34 & -4 &   525,00 & - 60,00 \\
		Verificatore    & 68 & +0 & 1.020,00 & +  0,00 \\
		\textbf{Totale Effettivo} & \multicolumn{2}{c}{\textbf{238}} & \multicolumn{2}{c}{\textbf{4.734,00}} \\
		\textbf{Delta} & \multicolumn{2}{c}{\textbf{-7}} & \multicolumn{2}{c}{\textbf{-120,00}} \\

		\rowcolor{white}\caption{Consuntivo per il periodo di Progettazione Architetturale}	\\

	\end{longtable}
	\subsubsection{Conclusione}
		Come riportato nella tabella del Consuntivo per il periodo di Progettazione Acrhitetturale, il preventivo orario per i ruoli di Responsabile, Amministratore e Verificatore si è rivelato sufficiente per svolgere il lavoro pianificato. \\
		Si è rivelato necessario dedicare più ore lavorative rispetto a quanto preventivato per il ruolo di Analista, mentre è stato impiegato un monte ore ridotto per i ruoli di Progettista e Programmatore. Di seguito sono riportate le motivazioni:
		\begin{itemize}
			\item \textbf{Analista:} la comprensione delle indicazioni ricevute nell'esito della \textit{Revisione dei Requisiti} e la conseguente correzione del documento dell'\textit{AdR{} 1.0.0}, hanno richiesto alcune ore di lavoro in più rispetto a quelle preventivate;
			\item \textbf{Progettista:} le molte ore di studio individuale impiegate nel periodo di Consolidamento dei Requisiti riguardo le tecnologie ritenute potenzialmente utili per lo sviluppo del prodotto, hanno permesso di poter impiegare meno ore di quanto pianificato per l'individuazione delle tecnologie e delle librerie adatte allo sviluppo del prodotto;
			\item \textbf{Programmatore:} grazie alle ore di studio individuali descritte nel punto precedente e all'adeguata scelta delle tecnologie e delle librerie, la codifica del Proof of Concept ha richiesto meno ore di lavoro rispetto a quelle preventivate.
		\end{itemize}
	\subsubsection{Preventivo a finire}
	Il bilancio finale è positivo. Ciò è dovuto soprattutto alla scelta iniziale delle tecnologie e delle librerie da utilizzare, che si è rivelata adeguata e ci ha permesso di risparmiare delle ore di lavoro nelle attività di progettazione e di codifica del Proof of Concept, rispetto a quanto avevamo preventivato. \\
	Intendiamo utilizzare l'ammontare risparmiato in questo periodo per assegnare alcune ore supplementari al ruolo dell'Analista nel periodo di Progettazione di Dettaglio e Codifica, in quanto potrebbe essere necessario apportare modifiche al documento dell'\textit{AdR{} 1.0.0}, seguendo eventuali indicazioni ricevute a seguito dell'esito della \textit{Revisione di Progettazione}.
	
	
	%Intendiamo utilizzare l'ammontare risparmiato in questo periodo per assegnare alcune ore aggiuntive ai ruoli di Progettista e Programmatore nel periodo di Progettazione di Dettaglio e Codifica, in quanto al momento siamo riusciti ad individuare più tecnologie e librerie valide ed adeguate per poter implementare il comando \textit{deploy} presente nel quarto incremento di quel periodo, ma non siamo ancora riusciti a decidere quale sia la scelta migliore; prevediamo quindi la necessità di dover impiegare del tempo in più per la valutazione degli aspetti positivi e negativi riguardo l'utilizzo potenziale di quelle tecnologie e librerie individuate in modo da poter effettuare la scelta migliore. \\























































