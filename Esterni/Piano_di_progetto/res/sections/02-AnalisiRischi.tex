\section{Analisi dei rischi}
Durante lo sviluppo di un progetto complesso è possibile incorrere in problematiche che potrebbero rallentare o impedire il normale proseguimento del progetto. Risulta quindi necessario effettuare un'approfondita attività di analisi dei fattori di rischio per cercare di evitare o rendere il più ininfluenti possibili le eventuali problematiche che potrebbero presentarsi.
Il risultato di questa analisi è presentata in forma tabellare dove ciascuna voce rappresenta un fattore di rischio ed è costruita tramite le seguenti quattro attività:
\begin{itemize}
	\item \textbf{Individuazione}: vengono identificati i potenziali fattori di rischio che potrebbero causare situazioni problematiche durante lo sviluppo del progetto;
	\item \textbf{Analisi}: viene studiato ciascun fattore di rischio. Questo studio consiste nell'assegnare a ciascun fattore la probabilità che esso si verifichi, un indice di gravità e l'impatto che avrebbe sul progetto nel caso in cui si verifichi;
	\item \textbf{Pianificazione di controllo e mitigazione}: si pianifica una metodologia per evitare che i rischi individuati si verifichino e si stabilisce preventivamente come procedere nel caso in cui il rischio si verifichi;
	%piano di contingenza
	\item \textbf{Monitoraggio}: attività continua nella quale la situazione è tenuta sotto controllo per prevenire il verificarsi dei rischi o, nel peggiore dei casi, per intervenire tempestivamente e mitigarli;
	%identificazione di metodologie di rilevamento dei rischi
\end{itemize}

Per semplificarne la gestione ed evitare possibili ambiguità si è deciso di raggruppare le varie tipologie di fattori di rischio in questo modo:
\begin{itemize}
	\item \textbf{RT}: Rischi Tecnologici;
	\item \textbf{RO}: Rischi Organizzativi;
	\item \textbf{RI}: Rischi Interpersonali;
	\item \textbf{RR}: Rischi legati ai Requisiti.
\end{itemize}


%\rowcolors{2}{lightRowColor}{darkRowColor}
\begin{longtable}{ 
		>{\centering}p{0.075\textwidth} 
		>{\centering}p{0.15\textwidth}
		>{\centering}p{0.3\textwidth} 
		>{\centering}p{0.3\textwidth} 
		>{\centering}p{0.15\textwidth} }
	
	\coloredTableHead
	\textbf{\color{white}Codice} & 
	\textbf{\color{white}Nome} & 
	\textbf{\color{white}Descrizione} & 
	\textbf{\color{white}Rilevamento} &
	\textbf{\color{white}Grado di rischio} 
	\tabularnewline  
	\endhead
	
	% contenuto tabella 
	% Codice & Nome & Descrizione & Rilevamento & Grado di rischio \\
	\rowcolor{lightRowColor}
	RT-1 & Tecnologie da utilizzare &
	  &
	  & \\
	 \rowcolor{lightRowColor}
	 \multicolumn{2}{c}{\textbf{Piano di contingenza}} &
	 \multicolumn{3}{c}{Testo} \\
	
	\rowcolor{darkRowColor}
	RT-2 & Guasti hardware &
	  &
	  & \\
	 \rowcolor{darkRowColor}
	 \multicolumn{2}{c}{\textbf{Piano di contingenza}} &
	 \multicolumn{3}{c}{Testo} \\
	
	\rowcolor{lightRowColor}
	RT-3 & Guasti software &
	  &
	  & \\
	 \rowcolor{lightRowColor}
	 \multicolumn{2}{c}{\textbf{Piano di contingenza}} &
	 \multicolumn{3}{c}{Testo} \\

	\rowcolor{darkRowColor}
	RO-1 & Calcolo tempistiche &  
	  &
	  & \\
	 \rowcolor{darkRowColor}
	 \multicolumn{2}{c}{\textbf{Piano di contingenza}} &
	 \multicolumn{3}{c}{Testo} \\
	
	\rowcolor{lightRowColor}
	RO-2 & Calcolo costi delle attività &  
	  &
	  & \\
	 \rowcolor{lightRowColor}
	 \multicolumn{2}{c}{\textbf{Piano di contingenza}} &
	 \multicolumn{3}{c}{Testo} \\
	
	\rowcolor{darkRowColor}
	RO-3 & Impegni accademici &
	  &
	  & \\
	 \rowcolor{darkRowColor}
	 \multicolumn{2}{c}{\textbf{Piano di contingenza}} &
	 \multicolumn{3}{c}{Testo} \\
	
	\rowcolor{lightRowColor}
	RO-4 & Impegni personali &
	  &
	  & \\
	 \rowcolor{lightRowColor}
	 \multicolumn{2}{c}{\textbf{Piano di contingenza}} &
	 \multicolumn{3}{c}{Testo} \\
	
	\rowcolor{darkRowColor}
	RO-5 & Ritardi &
	  &
	  & \\
	 \rowcolor{darkRowColor}
	 \multicolumn{2}{c}{\textbf{Piano di contingenza}} &
	 \multicolumn{3}{c}{Testo} \\
	
	\rowcolor{lightRowColor}
	RI-1 & Comunicazione interna &
	  &
	  & \\
	 \rowcolor{lightRowColor}
	 \multicolumn{2}{c}{\textbf{Piano di contingenza}} &
	 \multicolumn{3}{c}{Testo} \\
	
	\rowcolor{darkRowColor}
	RI-2 & Comunicazione esterna &
	  &
	  & \\
	 \rowcolor{darkRowColor}
	 \multicolumn{2}{c}{\textbf{Piano di contingenza}} &
	 \multicolumn{3}{c}{Testo} \\
	
	\rowcolor{lightRowColor}
	RI-3 & Contrasti interni &
	  &
	  & \\
	 \rowcolor{lightRowColor}
	 \multicolumn{2}{c}{\textbf{Piano di contingenza}} &
	 \multicolumn{3}{c}{Testo} \\
	
	\rowcolor{darkRowColor}
	RR-1 & Analisi dei requisiti incompleta &
	  &
	  & \\
	 \rowcolor{darkRowColor}
	 \multicolumn{2}{c}{\textbf{Piano di contingenza}} &
	 \multicolumn{3}{c}{Testo} \\
	
	\rowcolor{lightRowColor}
	RR-2 & Modifica dei requisiti &
	  &
	  & \\
	 \rowcolor{lightRowColor}
	 \multicolumn{2}{c}{\textbf{Piano di contingenza}} &
	 \multicolumn{3}{c}{Testo} \\
    	        
\end{longtable}
