\section*{Registro delle modifche} % con section* evito che la sezione sia considerata nell'indice 

\rowcolors{2}{lightRowColor}{darkRowColor}
\begin{longtable}{ 
		>{\centering}p{0.1\textwidth} 
		>{\centering}p{0.15\textwidth}
		>{\centering}p{0.2\textwidth} 
		>{\centering}p{0.15\textwidth} 
		>{}p{0.26\textwidth} }
		
	\coloredTableHead
	\textbf{\color{white}Versione} & 
	\textbf{\color{white}Data} & 
	\textbf{\color{white}Nominativo} & 
	\textbf{\color{white}Ruolo} &
	\textbf{\color{white}Descrizione} 
	\tabularnewline  
	\endhead
	
	% contenuto tabella 
	% ProvaVersione & ProvaData & ProvaPersona & \textit{ProvaRuolo} & ProvaAzione. \\
	0.0.8 & 2020-03-29 & \MP{} & \textit{Responsabile} & Stesura della sezione §5 \\
	0.0.7 & 2020-03-28 & \MP{} & \textit{Responsabile} & Stesura della sezione §4 \\
	0.0.6 & 2020-03-27 & \MP{} & \textit{Responsabile} & Conclusione stesura della struttura delle sezioni restanti e delle appendici \\
	0.0.5 & 2020-03-26 & \MP{} & \textit{Responsabile} & Inizio stesura della struttura delle sezioni restanti e delle appendici \\
	0.0.4 & 2020-03-25 & \MP{} & \textit{Responsabile} & Conclusione della tabella riassuntiva della sezione §2 \\
	0.0.3 & 2020-03-23 & \MP{} & \textit{Responsabile} & Stesura della prima parte e creazione della struttura della tabella riassuntiva della sezione §2 \\
	0.0.2 & 2020-03-20 & \MP{} & \textit{Responsabile} & Stesura della sezione §1 \\ 
    0.0.1 & 2020-03-18 & \MP{} & \textit{Responsabile} & Creazione della struttura del documento in \LaTeX{} \\ 
    	        
\end{longtable}