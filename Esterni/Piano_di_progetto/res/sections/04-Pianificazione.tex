\section{Pianificazione}
Lo sviluppo del progetto è costruito sulla base delle scadenze riportate nella sottosezione 1.5 ed è suddiviso nelle seguenti fasi:
\begin{itemize}
	\item analisi;
	\item consolidamento dei requisiti
	\item progettazione architetturale;
	\item progettazione di dettaglio e codifica;
	\item validazione e collaudo;
\end{itemize}

\subsection{Analisi}
\textbf{Periodo}: dal 2020-03-10 al 2020-04-13 \\
Questo periodo ha inizio con la formazione dei gruppi e termina con la scadenza per la consegna dei documenti relativi alla \textit{Revisione dei Requisiti}. \\
Le principali attività svolte in questo periodo sono:
\begin{itemize}
	\item \textbf{Strumenti di lavoro}: questa attività consiste nella scelta degli strumenti di lavoro da utilizzare per lo svolgimento del progetto;
	\item \textbf{\textit{\NdP{}}}: attività nella quale gli Amministratori redigono le \textit{\NdP{}}, documento in cui si specificano tutte le regole, le convenzioni e le tecnologie che i componenti del gruppo adotteranno durante tutto il corso del progetto;
	\item \textbf{\textit{\SdF{}}}: attività nella quale gli Analisti redigono lo \textit{\SdF}, documento in cui vengono analizzati i capitolati d'appalto elencando per ciascuno i punti positivi e negativi che li caratterizzano. Inoltre vengono indicate le motivazioni per le quali è stato scelto il capitolato C2 denominato \textit{Etherless} e sono stati esclusi i capitolati restanti. \\
	Questa attività è bloccante per l'inizio dell'\textit{\AdR{}};
	\item \textbf{\textit{\AdR{}}}: attività nella quale gli Analisti redigono \textit{l'\AdR}, documento essenziale in cui viene analizzato in maniera approfondita il capitolato scelto a seguito dello \textit{\SdF{}}, individuando le funzionalità e i casi d'uso previsti dal progetto;
	\item \textbf{\textit{\PdP{}}}: attività nella quale il Responsabile redige il \textit{\PdP}, documento in cui viene presentata la pianificazione del gruppo per lo sviluppo del progetto, un'analisi dei rischi e dei costi e dove vengono indicate le scadenze che il gruppo intende rispettare per la buona riuscita del progetto;
	\item \textbf{\textit{\PdQ{}}}: attività nella quale gli Analisti redigono il \textit{\PdQ}, documento in cui vengono indicate tutte le strategie di verifica e validazione che il gruppo intende adottare con lo scopo di garantire la qualità di processo e di prodotto;
	\item \textbf{\textit{\Glossario{}}}: attività nella quale viene redatto il \textit{\Glossario}, documento nel quale verranno elencati, chiariti ed approfonditi tutti i termini tecnici utilizzati nei documenti con lo scopo di evitare possibili ambiguità;
	\item \textbf{\textit{Lettera di Presentazione}}: attività nella quale viene redatta la \textit{Lettera di Presentazione} necessaria per la presentazione come fornitore del gruppo.
\end{itemize}

\subsection{Consolidamento dei requisiti}
\textbf{Periodo}: dal 2020-04-13 al 2020-04-20 \\
Questo periodo ha inizio dopo il termine del periodo di \textbf{Analisi} e termina il giorno della presentazione della \textit{Revisione dei Requisiti}. \\
In questo periodo l'attivita principale prevede un consolidamento e un miglioramento dei requisiti ottenuti conclusa la fase di analisi e una preparazione del materiale necessario alla presentazione del 2020-04-20. \\
Inoltre, qualora se ne presentasse la necessità, verranno apportate modifiche migliorative ai documenti redatti durante la fase precedente.

\subsection{Progettazione architetturale}
\textbf{Periodo}: dal 2020-04-13 al 2020-05-11 \\
Questo periodo ha inizio dopo il termine del periodo di Consolidamento dei requisiti e termina con la scadenza per la consegna dei documenti relativi alla \textit{Revisione di Progettazione}. \\
Le principali attività svolte in questo periodo sono:
\begin{itemize}
	\item \textbf{Incremento e verifica}: come prima cosa, analizzando l'esito della \textit{Revisione dei Requisiti} vengono svolte attività di incremento e verifica sui vari documenti redatti. \\
	L'incremento dell'\textit{\AdR{}} è il più importante perchè va completato prima di poter proseguire con il resto delle attività;
	\item \textbf{\textit{Specifica Tecnica}}: attività nella quale i Progettisti redigono il documento della \textit{Specifica Tecnica} dove vengono specificate le scelte progettuali ad alto livello.
	Questa attività è considerata bloccante perchè prima del termine di essa non è possibile proseguire con lo svolgimento del progetto;
	\item \textbf{\textit{Glossario}}: attività che prevede un miglioramento del \textit{\Glossario{}} aggiungendo nuovi termini oppure raffinando le definizioni di termini già presenti.
\end{itemize}

\subsection{Progettazione di dettaglio e codifica}
\textbf{Periodo}: dal 2020-05-11 al 2020-06-11 \\
Questo periodo ha inizio dopo il termine del periodo di \textit{Progettazione Architetturale} e termina con la scadenza per la consegna dei documenti relativi alla \textit{Revisione di Qualifica}. \\
Le principali attività svolte in questo periodo sono:
\begin{itemize}
	\item \textbf{Incremento e verifica}: come prima cosa, analizzando l'esito della \textit{Revisione di Progettazione} vengono svolte attività di incremento e verifica sui vari documenti redatti;
	\item \textbf{\textit{Definizione di Prodotto}}: attività nella quale viene redatto il documento della \textit{Definizione di Prodotto} che contiene una dettagliata progettazione architetturale seguendo il contenuto della \textit{Specifica Tecnica};
	\item \textbf{Codifica}: attività nella quale viene prodotto e verificato il codice seguendo quanto scritto nella \textit{Definizione di Prodotto};
	\item \textbf{\textit{Manuale Utente}}: attività nella quale viene redatto il \textit{Manuale Utente} contenente le informazioni su come funziona e su come si utilizza il prodotto;
	\item \textbf{\textit{Glossario}}: attività che prevede un miglioramento del \textit{\Glossario{}} aggiungendo nuovi termini oppure raffinando le definizioni di termini già presenti.
\end{itemize}

\subsection{Validazione e collaudo}
\textbf{Periodo}: dal 2020-06-11 al 2020-07-13 \\
Questo periodo ha inizio dopo il termine del periodo di \textit{Progettazione di Dettaglio e Codifica} e termina con la scadenza per la consegna dei documenti relativi alla \textit{Revisione di Accettazione}. \\
Le principali attività svolte in questo periodo sono:
\begin{itemize}
	\item \textbf{Incremento e verifica}: come prima cosa, analizzando l'esito della \textit{Revisione dei Qualifica} vengono svolte attività di incremento e verifica sui vari documenti redatti;
	\item \textbf{Validazione e Collaudo}: attività nella quale vengono eseguiti test e, se necessario, vengono apportati dei miglioramenti al prodotto per poter assicurare il soddisfacimento dei requisiti e dei vincoli qualitativi;
	\item \textbf{\textit{Manuale Utente}}: attività nella quale viene migliorato il documento del \textit{Manuale Utente};
	\item \textbf{\textit{Glossario}}: attività che prevede un miglioramento del \textit{\Glossario{}} aggiungendo nuovi termini oppure raffinando le definizioni di termini già presenti.
\end{itemize}