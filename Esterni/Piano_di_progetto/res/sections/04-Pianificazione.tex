\section{Pianificazione}
Lo sviluppo del progetto è costruito sulla base delle scadenze riportate nella sottosezione 1.5 ed è suddivisa nelle seguenti fasi:
\begin{itemize}
	\item Analisi;
	\item Consolidamento dei requisiti
	\item Progettazione architetturale;
	\item Progettazione di dettaglio e codifica;
	\item Validazione e collaudo;
\end{itemize}

\subsection{Analisi}
\textbf{Periodo}: dal 2020-03-10 al 2020-04-13 \\
Questa fase ha inizio con la formazione dei gruppi e termina con la scadenza per la consegna dei documenti relativi alla \textit{Revisione dei Requisiti}. \\
In questa fase le principali attività svolte sono:
\begin{itemize}
	\item \textbf{\textit{\NdP{}}}: attività nella quale gli Amministratori redigono le \NdP{}, documento in cui si specificano tutte le regole, le convenzioni e le tecnologie che i componenti del gruppo adotteranno durante tutto il corso del progetto;
	\item \textbf{\textit{\SdF{}}}: attività nella quale gli Analisti redigono lo \SdF, documento in cui vengono analizzati i capitolati d'appalto elencando per ciascuno i punti positivi e negativi che li caratterizzano. Inoltre vengono indicate le motivazioni per le quali è stato scelto il capitolato C2 denominato \textit{Etherless} ed è stato escluso i capitolati restanti. \\
	Questa attività è bloccante per l'inizio dell'\AdR{};
	\item \textbf{\textit{\AdR{}}}: attività nella quale gli Analisti redigono l'\AdR, documento essenziale in cui viene analizzato in maniera approfondita il capitolato scelto a seguito dello \SdF{}, individuando le funzionalità e i casi d'uso previsti dal progetto;
	\item \textbf{\textit{\PdP{}}}: attività nella quale il Responsabile redige il \PdP, documento in cui viene presentata la pianificazione del gruppo per lo sviluppo del progetto, un'analisi dei rischi e dei costi e dove vengono indicate le scadenze che il gruppo intende rispettare per la buona riuscita del progetto;
	\item \textbf{\textit{\PdQ{}}}: attività nella quale gli Analisti redigono il \PdQ, documento in cui vengono indicate tutte le strategie di verifica e validazione che il gruppo intende adottare con lo scopo di garantire la qualità di processo e di prodotto;
	\item \textbf{\textit{\Glossario{}}}: attività nella quale viene redatto il \Glossario, documento nel quale verranno elencati, chiariti ed approfonditi tutti i termini tecnici utilizzati nei documenti con lo scopo di evitare possibili ambiguità;
	\item \textbf{\textit{Lettera di Presentazione}}: attività nella quale viene redatta la Lettera di Presentazione necessaria per la presentazione come fornitore del gruppo.
\end{itemize}

\subsection{Consolidamento dei requisiti}
\textbf{Periodo}: dal 2020-04-13 al 2020-04-20 \\
Questa fase ha inizio dopo il termine della fase di \textit{Analisi} e termina il giorno della presentazione della \textit{Revisione dei Requisiti}. \\
In questa fase l'attivita principale prevede un consolidamento e un miglioramento dei requisiti ottenuti conclusa la fase di analisi e una preparazione del materiale necessario alla presentazione del 2020-04-20. \\
Inoltre, qualora se ne presentasse la necessità verranno apportate modifiche migliorative ai documenti redatti durante la fase precedente.

\subsection{Progettazione architetturale}
\textbf{Periodo}: dal 2020-04-13 al 2020-05-11 \\
Questa fase ha inizio dopo il termine della fase di \textit{Consolidamento dei requisiti} e termina con la scadenza per la consegna dei documenti relativi alla \textit{Revisione di Progettazione}. \\

\subsection{Progettazione di dettaglio e codifica}
\textbf{Periodo}: dal 2020-05-11 al 2020-06-11 \\
Questa fase ha inizio dopo il termine della fase di \textit{Progettazione architetturale} e termina con la scadenza per la consegna dei documenti relativi alla \textit{Revisione di Qualifica}

\subsection{Validazione e collaudo}
\textbf{Periodo}: dal 2020-06-11 al 2020-07-13 \\
Questa fase ha inizio dopo il termine della fase di \textit{Progettazione di dettaglio e codifica} e termina con la scadenza per la consegna dei documenti relativi alla \textit{Revisione di Accettazione}