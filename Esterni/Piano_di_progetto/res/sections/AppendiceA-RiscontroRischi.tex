\begin{appendices}
\section{Riscontro dei rischi}
	\rowcolors{2}{lightRowColor}{darkRowColor}
	\begin{longtable}{
		>{\centering}p{0.1\textwidth}
		>{\centering}p{0.15\textwidth}
		>{\centering\arraybackslash}p{0.3\textwidth}
		>{\centering\arraybackslash}p{0.3\textwidth} }

		\caption {Riscontro dei rischi} \\

		\coloredTableHead
		\textbf{\color{white}ID} &
		\textbf{\color{white}Periodo} &
		\textbf{\color{white}Scenario} &
		\textbf{\color{white}Mitigazione}
		\tabularnewline
		\endhead

		% Contenuto della tabella
		% ID & Periodo & Scenario & Manutenzione migliorativa \\
		RT-1
		&
		Analisi
		&
		Molti componenti del gruppo \Gruppo{} non presentavano conoscenze pregresse di alcuni concetti e di molte delle tecnologie necessarie per lo sviluppo del progetto, come ad esempio l'ambiente\ped{\textit{G}} Serverless\ped{\textit{G}}, AWS\ped{\textit{G}}, Truffle\ped{\textit{G}} e il concetto di blockchain\ped{\textit{G}}.
		&
		Il Proponente\ped{\textit{G}} e alcuni membri del gruppo che presentavano conoscenze pregresse in materia, si sono occupati di fornire materiale di studio in supporto agli altri membri del gruppo.\\

		RO-1
		&
		Analisi
		&
		A causa dell'inesperienza del gruppo nell'affrontare un progetto complesso, si sono presentati degli errori e delle cattive pratiche nello svolgimento del lavoro assegnato.
		&
		Il problema è stato mitigato dalla rotazione dei ruoli e dalle numerose riunioni che hanno permesso l'individuazione tempestiva e la correzione delle criticità sopracitate.\\
\hline
		RO-1
		&
		Progettazione Architetturale
		&
		A causa dell'inesperienza del gruppo non è stato semplice ed immediato riuscire a capire a fondo le indicazioni ricevute nell'esito della \textit{Revisione dei Requisiti} riguardo l'errata adesione al modello incrementale.
		&
		La collaborazione tra i componenti del gruppo, tramite esposizione dei propri dubbi e delle proprie perplessità, e uno studio più approfondito del modello incrementale hanno permesso di colmare le lacune dovute all'inesperienza e di risolvere questo problema.\\

		RO-1
		&
		Progettazione Architetturale
		&
		Sono state rilevate alcune difficoltà nell'organizzare il lavoro e nella rotazione dei ruoli.
		La difficoltà maggiore per i componenti del gruppo è stata riscontrata nel passaggio da un ruolo ad un altro, dovendo quindi ambientarsi nel nuovo ruolo ed organizzarsi per continuare il lavoro svolto da un altro componente del gruppo.
		&
		Per limitare il più possibile eventuali ritardi dovuti alla rotazione dei ruoli ciascun componente del gruppo ha tempestivamente comunicato agli altri, durante le frequenti riunioni interne, eventuali problematiche riscontrate per fare in modo di risolverle con l'aiuto di coloro i quali hanno già svolto quello stesso ruolo in precedenza.
		Questo approccio si è rivelato molto efficace, perché ha limitato la possibilità che errori simili o identici si siano verificati a seguito delle varie rotazioni. \\

		RO-3
		&
		Progettazione Architetturale
		&
		A causa di esami universitari o di altri impegni personali, alcuni componenti non sono riusciti a portare a termine alcuni compiti assegnati entro una certa scadenza, concordata all'interno del gruppo.
		&
		Per ridurre al minimo eventuali ritardi dovuti a questa problematica ciascun componente ha notificato per tempo al resto del gruppo il proprio impegno ed il gruppo si è organizzato ridistribuendo il lavoro ad altri componenti per riuscire a portare a termine quel compito entro la scadenza prefissata.  \\

		RI-1
		&
		Progettazione Architetturale
		&
		Durante alcune riunioni interne svolte in questo periodo nelle quali sono state prese delle decisioni talvolta importanti, alcuni componenti non hanno potuto partecipare all'incontro o sono dovuti uscire prima della fine.
		&
		Durante ciascuna riunione interna è stata decisa e concordata la data e l'ora della riunione successiva, in modo da notificare immediatamente eventuali momenti di irreperibilità e di riuscire ad essere tutti presenti all'incontro successivo.
		A seguito dell'impossibilità per un componente di partecipare ad un incontro, quest'ultimo ha sempre comunicato preventivamente l'avanzamento del proprio lavoro ed eventuali quesiti da porre, permettendo quindi al Responsabile si riportare tali infomazioni durante l'incontro.
		Per ciascun incontro è stato sempre nominato un Segretario col compito di redigere un Verbale\ped{\textit{G}} con tutto ciò che è stato detto e tutte le decisioni prese. Il componente assente ha perciò potuto informarsi sulle discussioni avvenute durante l'incontro, leggendo il relativo Verbale\ped{\textit{G}}. \\
\hline
		RO-3
		&
		Progettazione di Dettaglio e Codifica
		&
		A causa di esami universitari o di altri impegni personali, alcuni componenti non sono riusciti a portare a termine alcuni compiti assegnati entro una certa scadenza, oppure non hanno potuto presenziare agli incontri di gruppo.
		&
		Per ridurre al minimo eventuali ritardi dovuti a queste problematiche, ciascun componente ha notificato per tempo il resto del gruppo per potersi organizzare ridistribuendo il lavoro e portando a termine quel compito entro la scadenza prefissata.  \\

		RT-1
		&
		Progettazione di Dettaglio e Codifica
		&
		Si sono presentate alcune lacune nella conoscenza approfondita delle tecnologie scelte, in particolare nell'ambiente di testing del codice del prodotto\ped{\textit{G}}.
		&
		I membri del gruppo, data l'esperienza accumulata nel trascorrere del progetto, si sono documentati sugli argomenti nei quali presentassero lacune, presentando poi i risultati ottenuti al resto del gruppo.\\

		RT-4
		&
		Progettazione di Dettaglio e Codifica
		&
		Si sono presentate alcune lacune nella conoscenza dei design pattern\ped{\textit{G}} e degli stili architetturali, necessari per ottenere una progettazione efficace del prodotto\ped{\textit{G}}. I Progettisti si sono documentati sugli argomenti, anche in seguito alle indicazioni ricevute in riunione con il \RC{}. Tuttavia alcuni design pattern\ped{\textit{G}} non sono stati implementati correttamente, nello specifico \textit{Facade} e \textit{Command}. Questi errori quindi, mancati al rilevamento, sono stati riportati anche in sede di Product Baseline\ped{\textit{G}}, dove sono stati evidenziati dal professore.
		&
		I Progettisti si sono immediatamente impegnati a rivalutare e correggere l'implementazione dei design pattern\ped{\textit{G}} e di conseguenza la progettazione del modulo\ped{\textit{G}} in questione. Collaborando strettamente verranno ridotti al minimo i ritardi causati dalla problematica.\\
	\end{longtable}
\end{appendices}
