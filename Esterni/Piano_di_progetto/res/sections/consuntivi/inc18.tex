\subsubsection{18$^{\circ}$ Incremento}

	\subsubsubsection{Consuntivo}
		\begin{longtable}{
				>{\centering}p{0.25\textwidth}
				>{\centering}p{0.08\textwidth}
				>{\centering}p{0.08\textwidth}
				>{\centering}p{0.15\textwidth}
				>{\centering\arraybackslash}p{0.15\textwidth} }

			\coloredTableHead
			\textbf{\color{white}Ruolo} &
			\textbf{\color{white}Ore} &
			\textbf{\color{white}Delta ore} &
			\textbf{\color{white}Costo in \euro{}} &
			\textbf{\color{white}Delta costo}
			\tabularnewline
			\endhead

			% Contenuto della tabella
			% Ruolo & OreEffettive & DeltaOre & Costo & DeltaCosto \\
      Responsabile    & 1  & +0 & 30,00  & +  0,00 \\
      Amministratore  & 1  & +0 & 20,00  & +  0,00 \\
      Analista        & 0  & +0 & 0,00   & +  0,00 \\
      Progettista     & 3  & +0 & 66,00  & +  0,00 \\
      Programmatore   & 10 & +0 & 150,00 & +  0,00 \\
      Verificatore    & 15 & +0 & 225,00 & +  0,00 \\
			\textbf{Totale Effettivo} & \multicolumn{2}{c}{\textbf{30}} & \multicolumn{2}{c}{\textbf{491,00}} \\
			\textbf{Delta} & \multicolumn{2}{c}{\textbf{0}} & \multicolumn{2}{c}{\textbf{+0,00}} \\

			\rowcolor{white}\caption{Consuntivo per il 18$^{\circ}$ Incremento}	\\

		\end{longtable}

	\subsubsubsection{Conclusioni}
	In questo periodo ci siamo dedicati all'implementazione della funzionalità di \texttt{init} (visualizzazione informazioni utili per iniziare ad utilizzare l'applicativo), \texttt{whoami} (visualizzazione dell'indirizzo dell'utente attualmente autenticato nel sistema) e \texttt{search} (ricerca di una funzione). Queste funzionalità richiedono solamente modifiche al modulo \textit{Etherless-cli}, e sono state rispettate il monte ore e le scadenze previste.

	\subsubsubsection{Preventivo a finire rispetto al periodo}
	La pianificazione in questo incremento è stata rispettata e il bilancio risulta essere \textbf{pari}.

	\subsubsubsection{Preventivo a finire complessivo}
	Date le considerazioni precedenti, il preventivo complessivo resta invariato.
