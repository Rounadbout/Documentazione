\subsubsection{Periodo complessivo}

	\subsubsubsection{Consuntivo}
		\begin{longtable}{
			>{\centering}p{0.25\textwidth}
			>{\centering}p{0.08\textwidth}
			>{\centering}p{0.08\textwidth}
			>{\centering}p{0.15\textwidth}
			>{\centering\arraybackslash}p{0.15\textwidth} }
		
		\coloredTableHead
		\textbf{\color{white}Ruolo} &
		\textbf{\color{white}Ore} &
		\textbf{\color{white}Delta ore} &
		\textbf{\color{white}Costo in \euro{}} &
		\textbf{\color{white}Delta costo}
		\tabularnewline
		\endhead
		
		% Contenuto della tabella
		% Ruolo & OreEffettive & DeltaOre & Costo & DeltaCosto \\
		Responsabile    & 23 & +0 &   690,00 & +  0,00 \\
		Amministratore  & 26 & -6 &   520,00 & -  120,00 \\
		Analista        & 3 & +3 &   75,00 & + 75,00 \\
		Progettista     & 94 & +6 & 2068,00 & + 132,00 \\
		Programmatore   & 145 & -8 &   2175,00 &  -120,00 \\
		Verificatore    & 98 & +2 & 1470,00 & + 30,00 \\
		\textbf{Totale Effettivo} & \multicolumn{2}{c}{\textbf{389}} & \multicolumn{2}{c}{\textbf{6998,00}} \\
		\textbf{Delta} & \multicolumn{2}{c}{\textbf{-3}} & \multicolumn{2}{c}{\textbf{-3,00}} \\
		
		\rowcolor{white}\caption{Consuntivo per il periodo di Progettazione di Dettaglio e Codifica}	\\
		
	\end{longtable}
	
	\subsubsubsection{Conclusioni}
	Il bilancio finale è \textbf{positivo}, con un risparmio di \textbf{3,00\euro}. \\ Come riportato nella tabella del consuntivo per il periodo di Progettazione di Dettaglio e Codifica, il numero di ore preventivate per il ruolo di Responsabile si è dimostrato adeguato. \\ Segue un'analisi delle motivazioni per cui gli altri ruoli hanno necessitato di un numero di ore differente da quanto previsto: 
	\begin{itemize}
		\item \textbf{Amministratore}: il numero limitato di modifiche da apportare alle \NdP{} hanno portato ad un risparmio di ore;
		\item \textbf{Analista}: al contrario di quanto previsto, è stato necessario apportare ulteriori modifiche all'\AdR, questo ha comportato alcune ore di lavoro non preventivate;
		\item \textbf{Progettista}: la scarsa conoscenza dei design pattern\ped{\textit{G}} e l'esito della Product Baseline hanno portato ad un incremento considerevole delle ore di lavoro richieste da tale ruolo;
		\item \textbf{Programmatore}: il riutilizzo del prototipo creato per la \textit{Technology Baseline} e la ripetizione di pattern di comunicazione tra i vari moduli\ped{\textit{G}} hanno permesso di risparmiare alcune ore di lavoro;
		\item \textbf{Verificatore}: sono state necessarie alcune ore di lavoro non preventivate per verificare tutte le modifiche apportate alla documentazione;
	\end{itemize}
	
	\subsubsubsection{Preventivo a finire}	
	Date le considerazioni precedenti il preventivo complessivo resta invariato. In particolare il risparmio ottenuto in questo periodo risulta essere di portata alquanto ridotta, e per questo non porta ad alcuna ripianificazione. 
	
	