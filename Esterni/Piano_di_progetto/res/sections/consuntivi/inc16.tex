\subsubsection{16$^{\circ}$ Incremento}
		
	\subsubsubsection{Consuntivo}
	
		\begin{longtable}{
			>{\centering}p{0.25\textwidth}
			>{\centering}p{0.08\textwidth}
			>{\centering}p{0.08\textwidth}
			>{\centering}p{0.15\textwidth}
			>{\centering\arraybackslash}p{0.15\textwidth} }

		\coloredTableHead
		\textbf{\color{white}Ruolo} &
		\textbf{\color{white}Ore} &
		\textbf{\color{white}Delta ore} &
		\textbf{\color{white}Costo in \euro{}} &
		\textbf{\color{white}Delta costo}
		\tabularnewline
		\endhead
		
		% Contenuto della tabella
		% Ruolo & OreEffettive & DeltaOre & Costo & DeltaCosto \\
		Responsabile    & 3 & +0 &   90,00 & +  0,00 \\
		Amministratore  & 4 & +0 &   80,00 & -  0,00 \\
		Analista        & 0 & +0 &   0,00 & + 0,00 \\
		Progettista     & 2 & +0 & 44,00 & + 0,00 \\
		Programmatore   & 3 & +2 &   45,00 &  +30,00 \\
		Verificatore    & 2 & +0 & 30,00 & + 0,00 \\
		\textbf{Totale Effettivo} & \multicolumn{2}{c}{\textbf{14}} & \multicolumn{2}{c}{\textbf{289,00}} \\
		\textbf{Delta} & \multicolumn{2}{c}{\textbf{+2}} & \multicolumn{2}{c}{\textbf{+30,00}} \\
		
		\rowcolor{white}\caption{Consuntivo per il 16$^{\circ}$ Incremento}	\\
	\end{longtable}
		
	\subsubsubsection{Conclusioni}
	Durante questo incremento il gruppo si è occupato di creare la presentazione per la RQ e raffinare la corrispondente demo. Sono stati riscontrati alcuni errori per le funzionalità di deployment ed eliminazione di una funzione; la cui risoluzione ha richiesto alcune ore non pianificate per il ruolo di Programmatore. 
	
	\subsubsubsection{Preventivo a finire rispetto al periodo}
	Il bilancio è \textbf{negativo}, con un una spesa in eccesso di \textbf{30,00\euro{}}. Non si ritiene necessaria alcuna ripianificazione, poichè la somma non è significativa e viene in parte bilanciata dal risparmio ottenuto in alcuni degli incrementi precedenti. 
	
	\subsubsubsection{Preventivo a finire complessivo}	
	Date le considerazioni precedenti, il preventivo complessivo resta invariato.