\subsubsection{12$^{\circ}$ Incremento}
		
	\subsubsubsection{Consuntivo}
		\begin{longtable}{
				>{\centering}p{0.25\textwidth}
				>{\centering}p{0.08\textwidth}
				>{\centering}p{0.08\textwidth}
				>{\centering}p{0.15\textwidth}
				>{\centering\arraybackslash}p{0.15\textwidth} }
			
			\coloredTableHead
			\textbf{\color{white}Ruolo} &
			\textbf{\color{white}Ore} &
			\textbf{\color{white}Delta ore} &
			\textbf{\color{white}Costo in \euro{}} &
			\textbf{\color{white}Delta costo}
			\tabularnewline
			\endhead
			
			% Contenuto della tabella
			% Ruolo & OreEffettive & DeltaOre & Costo & DeltaCosto \\
			Responsabile    & 3 & +0 &   90,00 & +  0,00 \\
			Amministratore  & 3 & +0 &   60,00 & +  0,00 \\
			Analista        & 0 & +0 &   0,00 & + 0,00 \\
			Progettista     & 10 & +0 & 220,00 & + 0,00 \\
			Programmatore   & 45 & +0 &   675,00 &  +0,00 \\
			Verificatore    & 15 & +0 & 225,00 & + 0,00 \\
			\textbf{Totale Effettivo} & \multicolumn{2}{c}{\textbf{76}} & \multicolumn{2}{c}{\textbf{1270,00}} \\
			\textbf{Delta} & \multicolumn{2}{c}{\textbf{0}} & \multicolumn{2}{c}{\textbf{+0,00}} \\
			
			\rowcolor{white}\caption{Consuntivo per il 12$^{\circ}$ incremento}	\\
			
		\end{longtable}
		
	
	\subsubsubsection{Conclusioni}
	In questo periodo il gruppo si è dedicato principalmente all'implementazione della funzionalità di deployment di una funzione all'interno della piattaforma \textit{Etherless}. Essendo una delle funzionalità di maggiore importanza del prodotto, il gruppo aveva preventivato un numero di ore di codifica alquanto elevato rispetto agli altri incrementi. 
	
	\subsubsubsection{Preventivo a finire rispetto al periodo}
	La pianificazione in questo incremento è stata rispettata e il bilancio risulta essere \textbf{pari}. Il gruppo è riuscito a rispettare la pianificazione in modo migliore rispetto agli incrementi precedenti. 
		
	\subsubsubsection{Preventivo a finire complessivo}	
	Date le considerazioni precedenti, il preventivo complessivo resta invariato.