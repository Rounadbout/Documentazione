\subsubsection{15$^{\circ}$ Incremento}

	\subsubsubsection{Consuntivo}
		\begin{longtable}{
				>{\centering}p{0.25\textwidth}
				>{\centering}p{0.08\textwidth}
				>{\centering}p{0.08\textwidth}
				>{\centering}p{0.15\textwidth}
				>{\centering\arraybackslash}p{0.15\textwidth} }
			
			\coloredTableHead
			\textbf{\color{white}Ruolo} &
			\textbf{\color{white}Ore} &
			\textbf{\color{white}Delta ore} &
			\textbf{\color{white}Costo in \euro{}} &
			\textbf{\color{white}Delta costo}
			\tabularnewline
			\endhead
			
			% Contenuto della tabella
			% Ruolo & OreEffettive & DeltaOre & Costo & DeltaCosto \\
			Responsabile    & 3 & +0 &   90,00 & +  0,00 \\
			Amministratore  & 4 & -2 &   80,00 & -  40,00 \\
			Analista        & 0 & +0 &   0,00 & + 0,00 \\
			Progettista     & 2 & +2 & 44,00 & + 44,00 \\
			Programmatore   & 1 & +2 &   15,00 &  +30,00 \\
			Verificatore    & 2 & +0 & 30,00 & + 0,00 \\
			\textbf{Totale Effettivo} & \multicolumn{2}{c}{\textbf{12}} & \multicolumn{2}{c}{\textbf{259,00}} \\
			\textbf{Delta} & \multicolumn{2}{c}{\textbf{+2}} & \multicolumn{2}{c}{\textbf{+34,00}} \\
			
			\rowcolor{white}\caption{Consuntivo per il nono incremento}	\\
			
		\end{longtable}
			
	\subsubsubsection{Conclusioni}
	In questo periodo il gruppo si è occupato di scrivere una prima versione del Developer Manual e dello User Manual. Sono state inoltre apportate delle modifiche e incrementi al resto della documentazione. \\
	A seguito della Product Baseline è stato necessario utilizzare delle ore non preventivate per correggere alcuni errori presenti nella progettazione e applicare tali modifiche anche al prodotto software. Tale differenza dalla pianificazione è stata limitata dalla riduzione del numero di ore richieste da parte degli Amministratori. 
	
	\subsubsubsection{Preventivo a finire rispetto alla fase}
	Il bilancio è \textbf{negativo}, con un una spesa in eccesso di \textbf{34,00\euro{}}. Non si ritiene necessaria alcuna ripianificazione, poichè la somma non è significativa e viene in parte bilanciata dal risparmio ottenuto in alcuni degli incrementi precedenti. 
	
	\subsubsubsection{Preventivo a finire complessivo}	
	Date le considerazioni precedenti, il preventivo complessivo resta invariato.