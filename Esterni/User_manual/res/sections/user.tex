\section{Functionalities}
Etherless-cli is a command line application that allow the user to interact with the Etherless platform. The user can perform specific tasks using simple and easy commands.

\subsection{Command usage}
The commands syntax used is the following:
\begin{center}
	\code{etherless <command> [parameters..]}
\end{center}
Where:
\begin{itemize}
	\item command is the command name;
	\item parameters are some parameters the command may need.
\end{itemize}

\noindent The commands available on etherless-cli are:
\begin{itemize}
	\item signup
	\item init
	\item login [m] <value>
	\item history
	\item list [m]
	\item whoami
	\item search <keyword>
	\item info <function\_name>
	\item run <function\_name> [params..]
	\item logout
	\item delete <function\_name>
	\item deploy <function\_name> <path> <desc>
\end{itemize}

\subsection{Account management}
To uniquely identify a user and manage payments the Ethereum network is used. By logging in with an Ethereum account, you give the application the permission to detract the fees, needed for the commands execution, from your wallet. \\
Every Ethereum account is identified by an address and is accessible through a private key or mnemonic phrase; this application allow you to use both of them. \\
To execute some commands you will be asked to enter a password, which will be used to encrypt and decrypt a local copy of your wallet.

\subsubsection{Init}
To read the details about the signup and authentication process you can use the \texttt{init} command, with the following syntax:
\begin{center}
	\code{etherless init}
\end{center}

\subsubsection{Signup}
The \texttt{signup} command allow the user to create a new Ethereum account. The account has no ethers and can not perform all the functionality the product provides.
To fully use the account the user need to add some currency through online services such as Etherscan.
Use the command with the following syntax:

\begin{center}
	\code{etherless signup [save]}
\end{center}

\noindent The execution of this command shows the credentials of the newly created account, particularly:
\begin{itemize}
	\item an address that identifies the Ethereum account;
	\item the private key associated to the Ethereum account;
	\item the mnemonic phrase associated to the Ethereum account.
\end{itemize}
With the optional flag \texttt{save} you can choose to save the credentials into a file that will be created in the current directory. \\
Note that the signup command creates an account, but does not log in the user.

\subsubsection{Login}
The \texttt{login} command allows the user to log into the Ethereum network with an existing account.
This authentication process can be performed in two ways: with private key or with mnemonic phrase.
A password will be asked during the command execution, which is used to store a safe copy of the wallet locally. If the authentication process completes successfully a success message will be shown.

\subsubsubsection{Login with private key}
The user can log inside his wallet using his private key, the command syntax is the following: \\
\begin{center}
	\code{etherless login <private\_key>}
\end{center}

\subsubsubsection{Login with mnemonic phrase}
The user can log inside his Ethereum wallet using the associated mnemonic phrase; the command syntax is the following:

\begin{center}
	\code{etherless login -m <mnemonic\_phrase>}
\end{center}
The flag \texttt{-m} must be present and the mnemonic phrase must be inserted without quotation marks.

\subsubsubsection{Whoami}
The user can perfrom this command to get the address of the wallet associated with the current session. If the user tries to execute this command without being logged, an error message will be shown.

The command syntax is the following: \\
\begin{center}
	\code{etherless whoami}
\end{center}
After the execution of the command the address of the wallet associated with the current session will be displyed.

\subsubsubsection{Logout}
The user can run this command to log out from the Ethereum network; the command syntax is:
\begin{center}
	\code{etherless logout}
\end{center}
After executing this command, all the information about the previous user session will be deleted.

\subsection{List functions}
The application allows the users to see a list of all the functions available inside the platform or only the functions of their property. \\
\noindent In both cases for every function the following data will be provided:
\begin{itemize}
	\item \textbf{name: } the name of the function;
	\item \textbf{price: } the cost to run the function.
\end{itemize}

\subsubsubsection{List all functions}
The user can list all the functions available inside the Etherless platform using the command \texttt{list}, with the following syntax:
\begin{center}
	\code{etherless list}
\end{center}
\subsubsubsection{List only owned functions}
The user can list all the functions of their property inside the Etherless platform using the command \texttt{list}, with the following syntax:
\begin{center}
	\code{etherless list -m}
\end{center}

\subsection{Search for a function}
To list all the functions having a keyword in their name use the command \textit{search}, with the following syntax:
\begin{center}
	\code{etherless search <keyword>}
\end{center}

A list of functions will be displayed. For each function the following data will be provided:
\begin{itemize}
	\item \textbf{name: } the name of the function;
	\item \textbf{price: } the cost to run the function.
\end{itemize}

\subsection{Function details}
The user can request to visualize the details of a specific function through the \texttt{info} command, with the following syntax: \\
\begin{center}
	\code{etherless info <function\_name>}
\end{center}
Particularly, the following information will be shown:
\begin{itemize}
	\item \textbf{owner: } the address of the function owner;
	\item \textbf{signature: } the parameters that the function needs to run;
	\item \textbf{price: } the cost to run the function;
	\item \textbf{description: } a textual description of the function.
\end{itemize}

\subsection{Run a function}
The \texttt{run} command is used to run a specific function, with the following syntax: \\
\begin{center}
	\code{etherless run <function\_name> [params..]}
\end{center}
Run takes some times to retrieve the resut. Once the task is completed the result is displayed. If the request was not succesfull an error will be displayed instead and the cost will be refunded. If the function is executed correctly the price amount is detracted from the current user wallet.

\subsection{Delete a function}
The user can delete a previously deployed function through the \texttt{delete} command, with the following syntax:
\begin{center}
	\code{etherless delete <function\_name>}
\end{center}
After the execution of the command, the application will show an error or a success message, depending on the outcome of the operation.

\subsection{Deploy a function}
The user can deploy a function through the deploy command with the following syntax:
\begin{center}
	\code{etherless delete <function\_name> <path> <desc>}
\end{center}
\noindent where:
\begin{itemize}
	\item function\_name: is the name of the function to be deployed;
	\item path: is the relative path from the current position to the source file;
	\item desc: is a description of the function.
\end{itemize}
You can not specify the execution cost of your function, instead it will be chosen by the system. At the moment the execution price is fixed; we plan to determine it dynamically (based on the resource amount used by the function) in the future. After the execution of this command  the application will show an error or a success message depending on the outcome of the operation.
