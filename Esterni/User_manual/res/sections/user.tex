\section{Functionalities}
Etherless-cli is a command line application that allows the user to interact with the \textit{Etherless} platform. The user can perform specific tasks using simple and easy commands.

\subsection{Command syntax}
The commands syntax used in \textit{Etherless} is as follows:
\begin{center}
	\code{etherless <command\_name> [-flag] [params..]}
\end{center}
Where:
\begin{itemize}
	\item \texttt{command\_name}: is the command name to be executed;
	\item \texttt{flag}: is a flag attaching a particular behaviour to a certain command;
	\item \texttt{parameters}: are parameters the command may need to execute.
\end{itemize}
Moreover:
\begin{itemize}
	\item \texttt{<>}: terms within angular brackets are mandatory for the command to execute properly;
	\item \texttt{[]}: terms within square brackets are optional. This means that command execution can happen either with or without them;
	\item \texttt{-flag}: each term indicating a flag should be preceded by a dash.
	\item \texttt{option\_1 | option\_2}: elements separated by a vertical line mean you can choose among them;
\end{itemize}

\subsection{Available commands}
\noindent The commands available on \textit{Etherless} are:
\begin{itemize}
	\item \textbf{init}: returns information about the authentication process;
	\item \textbf{signup}: signs you up to the application;
	\item \textbf{login}: logs you into the application;
	\item \textbf{whoami}: returns information about your account;
	\item \textbf{logout}: logs you out of the application;
	\item \textbf{list}: lists all the functions you can execute;
	\item \textbf{info}: returns detailed information about a certain function;
	\item \textbf{search}: returns the search conducted on a certain keyword;
	\item \textbf{history}: returns the log of the commands you have executed;
	\item \textbf{run}: requests the execution of a function;
	\item \textbf{deploy}: requests the deployment of a function;
	\item \textbf{edit}: requests the modification of a function;
	\item \textbf{delete}: requests the deletion of a function.
\end{itemize}
You can find a detailed description of the above mentioned commands in section \ref{commands}.

\subsection{Account management}
\textit{Etherless} uses the Ethereum$_{G}$ network to uniquely identify a user and manage payments. This means that, in order to interact with the application, you need to log in with an Ethereum$_{G}$ account. By doing so, you give the application permission to detract the fees needed for the commands execution from your wallet. \\
Every Ethereum$_{G}$ account is identified by an address and is accessible either through a private key$_{G}$ or mnemonic phrase$_{G}$. \textit{Etherless} lets you use whichever you prefer to log into the application. \\
To execute some commands you will be asked to enter a password, which will be used to encrypt and decrypt a local copy of your wallet.

\subsection{Command explanation}\label{commands}
This section goes into detail regarding all the commands you can execute from your terminal window within the application environment.

\subsubsection{Init}
To get information about the signup and authentication process you can use the \texttt{init} command. The syntax is as follows:
\begin{center}
	\code{etherless init}
\end{center}

\subsubsection{Signup}
The \texttt{signup} command allows you to create a new Ethereum$_{G}$ account to log into the application. The account created has no Ethers$_{G}$ in the wallet associated with it. This means that, by simply creating an account, you cannot perform all the functionalities that \textit{Etherless} provides, since some of them require a certain amount of cryptocurrency to be executed. \\
To fully use the account you need to add some currency through other online services. \\
You can sing up to \textit{Etherless} using the following syntax:

\begin{center}
	\code{etherless signup [-save]}
\end{center}

\noindent The execution of this command returns the credentials of the newly created account, which are:
\begin{itemize}
	\item an address that identifies the Ethereum$_{G}$ account;
	\item the private key$_{G}$ associated with the Ethereum$_{G}$ account;
	\item the mnemonic phrase$_{G}$ associated with the Ethereum$_{G}$ account.
\end{itemize}
Appending the optional flag \texttt{save} to the command lets you save the credentials into a file that will be created in the current directory. \\
Note that the \texttt{signup} command creates an account, but it does not log you in. To do that, you have to use the designated \texttt{login} command.

\subsubsection{Login}
If you already have an Ethereum$_{G}$ account, you can log into \textit{Etherless} using the \texttt{login} command. There are two different ways to execute this command, either by entering the private key$_{G}$ associated with your Ethereum$_{G}$ account or the mnemonic phrase$_{G}$, as shown below: \\
\begin{center}
	\code{etherless login <your\_private\_key> | -m <your\_mnemonic\_phrase>}
\end{center}
During command execution you will be asked a password in order to store a safe copy of your wallet in the current directory. \\
If the authentication process completes successfully, a success message will be shown.

\subsubsubsection{Login with private key}
You can log inside your wallet using your private key$_{G}$. The command syntax to do this is the following:
\begin{center}
	\code{etherless login <your\_private\_key>}
\end{center}

\subsubsubsection{Login with mnemonic phrase}
You can log inside your Ethereum$_{G}$ wallet using the associated mnemonic phrase$_{G}$. The command syntax to do this is the following:
\begin{center}
	\code{etherless login -m <your\_mnemonic\_phrase>}
\end{center}
To correctly execute this command the flag \texttt{-m} must be present and the mnemonic phrase$_{G}$ must be inserted without quotation marks.

\subsubsection{Whoami}
By executing this command you get the address of the wallet associated with the current session. If you try to execute this command without being logged in, an error message will be shown. The command syntax is as follows: 
\begin{center}
	\code{etherless whoami}
\end{center}
After the execution of the command the address of the wallet associated with the current session will be displayed.

\subsubsection{Logout}
You can run this command to log out from the Ethereum$_{G}$ network. The command syntax is:
\begin{center}
	\code{etherless logout}
\end{center}
After executing this command, all the information about the previous user session will be deleted.

\subsubsection{List}
\textit{Etherless} allows you to list of all the functions available for execution inside the platform. Moreover, you can list all the functions of you property, if you have deployed any. \\
The syntax needed to do this is as follows:
\begin{center}
	\code{etherless list | list -m}
\end{center}
\noindent In both cases, for each function the following data will be provided:
\begin{itemize}
	\item \textbf{name}: the function name;
	\item \textbf{price}: the cost required to run the function.
\end{itemize}

\subsubsubsection{List all functions}
You can list all the functions available on \textit{Etherless} by typing:
\begin{center}
	\code{etherless list}
\end{center}
\subsubsubsection{List owned functions}
If you have deployed any function on \textit{Etherless}, you can get them back in a list by typing:
\begin{center}
	\code{etherless list -m}
\end{center}

\subsubsection{Info}
The \texttt{info} command returns the details of a specific function. To execute it, use the following syntax: \\
\begin{center}
	\code{etherless info <function\_name>}
\end{center}
In particular, the following information will be shown:
\begin{itemize}
	\item \textbf{owner}: the address of the function owner;
	\item \textbf{signature}: the parameters that the function needs to run;
	\item \textbf{price}: the cost to run the function;
	\item \textbf{description}: a brief description of the function.
\end{itemize}

\subsubsection{Search}
If you're looking for a certain function, use the command \texttt{search}.To execute this command, the following syntax is required:
\begin{center}
	\code{etherless search <keyword>}
\end{center}

A list of functions whose name matches the search will be displayed. For each function the following data will be provided:
\begin{itemize}
	\item \textbf{name}: the name of the function;
	\item \textbf{price}: the cost to run the function.
\end{itemize}

\subsubsection{History}
The command \texttt{history} displays a list of the commands you previously executed since logging in. To use this command, use the syntax:
\begin{center}
	\code{etherless history [-limit]}
\end{center}
Where \texttt{limit} is an optional flag you can use to indicate the maximum number of elements you want to see. \\
This command returns

\subsubsection{Run}
The \texttt{run} command is used to run a specific function, with the following syntax: \\
\begin{center}
	\code{etherless run <function\_name> [params..]}
\end{center}
Because \textit{Etherless} runs on the blockchain, execution takes some time to retrieve the result. Once the task is completed the result will be displayed. If the request was not successful an error will be displayed instead and the cost will be refunded. On the other hand, if the function is executed correctly the price amount is detracted from the current user wallet.

\subsubsection{Deploy}
You can deploy a function on AWS Lambda using the \texttt{deploy} command following the syntax:
\begin{center}
	\code{etherless delete <function\_name> <path> <desc>}
\end{center}
\noindent where:
\begin{itemize}
	\item \texttt{function\_name}: is the name of the function to be deployed;
	\item \texttt{path}: is the relative path from the current position to the source file;
	\item \texttt{desc}: is a description of the function.
\end{itemize}
At the moment of execution the price is fixed: the cost of your function will be chosen by the system. We plan to determine it dynamically (based on the amount of resources used by the function) in the future. After the execution of such command the application will either show an error or a success message depending on the outcome of the operation.

\subsubsection{Edit}
You can edit a previously deployed function of your own through the \texttt{edit} command, using the following syntax:
\begin{center}
	\code{etherless edit <function\_name>}
\end{center}

\subsubsection{Delete}
You can delete a previously deployed function of your own through the \texttt{delete} command, using the following syntax:
\begin{center}
	\code{etherless delete <function\_name>}
\end{center}
After the execution of this command, the application will either show an error or a success message, depending on the outcome of the operation.