\section{Functionalities}
Etherless-cli is a command line application that allows the user to interact with the \textit{Etherless} platform. The user can perform specific tasks using simple and easy commands.

\subsection{Command syntax}
The commands syntax used in \textit{Etherless} is as follows:
\begin{center}
	\code{etherless <command\_name> [-flag] [params..]}
\end{center}
Where:
\begin{itemize}
	\item \texttt{command\_name}: is the command name to be executed;
	\item \texttt{flag}: is a flag attaching a particular behaviour to a certain command;
	\item \texttt{parameters}: are parameters the command may need to execute.
\end{itemize}
Moreover:
\begin{itemize}
	\item \texttt{<>}: terms within angular brackets are mandatory for the command to execute properly;
	\item \texttt{[]}: terms within square brackets are optional. This means that command execution can happen either with or without them;
	\item \texttt{-flag}: each term indicating a flag should be preceded by at least one dash. Flags requiring two dashes to properly execute will be pointed out specifically in the command syntax;
	\item \texttt{option\_1 | option\_2}: elements separated by a vertical line mean you can choose among them;
\end{itemize}

\subsection{Available commands}
\noindent The commands available on \textit{Etherless} are:
\begin{itemize}
	\item \textbf{init}: returns information about the authentication process;
	\item \textbf{signup}: signs you up to the application;
	\item \textbf{login}: logs you into the application;
	\item \textbf{whoami}: returns information about your account;
	\item \textbf{logout}: logs you out of the application;
	\item \textbf{list}: lists all the functions you can execute;
	\item \textbf{info}: returns detailed information about a certain function;
	\item \textbf{search}: returns the search conducted on a certain keyword;
	\item \textbf{history}: returns the log of the requests you have made;
	\item \textbf{run}: requests the execution of a function;
	\item \textbf{deploy}: requests the deployment of a function;
	\item \textbf{edit}: requests the modification of a function;
	\item \textbf{delete}: requests the deletion of a function.
\end{itemize}
You can find a detailed description of the above mentioned commands in section \ref{commands}.

\subsection{Account management}
\textit{Etherless} uses the Ethereum$_{G}$ network to uniquely identify a user and manage payments. This means that, in order to interact with the application, you need to log in with an Ethereum$_{G}$ account. By doing so, you give the application permission to detract the fees needed for the commands execution from your wallet. \\
Every Ethereum$_{G}$ account is identified by an address and is accessible either through a private key$_{G}$ or mnemonic phrase$_{G}$. \textit{Etherless} lets you use whichever you prefer to log into the application. \\
To execute some commands you will be asked to enter a password, which will be used to encrypt and decrypt a local copy of your wallet.

\subsection{Command explanation}\label{commands}
This section goes into detail regarding all the commands you can execute from your terminal window within the application environment.

\subsubsection{Init}
To get information about the signup and authentication process you can use the \texttt{init} command. The syntax is as follows:
\begin{center}
	\code{etherless init}
\end{center}
\subsubsubsection*{Usage}
If you don't know how to sign up or log in and want to get more information about it, just type from your terminal window the \texttt{init} command you see in the above paragraph.
You will get back a list of useful commands you can execute and the respective explanation of what they do.

\subsubsection{Signup}
The \texttt{signup} command allows you to create a new Ethereum$_{G}$ account to log into the application. The account created has no Ethers$_{G}$ in the wallet associated with it. This means that, by simply creating an account, you cannot perform all the functionalities that \textit{Etherless} provides, since some of them require a certain amount of cryptocurrency to be executed. \\
To fully use the account you need to add some currency through other online services. The \href{https://ethereum.org/en/wallets/}{Ethereum$_{G}$ official website} provides some useful references to do this.\\
You can sign up to \textit{Etherless} using the following syntax:
\begin{center}
	\code{etherless signup [---save]}
\end{center}
\noindent The execution of this command returns the credentials of the newly created account, which are:
\begin{itemize}
	\item an address that identifies the Ethereum$_{G}$ account;
	\item the private key$_{G}$ associated with the Ethereum$_{G}$ account;
	\item the mnemonic phrase$_{G}$ associated with the Ethereum$_{G}$ account.
\end{itemize}
Appending the optional flag \texttt{---save} to the command lets you save the credentials into a file that will be created in the current directory. \\
Note that the \texttt{signup} command creates an account, but it does not log you in. To do that, you have to use the designated \texttt{login} command.
\subsubsubsection*{Usage}
Let's suppose you want to sign up without saving your credentials. You should then just type from your terminal window:
\begin{center}
	\code{etherless signup}
\end{center}
On the other hand, if you want your credentials to be saved on a file in the current directory, you should type:
\begin{center}
	\code{etherless signup ---save}
\end{center}
Note that saving your credentials on a file does not automate the authentication process, meaning when logging in you still have to insert them manually.

\subsubsection{Login}
If you already have an Ethereum$_{G}$ account, you can log into \textit{Etherless} using the \texttt{login} command. There are two different ways to execute this command, either by entering the private key$_{G}$ associated with your Ethereum$_{G}$ account or the mnemonic phrase$_{G}$, as shown below: \\
\begin{center}
	\code{etherless login <your\_private\_key> | -m <your\_mnemonic\_phrase>}
\end{center}
During command execution you will be asked a password in order to store a safe copy of your wallet in the current directory. \\
If the authentication process completes successfully, a success message will be shown.

\subsubsubsection{Login with private key}
You can log inside your wallet using your private key$_{G}$. The command syntax to do this is the following:
\begin{center}
	\code{etherless login <your\_private\_key>}
\end{center}

\subsubsubsection{Login with mnemonic phrase}
You can log inside your Ethereum$_{G}$ wallet using the associated mnemonic phrase$_{G}$. The command syntax to do this is the following:
\begin{center}
	\code{etherless login -m <your\_mnemonic\_phrase>}
\end{center}

\subsubsubsection*{Usage}
Let's suppose you already have an Ethereum$_{G}$ account and you want to log in using your mnemonic phrase$_{G}$, which is \texttt{this is one mnemonic phrase example}. To do this, just type from your terminal window:
\begin{center}
	\code{etherless login -m this is one mnemonic phrase example}
\end{center}

On the other hand, you might want to use your private key$_{G}$ to log in. For the sake of the example,  let's suppose your private key$_{G}$ is 00000...001. To do this, just type from your terminal window:
\begin{center}
	\code{etherless login 0000...001}
\end{center}

\subsubsection{Whoami}
By executing this command you get the address of the wallet associated with the current session. If you try to execute this command without being logged in, an error message will be shown. The command syntax is as follows: 
\begin{center}
	\code{etherless whoami}
\end{center}
Following the command execution the address of the wallet associated with the current session will be displayed.
\subsubsubsection*{Usage}
Let's suppose you have logged into \textit{Etherless} and, for some reason, you need to retrieve information regarding your account address. In order to get that information just type from your terminal window the \texttt{whoami} command found in the above paragraph.

\subsubsection{Logout}
You can run this command to log out from the Ethereum$_{G}$ network. The command syntax is:
\begin{center}
	\code{etherless logout}
\end{center}
After executing this command, all the information about the previous user session will be deleted.
\subsubsubsection*{Usage}
Suppose you have logged into \textit{Etherless}, deployed some functions and executed others. You had finished doing what you had to and you want to log out from the platform. To do that, just type from your terminal window the \texttt{logout} command in the paragraph above.

\subsubsection{List}\label{list}
\textit{Etherless} allows you to list all the functions available for execution inside the platform. Moreover, you can list all the functions of you property using the \texttt{-m} flag, if you have deployed any. \\
Respectively, the syntax needed to do this is as follows:
\begin{center}
	\code{etherless list [-m]}
\end{center}
\noindent In both cases, for each function the following data will be provided:
\begin{itemize}
	\item \textbf{name}: the function name;
	\item \textbf{signature}: the parameters the function needs to execute properly;
	\item \textbf{price}: the cost required to run the function.
\end{itemize}

\subsubsubsection{List all functions}
You can get back the name and the price of all the functions available on \textit{Etherless} by typing:
\begin{center}
	\code{etherless list}
\end{center}

\subsubsubsection{List owned functions}
If you have deployed any function on \textit{Etherless}, you can get them back in a list by typing:
\begin{center}
	\code{etherless list -m}
\end{center}

\subsubsubsection*{Usage}
Let's say you want to know which functions are available for execution on \textit{Etherless}. To get that information, just type from your terminal window:
\begin{center}
	\code{etherless list}
\end{center}
On the other hand, if you want to get information about all the functions you have deployed, type:
\begin{center}
	\code{etherless list -m}
\end{center}

\subsubsection{Info}
The \texttt{info} command returns the details of a specific function. To execute it, use the following syntax: \\
\begin{center}
	\code{etherless info <function\_name>}
\end{center}
In particular, the following information will be shown:
\begin{itemize}
	\item \textbf{owner}: the address of the function owner;
	\item \textbf{signature}: the parameters that the function needs to run;
	\item \textbf{price}: the cost to run the function;
	\item \textbf{description}: a brief description of the function.
\end{itemize}
\subsubsubsection*{Usage}
Let's say you want to get more information about a function called \texttt{subtract} (which you know is available on the cloud$_{G}$ for execution). To do this, type from your terminal window:
\begin{center}
	\code{etherless info subtract}
\end{center}
As stated above, after this command execution you will get back information such as the function owner, the function signature, its price and a brief description.

\subsubsection{Search}
You can use the \texttt{search} command if you're looking for a certain function. To execute this command, the following syntax is required:
\begin{center}
	\code{etherless search <keyword>}
\end{center}

A list of functions whose name matches the keyword used will be displayed. For each function, the following information will be provided:
\begin{itemize}
	\item \textbf{name}: the name of the function;
	\item \textbf{price}: the cost to run the function.
\end{itemize}
\subsubsubsection*{Usage}
Let's say you want to find a function to execute a subtraction. You decide to use the keyword \texttt{sub} instead of \texttt{subtract} to get back as many results as possible. To do this, type from your terminal window:
\begin{center}
	\code{etherless search sub}
\end{center}
You should get back all the functions whose name contains the keyword \texttt{sub}.

\subsubsection{History}
The \texttt{history} command displays a list of the all the request you made that were previously executed since logging into \textit{Etherless}. These requests are displayed with the respective results they produced. To use this command, follow the syntax:
\begin{center}
	\code{etherless history [---limit] [number\_of\_executions]}
\end{center}
Where \texttt{limit} is an optional flag you can use to indicate the maximum number of elements you want to be displayed. \\
\subsubsubsection*{Usage}
The \texttt{history} command can be executed in three ways, in particular:
\begin{itemize}
\item you can display all the requests you've made since logging in by typing from your terminal window:
\begin{center}
\code{etherless history}
\end{center}
\item Alternatively, you can choose to limit the number of elements you get back (let's say you want to see the last ten requests you've made) either by typing:
\begin{center}
\code{etherless history 10}
\end{center}
Or by typing:
\begin{center}
\code{etherless history ---limit 10}
\end{center}
\end{itemize}

\subsubsection{Run}
The \texttt{run} command is used to run a specific function that has already been deployed on AWS Lambda. You can execute this command using the following syntax: \\
\begin{center}
	\code{etherless run <function\_name> [parameter\_list]}
\end{center}
Since \textit{Etherless} runs on the blockchain$_{G}$, execution takes some time to retrieve the result. Once the task is completed the result will be displayed. If the function is executed correctly the price amount is detracted from the current user wallet. On the other hand, if the request is not successful, an error will be displayed instead and, if the error was caused by the system, the cost will be refunded.
\subsubsubsection*{Usage}
Let's say you want to run a subtraction function called \texttt{subtract}. You know that in order to execute this function needs two parameters in the form of numbers. To correctly execute this command, just type from your terminal window:
\begin{center}
	\code{etherless run subtract 3 2}
\end{center}
After waiting some time for the command to execute (which you will be notified about) you should get back the result of said subtraction, which in this case should be \texttt{1}. \\
Note that you can only execute functions which have already been deployed on the cloud$_{G}$. To verify which functions are available for execution, use the \texttt{list} command (see section \ref{list} for more information about this).

\subsubsection{Deploy}\label{deploy}
You can deploy a function on AWS Lambda using the \texttt{deploy} command following the syntax:
\begin{center}
	\code{etherless deploy <function\_name> <path> <desc>}
\end{center}
\noindent Where:
\begin{itemize}
	\item \texttt{function\_name}: is the name of the function to be deployed. This field cannot be longer than 30 characters;
	\item \texttt{path}: is the relative path to the source file/folder from the current directory. If the specified path is relative to a single source file, the deployment will be performed without considering any external dependencies. For more information about deploying a function with dependencies, see section \ref{dep};
	\item \texttt{desc}: is the function's description. This field cannot exceed 150 characters.
\end{itemize}
At the moment of execution the price is fixed, that is to say the cost of your function will be chosen by the system. In the future we plan to determine it dynamically, based on the amount of resources required by the function to execute. \\
As a result the application will either show an error or a success message depending on the outcome of the operation.
\subsubsubsection{Deploy with dependencies}\label{dep}
If the specified \texttt{path} is relative to a directory, the latter should contain the following named files:
\begin{itemize}
\item \textbf{index.js}: the source code file;
\item \textbf{package.json}: the file used to handle external dependencies. For more information about this, consult \href{https://docs.npmjs.com/files/package.json}{the official npm documentation};
\item \textbf{package-lock.json}: another file used to handle external dependencies. For more information about this, consult \href{https://docs.npmjs.com/configuring-npm/package-lock-json.html}{the official npm documentation}.
\end{itemize}
\subsubsubsection*{Usage}
Let's say you want to deploy a multiplication function called \texttt{mul}, whose source code (a file named \texttt{mul.js}) is located in the same directory you're in. To deploy such function on AWS Lambda using \textit{Etherless}, the following steps are required:
\begin{itemize}
\item make sure no other function with the same name has been deployed. To do so, use the \texttt{list} command (see section \ref{list} for more information). If your function's name already exists, choose a different name;
\item if the name you have chosen to deploy your function is unique, type from your terminal window:
\begin{center}
	\code{etherless deploy mul mul.js "This is a multiplication function"}
\end{center}
Since the file \texttt{mul.js} is located in the current directory, only its name is required for the command to properly execute. However, supposing \texttt{mul.js} is contained within a folder named \texttt{source\_code} in the current directory, the \texttt{deploy} command syntax would change as follows:
\begin{center}
	\code{etherless deploy mul ./source\_code/mul.js "This is a multiplication function"}
\end{center}
Meaning \texttt{path} should contain the relative path to the source code from the current directory. \\
However, should you need to deploy a function with its external dependencies (see section \ref{dep}), just type the folder path which contains the function you want to deploy a as the \texttt{path}. So for example, if your directory called \texttt{source\_code} contains the function you want to deploy and its dependencies files, just type from your terminal window:
\begin{center}
	\code{etherless deploy mul ./source\_code "This is a multiplication function"}
\end{center}
\end{itemize}

\subsubsection{Edit}
You can edit a previously deployed function of your own through the \texttt{edit} command. To edit both a function's source code and its description, just type from your terminal window:
\begin{center}
	\code{etherless edit <function\_name> <path> <"new description">}
\end{center}
Where:
\begin{itemize}
\item \texttt{function\_name}: is the name of the function you want to edit;
\item \texttt{path}: is the relative path to the source file from the current directory. For more information regarding \texttt{path}, see section \ref{deploy};
\item \texttt{new description}: is a string containing the updated description you want to upload. If your description only consists of one word, quotation marks can be omitted.
 \end{itemize}

On the other hand, you can choose to only edit the function's description or its source code. Respectvely, the syntax to do so is as follows:
\begin{center}
	\code{etherless edit <function\_name> -d "new description" | -s <path>}
\end{center}
\subsubsubsection{Edit a function's description}
In order to edit a function's description, the syntax to be used is as follows:
\begin{center}
	\code{etherless edit <function\_name> -d "new description"}
\end{center}
\textbf{N.B.} the new description must be at most 150 characters long. \\

\subsubsubsection{Edit a function's source code}
To edit a function's source code, the syntax to be used is as follows:
\begin{center}
	\code{etherless edit <function\_name> -s <path>}
\end{center}

\subsubsubsection*{Usage}
Suppose some time ago you deployed a function called \texttt{mul} and now you want to change its behaviour. In particular, you want to update its source code, which is contained in a file called \texttt{mul.js} in the current directory. To do that, type from your terminal window:
\begin{center}
	\code{etherless edit mul -s mul.js}
\end{center}
Now that you've updated your function's behaviour, you want the description to match. To change a function's description, just type from your terminal window:
\begin{center}
	\code{etherless edit mul -d "Now the mul function does this and this and that"}
\end{center}
However, if you want to modify both the source code and the function's description at the same time, just type from your terminal window:
\begin{center}
	\code{etherless edit mul mul.js "Now the mul function does this and this and that"}
\end{center}
The Edit function can handle functions external dependencies the same way the Deploy function does (see \ref{dep}). 

\subsubsection{Delete}
You can delete a previously deployed function of your own through the \texttt{delete} command, using the following syntax:
\begin{center}
	\code{etherless delete <function\_name>}
\end{center}
After the execution of this command, the application will either show an error or a success message, depending on the outcome of the operation.
\subsubsubsection*{Usage}
Let's say you want to delete a function called \texttt{mul} you previously deployed. To do that, just type from your terminal window:
\begin{center}
	\code{etherless delete mul}
\end{center}