\section{Verbale}
\subsection{Richiesta di chiarimenti riguardo il capitolato C2 - \NomeProgetto}
I membri del gruppo hanno esposto una serie di domande volte a chiarire dei dubbi su parti del capitolato e varie tecnologie da utilizzare.
Di seguito si trova un riassunto delle informazioni estrapolate dalla discussione:
\begin{description}
	\item[Ambiente di sviluppo:] la scelta dell'ambiente di sviluppo per la programmazione degli smart contract in Solidity è stata lasciata al team. E' stato consigliato l'uso di un programma di scripting invece di un IDE. Alcune valide opzioni possono essere Visual Studio Code e Truffle;
	\item[Identificazione delle funzioni:] Le funzioni possono essere identificate con il nome;
	\item[Libreria per comunicazione con nodi Ethereum:] La libreria proposta è Web3 ma non è vincolante. Possono essere valutate delle alternative come ethers.js, la scelta necessita di una attenta valutazione;
	\item[Costo delle funzioni:] Il costo della funzione può essere implementato in diversi modi: da una stima calcolata dallo sviluppatore, pensato come un sistema di crediti, oppure come escrow. La possibilità di stabilire un costo differente per la modifica e la rimozione delle funzioni è un  dettaglio implementativo che deve essere valutato. Lo stesso vale anche per il deploy delle funzioni;
	\item[Caricamento delle funzioni sulla piattaforma:] il comportamento della funzione caricata dallo sviluppatore non deve inficiare il funzionamento del sistema. Non è necessario applicare l'analisi statica sulla funzione ma è invece consigliato esercitare controlli sulla infrastruttura e sugli effetti che il codice vi può provocare applicando delle limitazioni;
	\item[Gestione del codice di deploy:] valutare la possibilità di non usare in questo caso gli smart contrct, ma effettuare una scelta implementativa alternativa;
\end{description}

\subsection{Tecnologie individuate}
Sono state individuate una serie di tecnologie da utilizzare per lo sviluppo:
\begin{itemize}
	\item \textbf{Solidity:} come linguaggio per l'implementazione degli smart contract;
	\item \textbf{Airbnb Javascript:} come standard che delinea il modo in cui il codice sarà scritto;
\end{itemize}
