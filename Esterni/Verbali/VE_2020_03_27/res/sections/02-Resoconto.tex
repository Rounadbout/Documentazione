\section{Verbale}
\subsection{Richiesta di chiarimenti riguardo il capitolato C2 - \NomeProgetto}
I membri del gruppo hanno esposto una serie di domande volte a chiarire dei dubbi su parti del capitolato\ped{\textit{G}} e varie tecnologie da utilizzare.
Di seguito si trova un riassunto delle informazioni estrapolate dalla discussione:
\begin{description}
	\item[Ambiente di sviluppo\ped{\textit{G}}:] la scelta dell'ambiente di sviluppo\ped{\textit{G}} per la programmazione degli smart contract\ped{\textit{G}} in Solidity\ped{\textit{G}} è stata lasciata al team. E' stato consigliato l'uso di un programma di scripting invece di un IDE\ped{\textit{G}}. Alcune valide opzioni possono essere Visual Studio Code e Truffle\ped{\textit{G}};
	\item[Identificazione delle funzioni:] le funzioni possono essere identificate con un nome unico;
	\item[Libreria per comunicazione con nodi Ethereum\ped{\textit{G}}:] la libreria proposta è Web3, ma non è vincolante. Possono essere valutate delle alternative come ethers.js, la scelta necessita di una attenta valutazione;
	\item[Costo delle funzioni:] il costo della funzione può essere implementato in diversi modi: da una stima calcolata dallo sviluppatore, pensato come un sistema di crediti, oppure come escrow\ped{\textit{G}}. La possibilità di stabilire un costo differente per la modifica e la rimozione delle funzioni è un  dettaglio implementativo che deve essere valutato. Lo stesso vale anche per il deploy\ped{\textit{G}} delle funzioni;
	\item[Caricamento delle funzioni sulla piattaforma:] il comportamento della funzione caricata dallo sviluppatore non deve inficiare il funzionamento del sistema. Non è necessario applicare l'analisi statica sulla funzione, ma è invece consigliato esercitare controlli sull'infrastruttura e sugli effetti che il codice vi può provocare applicando delle limitazioni;
	\item[Gestione del codice di deploy\ped{\textit{G}}:] valutare la possibilità di non usare in questo caso gli smart contract\ped{\textit{G}}, ma effettuare una scelta implementativa alternativa.
\end{description}

\subsection{Tecnologie individuate}
Sono state individuate una serie di tecnologie da utilizzare per lo sviluppo:
\begin{itemize}
	\item \textbf{Solidity\ped{\textit{G}}:} come linguaggio per l'implementazione degli smart contract\ped{\textit{G}};
	\item \textbf{Airbnb Javascript}\ped{\textit{G}}\textbf{:} come standard che delinea il modo in cui il codice sarà scritto.
\end{itemize}
