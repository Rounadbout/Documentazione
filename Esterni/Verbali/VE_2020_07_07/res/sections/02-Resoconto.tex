\section{Verbale}
\subsection{Aggiornamento sullo stato di sviluppo del prodotto}
Per prima cosa i Proponenti\ped{\textit{G}} sono stati aggiornati sullo stato di sviluppo del prodotto\ped{\textit{G}}. In particolare sono state analizzate tutte le funzionalità sviluppate, comprese quelle relative ai requisiti opzionali e desiderabili concordati. 

\subsection{Dimostrazione delle funzionalità implementate}
I comandi considerati durante tale dimostrazione sono stati quelli sviluppati nell'ultimo periodo, ossia:
\begin{itemize}
	\item modifica del codice sorgente di una funzione (\texttt{edit -s}); 
	\item modifica della descrizione di una funzione (\texttt{edit -d}); 
	\item visualizzazione delle passate richieste di esecuzione dell'utente (\texttt{history}).
\end{itemize}
Durante la dimostrazione i Proponenti\ped{\textit{G}} hanno richiesto alcuni dettagli riguardanti la gestione delle richieste eseguite dal modulo \textit{Etherless-cli}. Sono state inoltre considerate alcune situazioni limite a cui il modulo \textit{Etherless-server} potrebbe essere soggetto, e la modalità di gestione prevista. \\ 
Il gruppo ha poi discusso con i Proponenti\ped{\textit{G}} la possibilità di implementare il deployment\ped{\textit{G}} di funzioni aventi dipendenze verso librerie esterne (requisito opzionale). È stato concordato l'impegno da parte del gruppo di sviluppare tale funzionalità. In particolare, in questo caso si considera sempre un unico file sorgente, denominato "index.js", a cui dovranno essere associati ulteriori due file: package.json e package\_lock.json, i quali descrivono le dipendenze che la funzione considerata presenta. \\
Durante la dimostrazione è stato possibile constatare la mancanza di alcuni controlli nel modulo\ped{\textit{G}} \textit{Etherless-cli}, che il gruppo si impegna a terminare prima del prossimo incontro. \\ 
Complessivamente i Proponenti\ped{\textit{G}} sono stati soddisfatti nel constatare che oltre ad aver soddisfatto tutti i requisiti obbligatori il gruppo si sta impegnando nell'implementare anche i rimanenti requisiti opzionali e desiderabili individuati. 

\subsection{Prossima riunione}
Per permettere una dimostrazione completa ai Proponenti\ped{\textit{G}}, riguardante il prodotto completo, è stata fissata un'ulteriore riunione come segue: 
\begin{itemize}
	\item \textbf{Luogo}: chiamata tramite Zoom; 
	\item \textbf{Data}: 2020-07-15; 
	\item \textbf{Ora di inizio}: 14.00; 
	\item \textbf{Tipologia}: riunione esterna; 
\end{itemize}