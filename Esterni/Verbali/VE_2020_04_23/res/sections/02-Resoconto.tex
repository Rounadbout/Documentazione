\section{Verbale}
\subsection{Presentazione e discussione di alcune scelte architetturali}
I membri del gruppo hanno esposto una serie di scelte architetturali a \Proponente.
Di seguito il riassunto delle informazioni estrapolate dalla discussione:
\begin{description}
	\item [Storage delle informazioni delle funzioni caricate dagli sviluppatori:] il gruppo ha proposto di mantenere alcune informazioni basilari riguardanti le funzioni (come nome e prezzo) nello storage dello smart contract\ped{\textit{G}} di Etherless-smart, mentre il codice vero e proprio e altre informazioni riguardanti queste ultime in un'unico account AWS\ped{\textit{G}}.
	\item [Utilizzo del servizio AWS\ped{\textit{G}} Elastic Beanstalk\ped{\textit{G}}:] il gruppo ha proposto di utilizzare questo servizio di orchestrazione offerto da AWS\ped{\textit{G}}. Questa proposta é stata ben ricevuta dal Proponente\ped{\textit{G}}, a patto di valutare il peso dell'aggiunta di questo elemento all'infrastruttura di progetto;
	\item [Necessità di bloccare l'accesso alle funzionalità Etherless per utenti non registrati:] é stata discussa l'eventuale necessità di bloccare l'accesso alle funzionalità di Etherless -come l'esecuzione delle funzioni- a utenti non registrati all'applicativo. Ciò è risultato non necessario, a patto di avere un controllo sicuro sul pagamento: se un utente non registrato, ad esempio, richiede l'esecuzione di una funzione, questo gli viene permesso a patto che costui effetui preventivamente il pagamento per l'operazione richiesta;
	\item [OpenZeppelin\ped{\textit{G}} per gli smart contract\ped{\textit{G}} upgradable:] verranno utilizzati i tool offerti da OpenZeppelin\ped{\textit{G}} per agevolare l'implementazione di smart contract\ped{\textit{G}} upgradable;
	\item [Numero di smart contract\ped{\textit{G}} implementati:] il gruppo si è riservato di valutare se organizzare il modulo Etherless-smart in un singolo smart contract\ped{\textit{G}} o in multipli che interagiscono tra loro, in base al gas cost delle operazioni nei due casi;
	\item [Necessità di installazione e disinstallazione da comando singolo:] il proponente ha ribadito la necessità per gli utenti di installare e disinstallare l'applicativo con un singolo comando da CLI\ped{\textit{G}} e, di conseguenza, la necessità di implementare uno script che si occupi in automatico di queste operazioni.
\end{description}
