\section{Verbale}
\subsection{Discussione correzioni Analisi dei Requisiti}
Il gruppo ha presentato al docente i propri dubbi, legati ad alcuni punti da correggere del documento di \AdR{} \textit{v1.0.0}.
Nello specifico i dubbi trattati ed i casi d'uso\ped{\textit{G}} interessati sono i seguenti:
\subsubsection*{UC5 - Login manuale}
\begin{itemize}
	\item Separazione dell'inserimento delle credenziali di accesso in due casi d'uso\ped{\textit{G}} (non sottocasi), che trattino separatamente i due tipi di login;
	\item risultato del login sempre positivo, se il formato è corretto.
\end{itemize}
È stato quindi deciso di impostare l'inserimento della password come sottocaso del padre della gerarchia.
\subsubsection*{UC11 - Ricerca funzione per nome}
\begin{itemize}
	\item Separazione della ricerca di una funzione per nome dal risultato ottenuto da tale ricerca, in due casi d'uso\ped{\textit{G}} diversi.
\end{itemize}
\subsubsection*{Discussioni generali}
\begin{itemize}
	\item Il docente ha evidenziato la ridondanza e scorrettezza, presenti nella scelta di rappresentare i casi d'uso\ped{\textit{G}} nei loro stessi diagrammi.
	\item il docente ha evidenziato sulla necessità di focalizzarsi sulle pre/post condizioni di ogni possibile caso d'uso\ped{\textit{G}} per stabilire se è possibile accorpare questi ultimi sotto un unico caso. Una semplice similitudine tra due casi d'uso\ped{\textit{G}} non è sufficiente a giustificare un accorpamento.
\end{itemize}

\subsection{Accordo sulla data di presentazione Technology Baseline}
Il docente ha informato il gruppo della possibilità, e necessità, di prenotarsi per un incontro di presentazione della Technology Baseline\ped{\textit{G}} entro il 2020-05-08. Il gruppo ha poi deciso, sotto proposta del docente, per il giorno 2020-05-07 alle ore 13.10.
