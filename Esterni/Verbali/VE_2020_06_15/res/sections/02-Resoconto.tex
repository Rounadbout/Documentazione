\section{Verbale}
\subsection{Aggiornamento sullo stato di sviluppo del prodotto}
Per prima cosa i Proponenti sono stati aggiornati sullo stato di sviluppo del prodotto. Particolare attenzione è stata posta alla modalità con cui viene gestito il deployment di una funzione (funzionalità sviluppata nell'ultimo periodo). I Proponenti hanno poi concordato nell'utilizzare un prezzo fisso per l'esecuzione delle funzioni.
\\
I Proponenti sono stati molto soddisfatti nel constatare che tutti i requisiti obbligatori sono stati implementati in modo decisamente soddisfacente.

\subsection{Dimostrazione delle funzionalità implementate}
Durante la dimostrazione delle funzionalità, sono state presentate diverse caratteristiche del prodotto, tra le quali:
\begin{itemize}
	\item comunicazione tra i moduli \textit{Etherless-cli} e 	\textit{Etherless-server} unicamente attraverso \textit{Etherless-smart};
	\item eventuali problematiche di sicurezza individuate durante lo sviluppo;
	\item utilizzo del protocollo IPFS, in supporto alla procedura di deployment di una funzione;
	\item analisi del processo di parsing del file sorgente prima del deployment.
\end{itemize}
I comandi considerati durante tale dimostrazione sono stati quelli sviluppati nell'ultimo periodo, ossia:
\begin{itemize}
	\item login con mnemonic phrase (comando "login -m");
	\item visualizzazione dei dettagli di una funzione (comando "info");
	\item visualizzazione della lista di funzioni presenti all'interno del sistema (comando "list");
	\item deployment di una funzione (comando "deploy");
	\item eliminazione di una funzione (comando "delete").
\end{itemize}
A seguito della dimostrazione, i Proponenti hanno proposto di utilizzare una rete Ethereum locale ai fini della demo finale del prodotto, in modo da evitare problematiche riguardanti il tempo di esecuzione delle transazioni nella blockchain. \\
Infine il gruppo ha esposto gli obiettivi che si prefigge di soddisfare nel prossimo periodo: tra i quali:
\begin{itemize}
	\item rafforzamento della sicurezza del prodotto;
	\item miglioramento della user experience.
\end{itemize}
