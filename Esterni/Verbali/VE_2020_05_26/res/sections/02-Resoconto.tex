\section{Verbale}
\subsection{Chiarimento di alcuni dubbi riguardo l'esito della revisione di progettazione}
Il gruppo ha esposto dei dubbi legati ad alcune indicazioni ricevute nell'esito della revisione di progettazione, per averne un chiarimento da parte del Committente\ped{\textit{G}}. \\
Nello specifico le domande poste riguardano:

	\subsubsection*{Piano di Progetto}
	Abbiamo richiesto un chiarimento riguardo le segnalazioni sull'utilizzo non corretto del modello di sviluppo incrementale, proponendo una possibile soluzione rispetto alla nostra interpretazione; abbiamo proposto di orientare i vari incrementi alle funzionalità di prodotto software sviluppate e non allo sviluppo dei singoli moduli all'interno di ciascun periodo, come abbiamo fatto fino ad ora. \\
	Il \TV{} come prima cosa ci ha fatto ragionare sulla definizione di sviluppo incrementale, sui vantaggi che questo modello porta e su tutte le problematiche che potrebbero insorgere se utilizzato in modo non adeguato; in questo modo ci ha fatto comprendere le motivazioni che ci hanno portato ad un utilizzo non corretto di questo modello di sviluppo. \\
	Lo sviluppo incrementale non deve essere nascosto all'interno di un periodo ma dovrebbe essere quindi organizzato per tappe incrementali, cercando di mitigare i rischi che potrebbero insorgere (inesperienza, etc..); inoltre queste tappe dovrebbero essere indipendenti dalle scadenze poste per le varie revisioni e non dovrebbero essere troppo distanti l'una dall'altra, soprattutto per chi ha poca esperienza sull'utilizzo di questo modello. \\
	Abbiamo capito che la pianificazione definita inizialmente nella sezione apposita del \PdP{} non va modificata; proprio per questo abbiamo compreso la necessità che nel consuntivo di periodo venga analizzato il punto di maturazione del prodotto e non solamente i discostamenti delle ore e del costo effettivo rispetto a quanto preventivato. \\
	Inoltre, rispetto a quanto riportato nel consuntivo di periodo, le eventuali modifiche da apportare alla pianificazione dei periodi rimanenti andrebbero riportate nel preventivo a finire. \\
	In questo modo è possibile analizzare i periodi passati per raffinare la pianificazione dei periodi successivi ed essere più efficienti nello svolgere i vari incrementi. \\
	Intendiamo quindi definire degli incrementi focalizzati sulle funzionalità del prodotto software e andare ad effettuare la stesura del consuntivo di periodo e del preventivo a finire in maniera più adatta rispetto al modello di sviluppo incrementale.
	 
	
	\subsubsection*{Piano di Qualifica}
	Abbiamo richiesto un chiarimento riguardo la segnalazione sull'utilizzo errato del titolo "Test di verifica" nel \PdQ ; ci siamo resi conto che questo titolo è impreciso e abbiamo chiesto un parere riguardo una possibile soluzione a cui abbiamo pensato, ovvero creare una sezione "Test" che conterrà la specifica e lo stato attuale per ciascun test che indendiamo definire. \\
	Il \TV{} anche qui ci ha fatto ragionare su che cos'è il \PdQ, in poche parole possiamo intenderlo come il nostro passaporto competitivo. \\
	Abbiamo capito che la principale differenza tra concorrenti in una fornitura non è identificata dall'\AdR{} ma bensì dalla qualità del prodotto e dal modo di lavorare, possiamo quindi dire che ciò che ci rende più bravi degli altri è la qualità che noi portiamo in più dei nostri concorrenti. \\
	Abbiamo ragionato anche sul fatto che la qualità di prodotto la si misura su quanto ho testato e verificato. Le varie tipologie di test ed il numero di test effettuati ci permettono di misurare la qualità del codice che abbiamo scritto e per questo motivo è importante che nel \PdQ{} ci sia una sezione apposita sui test. \\
	È inoltre importante che nella parte del \PdQ{} che decrive i test siano presenti obiettivi metrici chiari e non solo un elenco di test, in modo da poter definire degli indicatori all'interno di un cruscotto che ci possano informare sempre su come stiamo procedendo. \\
	Intendiamo quindi creare una sezione apposita per i test nel \PdQ{} e andare a definire degli obiettivi metrici chiari per poter comporre un cruscotto come è stato descritto in precedenza.
	
	\subsubsection*{Changelog}
	Abbiamo richiesto un chiarimento riguardo la segnalazione ricevuta sugli scatti di versione effettuati a seguito di azioni di modifica, prima della loro verifica di validità. \\
	Ci siamo resi conto di questo errore dopo la segnalazione in sede di revisione dei requisiti e avevamo perciò deciso di verificare le modifiche prima di eseguire lo scatto di versione; in sede di revisione di progettazione però ci è stata fatta la stessa critica. \\
	Abbiamo quindi chiesto un parere riguardo l'idea di aggiungere le informazioni sulla verifica delle modifiche apportate al documento (verificatore e data della verifica) a ciascuna voce presente nel registro delle modifiche, in modo da rendere chiaro che lo scatto di versione avviene a seguito di una verifica della modifica. \\
	Il \TV{} ci ha fatto ragionare sulla motivazione per cui uno scatto di versione prima della verifica di validità non è corretto rispetto al modello di sviluppo che abbiamo deciso di adottare. \\
	Abbiamo capito che il registro delle modifiche serve per tenere traccia delle modifiche verificate ed accettate, quindi valide per essere aggiunte all'interno, in questo caso, del documento. \\
	Facendo in questo modo abbiamo un invariante molto importante, ovvero la certezza che il documento contenga solo cose corrette e che non contenga degli errori. \\
	Per rendere questo meccanismo più efficiente sarebbe desiderabile automatizzare il più possibile il processo di verifica, definendo come si verificherà una modifica nello stesso momento in cui viene assegnato ad un componente il compito di apportare quella modifica. \\
	In questo modo è possibile che il processo di verifica possa iniziare (utilizzando qualche strumento automatico quando possibile) immediatamente dopo che una modifica è stata apportata. \\
	Abbiamo quindi compreso l'importanza di definire come effettuare la verifica nello stesso momento in cui andiamo ad assegnare la verifica a qualcuno. \\
	Intendiamo infine andare ad incrementare la versione di ciascun documento solamente a seguito di una pull request completata, ovvero quando tutte le modifiche apportate sono state verificate e non sono stati riscontrati errori, e aggiungere nel registro delle modifiche il nome del verificatore e la data nella quale è stata effettuata la verifica.

	
	
	