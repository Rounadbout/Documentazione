\subsection{Requisiti di qualità}

\def\arraystretch{1.75}
\rowcolors{2}{lightRowColor}{darkRowColor}
\begin{longtable}{ 
		>{\centering}p{0.10\textwidth} 
		>{}p{0.35\textwidth} 
		>{\centering}p{0.25\textwidth}
		>{\centering}p{0.15\textwidth} }
	
	\caption{Tabella dei requisiti di qualità} \\ 
	\coloredTableHead
	\textbf{\color{white}Requisito} & 
	\centering\textbf{\color{white}Descrizione} & 
	\centering\textbf{\color{white}Classificazione} &
	\textbf{\color{white}Fonti} 
	\endfirsthead
	 
 	 \rowcolor{white}\caption[]{(continua)}\\
	 \coloredTableHead 
	 \textbf{\color{white}Requisito} &
	 \centering\textbf{\color{white}Descrizione} &
	 \centering\textbf{\color{white}Classificazione} &
	 \textbf{\color{white}Fonti} 
	 \endhead
	
	% contenuto tabella 
	R1Q1 &  La progettazione e la codifica devono rispettare le norme e 
			le metriche definite nei documenti 
			\textit{Norme di Progetto 2.0.0} 
			e \textit{Piano di Qualifica 1.0.0}. 							& \ob & Interno \tabularnewline
	R1Q2 & Il sistema deve essere pubblicato con licenza MIT\ped{\textit{G}}. 				& \ob & Capitolato \tabularnewline
	R1Q3 & Il codice sorgente di \textit{Etherless} deve essere pubblicato
			e versionato usando 
			GitHub\ped{\textit{G}} o GitLab\ped{\textit{G}}.					& \ob & Capitolato \tabularnewline
	R1Q4 & Deve essere redatto un manuale sviluppatore. 						& \ob & Capitolato \tabularnewline
	R1Q4.1 & Il manuale sviluppatore deve contenere le informazioni per
				eseguire e fare il 
				deploy\ped{\textit{G}} dei moduli\ped{\textit{G}}.			& \ob & Capitolato \tabularnewline
	R1Q5 & Deve essere redatto un manuale utente. 							& \ob & Capitolato \tabularnewline
	R1Q5.1 & Il manuale utente deve contenere tutte le informazioni
				necessarie all'utente finale per utilizzare correttamente 
				il sistema. 													& \ob & Capitolato \tabularnewline
	R1Q6 & La documentazione per l'utilizzo del software deve essere 
		 	scritta in lingua inglese.										& \ob & Verbale 2020-03-18, VE\_1.2  \tabularnewline
	R1Q7 & Nella scrittura del codice Javascript\ped{\textit{G}} deve essere seguita 
			la guida sullo stile di programmazione Airbnb\ped{\textit{G}} Javascript\ped{\textit{G}}
			style guide. 													& \ob & Verbale 2020-03-27, VE\_2.2 \tabularnewline
	R1Q8 & Lo sviluppo del codice Javascript\ped{\textit{G}} deve essere supportato 
			dal software di analisi statica del codice 
			ESLint\ped{\textit{G}}.											& \ob & Capitolato \tabularnewline
	R1Q9 & Deve essere utilizzato il meccanismo delle promise/async-await\ped{\textit{G}} 
			come approccio principale. 										& \ob & Capitolato \tabularnewline
	R1Q10 & Il progetto deve utilizzare i seguenti ambienti di sviluppo: 
			ambiente di sviluppo locale, ambiente di testing e ambiente 
			di staging\ped{\textit{G}}. 														& \ob & Capitolato \tabularnewline
	R1Q10.1 & Gli ambienti per la fase di sviluppo locale e testing possono 
	fare utilizzo della rete TestRPC\ped{\textit{G}} fornita dal framework\ped{\textit{G}} Truffle\ped{\textit{G}}.  & \de & Capitolato \tabularnewline
	R1Q10.2 & Per la fase di staging\ped{\textit{G}} è desiderabile l'utilizzo della rete 
	Ethereum\ped{\textit{G}} Ropsten\ped{\textit{G}}.				& \de & Capitolato \tabularnewline
	R1Q10.3 & Durante la fase di staging\ped{\textit{G}} l'applicativo deve essere 
	pubblicamente accessibile. 										& \ob & Capitolato \tabularnewline
	R1Q10.4 & Al termine del progetto il prodotto deve essere pronto 
	per la produzione. 												& \ob & Capitolato \tabularnewline
	R1Q10.4.1 & L'ambiente di produzione deve fare utilizzo dell'Ethereum\ped{\textit{G}}
	main network. 													& \op & Capitolato \tabularnewline
	
\end{longtable}

