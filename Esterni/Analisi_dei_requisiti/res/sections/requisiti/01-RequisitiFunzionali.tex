
\subsection{Requisiti funzionali}

\def\arraystretch{1.75}
\rowcolors{2}{lightRowColor}{darkRowColor}
\begin{longtable}{ 
		>{\centering}p{0.10\textwidth} 
		>{}p{0.35\textwidth} 
		>{\centering}p{0.25\textwidth}
		>{\centering}p{0.15\textwidth} }
	
	\caption{Tabella dei requisiti funzionali} \\
	\coloredTableHead
	\textbf{\color{white}Requisito} & 
	\centering\textbf{\color{white}Descrizione} & 
	\centering\textbf{\color{white}Classificazione} &
	\textbf{\color{white}Fonti} 
	\endfirsthead
	
	\rowcolor{white}\caption[]{(continua)}\\
	\coloredTableHead 
	\textbf{\color{white}Requisito} &
	\centering\textbf{\color{white}Descrizione} &
	\centering\textbf{\color{white}Classificazione} &
	\textbf{\color{white}Fonti} 
	\endhead

	% init
	R2F1 & L'utente può leggere una breve guida iniziale riguardante l'applicativo
			e i comandi per effettuare l'accesso. 									& \de & Interno \\ UC1 \tabularnewline

	% help
	R2F2 & L'utente può richiedere di visualizzare una descrizione più approfondita
		 per ogni comando messo a disposizione da \textit{Etherless-cli}\ped{\textit{G}}.			& \de & Interno \\ UC2 \tabularnewline
	R2F2.1 & Per ottenere informazioni specifiche su un comando, l'utente deve
		inserire il nome del comando di interesse seguito dal flag \texttt{-{}-help}.	& \de & Interno \\ UC2 \tabularnewline

	% signup
	R1F3 & Un utente non registrato può richiedere la creazione di un nuovo account
			 all'interno della rete Ethereum\ped{\textit{G}}.						& \ob & Capitolato \\ UC3 \tabularnewline
	R1F3.1 & Una volta creato il nuovo account, il sistema deve mostrare nella 
			CLI\ped{\textit{G}} le credenziali a esso relative.										& \ob & Interno \\ UC3.1 \tabularnewline
	R1F3.1.1 & A seguito del completamento della procedura di registrazione viene
			mostrato l'address associato al nuovo account creato. 					& \ob & Interno \\ UC3.1 \tabularnewline
	R1F3.1.2 & A seguito del completamento della procedura di registrazione viene
			mostrata la private key\ped{\textit{G}} associata al nuovo account creato. 				& \ob & Interno \\ UC3.1 \tabularnewline
	R2F3.1.3 & A seguito del completamento della procedura di registrazione viene
			mostrata la mnemonic phrase\ped{\textit{G}} associata al nuovo account creato. 			& \de & Interno \\ UC3.1 \tabularnewline
	R2F3.2 & L'utente può richiedere il salvataggio su file delle credenziali 
			dell'account creato durante la procedura di registrazione.				& \de & Interno \\ UC3.2 \tabularnewline

	%login
	R1F4 & Un utente può effettuare il login. 										& \ob & Capitolato \\ UC4 \tabularnewline
	R1F4.1 & Per effettuare la procedura di login è richiesto che l'utente inserisca una 
		password, con cui verrà cifrato il proprio wallet\ped{\textit{G}}. 							& \ob & Interno \\ UC4.1 \tabularnewline
	R1F4.2 & Per completare la proceduta di login manuale l'utente deve inserire
			 la propria private key\ped{\textit{G}}. 								& \ob & Interno \\ UC5 \tabularnewline
	R1F4.2.1 & Nel caso in cui l'utente tenti di autenticarsi con una private key in formato errato 
		deve essere mostrato un messaggio di errore. 								& \ob & Interno \\ UC5.2 \tabularnewline
	R2F4.3 & L'utente può decidere di completare la procedura di login manuale 
			 utilizzando la propria mnemonic phrase\ped{\textit{G}} al posto della private key.		& \de & Interno \\ UC6 \tabularnewline
 	R2F4.3.1 & Nel caso in cui l'utente tenti di autenticarsi con una mnemonic phrase in formato errato 
 	deve essere mostrato un messaggio di errore. 								& \ob & Interno \\ UC6.2 \tabularnewline
	
	% logout
	R1F5 & L'utente può effettuare il logout. 										& \ob & Capitolato \\ UC7 \tabularnewline 
	
	% whoami
	R2F6 & L'utente può richiedere di visualizzare l'address 
			associato alla sessione corrente. 										& \de & Interno \\ UC8 \tabularnewline
	
	% info
	R1F7 & L'utente può richiedere di visualizzare le informazioni dettagliate di una funzione
		tramite il comando \info{}.													& \ob & Interno \\ UC9 \tabularnewline
	R1F7.1 & Per visualizzare la descrizione di una funzione l'utente deve inserire 
		il nome della funzione di interesse.											& \ob & Interno \\ UC9.1 \tabularnewline
	R1F7.2 & Nel caso in cui l'utente richieda di visualizzare la descrizione di una 
		funzione non presente nel sistema, deve essere mostrato un messaggio di
		errore.															 			& \ob & Interno \\ UC25 \tabularnewline
	
	% search 
	R2F8 & Il sistema deve permettere all'utente di cercare una funzione 
		attraverso una keyword. 														& \de & Interno \\ UC10 \tabularnewline
	R2F8.1 & Per effettuare la ricerca è necessario che l'utente inserisca 
		una keyword. 																& \de & Interno \\ UC10.1 \tabularnewline
	R2F8.2 & A seguito di una ricerca il sistema deve mostrare la lista di
	 tutte le funzioni che presentano la keyword indicata 
	 all'interno del proprio nome.													& \de & Interno \\ UC11 \tabularnewline
	R2F8.2.1 & La visualizzazione di un risultato di ricerca include
		 la firma della funzione.													& \de & Interno \\ UC11.1 \tabularnewline
  	R2F8.2.2 & La visualizzazione di un risultato di ricerca include
		  il costo di esecuzione della funzione.										& \de & Interno \\ UC11.1 \tabularnewline
  	R2F8.3 & Se una ricerca non porta a nessun risultato deve essere mostrato un 
		messaggio di errore. 														& \de & Interno \\ UC12 	\tabularnewline	
	
	% run 
	R1F9 & L'utente deve essere in grado di eseguire le funzioni messe a 
		disposizione da \textit{Etherless} attraverso il comando \run{}.				& \ob & Capitolato \\ UC13 \tabularnewline
	R1F9.1 & Per eseguire una funzione è necessario inserire il relativo nome. 		& \ob & Capitolato \\ UC13.1 \tabularnewline
	R1F9.1.1 & Nel caso in cui il nome inserito a seguito del comando \run{} non 
		corrisponda ad alcuna funzione presente nel sistema, deve essere 
		visualizzato un messaggio di errore.										& \ob & Interno \\ UC13.4 \tabularnewline 
	R1F9.2 & L'esecuzione di una funzione necessita dell'inserimento dei parametri 
		necessari per la sua esecuzione.												& \ob & Capitolato \\ UC13.2 \tabularnewline
	R1F9.2.1 & Se l'utente tenta di eseguire una funzione inserendo un numero 
		di parametri che non coincide con quanto richiesto, deve essere 
		visualizzato un messaggio di errore. 										& \ob & Interno \\ UC13.5 \tabularnewline
	R1F9.2.2 & Se l'utente tenta di eseguire una funzione inserendo almeno un parametro 
		con tipo differente da quanto indicato nella firma della funzione, deve essere 
		visualizzato un messaggio di errore. 										& \ob & Interno \\ UC13.6 \tabularnewline
	R1F9.3 & A seguito dell'esecuzione di una funzione il sistema deve mostrare 
		all'utente i relativi risultati. 											& \ob & Capitolato \\ UC13.3 \tabularnewline
	R1F9.4 & Nel caso in cui l'utente richieda di eseguire una funzione senza 
		avere credito sufficiente, deve essere mostrato un messaggio di errore.		& \ob & Interno \\ UC19 \tabularnewline
	
	% list  --
	R1F10 & L'utente deve essere in grado di visualizzare tutte le funzioni 
		disponibili in \textit{Etherless} tramite il comando \lista{}. 				& \ob & Capitolato \\ UC20 \tabularnewline
	R2F10.1 & L'utente può richiede di visualizzare solo le funzioni da 
		lui caricate tramite l'utilizzo di un apposito flag.						& \de & Interno \\ UC21 \tabularnewline
	R1F10.2 & La visualizzazione di un elemento della lista ottenuta a seguito 
		del comando \lista{} include la firma della funzione. 						& \ob & Interno \\ UC20.1 \\ UC21.1 \tabularnewline
	R1F10.3 & La visualizzazione di un elemento della lista ottenuta a seguito 
		del comando \lista{} include il costo di esecuzione della funzione. 			& \ob & Interno \\ UC20.1 \\ UC21.1 \tabularnewline
	R1F10.5 & Se il comando list non porta ad alcun tipo di risultato, viene mostrato 
		un apposito errore. & \ob & Interno \\ UC22 \\ UC23 \tabularnewline

	%deploy
	R1F11 & L'utente deve essere in grado di eseguire il deploy\ped{\textit{G}} di una propria
		funzione all'interno della piattaforma \textit{Etherless}. 					& \ob & Capitolato \\ UC14 \tabularnewline
	R1F11.1 & Per eseguire il deploy\ped{\textit{G}} l'utente deve inserire il percorso del file 
		contenente il codice della funzione. 										& \ob & Capitolato \\ UC14.1 \tabularnewline
	R2F11.1.1 & Se il formato del file indicato durante la procedura di 
		deploy\ped{\textit{G}} non è supportato dall'applicativo deve essere
		mostrato un messaggio di errore.												& \de & Interno \\ UC14.5 \tabularnewline
	R1F11.1.2 & Se il file indicato durante la procedura di deploy\ped{\textit{G}}
	 	non esiste, deve essere visualizzato un messaggio di errore.					& \ob & Interno \\ UC14.4 \tabularnewline
	R1F11.2 & Per eseguire il deploy\ped{\textit{G}} l'utente deve inserire il nome della 
		funzione considerata. 														& \ob & Capitolato \\ UC14.2 \tabularnewline
	R1F11.2.1 & Nel caso in cui il nome della funzione di cui si tenta di fare 
		il deploy\ped{\textit{G}} sia troppo lungo, deve essere visualizzato 
		un messaggio di errore. 														& \ob & Interno \\ UC14.7 \tabularnewline
	R1F11.2.2 & Nel caso in cui il nome della funzione di cui si tenta di fare 
		il deploy\ped{\textit{G}} sia già usato nel sistema, deve essere visualizzato un messaggio 
		di errore.																	& \ob & Interno \\ UC14.6 \tabularnewline
	R2F11.3 & Per eseguire il deploy\ped{\textit{G}} l'utente deve inserire una descrizione 
		della funzione. 																& \de & Interno \\ UC14.3  \tabularnewline
	R2F11.3.1 & Se la descrizione inserita durante la procedura di deploy\ped{\textit{G}} supera la 
		lunghezza massima, deve essere mostrato un messaggio di errore. 				& \de & Interno \\ UC14.8  \tabularnewline
	R1F11.4 & Nel caso in cui l'utente tenti di eseguire il deploy\ped{\textit{G}} di una funzione
		senza avere il credito necessario, deve essere visualizzato un messaggio 
		di errore. 																	& \ob & Interno \\ UC19 \tabularnewline

	% modify 
	R2F12 & L'utente deve essere in grado di modificare le informazioni relative 
		ad una funzione da lui caricata. 											& \ob & Interno \\ UC15 \tabularnewline
	R2F12.1 & Per eseguire la procedura di modifica è necessario che l'utente 
		indichi il nome della funzione che vuole modificare. 						& \ob & Interno \\ UC15.1.1 \tabularnewline
	R2F12.1.1 & Nel caso in cui, durante la procedura di modifica, l'utente 
		inserisca il nome di una funzione non presente all'interno della piattaforma
		\textit{Etherless}, deve essere mostrato un messaggio di errore.				& \ob & Interno \\ UC15.1.2 \tabularnewline
	R2F12.1.2 & Nel caso in cui, durante la procedura di modifica, l'utente 
		inserisca il nome di una funzione che non è di sua proprietà, deve essere 
		mostrato un messaggio di errore.												& \ob & Interno \\ UC15.1.3 \tabularnewline
	R2F12.2 & Il sistema deve permettere all'utente di modificare la descrizione 
		associata ad una propria funzione. 											& \ob & Interno \\ UC15.2 \tabularnewline
	R2F12.2.1 & L'utente deve visualizzare un errore nel caso in cui, durante 
		la procedura di modifica, venga inserita una descrizione di lunghezza
		superiore a quella massima consentita. 										& \ob & Interno \\ UC15.4 \tabularnewline
	R2F12.3 & Il sistema deve permettere all'utente di aggiornare il codice di 
		una propria funzione. 														& \ob & Interno \\ UC15.3 \tabularnewline	
	R2F12.3.1 & Se il file indicato durante la procedura di aggiornamento del 
		codice di una funzione non esiste, deve essere mostrato un messaggio di 
		errore.																		& \ob & Interno \\ UC15.5 \tabularnewline
	R2F12.3.2 & Se il file indicato durante la procedura di aggiornamento del 
		codice di una funzione presenta un formato non supportato, deve 
		essere mostrato un messaggio di errore.										& \ob & Interno \\ UC15.6 \tabularnewline  
		
	% history
	R2F13 & L'utente deve essere in grado di visualizzare la propria cronologia 
		di richieste di esecuzione. 													& \de & Interno \\ UC16 \tabularnewline
	R2F13.1 & L'utente deve poter essere in grado di richiedere di visualizzare
		solo una porzione della propria cronologia di esecuzione. 					& \de & Interno \\ UC16.1 \tabularnewline
	R2F13.2 & La visualizzazione di un elemento della cronologia include 
		l'identificativo della richiesta di esecuzione. 								& \de & Interno \\ UC16.2 \tabularnewline
	R2F13.3 & La visualizzazione di un elemento della cronologia include 
		il nome della funzione richiesta. 											& \de & Interno \\ UC16.2 \tabularnewline				
	R2F13.4 & La visualizzazione di un elemento della cronologia include 
		il valore dei parametri indicati nella chiamata alla funzione.				& \de & Interno \\ UC16.2 \tabularnewline
	R2F13.5 & La visualizzazione di un elemento della cronologia include 
		il risultato della richiesta di esecuzione.									& \de & Interno \\ UC16.2 \tabularnewline
	R2F13.6 & La visualizzazione di un elemento della cronologia include 
		la data e l'orario della richiesta. 											& \de & Interno \\ UC16.2 \tabularnewline
	
	% delete
	R1F14 & L'utente deve essere in grado di eliminare una funzione da lui caricata. & \ob & Capitolato \\ UC18 \tabularnewline
	R1F14.1 & Per eseguire l'operazione di eliminazione l'utente deve inserire 
		il nome della funzione da eliminare. 										& \ob & Capitolato \\ UC18.1 \tabularnewline
	R1F14.1.1 & Nel caso in cui il nome inserito durante la procedura di eliminazione
		non si riferisca ad alcuna funzione presente all'interno del sistema, deve 
		essere mostrato un messaggio di errore.										& \ob & Interno \\ UC18.2 \tabularnewline
	R1F14.1.2 & Nel caso in cui la funzione considerata nella procedura di eliminazione
		non sia di proprietà dell'utente, deve essere visualizzato un messaggio 
		di errore.																	& \ob & Interno \\ UC18.3 \tabularnewline
	
	% altro 
	R1F15 & Gli smart contract\ped{\textit{G}} devono poter essere aggiornati. 				& \ob & Capitolato \tabularnewline
\end{longtable}