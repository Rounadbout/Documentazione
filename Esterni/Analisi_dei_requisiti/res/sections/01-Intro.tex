\section{Introduzione}
\subsection{Scopo del documento}
Il presente documento ha l'obiettivo di individuare le funzionalità e i casi d'uso previsti dal progetto Etherless, proposto dall'azienda Red Babel. Le informazioni qui riportate, individuate da un'approfondita analisi del capitolato stesso e dai successivi incontri con il proponente, rappresentano la base di partenza per la successiva fase di progettazione.
\subsection{Scopo del prodotto}
Si svuole sviluppare una piattaforma cloud che consenta agli sviluppatori di fare il deploy di funzioni Javascript e gestisca il pagamento per la loro esecuzione tramite la piattaforma Ethereum.\\
Il prodotto finale prevede quindi l'integrazione di due tecnologie, Serverless ed Ethereum.\\
Il lato Serverless si occupa dell'esecuzione delle funzioni fornite dagli sviluppatori. Tali funzioni vengono salvate ed eseguite in un servizio cloud esterno, quale Amazon Web Services.  \\La richiesta di utilizzo di una funzione e il successivo pagamento vengono invece gestiti tramite la piattaforma Ethereum sfruttando gli smart contracts. Il pagamento viene effettuato in ETH. Una percetuale significativa di ogni pagamento viene riservata agli amministratori del servizio. 
Lo sviluppatore e l'utente finale interagiscono con il prodotto tramite una CLI che prevede alcuni comandi intuitivi.
\subsection{Documenti complementari}
\subsubsection{Glossario}
	I termini tecnici utilizzati in questo documento sono contrassegnati da una ’G’ a pedice ed ulteriormente approfonditi nel Glossario denominato "Glossario Esterno 1.0.0", disponibile in allegato al presente documento.
\subsection{Riferimenti}
\subsubsection{Normativi}
\begin{itemize}
	\item \textbf{Norme di progetto}: \textit{Norme di progetto v1.0.0};
	\item \textbf{Verbale esterno con il proponente}:\textit{ VE\_2020\_03\_18};
	\item \textbf{Capitolato d'appalto Etherless}:\\\url{https://www.math.unipd.it/~tullio/IS-1/2019/Progetto/C2.pdf}.
\end{itemize}
\subsubsection{Informativi}
\begin{itemize}
	\item \textbf{Studio di fattibilità}: \textit{Studio di fattibilità v1.0.0};
	\item \textbf{Sito ufficiale Ethereum}: \url{https://ethereum.org/};
	\item \textbf{Sito del framework Serverless}: \url{https://serverless.com/};
	\item \textbf{Amazon Web Services}: \url{https://aws.amazon.com/}.
\end{itemize}