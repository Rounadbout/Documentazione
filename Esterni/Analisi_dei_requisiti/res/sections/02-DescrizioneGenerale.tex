\section{Descrizione generale}

\subsection{Obiettivi del progetto}
Il progetto\ped{\textit{G}} \NomeProgetto{} ha come obiettivo finale creare una piattaforma che consenta agli sviluppatori di fare il deploy\ped{\textit{G}} di funzioni JavaScript\ped{\textit{G}}, preoccupandosi solamente della relativa codifica e non dell'architettura sottostante. Allo stesso tempo, tali funzioni vengono messe a disposizione agli altri utenti, che possono eseguirle e pagare secondo un modello CaaS\ped{\textit{G}} (Computation as a Service), cioè solamente per il tempo e le risorse richieste dalla loro esecuzione.

\subsection{Funzionalità del prodotto}
L'applicativo deve fornire la possibilità agli sviluppatori di caricare nel cloud\ped{\textit{G}} le proprie funzioni e renderle disponibili secondo la modalità FaaS\ped{\textit{G}} (Function as a Service). Gli utenti finali possono usufruire di tali servizi pagando una certa quantità di ETH\ped{\textit{G}}, e gli sviluppatori non si devono quindi preoccupare della gestione dell'infrastruttura alla base dei servizi, ma possono guadagnare all'esecuzione di ogni funzione da loro caricata. \\
Nello specifico: 
	\begin{itemize}
		\item gli utenti finali e gli sviluppatori possono: 
		\begin{enumerate}[a.]
			\item autenticarsi all'interno della rete Ethereum\ped{\textit{G}}; 
			\item eseguire una funzione presente della piattaforma e visualizzarne i risultati; 
			\item elencare tutte le funzioni disponibili nella piattaforma; 
			\item visualizzare i dettagli di una determinata funzione; 
			\item visualizzare la propria cronologia di esecuzione di funzioni; 
			\item ricercare funzioni in base ad un termine di ricerca. 
		%	\item visualizzare i dettagli relativi all'esecuzione di una specifica funzione; 
		\end{enumerate}
		
		\item gli sviluppatori possono: 
		\begin{enumerate}[a.]
			\item caricare all'interno della piattaforma delle proprie funzioni Javascript\ped{\textit{G}}; 
			\item eliminare una funzione da loro precedentemente caricata; 
			\item modificare le informazioni e il codice relativo ad una loro funzione.
		\end{enumerate}
	\end{itemize}