\subsubsection{UC15.3 - Inserimento percorso file sorgente}
\begin{itemize}
	\item \textbf{Attori primari:} \us{};
	\item \textbf{Descrizione:} l'utente inserisce il comando \pedit{} \texttt{–c file\_path} indicando la volontà di voler modificare il codice associato alla funzione tramite il flag \texttt{-c}, e inserendo successivamente il percorso del file sorgente nel campo \texttt{file\_path}; 
	\item \textbf{Scenario principale:} l'utente inserisce il percorso del file contente il codice aggiornato dalla funzione; 
	\item \textbf{Estensioni:} 
	\begin{itemize}
		\item \textbf{UC15.5:} se l'utente inserisce il percorso di un file non presente viene visualizzato un apposito messaggio di errore;
		\item \textbf{UC15.6:} l’utente inserisce un percorso di file il cui formato non è supportato dal sistema. Viene di conseguenza visualizzato un messaggio di errore.
	\end{itemize}
	\item \textbf{Precondizione:} l’utente ha inserito all’interno della CLI\ped{\textit{G}} il comando \edit{}; 
	\item \textbf{Postcondizione:} l’utente ha inserito il percorso del file contenente il codice aggiornato della funzione.
\end{itemize}