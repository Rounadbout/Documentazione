\subsubsection{UC13.1 - Inserimento nome funzione}
\begin{itemize}
	\item \textbf{Attori primari:} \ua{};
	\item \textbf{Descrizione:} l’utente inserisce il nome della funzione che desidera eseguire nel campo \texttt{function\_name};
	\item \textbf{Scenario principale:} l'utente inserisce il nome della funzione; 
	\item \textbf{Estensioni:} 
	\begin{itemize}
		\item \textbf{UC13.4:} l’utente inserisce un nome non presente nel sistema, di conseguenza viene visualizzato un messaggio di errore.
	\end{itemize}
	\item \textbf{Precondizione:} l’utente ha digitato all’interno della CLI\ped{\textit{G}} il comando \run{};
	\item \textbf{Postcondizione:}  il campo \texttt{function\_name} contiene il nome della funzione.
\end{itemize}