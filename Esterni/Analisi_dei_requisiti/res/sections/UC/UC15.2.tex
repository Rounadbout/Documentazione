\subsubsection{UC15.2 - Inserimento nuova descrizione}
\begin{itemize}
	\item \textbf{Attori primari:} \us{};
	\item \textbf{Descrizione:} l'utente inserisce il comando \pedit{} \texttt{–d new\_desc} indicando la volontà di voler modificare la descrizione associata alla funzione tramite il flag \texttt{-d}, e inserendo successivamente la nuova descrizione nel campo new\_desc;
	Scenario principale: l’utente inserisce la nuova descrizione;   
	\item \textbf{Scenario principale:} l’utente inserisce la nuova descrizione;  
	\item \textbf{Estensioni:} 
	\begin{itemize}
		\item \textbf{UC15.4:} se l’utente inserisce una nuova descrizione che supera la lunghezza massima consentita, viene visualizzato un apposito messaggio di errore. 
	\end{itemize}
	\item \textbf{Precondizione:} l’utente ha inserito all’interno della CLI\ped{\textit{G}} il comando \edit{};
	\item \textbf{Postcondizione:} l’utente ha inserito correttamente la nuova descrizione.
\end{itemize}