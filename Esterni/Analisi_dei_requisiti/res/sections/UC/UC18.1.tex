\subsubsection{UC18.1 - Inserimento nome funzione}
\begin{itemize}
	\item \textbf{Attori primari:} \us{};
	\item \textbf{Descrizione:} al fine di eseguire la procedura di rimozione di una funzione è richiesto l’inserimento del relativo nome; 
	\item \textbf{Scenario principale:} dopo aver deciso di eliminare una funzione l’utente inserisce nella CLI\ped{\textit{G}} il relativo nome; 
	\item \textbf{Estensioni:} 
	\begin{itemize}
		\item \textbf{UC18.2:} se l’utente inserisce un nome che non si riferisce ad alcuna funzione della piattaforma \textit{Etherless} viene mostrato un errore adeguato;  
		\item \textbf{UC18.3:} se viene inserito un nome relativo ad una funzione non appartenente all’utente considerato, viene mostrato un relativo messaggio di errore. 
	\end{itemize}
	\item \textbf{Precondizione:} l’utente vuole rimuovere una determinata funzione dal sistema e ha già inserito il comando \delete{} nella CLI\ped{\textit{G}};  
	\item \textbf{Postcondizione:} l’utente ha inserito correttamente il nome della funzione che vuole rimuovere.  
\end{itemize}