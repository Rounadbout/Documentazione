\section{Test di Verifica}

\subsection{Test di Unità}
Il test di unità ha come obiettivo quello di determinare la correttezza e completezza, rispetto ai requisiti, di un programma visto come singolo modulo.\\
Questa tipologia di test verrà sviluppata in vista delle prossime revisioni.


\subsection{Test di Integrazione}
Il test di integrazione ha come obiettivo quello di verificare la correttezza funzionale nell’iterazione tra più moduli. In particolare:
\begin{enumerate}
	\item verifica sull'assemblamento dei vari moduli aggiunti incrementalmente;
	\item verifica sull'assemblamento di tutti i moduli allo stesso tempo.
\end{enumerate}
Questa tipologia di test verrà sviluppata in vista delle prossime revisioni.


\subsection{Test di Sistema}
Il test di sistema ha come obiettivo quello di testare particolari proprietà globali di esso. In particolare:
\begin{enumerate}
	\item \textbf{test di stress}: valutazione del sistema in condizioni di sovraccarico;
	\item \textbf{test di robustezza}: valutazione del sistema quando sono presenti dati non corretti;
	\item \textbf{test di sicurezza}: valutazione del sistema nella sua sicurezza.
\end{enumerate}
Questa tipologia di test verrà sviluppata in vista delle prossime revisioni.


\subsection{Test di Regressione}
Il test di regressione ha come obiettivo quello di verificare che ad ogni aggiornamento di un modulo software, la nuova versione mantenga le funzionalità di quella precedente.
Il particolare si applica attraverso l'esecuzione del programma nuovo e vecchio sugli stessi dati, confrontando successivamente i risultati ottenuti.\\
Questa tipologia di test verrà sviluppata in vista delle prossime revisioni.