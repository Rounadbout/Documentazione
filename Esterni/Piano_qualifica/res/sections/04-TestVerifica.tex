\section{Test di Verifica}

\subsection{Test di Unità}
Il test di unità ha l'obiettivo di determinare la correttezza e la completezza, rispetto ai requisiti, di un programma visto come singolo modulo\ped{\textit{G}}.\\
Per rispettare il livello qualitativo richiesto è necessario adempiere la seguente metrica:
\begin{itemize}
	\item{misurazione: numero di test soddisfatti;}
	\item{valore minimo accettabile: 100\%;}
	\item{valore preferibile: 100\%.}
\end{itemize}
Questa tipologia di test verrà sviluppata in vista delle prossime revisioni.


\subsection{Test di Integrazione}
Il test di integrazione ha l'obiettivo di verificare la correttezza funzionale nell’interazione tra più moduli\ped{\textit{G}}. In particolare questo tipo di test verifica:
\begin{enumerate}
	\item{l'assemblamento dei vari moduli\ped{\textit{G}} aggiunti incrementalmente;}
	\item{l'assemblamento di tutti i moduli\ped{\textit{G}} facenti parte del programma.}
\end{enumerate}
Per rispettare il livello qualitativo richiesto è necessario adempiere la seguente metrica:
\begin{itemize}
	\item{misurazione: numero di test soddisfatti;}
	\item{valore minimo accettabile: 100\%;}
	\item{valore preferibile: 100\%.}
\end{itemize}
Questa tipologia di test verrà sviluppata in vista delle prossime revisioni.


\subsection{Test di Sistema}
Il test di sistema ha l'obiettivo di testare particolari proprietà globali. In particolare:
\begin{enumerate}
	\item{\textbf{test di stress}: valutazione del sistema in condizioni di sovraccarico;}
	\item{\textbf{test di robustezza}: valutazione del sistema in presenza di dati non corretti;}
	\item{\textbf{test di sicurezza}: valutazione del livello di sicurezza del sistema.}
\end{enumerate}
Per rispettare il livello qualitativo richiesto è necessario adempiere la seguente metrica:
\begin{itemize}
	\item{misurazione: numero di test soddisfatti;}
	\item{valore minimo accettabile: 100\%;}
	\item{valore preferibile: 100\%.}
\end{itemize}
Questa tipologia di test verrà sviluppata in vista delle prossime revisioni.


\subsection{Test di Regressione}
Il test di regressione ha l'obiettivo di verificare che ad ogni aggiornamento di un modulo\ped{\textit{G}} software la nuova versione mantenga le funzionalità di quella precedente.
La sua applicazione consiste nell'esecuzione della nuova e della vecchia versione sullo stesso pool di dati, confrontando successivamente i risultati ottenuti per verificarne l'uguaglianza.\\
Per rispettare il livello qualitativo richiesto è necessario adempiere la seguente metrica:
\begin{itemize}
	\item misurazione: numero di test soddisfatti;
	\item valore minimo accettabile: 100\%;
	\item valore preferibile: 100\%.
\end{itemize}
Questa tipologia di test verrà sviluppata in vista delle prossime revisioni.