\section{Valutazioni per il miglioramento}
	Questa sezione riporta i problemi riscontrati dal gruppo \textit{Roundabout} durante il corso del progetto. Ogni problema viene valutato per trovare una possibile soluzione seguendo lo schema \textit{plan-do-check-act} per apportare un miglioramento continuo che sia il più efficace ed efficiente possibile. \\
	Verranno presentate quindi delle valutazioni dei problemi incontrati, suddivise per i periodi passati. \\
		I problemi incontrati possono essere divisi in 3 categorie:
		\begin{itemize}
			\item \textbf{organizzazione}: problemi relativi all'organizzazione ed alla comunicazione all'interno del gruppo;
			\item \textbf{ruoli}: problemi relativi allo svolgimento dei diversi ruoli;
			\item \textbf{strumenti di lavoro}: problemi relativi l'uso degli strumenti di lavoro.
		\end{itemize}
	Inoltre per ciascun periodo, ad eccezione dei periodi di Analisi e Consolidamento dei Requisiti, verranno presentate in forma tabellare tutte le modifiche migliorative apportate seguendo le segnalazioni presenti nell'esito della revisione formale al termine del periodo precedente.
	

	\subsection{Analisi e Consolidamento dei Requisiti}
	
		\subsubsection{Valutazioni sul periodo}
			\rowcolors{2}{lightRowColor}{darkRowColor}	
			\begin{longtable}{ 
				>{\centering}p{0.3\textwidth}
				>{\centering}p{0.3\textwidth}
				>{\centering\arraybackslash}p{0.3\textwidth}}
				
				\rowcolor{white} \caption {Valutazioni sui periodi di Analisi e Consolidamento dei Requisiti}	\\
		
				\coloredTableHead
				\textbf{\color{white}Problema} &
				\textbf{\color{white}Contromisura} &
				\textbf{\color{white}Riscontri} 
				\endfirsthead
				
				\rowcolor{white}\caption[]{(continua)}\\
				\coloredTableHead
				\textbf{\color{white}Problema} &
				\textbf{\color{white}Contromisura} &
				\textbf{\color{white}Riscontri} 
				\endhead
				
				\hline \multicolumn{3}{c}{\textit{Continua nella prossima pagina}} \\
				\endfoot
				\hline
				\endlastfoot	
								
				
		
				% Contenuto della tabella
				% Problema & Soluzione\\
				\hline
				\multicolumn{3}{c}{Organizzazione} \\
				\hline
				\textbf{Riunioni Interne}: si è rivelato un problema organizzativo l'impossibilità di vedersi fisicamente a causa della situazione di emergenza COVID-19\ped{\textit{G}}.
					&
				Abbiamo concordato di utilizzare maggiormente strumenti di collaborazione che consentono, oltre alla possibilità di effettuare videochiamate, una comunicazione semplificata per i diversi problemi che si possono verificare.
					&
				Gli strumenti di collaborazione online utilizzati si sono rivelati molto utili e ci hanno permesso di lavorare molto bene anche senza vederci fisicamente.
					\\
	
				\textbf{Appuntamenti}: problema a definire una calendarizzazione degli incontri tra i vari membri del gruppo.
					&
				Abbiamo definito che le riunioni interne saranno effettuate cadenzialmente due volte alla settimana il martedì e il venerdì, salvo esigenze particolari.
					&
				Tale programmazione delle riunioni ci ha permesso di rimanere sempre aggiornati sullo stato totale di avanzamento del progetto.
					\\
		
				\textbf{Riunioni Esterne}: durante la prima riunione effettuata con il \textit{Proponente}\ped{\textit{G}} a mezzo Skype\ped{\textit{G}}, si è valutato il problema comune di connessione instabile e conseguente perdita di parole durante la conversazione.
					&
				Risolto proponendo al \textit{Proponente}\ped{\textit{G}}	incontri telematici su piattaforma Zoom\ped{\textit{G}}, molto più leggera e con limitati problemi di chiamata.
					&
				Utilizzando la piattaforma Zoom\ped{\textit{G}} non si sono più presentati problemi di instabilità di connessione.
					\\
				
				\hline
				\multicolumn{3}{c}{Ruoli} \\
				\hline
				\textbf{Rivestire un ruolo}: il problema comune a tutti i ruoli è stato quello di doversi adattare ad una mentalità diversa in base al contesto richiesto, considerato il vincolo relativo alla copertura di un ruolo per membro, descritto nelle \textit{Norme di Progetto}.
					&
				Valutato che il maggior impatto di questa problematica si verifica nella fase iniziale di ogni cambio di ruolo, si è deciso di limitare le rotazioni indicativamente ogni due settimane cercando di non lasciare lavori in sospeso al membro successivo. In ogni caso vige il buon senso e la collaborazione reciproca.
					&
				L'esperienza appresa con l'avanzare del tempo e la conclusione dei lavori prima di cedere il ruolo al membro successivo hanno reso meno complicato il passaggio da un ruolo ad un altro. Questa problematica verrà senz'altro monitorata anche durante i periodi successivi.
					\\
				
			\end{longtable}
	
	\subsection{Progettazione Architetturale}
	
		\subsubsection{Segnalazioni RR}
			La tabella seguente presenta tutti i cambiamenti e miglioramenti apportati a seguito delle segnalazioni da parte del Committente\ped{\textit{G}} e del Proponente\ped{\textit{G}} presenti nell'esito della Revisione dei Requisiti.
			
			\rowcolors{2}{lightRowColor}{darkRowColor}	
			\begin{longtable}{ 
				>{\centering}p{0.2\textwidth} 
				>{\centering}p{0.35\textwidth}
				>{\centering\arraybackslash}p{0.35\textwidth}}
				
				\rowcolor{white} \caption {Risoluzione segnalazioni RR} \\
		
				\coloredTableHead
				\textbf{\color{white}Processo} &
				\textbf{\color{white}Segnalazione} &
				\textbf{\color{white}Soluzione}
				\endfirsthead
				
				\rowcolor{white}\caption[]{(continua)}\\
				\coloredTableHead
				\textbf{\color{white}Processo} &
				\textbf{\color{white}Segnalazione} &
				\textbf{\color{white}Soluzione}
				\endhead
				
				\hline \multicolumn{3}{c}{\textit{Continua nella prossima pagina}} \\
				\endfoot
				\hline
				\endlastfoot				
			
			
				% Contenuto della tabella
				% Processo & Segnalazione & Soluzione \\
		
				Documentazione
					&
				 Registro delle modifiche: uno “scatto” di versione conseguente a un'azione di modifica prima della sua verifica di validità, innesca rischi di iterazione che contrastano con l'approccio incrementale che dite vostro. 
					&
				Lo scatto di versione ora avviene solamente a seguito della verifica di validità della modifica eseguita.
					\\
			
				Documentazione
					&
				Riferimenti: essi sono utili se sono specifici (e non eccessivamente ampi o generici) e localizzabili (cioè identificati con precisione).
					&
				I riferimenti sono stati resi più specifici.
					\\
		
				Documentazione
					&
				Convenzioni redazionali: per evitare rischi di sovrapposizione tra documenti, converrà che il loro nome ne riporti anche il corrispondente numero di versione. Evitate espressioni come “il fine di … è \textbf{quello di}” (e similari), dove la parte in grassetto è fastidiosamente ridondante.
					&
				È stato aggiunto il numero di versione nel nome dei documenti e le frasi contenenti espressioni come “il fine di … è \textbf{quello di}” sono state riformulate.
					\\
		
				Documentazione
					&
				Stile tipografico: fate maggiore attenzione all'uso consistente delle iniziali maiuscole nei titoli delle parti di documento.
					&
				Sono stati modificati i titoli delle parti di documento che presentavano inconsistenze nell'utilizzo di iniziali maiuscole.
					\\
						
				Documentazione - Analisi dei Requisiti
					&
				Eliminare il diagramma di UC1. UC3 è completamente da rivedere.
					&
				Eliminato diagramma UC1, introdotto UC4 come sottocaso d'uso di UC3.   
					\\
			
				Documentazione - Analisi dei Requisiti
					&
				Non si può inserire il caso d'uso\ped{\textit{G}} nel proprio diagramma, ma i suoi sotto-casi (molti casi d'uso\ped{\textit{G}} presentano il medesimo errore). Inoltre, non è corretta l'estensione.
					&
				I diagrammi non corretti sono stati rimossi in tutto il documento. 
					\\
			
				Documentazione - Analisi dei Requisiti
					&
				UC3.1 non è chiaro.
					&
				La descrizione di UC3.1 è stata ampliata in modo da renderlo più chiaro. 
					\\
			
				Documentazione - Analisi dei Requisiti
					&
				UC5 non è corretta l'estensione. UC5.1: in realtà, la gerarchia andrebbe individuata su UC5, poiché stiamo parlando di due tipologie di login differenti e mutualmente esclusive. Conseguentemente, fig. 3.2.9 va eliminata.
					&
				È stata creata una gerarchia per la gestione del login, in particolare la gerarchia vede coma padre il caso d'uso\ped{\textit{G}} di login generale, e come figli le specializzazioni. 
					\\
			
				Documentazione - Analisi dei Requisiti
					&
				UC10 fonde sia la ricerca di una funzione che la visualizzazione del dettaglio. Rivedere.
					&
				La ricerca e la visualizzazione di una funzione sono stati divisi in due casi d'uso\ped{\textit{G}} distinti. 
					\\
			
				Documentazione - Analisi dei Requisiti
					&
				UC11.2 non può essere sotto-caso di UC11.
					&
				UC11.2 è andato a costituire un caso d'uso\ped{\textit{G}} primario. 
					\\
			
				Documentazione - Analisi dei Requisiti
					&
				UC13: i sotto-casi sono in realtà casi derivati.
					&
				I sotto-casi sono andati a formare dei casi d'uso\ped{\textit{G}} primari, in particolare non è stata identificata alcuna gerarchia, poichè i casi d'uso\ped{\textit{G}} non presentano alcuna azione in comune. 
					\\
							
				Documentazione - Analisi dei Requisiti
					&
				UC14, alcune estensioni sono senza direzione.
					&
				È stato corretto il diagramma errato. 
					\\
			
		 
				Documentazione - Norme di Progetto
					&
				 §2: tra le attività del processo di fornitura, considerate la cura dei rapporti con il Proponente\ped{\textit{G}}, particolarmente delicata in questo periodo di blocco degli spostamenti.
					&
				Sono stati inseriti dei riferimenti alla cura dei rapporti con il Proponente\ped{\textit{G}} nelle attività del processo di fornitura. E' stato inserito il processo di gestione dei cambiamenti tra i processi di supporto.
					\\
			
				Documentazione - Norme di Progetto
					&
				 §3: tra i processi di supporto, vi converrà considerare anche il processo di gestione dei cambiamenti, che presto diventerà essenziale per dare ordine alle attività correttive che conseguono alla rilevazione di un difetto da correggere. L'efficacia della trattazione delle metriche di qualità adottate potrebbe migliorare associandone ciascuna alle attività cui essa si riferisce, e alle corrispondenti procedure.
					&
				Sono state associate le metriche alle corrispondenti attività.
				E' stato fatto un aggiornamento sia per quanto riguarda la procedura di verifica e gestione del repository\ped{\textit{G}} che per quanto riguarda la struttura del repository\ped{\textit{G}} di progetto. Sono inoltre state aggiunte nuove norme e strumenti relativi alla codifica.
					\\
			 
				Documentazione - Piano di Progetto
					&
				§3-§4: vi è totale disallineamento tra la dichiarazione di adesione al modello di sviluppo incrementale e la pianificazione delle attività, che si incentra esclusivamente sulla scansione degli obblighi contrattuali (che per loro natura sono strettamente sequenziali), ignorando lo sviluppo degli incrementi, e quindi evidentemente non assegnandovi risorse congrue. §5: la criticità di cui sopra invalida la credibilità del preventivo che avete presentato.
					&
				E' stata aggiunta la pianificazione ad incrementi per i periodi di Progettazione Architetturale, Progettazione di Dettaglio e Codifica, Validazione e Collaudo. Sono state inoltre aggiunte le rispettive tabelle di pianificazione oraria interna agli incrementi e aggiornati i diagrammi di Gantt.
					\\
			
				Documentazione - Piano di Progetto
					&
				 §6: il consuntivo di periodo serve per ragionare, in corso d'opera, sulle ragioni degli scostamenti eventualmente rilevati, sulle loro possibili mitigazioni, e sui conseguenti raffinamenti di pianificazione da  effettuare nei periodi successivi, da riflettere poi nel “Preventivo a finire”. Questo tipo di ragionamento è particolarmente importante nel caso di sviluppo incrementale, come dovrebbe essere il vostro.
					&
				E' stato aggiunto il consuntivo di periodo per il periodo di Progettazione Architetturale ed è stata aggiornata la tabella di riscontro dei rischi.
					\\
		 
				 Documentazione - Piano di Qualifica
					&
				§2-§3: migliorabile la correlazione con le Norme, cui spetta la definizione delle metriche adottate per la misurazione della qualità e degli strumenti scelti per la loro valutazione, permettendo al PdQ di limitarsi credibilmente a fissare gli obiettivi quantitativi di qualità scelti per il progetto. Tali obiettivi riguardano anche i fattori di copertura dei test, che invece sembrate ignorare.
					&
				Sono stati inoltre aggiornati i valori delle metriche presenti nell'appendice B ed aggiunti i relativi valori corrispondenti alle metriche per la codifica. E' stata modificata la tabella nell'appendice A ed è stata aggiunta l'appendice C contenente i cambiamenti apportati ai documenti in seguito alle revisioni.
					\\
			
				Documentazione - Piano di Qualifica
					&
				§A: questo tipo di contenuti è di pertinenza delle Norme. §B: questi contenuti dovrebbe essere interpretati come attuazione del ciclo PDCA\ped{\textit{G}}, ma non è chiaro che questo sia il caso. §C: questo resoconto deve riflettere tutte le metriche adottate, meglio se tramite presentazione “a cruscotto”, con serie storiche e diagrammi a contenuto incrementale, piuttosto che tramite tabelle che “fotografano” gli eventi\ped{\textit{G}}, ma non li mettono in relazione tra loro.
					&
				E' stata spostata l'appendice contenente gli standard di qualità di processo e di prodotto\ped{\textit{G}} nelle \NdP{} e sono state eliminate le descrizioni associate alle metriche.
					\\

			\end{longtable}
		
		\subsubsection{Valutazioni sul periodo}
			\rowcolors{2}{lightRowColor}{darkRowColor}
			\begin{longtable}{ 
				>{\centering}p{0.3\textwidth}
				>{\centering}p{0.3\textwidth}
				>{\centering\arraybackslash}p{0.3\textwidth}}
				
				\rowcolor{white} \caption {Valutazioni sul periodo di Progettazione Architetturale}	\\
		
				\coloredTableHead
				\textbf{\color{white}Problema} &
				\textbf{\color{white}Contromisura} &
				\textbf{\color{white}Riscontri} 
				\endfirsthead
				
				\rowcolor{white}\caption[]{(continua)}\\
				\coloredTableHead
				\textbf{\color{white}Problema} &
				\textbf{\color{white}Contromisura} &
				\textbf{\color{white}Riscontri} 
				\endhead
				
				\hline \multicolumn{3}{c}{\textit{Continua nella prossima pagina}} \\
				\endfoot
				\hline
				\endlastfoot	
		
				% Contenuto della tabella
				% Problema & Soluzione\\
				%\multicolumn{3}{c}{Organizzazione} \\
				
				%\multicolumn{3}{c}{Ruoli} \\
				
				\hline \multicolumn{3}{c}{Strumenti di lavoro} \\ \hline
				\textbf{Apprendiento delle tecnologie}: si è rivelato un problema l'insufficiente conoscenza delle tecnologie da utilizzare per sviluppare l'applicativo. Nel dettaglio, queste sono: 		\begin{enumerate}
					\item{\LaTeX{}\ped{\textit{G}};}
					\item{Ethereum\ped{\textit{G}};}
					\item{Solidity\ped{\textit{G}};}
					\item{Truffle\ped{\textit{G}};}
					\item{Ganache\ped{\textit{G}};}
					\item{Ropsten\ped{\textit{G}};}
					\item{AWS Elastic Beanstalk\ped{\textit{G}};}
					\item{AWS Lambda\ped{\textit{G}};}
					\item{Serverless\ped{\textit{G}};}
					\item{Typescript\ped{\textit{G}};}
					\item{ESLint\ped{\textit{G}};}
					\item{ethers.js\ped{\textit{G}};}
					\item{Yargs\ped{\textit{G}};}
					\item{inquirer.js\ped{\textit{G}};}
					\item{Configstore\ped{\textit{G}}.}
				\end{enumerate}
					&
				Si è colmata questa mancanza tramite ricerca personale e studio autonomo.
					&
				Lo studio autonomo delle tecnologie ha permesso a tutti i membri di condividere informazioni con il resto del gruppo.
				Ciò ha permesso uno studio più approfondito e più rapido delle tecnologie individuate.
				Questa problematica riguardo lo studio delle tecnologie verrà monitorata anche durante i periodi successivi.
					\\
		
				\textbf{Omogeneità dei documenti prodotti\ped{\textit{G}}}: Considerato che la stesura di un documento può essere effettuata anche da più persone che ricoprono lo stesso ruolo in contemporanea, si è verificato il problema di omogeneità all'interno dei documenti
					&
				La soluzione migliore è stata quella di concordare assieme nelle \textit{Norme di Progetto} gli utilizzi di maiuscole, minuscole, corsivo, grassetto, etc.
					&
				L'aver concordato assieme gli utilizzi di maiuscole, minuscole, etc. e la lettura delle \textit{Norme di Progetto} da parte dei membri del gruppo ha permesso di limitare incoerenze all'interno dei documenti. È comunque necessario che ciascun componente utilizzi grande attenzione nella modifica dei documenti e nella verifica di essi in quanto, a causa dell'inesperienza, potrebbero verificarsi incoerenze e alcune norme definite e concordate potrebbero essere seguite in maniera errata.
					\\
				
			\end{longtable}
	
	\subsection{Progettazione di Dettaglio e Codifica}
	
		\subsubsection{Segnalazioni RP}
			La tabella seguente presenta tutti i cambiamenti e miglioramenti apportati a seguito delle segnalazioni da parte del Committente\ped{\textit{G}} e del Proponente\ped{\textit{G}} presenti nell'esito della Revisione di Progettazione.
			
			\rowcolors{2}{lightRowColor}{darkRowColor}	
			\begin{longtable}{ 
				>{\centering}p{0.2\textwidth} 
				>{\centering}p{0.35\textwidth}
				>{\centering\arraybackslash}p{0.35\textwidth}}
				
				\rowcolor{white} \caption {Risoluzione segnalazioni RP} \\
		
				\coloredTableHead
				\textbf{\color{white}Processo} &
				\textbf{\color{white}Segnalazione} &
				\textbf{\color{white}Soluzione}
				\endfirsthead
				
				\rowcolor{white}\caption[]{(continua)}\\
				\coloredTableHead
				\textbf{\color{white}Processo} &
				\textbf{\color{white}Segnalazione} &
				\textbf{\color{white}Soluzione}
				\endhead
				
				\hline \multicolumn{3}{c}{\textit{Continua nella prossima pagina}} \\
				\endfoot
				\hline
				\endlastfoot	
		
				% Contenuto della tabella
				% Processo & Segnalazione & Soluzione \\
		
				Documentazione
					&
				Registro delle modifiche: permane l’errore metodologico che vi porta a effettuare uno “scatto” di versione a seguito di azioni di modifica prima della loro verifica di validità. In termini di gestione dei contenuti del repository\ped{\textit{G}}, ciò che state facendo usa il versionamento come mero diario delle azioni svolte, indipendentemente dalla qualità del loro prodotto\ped{\textit{G}}. 
					&
				Per rendere chiaro che lo scatto di versione avviene solamente a seguito della verifica di validità di una modifica apportata è stato aggiunto nel changelog una colonna nella quale viene indicato il nome del componente che ha verificato la validità della modifica.
					\\
				
				Documentazione
					&
				Convenzioni redazionali: il nome dei documenti versionati non ne riporta anche il corrispondente numero di versione. Permangono poi occorrenze di espressioni come “il fine di … è quello di” (e similari). 
					&
				È stato aggiunto il numero di versione nel nome dei documenti versionati e sono state riformulate le frasi contenenti le  espressioni come “il fine di … è quello di” (e similari).
					\\
					
				Documentazione
					&
				Stile tipografico: permangono inconsistenze nell’uso di iniziali maiuscole nei titoli delle parti di documento. 
					&
				Sono stati controllati tutti i documenti ed è stato uniformato l'utilizzo delle maiuscole nei titoli delle parti di documento.
					\\
					
				Documentazione - Norme di Progetto
					&
				Permangono incongruenze nella struttura descrittiva dei processi, che è irragionevolmente difforme tra taluni processi, con conseguente confusione informativa.
					&
				È stata identificata una struttura definitiva per la descrizione dei processi, e tutti sono stati uniformati a quest’ultima.
					\\
					
				Documentazione - Analisi dei Requisiti
					&
				UC9 è da suddividere in due casi d’uso distinti: la ricerca per nome funzione e la visualizzazione. 
					&
				UC9 è stato scomposto in più casi d’uso\ped{\textit{G}}, in maniera da separare la ricerca di una funzione della visualizzazione dei risultati  .
					\\
					
				Documentazione - Analisi dei Requisiti
					&
				Quali differenza vi sono tra i comandi “info” e “search”? 
					&
				La differenza tra le informazioni mostrate dai due comandi è stata accentuata, andando a modificare anche il nome dei relativi casi d’uso\ped{\textit{G}}.
					\\
					
				Documentazione - Analisi dei Requisiti
					&
				UC15.1 ha due casi d’uso\ped{\textit{G}} derivati. Modellandoli in questo modo, state dicendo che le rispettive funzionalità sono tra loro mutuamente esclusive. 
					&
				Il UC15 è stato rimodellato in modo tale da permettere la modifica della descrizione e del codice contemporaneamente, in un unico comando. Per fare ciò, abbiamo previsto dei casi d’uso\ped{\textit{G}} per l’inserimento della nuova descrizione e l’inserimento del nuovo codice come sotto-casi diretti del UC15.
					\\
					
				Documentazione - Analisi dei Requisiti
					&
				Fig. 3.2.11: le estensioni non hanno verso. 
					&
				Abbiamo aggiunto il verso corretto alle estensioni. 
					\\
					
				Documentazione - Analisi dei Requisiti
					&
				R1V2 è un requisito funzionale.
					&
				Il requisito inizialmente identificato erroneamente come un requisito di vincolo è stato convertito in un requisito funzionale. 
					\\
				
				Documentazione - Piano di Progetto
					&
				§2 e §A: materiale diligente per presentazione ma ancora poco profondo per contenuto. 
					&
				Il materiale relativo ai rischi è stato approfondito.
					\\
					
				Documentazione - Piano di Progetto
					&
				§3: la specifica degli obiettivi incrementali è meglio collocata in premessa alla pianificazione. 
					&
				È stata introdotta in premessa alla pianificazione una tabella che specifica gli obiettivi incrementali pianificati.
					\\
					
				Documentazione - Piano di Progetto
					&
				§4: gli attuali contenuti suggeriscono uno sviluppo segmentato in macro-periodi, incrementali solo al loro interno, e non piuttosto la dichiarata incrementalità. 
					&
				Gli incrementi del periodo attuale e successivo sono stati leggermente modificati, in maniera da focalizzarsi sulle funzionalità dell’intero prodotto\ped{\textit{G}} e non sull’implementazione dei singoli moduli\ped{\textit{G}}.
					\\
					
				Documentazione - Piano di Progetto
					&
				§4-§5: questi contenuti sono resi poco credibili dall’errata interpretazione dello logica di sviluppo incrementale. 
					&
				La sezione 5 è stata approfondita in maniera da essere maggiormente focalizzata sugli incrementi individuati.
					\\
					
				Documentazione - Piano di Progetto
					&
				§6: l’incrementalità è così poco centrale nel vostro ragionamento, che il consuntivo di periodo non ne riporta traccia. 
					&
				Dal periodo attuale abbiamo iniziato a riportare un consuntivo per ogni incremento, in modo da seguire maggiormente il modello di sviluppo incrementale.
					\\
					
				Documentazione - Piano di Qualifica
					&
				§4: gravemente errato sul piano concettuale il titolo “test di verifica”; per renderne significativi i contenuti (cui manca la dichiarazione di specifici obiettivi metrici) serve esplicitare la correlazione con §B.2, attualmente omessa, e conseguentemente priva di visione d’insieme. 
					&
				È stata corretta la struttura dei test, assegnando per ogni tipologia un grafico. È ora possibile osservare l'andamento della copertura dei test in un qualsiasi punto di maturazione del software.
					\\
				
				Documentazione - Piano di Qualifica
					&
				§A: come già segnalato in sede di RR, tale materiale è utile se correlato con l’attuazione del ciclo PDCA\ped{\textit{G}} e dunque strutturato e misurato attraverso specifici obiettivi di miglioramento. 
					&
				§A è stato ristrutturato in modo che possa presentare l'attuazione del ciclo PDCA\ped{\textit{G}}.
					\\
					
				Documentazione - Piano di Qualifica
					&
				§C: questi contenuti sono meglio collocati in §A, in ottica di miglioramento. 
					&
				I contenuti di §C sono stati collocati in §A.
					\\
			\end{longtable}
			
		\subsubsection{Valutazioni sul periodo}
			\rowcolors{2}{lightRowColor}{darkRowColor}
			\begin{longtable}{ 
				>{\centering}p{0.3\textwidth}
				>{\centering}p{0.3\textwidth}
				>{\centering\arraybackslash}p{0.3\textwidth}}
				
				\rowcolor{white} \caption {Valutazioni sul periodo di Progettazione di Dettaglio e Codifica}		\\
		
				\coloredTableHead
				\textbf{\color{white}Problema} &
				\textbf{\color{white}Contromisura} &
				\textbf{\color{white}Riscontri} 
				\endfirsthead
				
				\rowcolor{white}\caption[]{(continua)}\\
				\coloredTableHead
				\textbf{\color{white}Problema} &
				\textbf{\color{white}Contromisura} &
				\textbf{\color{white}Riscontri} 
				\endhead
				
				\hline \multicolumn{3}{c}{\textit{Continua nella prossima pagina}} \\
				\endfoot
				\hline
				\endlastfoot	
		
				% Contenuto della tabella
				% Problema & Soluzione\\
				\hline \multicolumn{3}{c}{Organizzazione} \\ \hline
				\textbf{Comunicazioni interne}: a causa di impegni personali alcuni componenti talvolta non sono riusciti ad essere presenti ad alcune riunioni; inoltre questi impegni hanno causato in qualche momento dei rallentamenti nel proseguimento del lavoro.
					&
				Ci siamo impegnati tutti sin da subito a notificare con largo anticipo eventuali impegni e/o assenze a riunioni programmate, permettendo quindi di riprogrammare delle riunioni e di riorganizzare le risorse disponibili per evitare rallentamenti.
					&
				La notifica degli impegni e la possibilità di riprogrammare le riunioni ci hanno permesso di essere sempre tutti presenti alle riunioni e di riuscire a portare a termine tutti i compiti rispettando le scadenze prefissate.
					\\
				
				\hline \multicolumn{3}{c}{Strumenti di lavoro} \\ \hline
				\textbf{Apprendimento delle tecnologie}: durante lo sviluppo di test si sono presentate delle lacune nella conoscenza delle tecnologie scelte che hanno portato ad alcune difficoltà nello sviluppo di essi.
					&
				Lo studio individuale dei periodi precedenti non si è rivelato sufficiente per consentirci di sviluppare di test completi ed esaustivi. Ci siamo quindi impegnati in uno studio più approfondito delle tecnologie utilizzate, presentando e discutendo poi all'interno del gruppo con quanto appreso.
					&
				Questo studio più approfondito delle tecnologie ci ha permesso di comprendere il modo in cui sviluppare test efficaci ed esaustivi.
					\\
			\end{longtable}
		
		
	\subsection{Validazione e Collaudo}
		\subsubsection{Segnalazioni RQ}
			La tabella seguente presenta tutti i cambiamenti e miglioramenti apportati a seguito delle segnalazioni
			da parte del Committente\ped{\textit{G}} e del Proponente\ped{\textit{G}} presenti nell'esito della Revisione di Qualifica.
			
			\rowcolors{2}{lightRowColor}{darkRowColor}	
			\begin{longtable}{ 
					>{\centering}p{0.2\textwidth} 
					>{\centering}p{0.35\textwidth}
					>{\centering\arraybackslash}p{0.35\textwidth}}
				
				\rowcolor{white} \caption {Risoluzione segnalazioni RQ} \\
				
				\coloredTableHead
				\textbf{\color{white}Processo} &
				\textbf{\color{white}Segnalazione} &
				\textbf{\color{white}Soluzione}
				\endfirsthead
				
				\rowcolor{white}\caption[]{(continua)}\\
				\coloredTableHead
				\textbf{\color{white}Processo} &
				\textbf{\color{white}Segnalazione} &
				\textbf{\color{white}Soluzione}
				\endhead
				
				\hline \multicolumn{3}{c}{\textit{Continua nella prossima pagina}} \\
				\endfoot
				\hline
				\endlastfoot	
				
				% Contenuto della tabella
				% Processo & Segnalazione & Soluzione \\
				
				Documentazione - Registro delle Modifiche
				&
				Non compreso e non sanato l’errore metodologico già più volte segnalato in precedenza, segno di un modo di procedere frettoloso, che cerca l’avanzamento senza curarsi di aver prima posato buone fondamenta di metodo.
				&
				Aggiornata la metodologia di gestione del Registro delle modifiche, in modo che faccia diretto riferimento all’impatto della modifica stessa ed alla sua retrocompatibilità o meno. (Vedi \textit{Norme di Progetto 4.0.0} per info più specifiche)
				\\
				
				
				Documentazione - Stile Tipografico
				&
				Nel segno di quanto sopra, permangono anche e ancora inconsistenze nell’uso di iniziali maiuscole nei titoli delle parti di documento. 
				&
				I documenti sono stati revisionati con attenzione, e corrette le inconsistenze rilevate. 
				\\
				
				
				Documentazione - Stile Redazionale
				&
				Le appendici sono tali perché hanno una diversa numerazione, che rende superfluo l’uso del prefisso “Appendice” nel loro titolo.
				&
				Rimosso il prefisso “Appendice” dal titolo delle appendici.
				\\
				
				
				Documentazione - Verbali
				&
				Scarse invece le testimonianze di contatti con il Proponente\ped{\textit{G}}. 
				&
				Tentativo di comunicazione più frequente con il Proponente\ped{\textit{G}} nell’ultimo periodo, precedente la RA.
				\\
				
				
				Documentazione - Manuale Utente
				&
				Documento redatto in modo molto sbrigativo. Valutare un approccio in stile tutorial per guidare l’utente attraverso le funzionalità offerte dal prodotto\ped{\textit{G}}. 
				&
				Resa più chiara la sintassi usata per illustrare i comandi (che è stata anche spiegata più approfonditamente nella sezione apposita).
				Aggiunta una sottosezione ‘Usage’ all’interno della sezione esplicativa di ogni comando che fornisce degli esempi d’uso pratici stile ‘tutorial’.
				Approfondita la parte 'Troubleshooting error' con tutti gli errori che potrebbero comparire all’utente allo stato attuale di sviluppo.\\
				
				
				Documentazione - Manuale Sviluppatore
				&
				I riferimenti forniti sono troppo generici e quindi poco informativi.
				&
				Migliorati i riferimenti, rendendoli più precisi ed utili.
				\\
				
				
				Documentazione - Manuale Sviluppatore
				&
				Le informazioni presenti in §2.5 vanno estese e completate.
				&
				Estese di molto le informazioni relative alle singole funzionalità, con riferimenti al processo di gestione delle singole richieste e delle varie opzioni disponibili per i comandi stessi. 
				\\
				
				
				Documentazione - Manuale Sviluppatore
				&
				§3.3: ha tono di slogan, inadatto allo scopo. Cercate di fornire esempi concreti, altrimenti, eliminate la sezione.
				&
				Eliminata la sezione, in quanto semplice reindirizzamento verso le omonime sezioni dei singoli moduli\ped{\textit{G}}. 
				\\
				
				
				Documentazione - Manuale Sviluppatore
				&
				Quale valore informativo fornisce al lettore l’utilizzo del tipo “any” nei diagrammi? 
				&
				Ristrutturati i diagrammi ed inserite tipizzazioni più precise a sostituzione degli identificativi “any”.
				\\
				
				
				Documentazione - Piano di Qualifica
				&
				La risoluzione del difetto segnalato intorno al titolo di §4 è “furbetta” ma poco lungimirante, perché il titolo attuale è diventato “monco”, cioè non descrive i contenuti della sezione. 
				&
				Eliminato il titolo Test, il quale non fornisce informazioni aggiuntive riguardo alla sezione. Ciascun test viene definito in una propria sezione corredata da titolo. 
				\\
				
				
				Documentazione - Piano di Qualifica
				&
				I cruscotti di avanzamento associati a ogni blocco di test segnalano che lo stato corrente è gravemente arretrato rispetto alle attese minime correlate con l’accesso alla RQ. Non vi è traccia, tuttavia, di una presa d’atto di tale situazione.
				&
				Viene posta maggiore attenzione nel descrivere le motivazioni per le quali alcune parti di documento possono presentano carenze. Lo stato dei test notevolmente migliorato, sanando eventuali ritardi presenti nella precedente revisione.
				\\
				
				
				Documentazione - Piano di Qualifica
				&
				Il contenuto delle appendici A e B è buono per stile di presentazione e ampiezza, ma riflette un atteggiamento retrospettivo, ancora non maturo abbastanza da spostare lo sforzo sull’adozione di misure proattive di automiglioramento.
				&
				Viene posta maggiore attenzione nella scrittura delle parti di documento portate all'attenzione, adottando un pensiero lungimirante orientato all'automiglioramento.
				\\
				
				
			\end{longtable}
			
		\subsubsection{Valutazioni sul periodo}
			\rowcolors{2}{lightRowColor}{darkRowColor}
			\begin{longtable}{ 
					>{\centering}p{0.3\textwidth}
					>{\centering}p{0.3\textwidth}
					>{\centering\arraybackslash}p{0.3\textwidth}}
				
				\rowcolor{white} \caption {Valutazioni sul periodo di Validazione e Collaudo}		\\
				
				\coloredTableHead
				\textbf{\color{white}Problema} &
				\textbf{\color{white}Contromisura} &
				\textbf{\color{white}Riscontri} 
				\endfirsthead
				
				\rowcolor{white}\caption[]{(continua)}\\
				\coloredTableHead
				\textbf{\color{white}Problema} &
				\textbf{\color{white}Contromisura} &
				\textbf{\color{white}Riscontri} 
				\endhead
				
				\hline \multicolumn{3}{c}{\textit{Continua nella prossima pagina}} \\
				\endfoot
				\hline
				\endlastfoot	
				
				% Contenuto della tabella
				% Problema & Soluzione\\
				\hline \multicolumn{3}{c}{Organizzazione} \\ \hline
				\textbf{Concomitanza ultimi esami}: vista la sovrapposizione del periodo di Validazione e Collaudo alla sessione estiva degli esami universitari, è stato opportuno prevedere l'assenza di alcuni membri del gruppo per dedicare del tempo al completamento degli ultimi esami rimanenti.
				&
				Ci siamo impegnati a comunicare tra di noi le date necessarie allo studio personale, permettendo quindi di programmare al meglio il periodo, evitando così ritardi.
				&
				La contromisura adottata ha permesso di prevedere i possibili ritardi evitandoli pianificando le risorse disponibili.
				\\
				
			\end{longtable}
	%\subsection{Revisione di accettazione}
	
