\section{Qualità di Prodotto}
\subsection{Scopo}
Lo scopo della seguente sezione è quello di fornire le metriche utilizzate dal team \Gruppo{} per valutare la qualità del prodotto risultante dal progetto didattico. Lo standard di riferimento per tale valutazione è ISO/IEC 9126, che definisce e descrive le caratteristiche atte a produrre un prodotto di qualità.

\subsection{Funzionalità}
È la capacità del prodotto di fornire funzioni che riescano a soddisfare i requisiti presentati nell'\textit{\AdR}.
Le metriche usate per valutare tale capacità sono:

	\subsubsection*{Completezza dell'implementazione}
	È un valore percentuale che indica la completezza del prodotto in funzione del rispetto dei requisiti.
	\begin{itemize}
	\item{misurazione: $CI = (1- \displaystyle\frac{\#funzionalit\grave{a}\_non\_implementate}{\#funzionalit\grave{a}\_individuate})\times100$};
	\item {valore minimo accettabile: 100\%;}
	\item {valore preferibile: 100\%.}
	\end{itemize}
	
\subsection{Affidabilità}
È la capacità del prodotto di mantenere prestazioni elevate anche in caso di anomalie o situazioni critiche.
Le metriche usate per valutare tali capacità sono:

	\subsubsection*{Densità errori}
	È un valore percentuale che indica la resistenza del prodotto a fronte di malfunzionamenti.
	\begin{itemize}
		\item{misurazione: $DE = \displaystyle\frac{\#errori\_rilevati}{\#test\_eseguiti}\times100$}
		\item {valore minimo accettabile: < 10\%;}
		\item {valore preferibile: 0\%.}
	\end{itemize}

\subsection{Usabilità}
È la capacità del prodotto di essere capito, appreso ed usato dall'utente in tempi ragionevoli.
Le metriche usate per valutare tale capacità sono:

	\subsubsection*{Facilità di utilizzo}
	È un valore intero che indica la facilità con cui l'utente interagisce con il prodotto. Viene misurato mediante il numero di passi che l'utente deve eseguire per portare a termine un'operazione.
	\begin{itemize}
		\item {misurazione: numero intero;}
		\item {valore minimo accettabile: 5;}
		\item {valore preferibile: 2.}
		\end{itemize}
		
	\subsubsection*{Facilità di apprendimento}
	È un numero intero che indica la facilità con cui l'utente riesce ad apprendere ed utilizzare le funzionalità del prodotto. Viene rappresentata tramite il tempo in minuti che serve per comprenderle. La facilità di apprendimento del prodotto è data dalla media della facilità di apprendimento delle varie funzionalità.
	\begin{itemize}
		\item {misurazione: numero intero;}
		\item {valore minimo accettabile: 5;}
		\item {valore preferibile: 2.}
	\end{itemize}
	
\subsection{Manutenibilità}
È la capacità del prodotto di essere modificato includendo correzioni, miglioramenti od adattamenti. 
Le metriche usate per valutare tale capacità sono:

	\subsubsection*{Facilità di comprensione}
	È un valore percentuale che determina la facilità di comprensione del codice. Corrisponde al rapporto tra il numero di linee di commento e il numero di linee di codice.
	\begin{itemize}
		\item{misurazione: $R = \displaystyle\frac{\#linee\_di\_commento}{\#linee\_di\_codice}\times100$;}
		\item {valore minimo accettabile: 10\%;}
		\item {valore preferibile: 20\%.}
	\end{itemize}
	
	\subsubsection*{Semplicità delle classi}
	È un valore intero che indica il numero di metodi presenti in una classe. Viene usato al fine di avere classi che hanno uno scopo ben preciso e che seguono il principio del \textit{Single Responsibility Principle}.
		\begin{itemize}
			\item {misurazione: numero intero;}
			\item {valore minimo accettabile: < 10;}
			\item {valore preferibile: < 6.}
	\end{itemize}