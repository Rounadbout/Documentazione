\section{Qualità di Prodotto}
\subsection{Scopo}
Lo scopo della seguente sezione è quello di fornire le metriche utilizzate dal team \Gruppo{} per valutare la qualità del prodotto risultante dal progetto didattico. Lo standard di riferimento per tale valutazione è ISO/IEC 9126, che definisce e descrive le caratteristiche atte a produrre un prodotto di qualità.

\subsection{Funzionalità}
È la capacità del prodotto di fornire funzioni che riescano a soddisfare i requisiti presentati nell'\textit{\AdR}.
I valori di riferimento delle metriche usate per valutare tale capacità sono:

	\subsubsection*{Completezza dell'implementazione}
	\begin{itemize}
	\item{misurazione: $CI = (1- \displaystyle\frac{\#funzionalit\grave{a}\_non\_implementate}{\#funzionalit\grave{a}\_individuate})\times100$};
	\item {valore minimo accettabile: 100\%;}
	\item {valore preferibile: 100\%.}
	\end{itemize}
	
\subsection{Affidabilità}
È la capacità del prodotto di mantenere prestazioni elevate anche in caso di anomalie o situazioni critiche.
I valori di riferimento delle metriche usate per valutare tali capacità sono:

	\subsubsection*{Densità errori}
	\begin{itemize}
		\item{misurazione: $DE = \displaystyle\frac{\#errori\_rilevati}{\#test\_eseguiti}\times100$}
		\item {valore minimo accettabile: < 10\%;}
		\item {valore preferibile: 0\%.}
	\end{itemize}

\subsection{Usabilità}
È la capacità del prodotto di essere capito, appreso ed usato dall'utente in tempi ragionevoli.
I valori di riferimento delle metriche usate per valutare tale capacità sono:

	\subsubsection*{Facilità di utilizzo}
	\begin{itemize}
		\item {misurazione: numero intero;}
		\item {valore minimo accettabile: 5;}
		\item {valore preferibile: 2.}
		\end{itemize}
		
	\subsubsection*{Facilità di apprendimento}
	\begin{itemize}
		\item {misurazione: numero intero;}
		\item {valore minimo accettabile: 5;}
		\item {valore preferibile: 2.}
	\end{itemize}
	
\subsection{Manutenibilità}
È la capacità del prodotto di essere modificato includendo correzioni, miglioramenti od adattamenti. 
I valori di riferimento delle metriche usate per valutare tale capacità sono:

	\subsubsection*{Facilità di comprensione}
	\begin{itemize}
		\item{misurazione: $R = \displaystyle\frac{\#linee\_di\_commento}{\#linee\_di\_codice}\times100$;}
		\item {valore minimo accettabile: 10\%;}
		\item {valore preferibile: 20\%.}
	\end{itemize}
	
	\subsubsection*{Semplicità delle classi}
		\begin{itemize}
			\item {misurazione: numero intero;}
			\item {valore minimo accettabile: < 10;}
			\item {valore preferibile: < 6.}
	\end{itemize}