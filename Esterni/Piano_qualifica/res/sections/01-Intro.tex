\section{Introduzione}

\subsection{Premessa}
	Il \textit{Piano di Qualifica} è un documento di cui si prevede la stesura durante l'intera durata del progetto, adottando una modalità incrementale. Per questo motivo, non è da considerarsi equivalente ad un documento completo.


\subsection{Scopo del documento}
	Questo documento contiene tutte le strategie di verifica e validazione adottate dal gruppo \textit{Roundabout}, al fine di garantire la qualità di prodotto e processo. Per ottenere questo obiettivo viene applicato una verifica continua sui processi in corso e sulle attività svolte. Procedendo in questo modo si eviteranno più facilmente eventuali malformità e si consentirà una manutenzione qualitativamente migliore.


\subsection{Scopo del prodotto}
	L'applicativo che si vuole sviluppare è \textit{Etherless}, una piattaforma cloud\ped{\textit{G}} che sfrutta la tecnologia
	degli smart contract\ped{\textit{G}} caratteristica della rete Ethereum\ped{\textit{G}}. Lo scopo di \textit{Etherless} è duplice: da una
	parte permettere agli \textit{sviluppatori} di rilasciare funzioni Javascript\ped{\textit{G}} nel cloud\ped{\textit{G}}, dall'altra
	permettere agli \textit{utenti} di beneficiare di queste funzioni in seguito ad un pagamento per il loro
	uso. \textit{Etherless} è gestita e mantenuta dai suoi Amministratori.


\subsection{Glossario}
	Al fine di evitare possibili ambiguità, i termini tecnici utilizzati nei documenti formali vengono
	chiariti ed approfonditi nel \textit{Glossario Interno 1.0.0}. Per facilitare la lettura, i termini presenti in
	tale documento sono contrassegnati in tutto il resto della documentazione da una 'G' a pedice.

\subsection{Riferimenti}
	\subsubsection{Riferimenti normativi}
	\begin{itemize}
		\item \textbf{Norme di Progetto}: \textit{Norme di Progetto v1.0.0};
		\item \textbf{Capitolato d'appalto C2 - Etherless}: \\
		\url{https://www.math.unipd.it/~tullio/IS-1/2019/Progetto/C2.pdf}.
	\end{itemize}
	
	\subsubsection{Riferimenti informativi}
	\begin{itemize}
		\item \textbf{Standard ISO/IEC 9126}: \\
			\url{https://it.wikipedia.org/wiki/ISO/IEC_9126};
		\item \textbf{Standard ISO/IEC 15504}: \\
			\url{https://en.wikipedia.org/wiki/ISO/IEC_15504};
		\item \textbf{Ciclo di Deming}: \\
			\url{https://it.wikipedia.org/wiki/Ciclo_di_Deming};
		\item \textbf{Indice di Gulpease}: \\
			\url{https://it.wikipedia.org/wiki/Indice_Gulpease};
		\item \textbf{Slide Qualità di prodotto}: \\
			\url{https://www.math.unipd.it/~tullio/IS-1/2019/Dispense/L12.pdf};
		\item \textbf{Slide Qualità di processo}: \\
			\url{https://www.math.unipd.it/~tullio/IS-1/2019/Dispense/L13.pdf};
		\item \textbf{Slide Verifica e Validazione:
		\begin{itemize}
			\item \url{https://www.math.unipd.it/~tullio/IS-1/2019/Dispense/L14.pdf}; 
			\item \url{https://www.math.unipd.it/~tullio/IS-1/2019/Dispense/L15.pdf}; 
			\item \url{https://www.math.unipd.it/~tullio/IS-1/2019/Dispense/L16.pdf}.
		\end{itemize}}
	\end{itemize}