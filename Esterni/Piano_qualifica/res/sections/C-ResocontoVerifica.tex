\section{Resoconto delle attività di verifica}

	\subsection{Verifiche statiche}
		Ogni documento prodotto è stato analizzato da parte dei \textit{Verificatori}, adottando un metodo Walkthrough\ped{\textit{G}} ed Inspection\ped{\textit{G}}.\\
		Terminata questa analisi, in accordo con il redattore, si procede alla risoluzione di lacune eventualmente presenti.
	
	\subsection{Verifiche requisiti}
		Questo tipo di verifica è necessario per accertarsi che, la relazione tra casi d'uso\ped{\textit{G}}, requisiti e fonti non abbia discrepanze.
	
	\subsection{Verifiche automatizzate}
		Nella seguente tabella vengono riportati i valori di Gulpease calcolati per ogni documento.
		I calcoli sono stati effettuati escludendo le intestazioni e le note a piè di pagina, in modo da avere un risultato valido ed attendibile. L'esito della verifica è da intendersi \textit{Positivo} qualora l'indice di Gulpease abbia valore maggiore di 40.
		
		\rowcolors{2}{lightRowColor}{darkRowColor}
		\begin{longtable}{ 
				>{\centering}p{0.5\textwidth}
				>{\centering}p{0.1\textwidth} 
				>{\centering\arraybackslash}p{0.2\textwidth} }
			
			\caption {Verifica Gulpease documenti}		\\
			
			\coloredTableHead
			\textbf{\color{white}Documento} &
			\textbf{\color{white}Gulpease} &
			\textbf{\color{white}Esito}
			\tabularnewline
			\endhead
			
			% Contenuto della tabella
			% Documento & Gulpease & Esito \\
			\textbf{Analisi dei Requisiti v1.0.0} & 0 & Non Positivo \\
			\textbf{Glossario v1.0.0} & 0 & Non Positivo \\
			\textbf{Norme di Progetto v1.0.0} & 0 & Non Positivo \\
			\textbf{Piano di Progetto v1.0.0} & 0 & Non Positivo \\
			\textbf{Studio di Fattibilità v1.0.0} & 0 & Non Positivo \\
			\textbf{Verbale xyz v1.0.0} & 0 & Non Positivo \\
			\textbf{Verbale xyz v1.0.0} & 0 & Non Positivo \\
			\textbf{Verbale xyz v1.0.0} & 0 & Non Positivo \\
			\textbf{Verbale xyz v1.0.0} & 0 & Non Positivo \\
			\textbf{Verbale xyz v1.0.0} & 0 & Non Positivo \\
			
		\end{longtable}