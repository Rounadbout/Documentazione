I valori registrati per ogni periodo, relativi alle metriche per la Gestione Organizzativa, sono i seguenti:

\rowcolors{2}{lightRowColor}{darkRowColor}
\begin{longtable}{
		>{\centering}p{0.10\textwidth}
		>{\centering}p{0.2\textwidth}
		>{\centering}p{0.2\textwidth}
		>{\centering}p{0.2\textwidth}
		>{}p{0.2\textwidth} }

		\caption{Gestione Organizzativa} \\

	\coloredTableHead
	\textbf{\color{white}Metrica} &
	\textbf{\color{white}Analisi e Consolidamento dei Requisiti} &
	\textbf{\color{white}Progettazione Architetturale} &
	\textbf{\color{white}Progettazione di Dettaglio e Codifica} &
	\textbf{\color{white}Validazione e Collaudo}
	\tabularnewline
	\endhead

	% contenuto tabella
	% esempio: ProvaMetrica & ProvaRR & ProvaRP & ProvaRQ & ProvaRA. \\
	BAC & 15.036,00 & 15.036,00 & 15.036,00 & 15.036,00 \\
	EAC & 15.036,00 & 14.916,00 & 14.913,00 & 15.013,00 \\
	ETC & 15.036,00 & 10.182,00 & 3.181,00 & 0,00 \\
	PV & 0,00\\(0,00) & 4.854,00\\(4.854,00) & 11.855,00 (7.001,00) & 15.036,00 (3.181,00) \\
	AC & 0,00\\(0,00) & 4.734,00\\(4.734,00) & 11.732,00 (6.998,00) & 15.013,00 (3.281,00) \\
	EV & 0,00\\(0,00) & 4.854,00\\(4.854,00) & 11.855,00 (7.001,00) & 15.036,00 (3.181,00) \\
	CV & 0 & 120 & 123 & 23 \\
	SV & 0 & 0 & 0 & 0 \\

\end{longtable}

I valori tra parentesi rappresentano il costo del singolo periodo, mentre gli altri il costo complessivo fino a quel momento.
%*Tali valori sono calcolati allo stato corrente del lavoro, attualmente corrispondente al 15$^{\circ}$ incremento.

	\begin{figure}[H]
	\centering
		\begin{tikzpicture} [scale = 0.6]
			\begin{axis}[
				xlabel={\textbf{Revisione}},
				ylabel={\textbf{Euro \euro}},
				date coordinates in=x,
				ymin=9500,
				ymax=16000,
				xtick=data,
				xticklabel style={
					rotate=90,
					anchor=near xticklabel,
				},
			]
			\addplot table [col sep=comma,x=date,y=value,blue] {res/sections/B/gestione_organizzativa/BAC.csv};
			\addplot table [col sep=comma,x=date,y=value,blue] {res/sections/B/gestione_organizzativa/EAC.csv};
			\legend{$BAC$, $EAC$};
			\end{axis}
		\end{tikzpicture}
		\caption{Confronto tra BAC ed EAC}
	\end{figure}

	\begin{figure}[H]
		\centering
		\begin{tikzpicture} [scale = 0.6]
		\begin{axis}[
		xlabel={\textbf{Revisione}},
		ylabel={\textbf{Euro \euro}},
		date coordinates in=x,
		ymin=0,
		ymax=8000,
		xtick=data,
		xticklabel style={
			rotate=90,
			anchor=near xticklabel,
		},
		]
		\addplot table [col sep=comma,x=date,y=value,blue] {res/sections/B/gestione_organizzativa/PV.csv};
		\addplot table [col sep=comma,x=date,y=value,blue] {res/sections/B/gestione_organizzativa/AC.csv};
		\legend{$PV$, $AC$};
		\end{axis}
		\end{tikzpicture}
		\caption{Confronto tra PV ed AC}
	\end{figure}
