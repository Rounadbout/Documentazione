\subsubsection*{Indice di Gulpease}
	Nel seguente grafico vengono riportati i valori di Gulpease, calcolati per ogni documento, in differenti momenti di maturazione del progetto.
	\begin{figure}[H]
	\centering
		\begin{tikzpicture} [scale = 0.7]
			\begin{axis}[
					xlabel={\textbf{Tempo}},
					ylabel={\textbf{Indice di Gulpease}},
					date coordinates in=x,
					ymin=38,
					ymax=130,
					xtick=data,
					xticklabel style={
						rotate=90,
						anchor=near xticklabel,
					},
					xticklabel=\year-\month-\day,
				]
				\addplot table [col sep=comma,x=date,y=value,blue] {res/sections/B/gulpease/AdR.csv};
				\addplot table [col sep=comma,x=date,y=value,blue] {res/sections/B/gulpease/Glossario.csv};
				\addplot table [col sep=comma,x=date,y=value,blue] {res/sections/B/gulpease/NdP.csv};
				\addplot table [col sep=comma,x=date,y=value,blue] {res/sections/B/gulpease/PdP.csv};
				\addplot table [col sep=comma,x=date,y=value,blue] {res/sections/B/gulpease/PdQ.csv};
				\addplot table [col sep=comma,x=date,y=value,blue] {res/sections/B/gulpease/SdF.csv};		
				\addplot table [col sep=comma,x=date,y=value,blue] {res/sections/B/gulpease/Verbali.csv};
				\draw [line width=0.1, red](2020-4-1, 40)--(2020-6-30, 40);
				\legend{$Analisi\;dei\;Requisiti$ ,$Glossario$ ,$Norme\;di\;Progetto$, $Piano\;di\;Progetto$, $Piano\;di\;Qualifica$, $Studio\;di\;Fattibilita'$, $Media\;dei\;Verbali$};
			\end{axis}
		\end{tikzpicture} 
		\caption{Indice di Gulpease}
	\end{figure}

\subsubsection*{Correttezza ortografica}
	Questa metrica è relativa alla correttezza ortografica all'interno di un documento. Prima di ogni consegna vengono calcolati i risultati, che sono riportati nella seguente tabella:
	\rowcolors{2}{lightRowColor}{darkRowColor}
	\begin{longtable}{
			>{\centering}p{0.4\textwidth}
			>{\centering}p{0.1\textwidth}
			>{\centering}p{0.1\textwidth}
			>{\centering}p{0.1\textwidth}
			>{}p{0.1\textwidth} }
			
		\caption{Correttezza ortografica} \\
		
		\coloredTableHead
		\textbf{\color{white}Documento} &
		\textbf{\color{white}RR} &
		\textbf{\color{white}RP} &
		\textbf{\color{white}RQ} &
		\textbf{\color{white}RA}
		\tabularnewline
		\endhead
		
		% contenuto tabella
		% esempio: Documento & ValoreRR & ValoreRP & ValoreRQ & ValoreRA. \\
		Analisi dei Requisiti & 0 & 0 & 0 & - \\
		Glossario & 0 & 0 & 0 & - \\
		Norme di Progetto & 0 & 0 & 0 & - \\
		Piano di Progetto & 0 & 0 & 0 & - \\
		Piano di Qualifica & 0 & 0 & 0 & - \\
		Studio di Fattibilità & 0 & 0 & 0 & - \\
		Verbali & 0 & 0 & 0 & - \\
		
	\end{longtable}

\subsubsection*{Formula di Flesch}
	Nel seguente grafico vengono riportati i valori ottenuti dalla formula di Flesch, calcolati per ogni documento in lingua inglese, in differenti momenti di maturazione del progetto.
		\begin{figure}[H]
		\centering
		\begin{tikzpicture} [scale = 0.7]
		\begin{axis}[
		xlabel={\textbf{Tempo}},
		ylabel={\textbf{Valore formula di Flesch}},
		date coordinates in=x,
		ymin=30,
		ymax=70,
		xtick=data,
		xticklabel style={
			rotate=90,
			anchor=near xticklabel,
		},
		xticklabel=\year-\month-\day,
		]
		\addplot table [col sep=comma,x=date,y=value,blue] {res/sections/B/flesch/DevMan.csv};
		\addplot table [col sep=comma,x=date,y=value,blue] {res/sections/B/flesch/UserMan.csv};
		\draw [line width=0.1, red](2020-5-20, 50)--(2020-6-30, 50);
		\legend{$Developer\;Manual$, $User\;Manual$};
		\end{axis}
		\end{tikzpicture} 
		\caption{Valori formula di Flesch}
	\end{figure}