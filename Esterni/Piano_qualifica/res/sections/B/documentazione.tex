\paragraph{Indice di Gulpease}
	Nel seguente grafico vengono riportati i valori di Gulpease, calcolati per ogni documento in differenti momenti di maturazione del progetto.
	\begin{center}
		\begin{tikzpicture} [scale = 0.9]
			\begin{axis}[
					xlabel={\textbf{Tempo}},
					ylabel={\textbf{Indice di Gulpease}},
					date coordinates in=x,
					ymin=38,
					ymax=100,
					xtick=data,
					xticklabel style={
						rotate=90,
						anchor=near xticklabel,
					},
					xticklabel=\year-\month-\day,
				]
				\addplot table [col sep=comma,x=date,y=value,blue] {res/sections/B/gulpease/AdR.csv};
				\addplot table [col sep=comma,x=date,y=value,blue] {res/sections/B/gulpease/Glossario.csv};
				\addplot table [col sep=comma,x=date,y=value,blue] {res/sections/B/gulpease/NdP.csv};
				\addplot table [col sep=comma,x=date,y=value,blue] {res/sections/B/gulpease/PdP.csv};
				\addplot table [col sep=comma,x=date,y=value,blue] {res/sections/B/gulpease/PdQ.csv};
				\addplot table [col sep=comma,x=date,y=value,blue] {res/sections/B/gulpease/SdF.csv};		
				\addplot table [col sep=comma,x=date,y=value,blue] {res/sections/B/gulpease/Verbali.csv};
				\draw [line width=0.1, red](2020-4-1, 40)--(2020-5-20, 40);
				\legend{$Analisi dei Requisiti$ ,$Glossario$ ,$Norme di Progetto$, $Piano di Progetto$, $Piano di Qualifica$, $Studio di Fattibilità$, $Media dei Verbali$};
			\end{axis}
		\end{tikzpicture} \\
	\end{center}
	
	
\paragraph{Formula di Flesch}
	Nel seguente grafico vengono riportati i valori ottenuti applicando la formula di Flesch. I risultati vengono messi in relazione a istanti temporali diversi.
	\begin{center}
		\begin{tikzpicture} [scale = 0.9]
			\begin{axis}[
					xlabel={\textbf{Tempo}},
					ylabel={\textbf{Valore di Flesch}},
					date coordinates in=x,
					ymin=48,
					ymax=100,
					xtick=data,
					xticklabel style={
						rotate=90,
						anchor=near xticklabel,
					},
					xticklabel=\year-\month-\day,
				]
				\addplot table [col sep=comma,x=date,y=value,blue] {res/sections/B/flesch/AdR.csv};
				\addplot table [col sep=comma,x=date,y=value,blue] {res/sections/B/flesch/Glossario.csv};
				\addplot table [col sep=comma,x=date,y=value,blue] {res/sections/B/flesch/NdP.csv};
				\addplot table [col sep=comma,x=date,y=value,blue] {res/sections/B/flesch/PdP.csv};
				\addplot table [col sep=comma,x=date,y=value,blue] {res/sections/B/flesch/PdQ.csv};
				\addplot table [col sep=comma,x=date,y=value,blue] {res/sections/B/flesch/SdF.csv};		
				\addplot table [col sep=comma,x=date,y=value,blue] {res/sections/B/flesch/Verbali.csv};
				\draw [line width=0.1, red](2020-4-1, 50)--(2020-5-20, 50);
				\legend{$Analisi dei Requisiti$ ,$Glossario$ ,$Norme di Progetto$, $Piano di Progetto$, $Piano di Qualifica$, $Studio di Fattibilità$, $Media dei Verbali$};
			\end{axis}
		\end{tikzpicture} \\
	\end{center}