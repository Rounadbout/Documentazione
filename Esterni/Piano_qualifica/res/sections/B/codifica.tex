I valori registrati per ogni periodo relativi alle metriche per la codifica sono i seguenti:

\subsubsubsection*{Complessità ciclomatica}
I valori sono riportati secondo lo schema \textit{valore minimo trovato - valore massimo trovato} per ogni modulo\ped{\textit{G}}.
\rowcolors{2}{lightRowColor}{darkRowColor}
\begin{longtable}{
		>{\centering}p{0.15\textwidth}
		>{\centering}p{0.15\textwidth}
		>{\centering}p{0.15\textwidth}
		>{\centering}p{0.15\textwidth}
		>{\centering\arraybackslash}p{0.15\textwidth} }

		\caption{Complessità ciclomatica} \\

	\coloredTableHead
	\textbf{\color{white} Modulo} &
	\textbf{\color{white} RR} &
	\textbf{\color{white} RP} &
	\textbf{\color{white} RQ} &
	\textbf{\color{white}RA}
	\tabularnewline
	\endhead

	% contenuto tabella
	% esempio: Modulo & Min & Max \\
	Etherless-cli & 0 & 3-13 & 3-7 & 3-7 \\
	Etherless-smart & 0 & 2 & 2-4 & 2-4 \\
	Etherless-server & 0 & 3-7 & 1-10 & 1-10 \\

\end{longtable}

\pagebreak
\subsubsubsection*{Rapporto linee di codice per linee di commento (RCC)}
I valori sono calcolati facendo la media della percentuale di commenti presente in ogni file appartenente al modulo\ped{\textit{G}}.
\rowcolors{2}{lightRowColor}{darkRowColor}
\begin{longtable}{
		>{\centering}p{0.15\textwidth}
		>{\centering}p{0.15\textwidth}
		>{\centering}p{0.15\textwidth}
		>{\centering}p{0.15\textwidth}
		>{\centering\arraybackslash}p{0.15\textwidth} }

		\caption{Rapporto linee di codice per linee di commento} \\

	\coloredTableHead
	\textbf{\color{white} Modulo} &
	\textbf{\color{white} RR} &
	\textbf{\color{white} RP} &
	\textbf{\color{white} RQ} &
	\textbf{\color{white} RA}
	\tabularnewline
	\endhead

	% contenuto tabella
	% esempio: Modulo & Min & Max \\
	Etherless-cli & 0\% & 4\% & 10\% & 29\% \\
	Etherless-smart & 0\% & 13\% & 11\% & 37\% \\
	Etherless-server & 0\% & 6\% & 13\% & 40\% \\

\end{longtable}

\subsubsubsection*{Numero di parametri per metodo}
I valori sono riportati secondo lo schema \textit{valore minimo trovato - valore massimo trovato} per ogni modulo\ped{\textit{G}}.
\rowcolors{2}{lightRowColor}{darkRowColor}
\begin{longtable}{
		>{\centering}p{0.15\textwidth}
		>{\centering}p{0.15\textwidth}
		>{\centering}p{0.15\textwidth}
		>{\centering}p{0.15\textwidth}
		>{\centering\arraybackslash}p{0.15\textwidth} }

		\caption{Numero di parametri per metodo} \\

	\coloredTableHead
	\textbf{\color{white} Modulo} &
	\textbf{\color{white} RR} &
	\textbf{\color{white} RP} &
	\textbf{\color{white} RQ} &
	\textbf{\color{white}RA}
	\tabularnewline
	\endhead

	% contenuto tabella
	% esempio: Modulo & Min & Max \\
	Etherless-cli & 0 & 0-1 & 0-3 & 0-3 \\
	Etherless-smart & 0 & 0-2 & 0-5 & 0-4 \\
	Etherless-server & 0 & 0-2 & 0-4 & 0-4 \\

\end{longtable}


\subsubsubsection*{Numero di attributi per classe}
I valori sono riportati secondo lo schema \textit{valore minimo trovato - valore massimo trovato} per ogni modulo\ped{\textit{G}}.
\rowcolors{2}{lightRowColor}{darkRowColor}
\begin{longtable}{
		>{\centering}p{0.15\textwidth}
		>{\centering}p{0.15\textwidth}
		>{\centering}p{0.15\textwidth}
		>{\centering}p{0.15\textwidth}
		>{\centering\arraybackslash}p{0.15\textwidth} }

		\caption{Numero di attributi per classe} \\

	\coloredTableHead
	\textbf{\color{white} Modulo} &
	\textbf{\color{white} RR} &
	\textbf{\color{white} RP} &
	\textbf{\color{white} RQ} &
	\textbf{\color{white}RA}
	\tabularnewline
	\endhead

	% contenuto tabella
	% esempio: Modulo & Min & Max \\
	Etherless-cli & 0 & 1-2 & 1-4 & 1-4 \\
	Etherless-smart & 0 & 5 & 1-7 & 1-7 \\
	Etherless-server & 0 & 0 & 1-4 & 1-4 \\

\end{longtable}
