\paragraph{Percentuale di metriche soddisfatte (PMS)}
	Nel seguente grafico vengono riportati i valori PMS, calcolati in differenti momenti di maturazione del progetto.\\
	Nella prima fase del progetto si può osservare che il numero di metriche soddisfatte può essere inferiore al valore minimo accettabile, in quanto si prendono in analisi anche le metriche da sviluppare in futuro.
	\begin{center}
		\begin{tikzpicture} [scale = 0.9]
		\begin{axis}[
		xlabel={\textbf{Tempo}},
		ylabel={\textbf{Percentuale di metriche soddisfatte}},
		date coordinates in=x,
		ymin=0,
		ymax=100,
		xtick=data,
		xticklabel style={
			rotate=90,
			anchor=near xticklabel,
		},
		xticklabel=\year-\month-\day,
		]
		\addplot table [col sep=comma,x=date,y=value,blue] {res/sections/B/gestione_qualita/PMS.csv};
		\draw [line width=0.1, red](2020-3-05, 60)--(2020-5-20, 60);
		\legend{$PMS$};
		\end{axis}
		\end{tikzpicture} \\
	\end{center}