\paragraph{Percentuale di metriche soddisfatte (PMS)}
	Nel seguente grafico vengono riportati i valori PMS, calcolati in differenti momenti di maturazione del progetto.
	\begin{center}
		\begin{tikzpicture} [scale = 0.9]
		\begin{axis}[
		xlabel={\textbf{Tempo}},
		ylabel={\textbf{Percentuale di metriche soddisfatte}},
		date coordinates in=x,
		ymin=58,
		ymax=100,
		xtick=data,
		xticklabel style={
			rotate=90,
			anchor=near xticklabel,
		},
		xticklabel=\year-\month-\day,
		]
		\addplot table [col sep=comma,x=date,y=value,blue] {res/sections/B/gestione_qualita/PMS.csv};
		\draw [line width=0.1, red](2020-4-1, 60)--(2020-4-20, 60);
		\legend{$PMS$};
		\end{axis}
		\end{tikzpicture} \\
	\end{center}

	\begin{center}
		\textbf{Dati ancora da calcolare}
	\end{center}
		