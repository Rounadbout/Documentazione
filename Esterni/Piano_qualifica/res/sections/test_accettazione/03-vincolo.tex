\subsubsection{Test di vincolo}

\def\arraystretch{1.75}
\rowcolors{2}{lightRowColor}{darkRowColor}
\begin{longtable}{ 
		>{\centering}p{0.1\textwidth} 
		>{}p{0.5\textwidth} 
		>{\centering}p{0.17\textwidth}
		>{\centering}p{0.12\textwidth} }
	
	\caption{Tabella dei test di qualità} \\ 
	\coloredTableHead
	\textbf{\color{white}Test} & 
	\centering\textbf{\color{white}Requisito e Descrizione} & 
	\centering\textbf{\color{white}Implementato} &
	\textbf{\color{white}Superato} 
	\endfirsthead
	
	\rowcolor{white}\caption[]{(continua)}\\
	\coloredTableHead 
	\textbf{\color{white}Test} &
	\centering\textbf{\color{white}Requisito e Descrizione} &
	\centering\textbf{\color{white}Implementato} &
	\textbf{\color{white}Superato} 
	\endhead
	
	% contenuto tabella 
	TA1V1 & \textbf{R1V1}: Gli smart contract\ped{\textit{G}} devono essere scritti in Solidity\ped{\textit{G}} 			& No & No \tabularnewline
	TA1V2 & \textbf{R1V2}: Gli smart contract\ped{\textit{G}} devono poter essere aggiornati 				& No & No \tabularnewline
	TA1V3 & \textbf{R1V3}: L'applicativo deve essere sviluppato utilizzando TypeScript\ped{\textit{G}} 3.6 	& No & No \tabularnewline
	TA1V3.1 & \textbf{R1V3.1}: Deve essere utilizzato il meccanismo delle promise/async-await\ped{\textit{G}} 
			come approccio principale 										& No & No \tabularnewline
	TA1V4 & \textbf{R1V4}: Il modulo\ped{\textit{G}} \textit{Etherless-server} deve essere implementato 
			utilizzando il framework\ped{\textit{G}} Serverless\ped{\textit{G}} 							& No & No \tabularnewline
	TA1V5 & \textbf{R1V5}: Il progetto deve utilizzare i seguenti ambienti di sviluppo: 
			ambiente di sviluppo locale, ambiente di testing e ambiente 
			di staging 														& No & No \tabularnewline
	TA2V5.1 & \textbf{R2V5.1}: Gli ambienti per la fase di sviluppo locale e testing possono 
			fare utilizzo della rete TestRPC\ped{\textit{G}} fornita dal framework\ped{\textit{G}} Truffle\ped{\textit{G}}  & No & No \tabularnewline
	TA2V5.2 & \textbf{R2V5.2}: Per la fase di staging\ped{\textit{G}} è desiderabile l'utilizzo della rete 
			Ethereum\ped{\textit{G}} Ropsten\ped{\textit{G}}												& No & No \tabularnewline
	TA1V5.3 & \textbf{R1V5.3}: Durante la fase di staging\ped{\textit{G}} l'applicativo deve essere 
			pubblicamente accessibile 										& No & No \tabularnewline
	TA1V5.4 & \textbf{R1V5.4}: Al termine del progetto il prodotto deve essere pronto 
			per la produzione 												& No & No \tabularnewline
	TA3V5.4.1 & \textbf{R3V5.4.1}: L'ambiente di produzione deve fare utilizzo dell'Ethereum\ped{\textit{G}}
			main network 													& No & No \tabularnewline
	TA3V6 & \textbf{R3V6}: Il pagamento deve essere gestito tramite un meccanismo di escrow\ped{\textit{G}} & No & No \tabularnewline
	TA1V7 & \textbf{R1V7}: Deve essere possibile installare \textit{Etherless-cli} usando npm\ped{\textit{G}} (node package manager)	& No & No \tabularnewline
		
\end{longtable}