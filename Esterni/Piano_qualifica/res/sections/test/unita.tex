\subsection{Test di Unità}

	\subsubsection{Specifica}
		I Test di Unità hanno l'obiettivo di determinare la correttezza e la completezza, rispetto ai requisiti, di un programma visto come singolo modulo\ped{\textit{G}}.\\
		Per rispettare il livello qualitativo richiesto è necessario adempiere la seguente metrica:
		\begin{itemize}
			\item{misurazione: numero di test soddisfatti;}
			\item{valore minimo accettabile: 100\%;}
			\item{valore preferibile: 100\%.}
		\end{itemize}


	\subsubsection{Stato}
		Questa tipologia di test verrà ulteriormente sviluppata in vista della prossima revisione.
		\subsubsubsection{Test di Unità}
			\def\arraystretch{1.75}
\rowcolors{2}{lightRowColor}{darkRowColor}
\begin{longtable}{
		>{\centering}p{0.12\textwidth}
		>{}p{0.5\textwidth}
		>{\centering}p{0.17\textwidth}
		>{\centering}p{0.12\textwidth} }

	\caption{Tabella dei test di unitá} \\
	\coloredTableHead
	\textbf{\color{white}Id Test} &
	\centering\textbf{\color{white}Descrizione} &
	\centering\textbf{\color{white}Implementato} &
	\textbf{\color{white}Superato}
	\endfirsthead

	\rowcolor{white}\caption[]{(continua)}\\
	\coloredTableHead
	\textbf{\color{white}Id Test} &
	\centering\textbf{\color{white}Descrizione} &
	\centering\textbf{\color{white}Implementato} &
	\textbf{\color{white}Superato}
	\endhead

%CLI
		TU1.01 & Viene verificato il corretto salvataggio del wallet su file. &
		No &
		No \tabularnewline

		TU1.02 &
		Viene verificata la corretta identificazione dello stato di autenticazione dell’utente. &
		No &
		No \tabularnewline

		TU1.03 &
		Viene verificata la corretta creazione di un nuovo wallet. &
		No &
		No \tabularnewline

		TU1.04 &
		Viene verificata la corretta eliminazione delle credenziali salvate. &
		No &
		No \tabularnewline
	
		TU1.05 &
		Viene verificata la corretta decifrazione del wallet salvato. &
		No &
		No \tabularnewline

		TU1.06 &
		Viene verificata la corretta autenticazione tramite private-key. &
		Sì &
		Sì \tabularnewline

		TU1.07 &
		Viene verificata la corretta autenticazione tramite mnemonic-phrase. &
		Sì &
		Sì \tabularnewline

		TU1.08 &
		Viene verificata la corretta restituzione del address relativo all’account dell’utente a seguito di un’azione di login. &
		No &
		No \tabularnewline

		TU1.09 &
		Viene verificata la corretta connessione con un determinato wallet. &
		No &
		No \tabularnewline

		TU1.10 &
		Viene verificata la corretta restituzione della lista delle funzioni presenti all’interno dell’applicativo a livello della classe EtherlessContract. &
		No &
		No \tabularnewline

		TU1.11 &
		Viene verificata la corretta restituzione della lista delle funzioni di proprietà dell’utente a livello della classe EtherlessContract. &
		No &
		No \tabularnewline

		TU1.12 &
		Viene verificata la corretta restituzione dei dettagli di una determinata funzione a livello della classe EtherlessContract. &
		No &
		No \tabularnewline

		TU1.13 &
		Viene verificata la presenza di una funzione all’interno dell’applicativo a livello della classe EtherlessContract. &
		No &
		No \tabularnewline

		TU1.14 &
		Viene verificata la corretta restituzione della lista contenente uno storico delle esecuzioni passate a livello della classe EtherlessContract. &
		No &
		No \tabularnewline

		TU1.15 &
		Viene verificato il corretto invio di una richiesta di esecuzione a livello della classe EtherlessContract. &
		No &
		No \tabularnewline

		TU1.16 &
		Viene verificato il corretto invio di una richiesta di eliminazione a livello della classe EtherlessContract. &
		No &
		No \tabularnewline

		TU1.17 &
		Viene verificato il corretto invio di una richiesta di aggiornamento a livello della classe EtherlessContract. &
		No &
		No \tabularnewline

		TU1.18 &
		Viene verificato il corretto invio di una richiesta di deployment a livello della classe EtherlessContract. &
		No &
		No \tabularnewline

		TU1.19 &
		Viene verificato il corretto ascolto di eventi a livello della classe EtherlessContract. &
		No &
		No \tabularnewline

		TU1.20 &
		Viene verificato il corretto salvataggio di un file in IPFS. &
		Sì &
		Sì \tabularnewline

		TU1.21 &
		Viene verificato il corretto recupero di un file presente in IPFS. &
		Sì &
		Sì \tabularnewline

		TU1.22 &
		Viene verificata la presenza di una funzione con un determinato nome all’interno di un file JavaScript. &
		Sì &
		Sì \tabularnewline

		TU1.23 &
		Viene verificata la corretta restituzione della firma di una funzione all’interno di un file JavaScript. &
		Sì &
		Sì \tabularnewline

		TU1.24 &
		Viene verificata la corretta restituzione della sintassi di un comando CLI. &
		Sì &
		Sì \tabularnewline

		TU1.25 &
		Viene verificata la corretta restituzione della descrizione di un comando CLI. &
		Sì &
		Sì \tabularnewline
		
		
% SMART
TU2.1   &  Verifica l'assenza di funzioni appena fatto il deploy dello smart contract. &
Si & Si \tabularnewline

TU2.2   &  Verifica la corretta emissione dell'evento di richiesta di esecuzione di una funzione. &
Si & Si \tabularnewline

TU2.3   &  Verifica la corretta emissione dell'evento di richiesta di deployment di una funzione. &
Si & Si \tabularnewline

TU2.4   &  Verifica la corretta emissione dell'evento di richiesta di eliminazione di una funzione. &
Si & Si. \tabularnewline

TU2.5   &  Verifica la corretta emissione dell'evento di richiesta di modifica del codice una funzione. &
Si & Si \tabularnewline

TU2.6   &  Verifica la corretta emissione dell'evento di risposta alla richiesta di esecuzione di una funzione. &
Si & Si \tabularnewline

TU2.7   &  Verifica la corretta emissione dell'evento di risposta alla richiesta di di deployment di una funzione. &
Si & Si \tabularnewline

TU2.8   &  Verifica la corretta emissione dell'evento di risposta alla richiesta di eliminazione di una funzione. &
Si & Si \tabularnewline

TU2.9   &  Verifica la corretta emissione dell'evento di risposta alla di modifica del codice una funzione. &
No & No \tabularnewline

TU2.10  &  Verifica la corretta aggiunta della funzione caricata tra le funzioni disponibili. &
Si & Si \tabularnewline

TU2.11  &  Verifica la corretta eliminazione di una funzione, in caso di un fallito deployment. &
No & No \tabularnewline

TU2.12  &  Verifica la corretta eliminazione di una funzione, in caso di una richiesta di delete che ha avuto successo. &
No & No \tabularnewline

TU2.13  &  Verifica che la descrizione della funzione venga aggiornata correttamente. &
No & No \tabularnewline

TU2.14  &  Verifica il corretto incremento dell'id della richiesta. &
Si & Si \tabularnewline

TU2.15  &  Verifica la corretta restituzione della lista di tutte le funzioni aggiunte al sistema. &
No & No \tabularnewline

TU2.16  &  Verifica la corretta restituzione della lista di tutte le funzioni di un proprietario al sistema. &
No & No \tabularnewline

TU2.17  &  Verifica la corretta restituzione dei dettagli di una funzione del sistema. &
Si & Si \tabularnewline

TU2.18  &  Verifica la restituzione del costo di una funzione. &
No & No \tabularnewline

TU2.19  &  Verifica che venga impedita l'aggiunta di una funzione giá esistente. &
No & No \tabularnewline

TU2.20  &  Verifica il corretto inserimento di una funzione. &%storage
Si & Si \tabularnewline

TU2.21  &  Verifica la presenza una funzione nel sistema.	&
Si & Si \tabularnewline

TU2.22  &  Verifica la corretta rimozione di una funzione. &
No & No \tabularnewline

TU2.23  &  Verifica il corretto ritorno del prezzo di una funzione. &
Si & Si \tabularnewline

TU2.24  &  Verifica il corretto ritorno del proprietario di una funzione. &
Si & Si \tabularnewline

TU2.25  &  Verifica il corretto confronto tra due stringhe. &
No & No \tabularnewline

TU2.26  &  Verifica la corretta conversione da tipo uint256 a stringa. &
No & No \tabularnewline

TU2.27  &  Verifica la corretta conversione da tipo address a stringa. &
No & No \tabularnewline

TU2.28  &  Verifica la corretta costruzione della stringa contenente le informazioni di una funzione. &
No & No \tabularnewline

TU2.29  &  Verifica il passaggio di credito dal contratto ad un utente. & %escrow
No & No \tabularnewline

TU2.30  &  Verifica la memorizzazione corretta degli estremi di un pagamento escrow. &
No & No \tabularnewline

TU2.31  &  Verifica il ritorno del beneficiario di un pagamento escrow con un certo id. &
No & No \tabularnewline

TU2.32  &  Verifica il ritorno del pagante di un pagamento escrow con un certo id.  &
Si & S. \tabularnewline

TU2.33  &  Verifica il ritorno del prezzo di un pagamento escrow con un certo id.  &
Si & Si \tabularnewline

TU2.34  &  Verifica il limitatore di accesso per le funzioni di ritorno dei risultati delle richieste.	&
Si & Si \tabularnewline

TU2.35  &  Verifica il limitatore di accesso per il deposito di fondi.	&
Si & Si \tabularnewline

TU2.36  &  Verifica il limitatore di accesso per il recupero di fondi.	&
Si & Si \tabularnewline


% Server
TU3.01 & Viene verificato il corretto inserimento di una callback nell'oggetto EventDispatcher.  & Si & Si \tabularnewline

TU3.02 & Viene verificata la corretta rimozione di una callback dall'oggetto EventDispatcher.  & Si & Si \tabularnewline

TU3.03 & Viene verificato che l'inserimento di una callback già presente nell'oggetto EventDispatcher non comporta un reinserimento della callback stessa.& Si & Si \tabularnewline

TU3.04 & Viene verificato che la richiesta di rimozione di una callback non presente nell'oggetto EventDispatcher viene gestita correttamente. & Si & Si \tabularnewline

TU3.05 & Viene verificata la corretta esecuzione della procedura di dispatch da parte di un oggetto EventDispatcher.  & Si & Si \tabularnewline

TU3.06 & Viene verificato il corretto inserimento di una callback nell'EventDispatcher che gestisce la funzionalità di Run, presente nello SmartManager. & Si & Si \tabularnewline

TU3.07 & Viene verificato il corretto inserimento di una callback nell'EventDispatcher che gestisce la funzionalità di Deploy, presente nello SmartManager. & Si & Si \tabularnewline

TU3.08 & Viene verificato il corretto inserimento di una callback nell'EventDispatcher che gestisce la funzionalità di Delete, presente nello SmartManager. & No & No \tabularnewline

TU3.09 & Viene verificato il corretto inserimento di una callback nell'EventDispatcher che gestisce la funzionalità di Edit, presente nello SmartManager. & No & No \tabularnewline

TU3.10 & Viene verificata la corretta esecuzione della procedura di dispatch in seguito alla ricezione di un evento di tipo Run. & Si & Si \tabularnewline

TU3.11 & Viene verificata la corretta esecuzione della procedura di dispatch in seguito alla ricezione di un evento di tipo Deploy. & Si & Si \tabularnewline

TU3.12 & Viene verificata la corretta esecuzione della procedura di dispatch in seguito alla ricezione di un evento di tipo Delete. & No & No \tabularnewline

TU3.13 & Viene verificata la corretta esecuzione della procedura di dispatch in seguito alla ricezione di un evento di tipo Edit. & No & No \tabularnewline

TU3.14 & Viene verificata la corretta gestione del fallimento della procedura di ritorno del risultato ottenuto dall'elaborazione di un evento di tipo Run.  & Si & Si \tabularnewline

TU3.15 & Viene verificata la corretta gestione del fallimento della procedura di ritorno del risultato ottenuto dall'elaborazione di un evento di tipo Deploy.  & Si & Si \tabularnewline

TU3.16 & Viene verificata la corretta gestione del fallimento della procedura di ritorno del risultato ottenuto dall'elaborazione di un evento di tipo Delete.  & No & No \tabularnewline

TU3.17 & Viene verificata la corretta gestione del fallimento della procedura di ritorno del risultato ottenuto dall'elaborazione di un evento di tipo Edit.  & No & No \tabularnewline

TU3.18 & Viene verificato che la corretta elaborazione di un evento di tipo Run provochi il ritorno del risultato con i dati corretti. & Si & Si \tabularnewline

TU3.19 & Viene verificato che la corretta elaborazione di un evento di tipo Deploy provochi il ritorno del risultato con i dati corretti. & Si & Si \tabularnewline

TU3.20 & Viene verificato che la corretta elaborazione di un evento di tipo Delete provochi il ritorno del risultato con i dati corretti. & No & No \tabularnewline

TU3.21 & Viene verificato che la corretta elaborazione di un evento di tipo Edit provochi il ritorno del risultato con i dati corretti. & No & No \tabularnewline

TU3.22 & Viene verificata la corretta gestione dell'elaborazione di un evento Run fallita a causa di un errore a runtime. & Si & Si \tabularnewline

TU3.23 & Viene verificata la corretta gestione dell'elaborazione di un evento Run fallita a causa di un errore Lambda specifico, identificato da un codice descrittivo. & Si & Si \tabularnewline

TU3.24 & Viene verificata la corretta gestione dell'elaborazione di un evento Run fallita a causa di un errore Lambda sconosciuto. & Si & Si \tabularnewline

TU3.25 & Viene verificata la corretta gestione di un risultato valido ottenuto dall'operazione di fetch da IPFS. & Si & Si \tabularnewline

TU3.26 & Viene verificata la corretta gestione del fallimento dell'operazione di fetch da IPFS. & No & No \tabularnewline

TU3.27 & Viene verificata la corretta gestione di un risultato valido ottenuto dall'operazione di deployment su Lambda. & Si & Si \tabularnewline

TU3.28 & Viene verificata la corretta gestione del fallimento dell'operazione di deployment su Lambda. & Si & Si \tabularnewline

TU3.29 & Viene verificata la corretta gestione di un risultato valido ottenuto dall'operazione di delete su Lambda. & No & No \tabularnewline

TU3.30 & Viene verificata la corretta gestione del fallimento dell'operazione di delete su Lambda. & No & No \tabularnewline

TU3.31 & Viene verificata la corretta gestione di un risultato valido ottenuto dall'operazione di edit su Lambda. & No & No \tabularnewline

TU3.32 & Viene verificata la corretta gestione del fallimento dell'operazione di edit su Lambda. & No & No \tabularnewline


\end{longtable}

			\def\arraystretch{1.75}
\rowcolors{2}{lightRowColor}{darkRowColor}
\begin{longtable}{
		>{\centering}p{0.12\textwidth}
		>{}p{0.88\textwidth}}

	\caption{Tabella di tracciamento dei test - metodi} \\
	\coloredTableHead
	\textbf{\color{white}Id Test} &
	\centering\textbf{\color{white}Metodo} 
	\endfirsthead

	\rowcolor{white}\caption[]{(continua)}\\
	\coloredTableHead
	\textbf{\color{white}Id Test} &
	\centering\textbf{\color{white}Metodo}
	\endhead

	%Prova
	Prova id & Prova metodo \tabularnewline
	TU1.01 & etherless-cli/src/Session/EthereumUserSession.ts:saveWallet(password: string, wallet: Wallet) \\

	TU1.02 & etherless-cli/src/Session/EthereumUserSession.ts:isLogged() \\

	TU1.03 & etherless-cli/src/Session/EthereumUserSession.ts:signup() \\

	TU1.04 & etherless-cli/src/Session/EthereumUserSession.ts:logout() \\

	TU1.05 & etherless-cli/src/Session/EthereumUserSession.ts:restoreWallet(password: string) \\

	TU1.06 & etherless-cli/src/Session/EthereumUserSession.ts:loginWithPrivateKey(privatekey: string, password: string) \\

	TU1.07 & etherless-cli/src/Session/EthereumUserSession.ts:loginWithMnemonicPhrase(mnemonic: string, password: string) \\

	TU1.08 & etherless-cli/src/Session/EthereumUserSession.ts:getAddress() \\

	TU1.09 & etherless-cli/src/EtherlessContract/EthereumContract.ts:connect(wallet: Wallet) \\

	TU1.10 & etherless-cli/src/EtherlessContract/EthereumContract.ts:getAllFunctions() \\

	TU1.11 & etherless-cli/src/EtherlessContract/EthereumContract.ts:getMyFunctions(password: string) \\

	TU1.12 & etherless-cli/src/EtherlessContract/EthereumContract.ts:getFunctionInfo(name: string) \\

	TU1.13 & etherless-cli/src/EtherlessContract/EthereumContract.ts:existsFunction(name: string) \\

	TU1.14 & etherless-cli/src/EtherlessContract/EthereumContract.ts:getExecHistory(address: string) \\

	TU1.15 & etherless-cli/src/EtherlessContract/EthereumContract.ts:sendRunRequest(name: string, params: string) \\

	TU1.16 & etherless-cli/src/EtherlessContract/EthereumContract.ts:sendDeleteRequest(name: string) \\

	TU1.17 & etherless-cli/src/EtherlessContract/EthereumContract.ts:sendCodeUpdateRequest(name: string, filePath: string) \\

	TU1.18 & etherless-cli/src/EtherlessContract/EthereumContract.ts:sendDeployRequest(name: string, signature: string, sourceCID: string, desc. string) \\

	TU1.19 & etherless-cli/src/EtherlessContract/EthereumContract.ts:listenResponse(requestId: BigNumber) \\

	TU1.20 & etherless-cli/src/IPFS/IPFSFileManager.ts:save(buffer: Buffer) \\

	TU1.21 & etherless-cli/src/IPFS/IPFSFileManager.ts:get(cid: string) \\

	TU1.22 & etherless-cli/src/FileParser/JSFileParser.ts:existsFunction(funcName: string) \\

	TU1.23 & etherless-cli/src/FileParser/JSFileParser.ts:getFunctionSignature(funcName: string) \\

	TU1.24 & etherless-cli/src/Command/Command.ts:getCommand() \\

	TU1.25 & etherless-cli/src/Command/Command.ts:getDescription() \\
	% SMART
	TU2.1  &  EtherlessSmart::getFuncList()  \tabularnewline
	TU2.2  &  EtherlessSmart::runFunction(funcName:string, param:string)  \tabularnewline
	TU2.3  &  EtherlessSmart::deployFunction(funcName:string, signature:string, description:string, funcHash:string)  \tabularnewline
	TU2.4  &  EtherlessSmart::deleteFunction(funcName:string)  \tabularnewline
	TU2.5  &  EtherlessSmart::editFunction(funcName:string, funcHash:string)  \tabularnewline
	TU2.6  &  EtherlessSmart::runResult(message:string, id:uint256, successful:bool)  \tabularnewline
	TU2.7  &  EtherlessSmart::deployResult(message:string, name:string, id:uint256, successful:bool)  \tabularnewline
	TU2.8  &  EtherlessSmart::deleteResult(message:string, name:string, id:uint256, successful:bool)  \tabularnewline
	TU2.9  &  EtherlessSmart::editResult(message:string, name:string, id:uint256, successful:bool)  \tabularnewline
	TU2.10 &  EtherlessSmart::deployResult(message:string, name:string, id:uint256, successful:bool)  \tabularnewline
	TU2.11 &  EtherlessSmart::deployResult(message:string, name:string, id:uint256, successful:bool)  \tabularnewline
	TU2.12 &  EtherlessSmart::deleteResult(message:string, name:string, id:uint256, successful:bool)  \tabularnewline
	TU2.13 &  EtherlessSmart::editFunctionDesc(funcName:string, desc:string, funcHash:string)  \tabularnewline
	TU2.14 &  EtherlessSmart::getNewId()  \tabularnewline
	TU2.15 &  EtherlessSmart::getList()  \tabularnewline
	TU2.16 &  EtherlessSmart::getOwnedList(owner:address)  \tabularnewline
	TU2.17 &  EtherlessSmart::getInfo(funcName:string)  \tabularnewline
	TU2.18 &  EtherlessSmart::getCost(funcName:string)  \tabularnewline
	TU2.19 &  EtherlessSmart::deployFunction(funcName:string, signature:string, description:string, funcHash:string)  \tabularnewline
	TU2.20 &  EtherlessStorage::insertNewFunction(name:string, signature:string , price:uint256, dev:address, description:string )  \tabularnewline
	TU2.21 &  EtherlessSmart::existsFunction(fname:string)  \tabularnewline
	TU2.22 &  EtherlessStorage::removeFunction(toRemove:string)  \tabularnewline
	TU2.23 &  EtherlessStorage::getFuncPrice(funcName:string)  \tabularnewline
	TU2.24 &  EtherlessStorage::getFuncDev(funcName:string)  \tabularnewline
	TU2.25 &  EtherlessStorage::compareString(s1:string, s2:string)  \tabularnewline
	TU2.26 &  EtherlessStorage::uintToString(x:uint256)  \tabularnewline
	TU2.27 &  EtherlessStorage::addressToString(addr:address)  \tabularnewline
	TU2.28 &  EtherlessStorage::singleFuncJson(funcName:string, info:bool)  \tabularnewline
	TU2.29 &  EtherlessEscrow::withdraw(payee:string, index:uint256)  \tabularnewline
	TU2.30 &  EtherlessEscrow::deposit(sender:address, beneficiary:address , amount:uint256, index:uint256)  \tabularnewline
	TU2.31 &  EtherlessEscrow::getBeneficiary(index:uint256)  \tabularnewline
	TU2.32 &  EtherlessEscrow::getSender(index:uint256)  \tabularnewline
	TU2.33 &  EtherlessEscrow::depositOf(index:uint256)  \tabularnewline
	TU2.34 &  EtherlessSmart::onlyServer(invokedFrom:address)  \tabularnewline
	TU2.35 &  EtherlessEscrow::onlyOwner  \tabularnewline
	TU2.36 &  EtherlessEscrow::onlyOwner  \tabularnewline

\end{longtable}

