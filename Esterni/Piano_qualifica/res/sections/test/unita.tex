\subsection{Test di Unità}

	\subsubsection{Specifica}
		Il test di unità ha l'obiettivo di determinare la correttezza e la completezza, rispetto ai requisiti, di un programma visto come singolo modulo\ped{\textit{G}}.\\
		Per rispettare il livello qualitativo richiesto è necessario adempiere la seguente metrica:
		\begin{itemize}
			\item{misurazione: numero di test soddisfatti;}
			\item{valore minimo accettabile: 100\%;}
			\item{valore preferibile: 100\%.}
		\end{itemize}
	
	
	\subsubsection{Stato}
		Questa tipologia di test verrà sviluppata in vista delle prossime revisioni.