\subsection{Test di Unità}

	\subsubsection{Specifica}
		I Test di Unità hanno l'obiettivo di determinare la correttezza e la completezza, rispetto ai requisiti, di un programma visto come singolo modulo\ped{\textit{G}}.\\
		Per rispettare il livello qualitativo richiesto è necessario adempiere la seguente metrica:
		\begin{itemize}
			\item{misurazione: numero di test soddisfatti;}
			\item{valore minimo accettabile: 100\%;}
			\item{valore preferibile: 100\%.}
		\end{itemize}


	\subsubsection{Stato}
		\begin{figure}[H]
			\centering
			\begin{tikzpicture} [scale = 0.7]
			\begin{axis}[
			xlabel={\textbf{Tempo}},
			ylabel={\textbf{\% Test di unità}},
			date coordinates in=x,
			ymin=0,
			ymax=102,
			xtick=data,
			xticklabel style={
				rotate=90,
				anchor=near xticklabel,
			},
			xticklabel=\year-\month-\day,
			]
			\addplot table [col sep=comma,x=date,y=value,blue] {res/sections/test/unita/impl.csv};	% al 2020-06-11 sono stati individuati 93 Test
			\addplot table [col sep=comma,x=date,y=value,blue] {res/sections/test/unita/sup.csv};
			\draw [line width=0.1, red](2020-6-1, 100)--(2020-6-30, 100);
			\legend{$Test\;implementati$, $Test\;Superati$};
			\end{axis}
			\end{tikzpicture} 
			\caption{\% Test di unità}
		\end{figure}
		
		\def\arraystretch{1.75}
\rowcolors{2}{lightRowColor}{darkRowColor}
\begin{longtable}{
		>{\centering}p{0.12\textwidth}
		>{}p{0.5\textwidth}
		>{\centering}p{0.17\textwidth}
		>{\centering}p{0.12\textwidth} }

	\caption{Tabella dei test di unitá} \\
	\coloredTableHead
	\textbf{\color{white}Id Test} &
	\centering\textbf{\color{white}Descrizione} &
	\centering\textbf{\color{white}Implementato} &
	\textbf{\color{white}Superato}
	\endfirsthead

	\rowcolor{white}\caption[]{(continua)}\\
	\coloredTableHead
	\textbf{\color{white}Id Test} &
	\centering\textbf{\color{white}Descrizione} &
	\centering\textbf{\color{white}Implementato} &
	\textbf{\color{white}Superato}
	\endhead


	TU3.1 & Viene verificato il corretto inserimento di una callback nell'oggetto EventDispatcher.  & Si & Si \tabularnewline
	TU3.2 & Viene verificata la corretta rimozione di una callback dall'oggetto EventDispatcher.  & Si & Si \tabularnewline
	TU3.3 & Viene verificato che l'inserimento di una callback già presente nell'oggetto EventDispatcher non comporta un reinserimento della callback stessa.& Si & Si \tabularnewline
	TU3.4 & Viene verificato che la richiesta di rimozione di una callback non presente nell'oggetto EventDispatcher viene gestita correttamente. & Si & Si \tabularnewline
	TU3.5 & Viene verificata la corretta esecuzione della procedura di dispatch da parte di un oggetto EventDispatcher.  & Si & Si \tabularnewline
	TU3.6 & Viene verificato il corretto inserimento di una callback nell'EventDispatcher che gestisce la funzionalità di Run, presente nello SmartManager. & Si & Si \tabularnewline
	TU3.7 & Viene verificato il corretto inserimento di una callback nell'EventDispatcher che gestisce la funzionalità di Deploy, presente nello SmartManager. & No & No \tabularnewline
	TU3.8 & Viene verificato il corretto inserimento di una callback nell'EventDispatcher che gestisce la funzionalità di Delete, presente nello SmartManager. & No & No \tabularnewline
	TU3.9 & Viene verificato il corretto inserimento di una callback nell'EventDispatcher che gestisce la funzionalità di Edit, presente nello SmartManager. & No & No \tabularnewline
	TU3.10 & Viene verificata la corretta esecuzione della procedura di dispatch in seguito alla ricezione di un evento di tipo Run. & Si & Si \tabularnewline
	TU3.11 & Viene verificata la corretta esecuzione della procedura di dispatch in seguito alla ricezione di un evento di tipo Deploy. & No & No \tabularnewline
	TU3.12 & Viene verificata la corretta esecuzione della procedura di dispatch in seguito alla ricezione di un evento di tipo Delete. & No & No \tabularnewline
	TU3.13 & Viene verificata la corretta esecuzione della procedura di dispatch in seguito alla ricezione di un evento di tipo Edit. & No & No \tabularnewline
	TU3.14 & Viene verificata la corretta gestione delle situazioni di fallimento che possono accadere durante l'invio di un risultato a Etherless-smart.  & Si & Si \tabularnewline
	TU3.15 & Viene verificata la corretteza del risultato ottenuto dal processamento di un evento di tipo Run. & Si & Si \tabularnewline
	TU3.16 & Viene verificata la corretteza del risultato ottenuto dal processamento di un evento di tipo Deploy. & No & No \tabularnewline
	TU3.17 & Viene verificata la corretteza del risultato ottenuto dal processamento di un evento di tipo Delete. & No & No \tabularnewline
	TU3.18 & Viene verificata la corretteza del risultato ottenuto dal processamento di un evento di tipo Edit. & No & No \tabularnewline
	TU3.19 & Viene verificata la corretta gestione dell'elaborazione di un evento Run fallita a causa di un errore a runtime. & Si & Si \tabularnewline
	TU3.20 & Viene verificata la corretta gestione dell'elaborazione di un evento Run fallita a causa di un errore Lambda specifico, identificato da un codice descrittivo. & Si & Si \tabularnewline
	TU3.21 & Viene verificata la corretta gestione dell'elaborazione di un evento Run fallita a causa di un errore Lambda sconosciuto. & Si & Si \tabularnewline
	TU3.22 & Viene verificato che la corretta esecuzione di una funzione Lambda provochi il ritorno del risultato nel modo corretto. & Si & Si \tabularnewline
	TU3.23 & Viene verificata la corretta gestione di un risultato valido ottenuto dall'operazione di fetch da IPFS & No & No \tabularnewline
	TU3.24 & Viene verificata la corretta gestione del fallimento dell'operazione di fetch da IPFS. & No & No \tabularnewline
	TU3.25 & Viene verificata la corretta gestione di un risultato valido ottenuto dall'operazione di deployment su Lambda & No & No \tabularnewline
	TU3.26 & Viene verificata la corretta gestione del fallimento dell'operazione di deployment su Lambda. & No & No \tabularnewline
	TU3.27 & Viene verificata la corretta gestione di un risultato valido ottenuto dall'operazione di delete su Lambda & No & No \tabularnewline
	TU3.28 & Viene verificata la corretta gestione del fallimento dell'operazione di delete su Lambda. & No & No \tabularnewline
	TU3.29 & Viene verificata la corretta gestione di un risultato valido ottenuto dall'operazione di edit su Lambda & No & No \tabularnewline
	TU3.30 & Viene verificata la corretta gestione del fallimento dell'operazione di edit su Lambda. & No & No \tabularnewline
\end{longtable}

		\def\arraystretch{1.75}
\rowcolors{2}{lightRowColor}{darkRowColor}
\begin{longtable}{
		>{\centering}p{0.12\textwidth}
		>{}p{0.88\textwidth}}

	\caption{Tabella di tracciamento dei test - metodi} \\
	\coloredTableHead
	\textbf{\color{white}Id Test} &
	\centering\textbf{\color{white}Metodo} 
	\endfirsthead

	\rowcolor{white}\caption[]{(continua)}\\
	\coloredTableHead
	\textbf{\color{white}Id Test} &
	\centering\textbf{\color{white}Metodo}
	\endhead

	%Prova
	TU1.01 & EthereumUserSession::saveWallet(password: string, wallet: Wallet) \\

	TU1.02 & EthereumUserSession::isLogged() \\

	TU1.03 & EthereumUserSession::signup() \\

	TU1.04 & EthereumUserSession::logout() \\

	TU1.05 & EthereumUserSession::restoreWallet(password: string) \\

	TU1.06 & EthereumUserSession::loginWithPrivateKey(privatekey: string, password: string) \\

	TU1.07 & EthereumUserSession::loginWithMnemonicPhrase(mnemonic: string, password: string) \\

	TU1.08 & EthereumUserSession::getAddress() \\

	TU1.09 & EthereumContract::connect(wallet: Wallet) \\

	TU1.10 & EthereumContract::getAllFunctions() \\

	TU1.11 & EthereumContract::getMyFunctions(password: string) \\

	TU1.12 & EthereumContract::getFunctionInfo(name: string) \\

	TU1.13 & EthereumContract::existsFunction(name: string) \\

	TU1.14 & EthereumContract::getExecHistory(address: string) \\

	TU1.15 & EthereumContract::sendRunRequest(name: string, params: string) \\

	TU1.16 & EthereumContract::sendDeleteRequest(name: string) \\

	TU1.17 & EthereumContract::sendCodeUpdateRequest(name: string, filePath: string) \\

	TU1.18 & EthereumContract::sendDeployRequest(name: string, signature: string, sourceCID: string, desc. string) \\

	TU1.19 & EthereumContract::listenResponse(requestId: BigNumber) \\

	TU1.20 & IPFSFileManager::save(buffer: Buffer) \\

	TU1.21 & IPFSFileManage::get(cid: string) \\

	TU1.22 & JSFileParser::existsFunction(funcName: string) \\

	TU1.23 & JSFileParser::getFunctionSignature(funcName: string) \\

	TU1.24 & Command::getCommand() \\

	TU1.25 & Command::getDescription() \\
	
	
	% Server
	TU3.01 & EventDispatcher::attach(callback: any) \\
	
	TU3.02 & EventDispatcher::detach(callback: any) \\
	
	TU3.03 & EventDispatcher::attach(callback: any) \\
	
	TU3.04 & EventDispatcher::detach(callback: any) \\
	
	TU3.05 & EventDispatcher::dispatch(data: EventData) \\
	
	TU3.06 & SmartManager:onRun(callback: any) \\
	
	TU3.07 & SmartManager:onDeploy(callback: any) \\
	
	TU3.08 & SmartManager:onDelete(callback: any) \\
	
	TU3.09 & SmartManager:onEdit(callback: any) \\
	
	TU3.10 & EventDispatcher::dispatch(data: RunEventData) \\
	
	TU3.11 & EventDispatcher::dispatch(data: DeployEventData) \\
	
	TU3.12 & EventDispatcher::dispatch(data: DeleteEventData) \\
	
	TU3.13 & EventDispatcher::dispatch(data: EditEventData) \\
	
	TU3.14 & SmartManager::sendRunResponse(message: string, id: BigNumber, success: boolean) \\
	
	TU3.15 & SmartManager::sendDeployResponse(message: string, functionName: string, id: BigNumber, success: boolean) \\
	
	TU3.16 & SmartManager::sendDeleteResponse(message: string, functionName: string, id: BigNumber, success: boolean) \\
	
	TU3.17 & SmartManager::sendEditResponse(message: string, functionName: string, id: BigNumber, success: boolean) \\
	
	TU3.18 & EventProcessor::processRunEvent(data: RunEventData) \\
	
	TU3.19 & EventProcessor::processDeployEvent(data: DeployEventData) \\
	
	TU3.20 & EventProcessor::processDeleteEvent(data: DeleteEventData) \\
	
	TU3.21 & EventProcessor::processEditEvent(data: EditEventData) \\
	
	TU3.22 & AWSManager::invokeLambda(functionName: string, parameters: []) \\
	
	TU3.23 & AWSManager::invokeLambda(functionName: string, parameters: []) \\
	
	TU3.24 & AWSManager::invokeLambda(functionName: string, parameters: []) \\
	
	TU3.25 & IPFSManager::getFileContent(hash: string) \\
	
	TU3.26 & IPFSManager::getFileContent(hash: string) \\
	
	TU3.27 & AWSManager::deploy(functionName: string, fileContent: Buffer) \\
	
	TU3.28 & AWSManager::deploy(functionName: string, fileContent: Buffer) \\
	
	TU3.29 & AWSManager::delete(functionName: string) \\
	
	TU3.30 & AWSManager::delete(functionName: string) \\
	
	TU3.31 & AWSManager::edit(functionName: string, fileContent: Buffer) \\
	
	TU3.32 & AWSManager::edit(functionName: string, fileContent: Buffer) \\
	
\end{longtable}

