\subsection{Test di Integrazione}

	\subsubsection{Specifica}
		I Test di Integrazione hanno l'obiettivo di verificare la correttezza funzionale nell’interazione tra più moduli\ped{\textit{G}}. In particolare questa tipologia di test verifica:
		\begin{enumerate}
			\item{l'assemblamento dei vari moduli\ped{\textit{G}} aggiunti incrementalmente;}
			\item{l'assemblamento di tutti i moduli\ped{\textit{G}} facenti parte del programma.}
		\end{enumerate}
		Per rispettare il livello qualitativo richiesto è necessario adempiere la seguente metrica:
		\begin{itemize}
			\item{misurazione: numero di test soddisfatti;}
			\item{valore minimo accettabile: 100\%;}
			\item{valore preferibile: 100\%.}
		\end{itemize}


	\subsubsection{Stato}
		\def\arraystretch{1.75}
\rowcolors{2}{lightRowColor}{darkRowColor}
\begin{longtable}{
		>{\centering}p{0.12\textwidth}
		>{}p{0.5\textwidth}
		>{\centering}p{0.17\textwidth}
		>{\centering}p{0.12\textwidth} }

	\caption{Tabella dei Test di Integrazione} \\
	\coloredTableHead
	\textbf{\color{white}Test} &
	\centering\textbf{\color{white}Requisito e Descrizione} &
	\centering\textbf{\color{white}Implementato} &
	\textbf{\color{white}Superato}
	\endfirsthead

	\rowcolor{white}\caption[]{(continua)}\\
	\coloredTableHead
	\textbf{\color{white}Test} &
	\centering\textbf{\color{white}Descrizione} &
	\centering\textbf{\color{white}Implementato} &
	\textbf{\color{white}Superato}
	\endhead

  	TI1 & Viene verificata la corretta creazione di un nuovo account Ethereum. & Si & Si \tabularnewline

  	TI2 & Viene verificata la disponibilità di autenticazione con credenziali già esistenti. & Si & Si \tabularnewline

    TI3 & Viene verificata la corretta restituzione dei dettagli di una funzione disponibile nella piattaforma. & Si & Si \tabularnewline

  	TI4 & Viene verificata la corretta restituzione di una lista delle funzioni disponibili in Etherless. & Si & Si \tabularnewline

  	TI5 & Viene verificata la corretta esecuzione di una funzione disponibile in Etherless. & Si & Si \tabularnewline

  	TI6 & Viene verificato il corretto caricamento di una funzione in Etherless. & Si & Si \tabularnewline

  	TI7 & Viene verificata la corretta eliminazione di una funzione da Etherless. & Si & Si \tabularnewline

  	TI8 & Viene verificato il corretto aggiornamento del codice di una funzione presente in Etherless. & Si & Si \tabularnewline

  	TI9 & Viene verificata la  corretta restituzione della cronologia delle operazioni effettuate dall'utente. & Si & Si \tabularnewline
  	
    TI10 & Viene verificata la funzionalità di ricerca di una funzione tramite keyword. & Si & Si \tabularnewline
    
    TI11 & Viene verificata la corretta gestione di un risultato, in risposta ad una richiesta di Run. & Si & Si \tabularnewline
    
    TI12 & Viene verificata la corretta gestione di un'eccezione, in risposta ad una richiesta di Run. & Si & Si \tabularnewline
    
    TI13 & Viene verificata la corretta gestione del fallimento di una richiesta di Run. & Si & Si \tabularnewline
    
    TI14 & Viene verificata la corretta gestione di un risultato, in risposta ad una richiesta di Deploy\ped{\textit{G}}. & Si & Si \tabularnewline
    
    TI15 & Viene verificata la corretta gestione di un'eccezione, in risposta ad una richiesta di Deploy\ped{\textit{G}}. & Si & Si \tabularnewline
    
    TI16 & Viene verificata la corretta gestione di un'eccezione di IPFS\ped{\textit{G}}, in risposta ad una richiesta di Deploy\ped{\textit{G}}. & Si & Si \tabularnewline
    
    TI17 & Viene verificata la corretta gestione di un risultato, in risposta ad una richiesta di Delete. & Si & Si \tabularnewline
    
    TI18 & Viene verificata la corretta gestione di un'eccezione, in risposta ad una richiesta di Delete. & Si & Si \tabularnewline
    
    TI19 & Viene verificata la corretta gestione di un risultato, in risposta ad una richiesta di Edit. & Si & Si \tabularnewline
    
    TI20 & Viene verificata la corretta gestione di un'eccezione, in risposta ad una richiesta di Edit. & Si & Si \tabularnewline

  \end{longtable}

