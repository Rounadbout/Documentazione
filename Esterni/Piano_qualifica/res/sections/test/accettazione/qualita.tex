\def\arraystretch{1.75}
\rowcolors{2}{lightRowColor}{darkRowColor}
\begin{longtable}{ 
		>{\centering}p{0.1\textwidth} 
		>{}p{0.5\textwidth} 
		>{\centering}p{0.17\textwidth}
		>{\centering}p{0.12\textwidth} }
	
	\caption{Tabella dei test di qualità} \\ 
	\coloredTableHead
	\textbf{\color{white}Test} & 
	\centering\textbf{\color{white}Requisito e Descrizione} & 
	\centering\textbf{\color{white}Implementato} &
	\textbf{\color{white}Superato} 
	\endfirsthead
	 
 	\rowcolor{white}\caption[]{(continua)}\\
	\coloredTableHead 
	\textbf{\color{white}Test} &
	\centering\textbf{\color{white}Requisito e Descrizione} &
	\centering\textbf{\color{white}Implementato} &
	\textbf{\color{white}Superato} 
	\endhead
	
	% contenuto tabella
	TA1Q1 & \textbf{R1Q1}: La progettazione e la codifica devono rispettare le norme e le metriche definite nei documenti \textit{Norme di Progetto 2.0.0} e \textit{Piano di Qualifica 1.0.0}.																																									& No & No \tabularnewline
	TA1Q2 & \textbf{R1Q2}: Il sistema deve essere pubblicato con licenza MIT\ped{\textit{G}}. 																				& No & No \tabularnewline
	TA1Q3 & \textbf{R1Q3}: Il codice sorgente di \textit{Etherless} deve essere pubblicato e versionato usando GitHub\ped{\textit{G}} o GitLab\ped{\textit{G}}.				& No & No \tabularnewline
	TA1Q4 & \textbf{R1Q4}: Deve essere redatto un manuale sviluppatore. 																									& No & No \tabularnewline
	TA1Q4.1 & \textbf{R1Q4.1}: Il manuale sviluppatore deve contenere le informazioni per	eseguire e fare il deploy\ped{\textit{G}} dei moduli\ped{\textit{G}}.			& No & No \tabularnewline
	TA1Q5 & \textbf{R1Q5}: Deve essere redatto un manuale utente. 																											& No & No \tabularnewline
	TA1Q5.1 & \textbf{R1Q5.1}: Il manuale utente deve contenere tutte le informazioni	necessarie all'utente finale per utilizzare correttamente	il sistema. 			& No & No \tabularnewline
	TA1Q6 & \textbf{R1Q6}: La documentazione per l'utilizzo del software deve essere scritta in lingua inglese.																& No & No \tabularnewline
	TA1Q7 & R1Q7 Nella scrittura del codice Javascript\ped{\textit{G}} deve essere seguita la guida sullo stile di programmazione Airbnb\ped{\textit{G}} Javascript\ped{\textit{G}}style guide. 																																									& No & No \tabularnewline
	TA1Q8 & \textbf{R1Q8}: Lo sviluppo del codice Javascript\ped{\textit{G}} deve essere supportato dal software di analisi statica del codice ESLint\ped{\textit{G}}.		& No & No \tabularnewline
	TA1Q9 & \textbf{R1Q9}: Deve essere utilizzato il meccanismo delle promise/async-await\ped{\textit{G}} come approccio principale. 										& No & No \tabularnewline
	TA1Q10 & \textbf{R1Q10}: Il progetto deve utilizzare i seguenti ambienti di sviluppo: ambiente di sviluppo locale, ambiente di testing e ambiente di staging. 			& No & No \tabularnewline
	TA1Q10.1 & \textbf{R1Q10.1}: Gli ambienti per la fase di sviluppo locale e testing possono fare utilizzo della rete TestRPC\ped{\textit{G}} fornita dal framework\ped{\textit{G}} Truffle\ped{\textit{G}}. 																																				& No & No \tabularnewline
	TA1Q10.2 & \textbf{R1Q10.2}: Per la fase di staging\ped{\textit{G}} è desiderabile l'utilizzo della rete Ethereum\ped{\textit{G}} Ropsten\ped{\textit{G}}.				& No & No \tabularnewline
	TA1Q10.3 & \textbf{R1Q10.3}: Durante la fase di staging\ped{\textit{G}} l'applicativo deve essere pubblicamente accessibile. 											& No & No \tabularnewline
	TA1Q10.4 & \textbf{R1Q10.4}: Al termine del progetto il prodotto\ped{\textit{G}} deve essere pronto per la produzione. 																	& No & No \tabularnewline
	TA1Q10.4.1 & \textbf{R1Q10.4.1}: L'ambiente di produzione deve fare utilizzo dell'Ethereum\ped{\textit{G}} main network. 												& No & No \tabularnewline

\end{longtable}

