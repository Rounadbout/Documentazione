\section{Valutazioni per il miglioramento}
	Questa sezione riporta i problemi riscontrati dal gruppo \textit{Roundabout} durante il corso del progetto. Ogni problema viene valutato per trovare una possibile soluzione e quindi un miglioramento il più efficace ed efficiente possibile. \\
	Si espongono di seguito i problemi incontrati divisi in 3 raggruppamenti:
	\begin{itemize}
		\item \textbf{organizzazione}: problemi relativi all’organizzazione e la comunicazione all’interno del gruppo;
		\item \textbf{ruoli}: problemi relativi allo svolgimento dei diversi ruoli;
		\item \textbf{strumenti di lavoro}: problemi relativi l’uso degli strumenti utilizzati.
	\end{itemize}

\subsection{Valutazioni sull'organizzazione}
	\rowcolors{2}{lightRowColor}{darkRowColor}
	\begin{longtable}{ 
		>{\centering}p{0.4\textwidth} 
		>{\centering\arraybackslash}p{0.5\textwidth}}
	
		\caption {Valutazioni Organizzazione}		\\
		
		\coloredTableHead
		\textbf{\color{white}Problema} &
		\textbf{\color{white}Soluzione}
		\tabularnewline  
		\endhead
		
		% Contenuto della tabella
		% Problema & Soluzione\\
		\textbf{Riunioni Interne}: si è rivelato un problema organizzativo l'impossibilità di vedersi fisicamente a causa della situazione di emergenza COVID-19\ped{\textit{G}} & Abbiamo concordato di utilizzare maggiormente strumenti di collaborazione che consentono, oltre alla possibilità di effettuare videochiamate, una comunicazione semplificata per i diversi problemi che si possono verificare. \\
		
		\textbf{Appuntamenti}: Problema a definire una calendarizzazione degli incontri tra i vari membri del gruppo & Abbiamo definito che le riunioni interne saranno effettuate cadenzialmente due volte alla settimana il martedì e il venerdì, salvo esigenze particolari.\\
		
		\textbf{Riunioni Esterne}: Durante la prima riunione effettuata con il \textit{Proponente}\ped{\textit{G}} a mezzo Skype\ped{\textit{G}}, si è valutato il problema comune di connessione instabile e conseguente perdita di parole durante la conversazione. & Risolto proponendo al \textit{Proponente}\ped{\textit{G}}	incontri telematici su piattaforma Zoom\ped{\textit{G}}, molto più leggera e con limitati problemi di chiamata.\\
				
		\end{longtable}

\subsection{Valutazioni sui ruoli}
	\rowcolors{2}{lightRowColor}{darkRowColor}
	\begin{longtable}{ 
			>{\centering}p{0.4\textwidth} 
			>{\centering\arraybackslash}p{0.5\textwidth} }
		
		\caption {Valutazioni Ruoli}		\\
		
		\coloredTableHead
		\textbf{\color{white}Problema} &
		\textbf{\color{white}Soluzione}
		\tabularnewline  
		\endhead
		
		% Contenuto della tabella
		% Problema & Soluzione\\
		\textbf{Rivestire un ruolo}: Il problema comune a tutti i ruoli è stato quello di doversi adattare ad una mentalità diversa in base al contesto richiesto, considerato il vincolo che ogni membro dovrà ricoprire un ruolo descritto nelle \textit{Norme di Progetto}. & Valutato che il maggior impatto di questa problematica si verifica nella fase iniziale di ogni cambio di ruolo, si è deciso di limitare le rotazioni indicativamente ogni due settimane cercando di non lasciare lavori in sospeso al membro successivo. In ogni caso vige il buon senso e la collaborazione reciproca. \\
		
	\end{longtable}


\subsection{Valutazioni sugli strumenti di lavoro}
	\rowcolors{2}{lightRowColor}{darkRowColor}
	\begin{longtable}{ 
			>{\centering}p{0.4\textwidth} 
			>{\centering\arraybackslash}p{0.5\textwidth} }
		
		\caption {Valutazioni Strumenti di Lavoro}		\\
		
		\coloredTableHead
		\textbf{\color{white}Problema} &
		\textbf{\color{white}Soluzione}
		\tabularnewline  
		\endhead
		
		% Contenuto della tabella
		% Problema & Soluzione\\
		\textbf{\LaTeX{}}: si è rivelato un problema l'utilizzo di questo strumento, in quanto la maggior parte del gruppo \textit{Roundabout} non lo aveva mai utilizzato prima. & La soluzione è stata quella di usufruire dell'esperienza maturata da parte di alcuni membri del gruppo per apprendere le basi di utilizzo: prima creando un template standard, poi illustrandolo assieme ad alcuni comandi che avremmo utilizzato con maggiore frequenza. \\
		
		\textbf{Ethereum\ped{\textit{G}}}: si è rivelato un problema la non conoscenza di questa piattaforma & Si è colmata questa mancanza tramite ricerca personale e studio autonomo. \\
		
		\textbf{Omogeneità dei documenti prodotti}: Considerato che la stesura di un documento può essere effettuata anche da più persone che ricoprono lo stesso ruolo in contemporanea, si è verificato il problema di omogeneità all'interno dei documenti & La soluzione migliore è stata quella di concordare assieme nelle \textit{Norme di Progetto} gli utilizzi di maiuscole, minuscole, corsivo, grassetto, etc. \\
		
	\end{longtable}
