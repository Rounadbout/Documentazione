\section*{Registro delle modifiche} % con section* evito che la sezione sia considerata nell'indice

\rowcolors{2}{lightRowColor}{darkRowColor}
\begin{longtable}{
		>{\centering}p{0.1\textwidth}
		>{\centering}p{0.15\textwidth}
		>{\centering}p{0.2\textwidth}
		>{\centering}p{0.15\textwidth}
		>{}p{0.26\textwidth} }

	\coloredTableHead
	\textbf{\color{white}Versione} &
	\textbf{\color{white}Data} &
	\textbf{\color{white}Nominativo} &
	\textbf{\color{white}Ruolo} &
	\textbf{\color{white}Descrizione}
	\tabularnewline
	\endhead

	% contenuto tabella
	% esempio: ProvaVersione & ProvaData & ProvaPersona & \textit{ProvaRuolo} & ProvaAzione. \\
	0.0.7 & 2020-04-02 & \LB{} & \textit{Verificatore} & Stesura \textsection{4} e \textsection{5}. \\
	0.0.6 & 2020-04-01 & \LB{} & \textit{Verificatore} & Stesura \textsection{C}. \\
	0.0.5 & 2020-04-01 & \NF{} & \textit{Progettista} & Stesura \textsection 2.1, \textsection2.2, \textsection2.3. \\
	0.0.4 & 2020-03-30 & \LB{} & \textit{Verificatore} & Stesura \textsection{B}. \\
	0.0.3 & 2020-03-26 & \LB{} & \textit{Verificatore} & Stesura \textsection1. \\
	0.0.2 & 2020-03-21 & \NF{} & \textit{Progettista} & Organizzazione struttura documento. \\
   	0.0.1 & 2020-03-20 & \LB{} & \textit{Amministratore} & Creazione documento \LaTeX{}.

\end{longtable}
