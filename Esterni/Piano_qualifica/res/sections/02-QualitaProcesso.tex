\section{Qualità di Processo}

\subsection{Scopo}
Lo scopo della seguente sezione è quello di illustrare le metriche adottate per garantire qualità nello svolgimento dei processi precedentemente scelti tra quelli proposti dallo standard ISO/IEC 12207:1995 ed illustrati nel documento \textit{\NdP}. Lo standard scelto per attuare la valutazione dei processi e garantirne la qualità è ISO/IEC 15504, con particolare attenzione all'applicazione del metodo di gestione PDCA al fine di ricercare un miglioramento continuo nel corso di tutto il progetto didattico.

\subsection{Processi Primari}

	\subsubsection{Processo di Sviluppo}
	
		\subsubsubsection{Analisi dei Requisiti}
		Le metriche usate per l'attività di analisi sono le seguenti:
			\subsubsubsection*{Percentuale dei Requisiti Obbligatori Soddisfatti (PROS)}
			È un valore percentuale che indica la quantità di requisiti obbligatori adempiti nel corso del progetto.
			 \begin{itemize}
			\item{Misurazione: PROS = $\displaystyle\frac{\mbox{requisiti obbligatori soddisfatti}}{\mbox{requisiti obbligatori totali}}$;}
			\item{Valore minimo accettabile: 100\%;}
			\item{Valore preferibile: 100\%.}
		\end{itemize}

		\subsubsubsection{Progettazione}
		Le metriche usate per l'attività di progettazione sono le seguenti:
		
			\subsubsubsection*{Structural Fan-In (SFIN)}
			È un valore intero che indica quante componenti utilizzano un dato modulo. Uno \textit{SFIN} alto sta a significare un consistente riuso della componente.
		\begin{itemize}
			\item{Misurazione: valore intero che indica il conteggio delle componenti;}
			\item{Valore minimo accettabile: $\geq$ 0;}
			\item{Valore preferibile: $\geq$ 1.}
		\end{itemize}
	
			\subsubsubsection*{Structural Fan-Out (SFOUT)}
			È un valore intero che indica quante componenti vengono utilizzate dalla componente in questione. Uno \textit{SFOUT} alto sta a significare un significativo accoppiamento della componente.
		\begin{itemize}
			\item{Misurazione: valore intero che indica il conteggio delle componenti;}
			\item{Valore minimo accettabile: = 0;}
			\item{Valore preferibile: $\leq$ 6.}
		\end{itemize}
		
		\subsubsubsection{Codifica}
		Le metriche usate per l'attività di codifica sono le seguenti:
		
			\subsubsubsection*{Coupling Between Objects (CBO)}
			È un valore intero che indica il grado di accoppiamento tra le classi di oggetti. Una classe A si dice \textit{accoppiata} ad una classe B se A usa metodi o variabili definite in B.
		\begin{itemize}
			\item{Misurazione: valore intero;}
			\item{Valore minimo accettabile: 0 $\leq$ CBO $\leq$ 6;}
			\item{Valore preferibile: 0 $\leq$ CBO $\leq$ 1.}
		\end{itemize}
			
			\subsubsubsection*{Complessità ciclomatica}
			È una metrica utilizzata per stimare la complessità di funzioni, moduli\ped{\textit{G}}, metodi o classi di un programma. Ciò viene fatto mediante la determinazione del numero dei cammini linearmente indipendenti attraverso il grafo di controllo di flusso. Un valore troppo elevato indica un'eccessiva complessità del codice che rende conseguentemente l'attività di manutenzione più difficile. Allo stesso tempo, un valore eccessivamente basso potrebbe essere indicatore di una scarsa efficienza dei metodi.
		\begin{itemize}
			\item{Misurazione: valore intero;}
			\item{Valore minimo accettabile: 1 $\leq$ complessità ciclomatica $\leq$ 15;}
			\item{Valore preferibile: 1 $\leq$ complessità ciclomatica $\leq$ 10.}
		\end{itemize}
			
			\subsubsubsection*{Rapporto linee di codice per linee di commento}
			Indica il rapporto tra le righe di codice e le righe di commento ad esso corrispondenti, escludendo le righe vuote, al fine di stimare il livello di difficoltà di manutenibilità del codice. Un rapporto troppo basso, infatti, potrebbe essere indicatore di scarse informazioni necessarie alla comprensione del codice scritto.
		\begin{itemize}
			\item{Misurazione: valore decimale;}
			\item{Valore minimo accettabile: $\displaystyle\frac{\mbox{linee di codice}}{\mbox{linee di commento}} \geq$ 0.25;}
			\item{Valore preferibile: $\displaystyle\frac{\mbox{linee di codice}}{\mbox{linee di commento}} \geq$ 0.30.}
		\end{itemize}

			\subsubsubsection*{Livello di annidamento}
			È un valore intero che indica il livello di annidamento nei vari metodi tenendo conto della presenza di strutture di controllo annidate: un alto livello di annidamento risulta in codice complesso e di difficile manutenzione.
		\begin{itemize}
			\item{Misurazione: valore intero;}
			\item{Valore minimo accettabile: 1 $\leq$ livello annidamento $\leq$ 7;}
			\item{Valore preferibile: 1 $\leq$ livello annidamento $\leq$ 3.}
		\end{itemize}
			
			\subsubsubsection*{Numero di parametri per metodo}
			È un numero intero che indica il numero di parametri di un metodo. Un valore troppo elevato denota un grado di complessità eccessivamente elevato del metodo
		\begin{itemize}
			\item{Misurazione: valore intero;}
			\item{Valore minimo accettabile: 0 $\leq$ numero totale attributi $\leq$ 8;}
			\item{Valore preferibile: 0 $\leq$ numero totale attributi $\leq$ 4.}
		\end{itemize}
			
			\subsubsubsection*{Numero di attributi per classe}
			È un numero intero che indica il numero totale di attributi presenti all'interno di una classe. Un valore troppo elevato denota un carico di responsabilità eccessiva per la classe, suggerendo di scomporre la stessa seguendo principi quali il \textit{Single Responsibility Principle}.
		\begin{itemize}
			\item{Misurazione: valore intero;}
			\item{Valore minimo accettabile: 1 $\leq$ numero totale attributi $\leq$ 15;}
			\item{Valore preferibile: 1 $\leq$ numero totale attributi $\leq$ 8.}
		\end{itemize}

\subsection{Processi di Supporto}

	\subsubsubsection{Documentazione}
	Le metriche usate per l'attività di documentazione sono le seguenti:
	
		\subsubsubsection*{Indice di Gulpease}
		È un indice che valuta la leggibilità del testo. È tarato sulla lingua italiana e considera la lunghezza delle parole, il numero delle frasi ed il numero delle parole totali. Il valore risultante è compreso tra 0 e 100, dove un valore di indice più alto corrisponde ad un indice di leggibilità più semplice.
		 \begin{itemize}
			\item{Misurazione: G = 89 + $\displaystyle\frac{300 \cdot{} (\mbox{numero delle frasi}) - 10 \cdot{} (\mbox{numero delle lettere})}{\mbox{numero delle parole}}$;}
			\item{Valore minimo accettabile: $\geq$ 40;}
			\item{Valore preferibile: $\geq$ 60.}
		\end{itemize}

		\subsubsubsection*{Correttezza ortografica}
		Tutti i documenti devono essere privi di errori grammaticali od ortografici.
		\begin{itemize}
			\item{Misurazione: numero intero che indica il numero di errori presenti nel testo;}
			\item{Valore minimo accettabile: 0;}
			\item{Valore preferibile: 0.}
		\end{itemize}
		
		\subsubsubsection*{Formula di Flesch}
		E' un indice che valuta la leggibilità di un testo in lingua inglese. Più questo indice è alto e più il testo risulta semplice da leggere.
		 \begin{itemize}
			\item{Misurazione: F = 206,835 - (84,6 $\cdot$ numero medio di sillabe per parola) - (1,015 $\cdot$ numero medio di parole per frase);}
			\item{Valore minimo accettabile: $\geq$ 50;}
			\item{Valore preferibile: $\geq$ 60.}
		\end{itemize}

					
	\subsubsubsection{Verifica}
	Le metriche usate per l'attività di verifica sono le seguenti:
	
		\subsubsubsection*{Code Coverage}
		 È la percentuale di linee di codice che sono state eseguite dai test dopo un’esecuzione.
		 \begin{itemize}
			\item{Misurazione: CC = $\displaystyle\frac{\mbox{linee di codice eseguite dal test}}{\mbox{linee di codice totali}}$;}
			\item{Valore minimo accettabile: 80\%;}
			\item{Valore preferibile: 100\%.}
		\end{itemize}

\subsection{Processi Organizzativi}

	\subsubsubsection{Gestione Organizzativa}
	Le metriche usate per la gestione organizzativa sono le seguenti:
	
		\subsubsubsection*{Budget at Completion (BAC)}
		Equivale al budget allocato inizialmente per il progetto.
		\begin{itemize}
			\item{Misurazione: numero intero;}
			\item{Valore minimo accettabile: valore del preventivo con un errore massimo del 5\%, ovvero \textit{preventivo}-5\% $\leq$ \textit{BAC} $\leq$ \textit{preventivo}+5\%;}
			\item{Valore preferibile: pari al preventivo.}
		\end{itemize}
		
		\subsubsubsection*{Estimated at Completion (EAC)}
		Equivale al BAC rivisto allo stato corrente del progetto, ovvero è la somma del costo sostenuto fino a quel momento e la stima del costo ancora da sostenere.
		\begin{itemize}
			\item{Misurazione: EAC = AC + ETC;}
			\item{Valore minimo accettabile: valore del preventivo con un errore massimo del 5\%, ovvero \textit{preventivo}-5\% $\leq$ \textit{BAC} $\leq$ \textit{preventivo}+5\%;}
			\item{Valore preferibile: pari al preventivo.}
		\end{itemize}
		
		\subsubsubsection*{Estimate to Complete (ETC)}
		Valore stimato per la realizzazione delle rimanenti attività necessarie al completamento del progetto.
		\begin{itemize}
			\item{Misurazione: numero intero;}
			\item{Valore minimo accettabile: $\leq$ preventivo;}
			\item{Valore preferibile: < preventivo.}
		\end{itemize}
		
		\subsubsubsection*{Planned Value (PV)}
		Corrisponde al costo pianificato per realizzare le attività di progetto fino a quel momento.
		\begin{itemize}
			\item{Misurazione: \% di lavoro pianificato $\cdot$ BAC;}
			\item{Valore minimo accettabile: $\geq$ 0;}
			\item{Valore preferibile: $\geq$ 0.}
		\end{itemize}
		
		\subsubsubsection*{Actual Cost (AC)}
		Indica la quantità di budget spesa al momento del calcolo.
		\begin{itemize}
			\item{Misurazione: numero intero;}
			\item{Valore minimo accettabile: 0 $\leq$ \textit{AC} < \textit{BAC};}
			\item{Valore preferibile: 0 $\leq$ \textit{AC} < \textit{PV}.}
		\end{itemize}
		
		\subsubsubsection*{Earned Value (EV)}
		Indica la quantità di lavoro compiuta al momento del calcolo, ovvero il valore prodotto dal progetto alla data corrente.
		\begin{itemize}
			\item{Misurazione: EV =  \% di lavoro completato $\cdot$ BAC;}
			\item{Valore minimo accettabile: $\geq$ 0;}
			\item{Valore preferibile: $\geq$ 0;}
		\end{itemize}

		\subsubsubsection*{Cost Variance (CV)}
		Misura l'andamento del budget nel corso di un progetto software. In particolare, la \textit{Cost Variance} è la differenza tra \textit{Earned Value} e \textit{Actual Cost}, ovvero tra ciò che si aveva pianificato di spendere e ciò che si è effettivamente speso nel corso del progetto.
		Se la \textit{Cost Variance} ha valore negativo significa che si è \textit{over budget}, se è nulla si è \textit{on budget}, mentre se è positiva si è \textit{under budget}.
		\begin{itemize}
			\item{Misurazione: CV = EV - AC;}
			\item{Valore minimo accettabile: 0;}
			\item{Valore preferibile: > 0;}
		\end{itemize}
		
		\subsubsubsection*{Schedule Variance (SV)}
		Indica lo stato di avanzamento di un progetto software rispetto alla schedulazione delle attività e viene calcolata mediante la differenza tra \textit{Earned Value} e \textit{Planned Value}.
		Se la \textit{Schedule Variance} ha valore negativo significa che lo stato di avanzamento del progetto è in ritardo rispetto alla pianificazione, se è nulla lo stato di avanzamento del progetto è nei tempi previsti, mentre se è positiva significa che si è in anticipo rispetto alla pianificazione.
		\begin{itemize}
			\item{Misurazione: SV = EV - PV;}
			\item{Valore minimo accettabile: 0;}
			\item{Valore preferibile: > 0.}
		\end{itemize}
						
		\subsubsubsection*{Correlazione tra CV e SV}
		Lo stato di un progetto è esprimibile dalla correlazione tra \textit{Cost Variance} e \textit{Schedule Variance}, in particolare:
		\begin{enumerate}
			\item{\textbf{SV e CV positive}: il progetto è in anticipo rispetto alla pianificazione e rientra nel budget previsto;}
			\item{\textbf{SV positiva, CV negativa}: il progetto è in anticipo rispetto alla pianificazione ma ha superato il budget allocato;}
			\item{\textbf{SV negativa, CV positiva}: il progetto è in ritardo rispetto alla pianificazione ma rientra nel budget previsto;}
			\item{\textbf{SV e CV negative}: il progetto è in ritardo rispetto alla pianificazione e ha superato il budget previsto.}
		\end{enumerate}

	\subsubsubsection{Gestione della Qualità}
	Le metriche usate per la gestione dell qualità sono le seguenti:
	
		\subsubsubsection*{Percentuale di metriche soddisfatte (PMS)}
		E' un valore percentuale che indica quante metriche raggiungono soglie accettabili sul totale delle metriche considerate. Un valore percentuale basso potrebbe essere indicatore di scarsa qualità, incorrettezza di calcolo o scelta metriche poco adeguate.
		\begin{itemize}
			\item{PMS = $\displaystyle\frac{\mbox{numero di metriche soddisfatte}}{\mbox{numero totale di metriche}}$}
			\item{Valore minimo accettabile: 60\%;}
			\item{Valore preferibile: 90\%.}
		\end{itemize}
