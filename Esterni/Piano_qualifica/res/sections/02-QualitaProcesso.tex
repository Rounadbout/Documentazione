\section{Qualità di Processo}

\subsection{Scopo}
Lo scopo della seguente sezione è quello di elencare le metriche adottate dal team \Gruppo{} per valutare la maturità dei processi descritti nel documento \textit{\NdP} e garantire qualità nel loro svolgimento. Lo standard scelto per perseguire tale scopo è ISO/IEC 15504. Inoltre, viene posta particolare attenzione sull'applicazione del metodo di gestione PDCA per ricercare un miglioramento continuo nel corso di tutto il progetto didattico.

\subsubsection{Struttura}
I valori di riferimento delle metriche illustrate nel seguente documento presentano la struttura:
\subsubsubsection*{Nome metrica}
\begin{itemize}
	\item{\textbf{Descrizione}: breve descrizione circa lo scopo della metrica;}
	\item{\textbf{misurazione}: valore mediante il quale viene espressa la metrica o -se presente- formula per calcolarla;}
	\item{\textbf{valore minimo accettabile}: limite inferiore sul valore minimo che la metrica deve assumere per ritenersi soddisfatta;}
	\item{\textbf{valore preferibile}: valore della metrica consigliato (ma non vincolante) per il perseguimento della qualità.}
\end{itemize}
La descrizione di tali metriche è invece consultabile nel documento \NdP.

\subsection{Processi Primari}

	\subsubsection{Processo di Sviluppo}
	
		\subsubsubsection{Analisi dei Requisiti}
		I valori di riferimento delle metriche usate per l'attività di analisi sono le seguenti:
			\subsubsubsection*{Percentuale dei Requisiti Obbligatori Soddisfatti (PROS)}
				\begin{itemize}
					\item{misurazione: $PROS = \displaystyle\frac{\#requisiti\_obbligatori\_soddisfatti}{\#requisti\_obbligatori\_totali} \times 100$;}
					\item{valore minimo accettabile: 100\%;}
					\item{valore preferibile: 100\%.}
				\end{itemize}

		\subsubsubsection{Progettazione}
		I valori di riferimento delle metriche usate per l'attività di progettazione sono le seguenti:
		
			\subsubsubsection*{Coupling Between Objects (CBO)}
				\begin{itemize}
					\item{misurazione: valore intero;}
					\item{valore minimo accettabile: 0 $\leq$ CBO $\leq$ 6;}
					\item{valore preferibile: 0 $\leq$ CBO $\leq$ 1.}
				\end{itemize}

		
			\subsubsubsection*{Structural Fan-In (SFIN)}
				\begin{itemize}
					\item{misurazione: valore intero;}
					\item{valore minimo accettabile: $\geq$ 0;}
					\item{valore preferibile: $\geq$ 1.}
				\end{itemize}
	
			\subsubsubsection*{Structural Fan-Out (SFOUT)}
				\begin{itemize}
					\item{misurazione: valore intero;}
					\item{valore minimo accettabile: = 0;}
					\item{valore preferibile: $\leq$ 6.}
				\end{itemize}
		
		\subsubsubsection{Codifica}
		I valori di riferimento delle metriche usate per l'attività di codifica sono le seguenti:
					
		\subsubsubsection*{Complessità ciclomatica}
			\begin{itemize}
				\item{misurazione: $v(G) = e - n + 2p$, dove: 
				\begin{itemize}
					\item{v(G)}: complessità ciclomatica del grafo G; 
					\item{e}: numero di archi del grafo; 
					\item{n}: numero di nodi del grafo; 
					\item{p}: numero di componenti connesse. 
				\end{itemize} }
				\item{valore minimo accettabile: 1 $\leq$ complessità ciclomatica $\leq$ 15;}
				\item{valore preferibile: 1 $\leq$ complessità ciclomatica $\leq$ 10.}
			\end{itemize}
			
		\subsubsubsection*{Rapporto linee di codice per linee di commento (RCC)}
			\begin{itemize}
				\item{misurazione: $RCC = \displaystyle\frac{\#linee\_totali}{\#linee\_di\_commento} $;}
				\item{valore minimo accettabile: RCC $\geq$ 0.25;}
				\item{valore preferibile: RCC $\geq$ 0.30.}
			\end{itemize}

		\subsubsubsection*{Livello di annidamento}
			\begin{itemize}
				\item{misurazione: valore intero;}
				\item{valore minimo accettabile: 1 $\leq$ livello annidamento $\leq$ 7;}
				\item{valore preferibile: 1 $\leq$ livello annidamento $\leq$ 3.}
			\end{itemize}
			
		\subsubsubsection*{Numero di parametri per metodo}
			\begin{itemize}
				\item{misurazione: valore intero;}
				\item{valore minimo accettabile: 0 $\leq$ numero totale attributi $\leq$ 8;}
				\item{valore preferibile: 0 $\leq$ numero totale attributi $\leq$ 4.}
			\end{itemize}
			
		\subsubsubsection*{Numero di attributi per classe}
			\begin{itemize}
				\item{misurazione: valore intero;}
				\item{valore minimo accettabile: 1 $\leq$ numero totale attributi $\leq$ 15;}
				\item{valore preferibile: 1 $\leq$ numero totale attributi $\leq$ 8.}
			\end{itemize}

\subsection{Processi di Supporto}

	\subsubsubsection{Documentazione}
	I valori di riferimento delle metriche usate per l'attività di documentazione sono le seguenti:
	
		\subsubsubsection*{Indice di Gulpease}
			 \begin{itemize}
				\item{misurazione: $ IG=89+\frac{300 \times \#frasi -10\times \#lettere}{\#parole} $;}
				\item{valore minimo accettabile: $\geq$ 40;}
				\item{valore preferibile: $\geq$ 60.}
			\end{itemize}

		\subsubsubsection*{Correttezza ortografica}
			\begin{itemize}
				\item{misurazione: numero intero che indica il numero di errori presenti nel testo;}
				\item{valore minimo accettabile: 0;}
				\item{valore preferibile: 0.}
			\end{itemize}
		
		\subsubsubsection*{Formula di Flesch}
			 \begin{itemize}
				\item{misurazione: $ F = 206,835 - (84,6 \times S) - (1,015 \times P) $, dove: 
				\begin{itemize}
					\item S indica il numero medio di sillabe per parola; 
					\item P indica il numero medio di parole per frase. 
				\end{itemize}}
				\item{valore minimo accettabile: $\geq$ 50;}
				\item{valore preferibile: $\geq$ 60.}
			\end{itemize}

	\subsubsubsection{Gestione della Qualità}
	I valori di riferimento delle metriche usate per la gestione della qualità sono le seguenti:
	
		\subsubsubsection*{Percentuale di metriche soddisfatte (PMS)}
			\begin{itemize}
				\item{$PMS=\displaystyle\frac{\#metriche\_soddisfatte}{\#totale\_di\_metriche} \times 100$;}
				\item{valore minimo accettabile: 60\%;}
				\item{valore preferibile: 90\%.}
			\end{itemize}
					
	\subsubsubsection{Verifica}
	I valori di riferimento delle metriche usate per l'attività di verifica sono le seguenti:
	
		\subsubsubsection*{Code Coverage}
		 	\begin{itemize}
				\item{misurazione: $ CC = \displaystyle\frac{\#linee\_di\_codice\_eseguite\_dal\_test}{\#linee\_di\_codice\_totali} \times 100$;}
				\item{valore minimo accettabile: 80\%;}
				\item{valore preferibile: 100\%.}
			\end{itemize}

\subsection{Processi Organizzativi}

	\subsubsubsection{Gestione Organizzativa}
	I valori di riferimento delle metriche usate per la gestione organizzativa sono le seguenti:
	
		\subsubsubsection*{Budget at Completion (BAC)}
			\begin{itemize}
				\item{misurazione: numero intero;}
				\item{valore minimo accettabile: valore del preventivo con un errore massimo del 5\%, ovvero \textit{preventivo}-5\% $\leq$ \textit{BAC} $\leq$ \textit{preventivo}+5\%;}
				\item{valore preferibile: pari al preventivo.}
			\end{itemize}
		
		\subsubsubsection*{Estimated at Completion (EAC)}
			\begin{itemize}
				\item{misurazione: $EAC = AC + ETC$;}
				\item{valore minimo accettabile: valore del preventivo con un errore massimo del 5\%, ovvero \textit{preventivo}-5\% $\leq$ \textit{BAC} $\leq$ \textit{preventivo}+5\%;}
				\item{valore preferibile: pari al preventivo.}
			\end{itemize}
		
		\subsubsubsection*{Estimate to Complete (ETC)}
			\begin{itemize}
				\item{misurazione: numero intero;}
				\item{valore minimo accettabile: $\leq$ preventivo;}
				\item{valore preferibile: < preventivo.}
			\end{itemize}
		
		\subsubsubsection*{Planned Value (PV)}
			\begin{itemize}
				\item{misurazione: $PV = \%lavoro\_pianificato \times BAC$;}
				\item{valore minimo accettabile: $\geq$ 0;}
				\item{valore preferibile: $\geq$ 0.}
			\end{itemize}
		
		\subsubsubsection*{Actual Cost (AC)}
			\begin{itemize}
				\item{misurazione: numero intero;}
				\item{valore minimo accettabile: 0 $\leq$ \textit{AC} < \textit{BAC};}
				\item{valore preferibile: 0 $\leq$ \textit{AC} < \textit{PV}.}
			\end{itemize}
		
		\subsubsubsection*{Earned Value (EV)}
			\begin{itemize}
				\item{misurazione: $EV = \%lavoro\_completato \times BAC$;}
				\item{valore minimo accettabile: $\geq$ 0;}
				\item{valore preferibile: $\geq$ 0.}
			\end{itemize}

		\subsubsubsection*{Cost Variance (CV)}
			\begin{itemize}
				\item{misurazione: $CV = EV - AC$;}
				\item{valore minimo accettabile: 0;}
				\item{valore preferibile: > 0.}
			\end{itemize}
		
		\subsubsubsection*{Schedule Variance (SV)}
			\begin{itemize}
				\item{misurazione: $SV = EV - PV$;}
				\item{valore minimo accettabile: 0;}
				\item{valore preferibile: > 0.}
			\end{itemize}
						
		\subsubsubsection*{Correlazione tra CV e SV}
		Lo stato di un progetto è esprimibile dalla correlazione tra \textit{Cost Variance} e \textit{Schedule Variance}, in particolare:
		\begin{enumerate}
			\item{\textbf{SV e CV positive}: il progetto è in anticipo rispetto alla pianificazione e rientra nel budget previsto;}
			\item{\textbf{SV positiva, CV negativa}: il progetto è in anticipo rispetto alla pianificazione ma ha superato il budget allocato;}
			\item{\textbf{SV negativa, CV positiva}: il progetto è in ritardo rispetto alla pianificazione ma rientra nel budget previsto;}
			\item{\textbf{SV e CV negative}: il progetto è in ritardo rispetto alla pianificazione e ha superato il budget previsto.}
		\end{enumerate}