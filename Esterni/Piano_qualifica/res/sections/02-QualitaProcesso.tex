\section{Qualità di Processo}

\subsection{Scopo}
Lo scopo della seguente sezione è quello di elencare le metriche adottate dal team \Gruppo{} per valutare la maturità dei processi descritti nel documento \textit{\NdP} e garantire qualità nel loro svolgimento. Lo standard scelto per perseguire tale scopo è ISO/IEC 15504. Inoltre, viene posta particolare attenzione sull'applicazione del metodo di gestione PDCA per ricercare un miglioramento continuo nel corso di tutto il progetto didattico.

\subsubsection{Struttura}
Le metriche illustrate nel seguente documento presentano la struttura:
\subsubsubsection*{Nome metrica}
\begin{itemize}
	\item{\textbf{Descrizione}: breve descrizione circa lo scopo della metrica;}
	\item{\textbf{misurazione}: valore mediante il quale viene espressa la metrica o -se presente- formula per calcolarla;}
	\item{\textbf{valore minimo accettabile}: limite inferiore sul valore minimo che la metrica deve assumere per ritenersi soddisfatta;}
	\item{\textbf{valore preferibile}: valore della metrica consigliato (ma non vincolante) per il perseguimento della qualità.}
\end{itemize}

\subsection{Processi Primari}

	\subsubsection{Processo di Sviluppo}
	
		\subsubsubsection{Analisi dei Requisiti}
		Le metriche usate per l'attività di analisi sono le seguenti:
			\subsubsubsection*{Percentuale dei Requisiti Obbligatori Soddisfatti (PROS)}
			È un valore percentuale che indica la quantità di requisiti obbligatori adempiti nel corso del progetto.
		\begin{itemize}
			\item{misurazione: PROS = $\displaystyle\frac{\mbox{requisiti obbligatori soddisfatti}}{\mbox{requisiti obbligatori totali}}$;}
			\item{valore minimo accettabile: 100\%;}
			\item{valore preferibile: 100\%.}
		\end{itemize}

		\subsubsubsection{Progettazione}
		Le metriche usate per l'attività di progettazione sono le seguenti:
		
			\subsubsubsection*{Coupling Between Objects (CBO)}
			È un valore intero che indica il grado di accoppiamento tra le classi di oggetti. Una classe A si dice \textit{accoppiata} ad una classe B, se A usa metodi o variabili definite in B.
		\begin{itemize}
			\item{misurazione: valore intero;}
			\item{valore minimo accettabile: 0 $\leq$ CBO $\leq$ 6;}
			\item{valore preferibile: 0 $\leq$ CBO $\leq$ 1.}
		\end{itemize}

		
			\subsubsubsection*{Structural Fan-In (SFIN)}
			È un valore intero che indica quante componenti utilizzano un dato modulo. Uno \textit{SFIN} alto sta a significare un consistente riuso della componente.
		\begin{itemize}
			\item{misurazione: valore intero che indica il conteggio delle componenti;}
			\item{valore minimo accettabile: $\geq$ 0;}
			\item{valore preferibile: $\geq$ 1.}
		\end{itemize}
	
			\subsubsubsection*{Structural Fan-Out (SFOUT)}
			È un valore intero che indica quante componenti vengono utilizzate dalla componente in questione. Uno \textit{SFOUT} alto sta a significare un significativo accoppiamento della componente.
		\begin{itemize}
			\item{misurazione: valore intero che indica il conteggio delle componenti;}
			\item{valore minimo accettabile: = 0;}
			\item{valore preferibile: $\leq$ 6.}
		\end{itemize}
		
		\subsubsubsection{Codifica}
		Le metriche usate per l'attività di codifica sono le seguenti:
					
		\subsubsubsection*{Complessità ciclomatica}
		È una metrica utilizzata per stimare la complessità di funzioni, moduli\ped{\textit{G}}, metodi o classi di un programma. Ciò viene fatto mediante la determinazione del numero dei cammini linearmente indipendenti attraverso il grafo di controllo di flusso.
		\begin{itemize}
			\item{misurazione: valore intero;}
			\item{valore minimo accettabile: 1 $\leq$ complessità ciclomatica $\leq$ 15;}
			\item{valore preferibile: 1 $\leq$ complessità ciclomatica $\leq$ 10.}
		\end{itemize}
			
		\subsubsubsection*{Rapporto linee di codice per linee di commento}
		Indica il rapporto tra le righe di codice e le righe di commento ad esso corrispondenti, escludendo le righe vuote, al fine di stimare il livello di difficoltà di manutenibilità del codice. 	
		\begin{itemize}
			\item{misurazione: valore decimale;}
			\item{valore minimo accettabile: $\displaystyle\frac{\mbox{linee di codice}}{\mbox{linee di commento}} \geq$ 0.25;}
			\item{valore preferibile: $\displaystyle\frac{\mbox{linee di codice}}{\mbox{linee di commento}} \geq$ 0.30.}
		\end{itemize}

		\subsubsubsection*{Livello di annidamento}
		È un valore intero che indica il livello di annidamento nei vari metodi tenendo conto della presenza di strutture di controllo adibite a tale mansione.		
		\begin{itemize}
			\item{misurazione: valore intero;}
			\item{valore minimo accettabile: 1 $\leq$ livello annidamento $\leq$ 7;}
			\item{valore preferibile: 1 $\leq$ livello annidamento $\leq$ 3.}
		\end{itemize}
			
		\subsubsubsection*{Numero di parametri per metodo}
		È un numero intero che indica il numero di parametri di un metodo. 
		\begin{itemize}
			\item{misurazione: valore intero;}
			\item{valore minimo accettabile: 0 $\leq$ numero totale attributi $\leq$ 8;}
			\item{valore preferibile: 0 $\leq$ numero totale attributi $\leq$ 4.}
		\end{itemize}
			
		\subsubsubsection*{Numero di attributi per classe}
		È un numero intero che indica il numero totale di attributi presenti all'interno di una classe.				\begin{itemize}
			\item{misurazione: valore intero;}
			\item{valore minimo accettabile: 1 $\leq$ numero totale attributi $\leq$ 15;}
			\item{valore preferibile: 1 $\leq$ numero totale attributi $\leq$ 8.}
		\end{itemize}

\subsection{Processi di Supporto}

	\subsubsubsection{Documentazione}
	Le metriche usate per l'attività di documentazione sono le seguenti:
	
		\subsubsubsection*{Indice di Gulpease}
		È un indice che valuta la leggibilità del testo tarato sulla lingua italiana. Il valore risultante è compreso tra 0 e 100, un valore di indice più alto corrisponde ad un indice di leggibilità più semplice.
		 \begin{itemize}
			\item{misurazione: G = 89 + $\displaystyle\frac{300 \cdot{} (\mbox{numero delle frasi}) - 10 \cdot{} (\mbox{numero delle lettere})}{\mbox{numero delle parole}}$;}
			\item{valore minimo accettabile: $\geq$ 40;}
			\item{valore preferibile: $\geq$ 60.}
		\end{itemize}

		\subsubsubsection*{Correttezza ortografica}
		Tutti i documenti devono essere privi di errori grammaticali od ortografici.
		\begin{itemize}
			\item{misurazione: numero intero che indica il numero di errori presenti nel testo;}
			\item{valore minimo accettabile: 0;}
			\item{valore preferibile: 0.}
		\end{itemize}
		
		\subsubsubsection*{Formula di Flesch}
		E' un indice che valuta la leggibilità di un testo in lingua inglese. Più questo indice è alto e più il testo risulta semplice da leggere.
		 \begin{itemize}
			\item{misurazione: F = 206,835 - (84,6 $\cdot$ numero medio di sillabe per parola) - (1,015 $\cdot$ numero medio di parole per frase);}
			\item{valore minimo accettabile: $\geq$ 50;}
			\item{valore preferibile: $\geq$ 60.}
		\end{itemize}

	\subsubsubsection{Gestione della Qualità}
	Le metriche usate per la gestione dell qualità sono le seguenti:
	
		\subsubsubsection*{Percentuale di metriche soddisfatte (PMS)}
		E' un valore percentuale che indica quante metriche raggiungono soglie accettabili sul totale delle metriche considerate.
		\begin{itemize}
			\item{PMS = $\displaystyle\frac{\mbox{numero di metriche soddisfatte}}{\mbox{numero totale di metriche}}$}
			\item{valore minimo accettabile: 60\%;}
			\item{valore preferibile: 90\%.}
		\end{itemize}
					
	\subsubsubsection{Verifica}
	Le metriche usate per l'attività di verifica sono le seguenti:
	
		\subsubsubsection*{Code Coverage}
		 È la percentuale di linee di codice che sono state eseguite dai test dopo un’esecuzione.
		 \begin{itemize}
			\item{misurazione: CC = $\displaystyle\frac{\mbox{linee di codice eseguite dal test}}{\mbox{linee di codice totali}}$;}
			\item{valore minimo accettabile: 80\%;}
			\item{valore preferibile: 100\%.}
		\end{itemize}

\subsection{Processi Organizzativi}

	\subsubsubsection{Gestione Organizzativa}
	Le metriche usate per la gestione organizzativa sono le seguenti:
	
		\subsubsubsection*{Budget at Completion (BAC)}
		Equivale al budget allocato inizialmente per il progetto.
		\begin{itemize}
			\item{misurazione: numero intero;}
			\item{valore minimo accettabile: valore del preventivo con un errore massimo del 5\%, ovvero \textit{preventivo}-5\% $\leq$ \textit{BAC} $\leq$ \textit{preventivo}+5\%;}
			\item{valore preferibile: pari al preventivo.}
		\end{itemize}
		
		\subsubsubsection*{Estimated at Completion (EAC)}
		Equivale al BAC rivisto allo stato corrente del progetto, ovvero è la somma del costo sostenuto fino a quel momento e la stima del costo ancora da sostenere.
		\begin{itemize}
			\item{misurazione: EAC = AC + ETC;}
			\item{valore minimo accettabile: valore del preventivo con un errore massimo del 5\%, ovvero \textit{preventivo}-5\% $\leq$ \textit{BAC} $\leq$ \textit{preventivo}+5\%;}
			\item{valore preferibile: pari al preventivo.}
		\end{itemize}
		
		\subsubsubsection*{Estimate to Complete (ETC)}
		Valore stimato per la realizzazione delle rimanenti attività necessarie al completamento del progetto.
		\begin{itemize}
			\item{misurazione: numero intero;}
			\item{valore minimo accettabile: $\leq$ preventivo;}
			\item{valore preferibile: < preventivo.}
		\end{itemize}
		
		\subsubsubsection*{Planned Value (PV)}
		Corrisponde al costo pianificato per realizzare le attività di progetto fino a quel momento.
		\begin{itemize}
			\item{misurazione: \% di lavoro pianificato $\cdot$ BAC;}
			\item{valore minimo accettabile: $\geq$ 0;}
			\item{valore preferibile: $\geq$ 0.}
		\end{itemize}
		
		\subsubsubsection*{Actual Cost (AC)}
		Indica la quantità di budget spesa al momento del calcolo.
		\begin{itemize}
			\item{misurazione: numero intero;}
			\item{valore minimo accettabile: 0 $\leq$ \textit{AC} < \textit{BAC};}
			\item{valore preferibile: 0 $\leq$ \textit{AC} < \textit{PV}.}
		\end{itemize}
		
		\subsubsubsection*{Earned Value (EV)}
		Indica la quantità di lavoro compiuta al momento del calcolo, ovvero il valore prodotto dal progetto alla data corrente.
		\begin{itemize}
			\item{misurazione: EV =  \% di lavoro completato $\cdot$ BAC;}
			\item{valore minimo accettabile: $\geq$ 0;}
			\item{valore preferibile: $\geq$ 0;}
		\end{itemize}

		\subsubsubsection*{Cost Variance (CV)}
		Misura l'andamento del budget nel corso di un progetto software. In particolare, la \textit{Cost Variance} è la differenza tra \textit{Earned Value} e \textit{Actual Cost}, ovvero tra ciò che si aveva pianificato di spendere e ciò che si è effettivamente speso nel corso del progetto.
		Se la \textit{Cost Variance} ha valore negativo significa che si è \textit{over budget}, se è nulla si è \textit{on budget}, mentre se è positiva si è \textit{under budget}.
		\begin{itemize}
			\item{misurazione: CV = EV - AC;}
			\item{valore minimo accettabile: 0;}
			\item{valore preferibile: > 0;}
		\end{itemize}
		
		\subsubsubsection*{Schedule Variance (SV)}
		Indica lo stato di avanzamento di un progetto software rispetto alla schedulazione delle attività e viene calcolata mediante la differenza tra \textit{Earned Value} e \textit{Planned Value}.
		Se la \textit{Schedule Variance} ha valore negativo significa che lo stato di avanzamento del progetto è in ritardo rispetto alla pianificazione, se è nulla lo stato di avanzamento del progetto è nei tempi previsti, mentre se è positiva significa che si è in anticipo rispetto alla pianificazione.
		\begin{itemize}
			\item{misurazione: SV = EV - PV;}
			\item{valore minimo accettabile: 0;}
			\item{valore preferibile: > 0.}
		\end{itemize}
						
		\subsubsubsection*{Correlazione tra CV e SV}
		Lo stato di un progetto è esprimibile dalla correlazione tra \textit{Cost Variance} e \textit{Schedule Variance}, in particolare:
		\begin{enumerate}
			\item{\textbf{SV e CV positive}: il progetto è in anticipo rispetto alla pianificazione e rientra nel budget previsto;}
			\item{\textbf{SV positiva, CV negativa}: il progetto è in anticipo rispetto alla pianificazione ma ha superato il budget allocato;}
			\item{\textbf{SV negativa, CV positiva}: il progetto è in ritardo rispetto alla pianificazione ma rientra nel budget previsto;}
			\item{\textbf{SV e CV negative}: il progetto è in ritardo rispetto alla pianificazione e ha superato il budget previsto.}
		\end{enumerate}