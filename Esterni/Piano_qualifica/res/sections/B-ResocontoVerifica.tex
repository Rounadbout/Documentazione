\section{Resoconto delle attività di verifica}
	Questa sezione riporta un resoconto di tutte le metriche per le quali è stato possibile svolgere una misurazione allo stato attuale del progetto.

	\subsection{Qualità di Processo}
		\subsubsection{Processi Primari}
			\subsubsubsection{Analisi dei Requisiti}
				I valori registrati per ogni periodo relativi alle metriche per l'analisi dei requisiti sono i seguenti:
			\subsubsubsection*{PROS}	%(Percentuale dei Requisiti Obbligatori Soddisfatti)
			\rowcolors{2}{lightRowColor}{darkRowColor}
\begin{longtable}{
		>{\centering}p{0.2\textwidth}
		>{\centering}p{0.2\textwidth}
		>{\centering}p{0.2\textwidth}
		>{\centering}p{0.2\textwidth}
		>{}p{0.2\textwidth} }
		
		\caption{Percentuale requisiti obbligatori soddisfatti (PROS)} \\

	\coloredTableHead
	\textbf{\color{white} Metrica} &
	\textbf{\color{white} RR} &
	\textbf{\color{white} RP} &
	\textbf{\color{white} RQ} &
	\textbf{\color{white}RA}
	\tabularnewline
	\endhead

	% contenuto tabella
	% esempio: Modulo & Min & Max \\
	PROS & 0 & 42\% & - & - \\
	
\end{longtable}

				
			%\subsubsubsection{Progettazione}
				%I valori registrati per ogni periodo relativi alle metriche per la progettazione sono i seguenti:
			%\subsubsubsection*{CBO}		%Grafico
				%Questa metrica viene valutata nelle fasi successive del ciclo di vita del software.
			%\subsubsubsection*{SFIN}
				%Questa metrica viene valutata nelle fasi successive del ciclo di vita del software.
			%\subsubsubsection*{SFOUT}
				%Questa metrica viene valutata nelle fasi successive del ciclo di vita del software.
			\subsubsubsection{Codifica}
				I valori registrati per ogni periodo relativi alle metriche per la codifica sono i seguenti:

\subsubsubsection*{Complessità ciclomatica}
I valori sono riportati secondo lo schema \textit{valore minimo trovato - valore massimo trovato} per ogni modulo\ped{\textit{G}}.
\rowcolors{2}{lightRowColor}{darkRowColor}
\begin{longtable}{
		>{\centering}p{0.15\textwidth}
		>{\centering}p{0.15\textwidth}
		>{\centering}p{0.15\textwidth}
		>{\centering}p{0.15\textwidth}
		>{\centering\arraybackslash}p{0.15\textwidth} }

		\caption{Complessità ciclomatica} \\

	\coloredTableHead
	\textbf{\color{white} Modulo} &
	\textbf{\color{white} RR} &
	\textbf{\color{white} RP} &
	\textbf{\color{white} RQ} &
	\textbf{\color{white}RA}
	\tabularnewline
	\endhead

	% contenuto tabella
	% esempio: Modulo & Min & Max \\
	Etherless-cli & 0 & 3-13 & 3-7 & - \\
	Etherless-smart & 0 & 2 & 2-4 & - \\
	Etherless-server & 0 & 3-7 & 1-10 & - \\

\end{longtable}

\pagebreak
\subsubsubsection*{Rapporto linee di codice per linee di commento (RCC)}
I valori sono calcolati facendo la media della percentuale di commenti presente in ogni file appartenente al modulo\ped{\textit{G}}.
\rowcolors{2}{lightRowColor}{darkRowColor}
\begin{longtable}{
		>{\centering}p{0.15\textwidth}
		>{\centering}p{0.15\textwidth}
		>{\centering}p{0.15\textwidth}
		>{\centering}p{0.15\textwidth}
		>{\centering\arraybackslash}p{0.15\textwidth} }

		\caption{Rapporto linee di codice per linee di commento} \\

	\coloredTableHead
	\textbf{\color{white} Modulo} &
	\textbf{\color{white} RR} &
	\textbf{\color{white} RP} &
	\textbf{\color{white} RQ} &
	\textbf{\color{white}RA}
	\tabularnewline
	\endhead

	% contenuto tabella
	% esempio: Modulo & Min & Max \\
	Etherless-cli & 0 & 0.04 & 0.10 & - \\
	Etherless-smart & 0 & 0.13 & 0.11 & - \\
	Etherless-server & 0 & 0.06 & 0.13 & - \\

\end{longtable}

\subsubsubsection*{Numero di parametri per metodo}
I valori sono riportati secondo lo schema \textit{valore minimo trovato - valore massimo trovato} per ogni modulo\ped{\textit{G}}.
\rowcolors{2}{lightRowColor}{darkRowColor}
\begin{longtable}{
		>{\centering}p{0.15\textwidth}
		>{\centering}p{0.15\textwidth}
		>{\centering}p{0.15\textwidth}
		>{\centering}p{0.15\textwidth}
		>{\centering\arraybackslash}p{0.15\textwidth} }

		\caption{Numero di parametri per metodo} \\

	\coloredTableHead
	\textbf{\color{white} Modulo} &
	\textbf{\color{white} RR} &
	\textbf{\color{white} RP} &
	\textbf{\color{white} RQ} &
	\textbf{\color{white}RA}
	\tabularnewline
	\endhead

	% contenuto tabella
	% esempio: Modulo & Min & Max \\
	Etherless-cli & 0 & 0-1 & 0-3 & - \\
	Etherless-smart & 0 & 0-2 & 0-5 & - \\
	Etherless-server & 0 & 0-2 & 0-4 & - \\

\end{longtable}


\subsubsubsection*{Numero di attributi per classe}
I valori sono riportati secondo lo schema \textit{valore minimo trovato - valore massimo trovato} per ogni modulo\ped{\textit{G}}.
\rowcolors{2}{lightRowColor}{darkRowColor}
\begin{longtable}{
		>{\centering}p{0.15\textwidth}
		>{\centering}p{0.15\textwidth}
		>{\centering}p{0.15\textwidth}
		>{\centering}p{0.15\textwidth}
		>{\centering\arraybackslash}p{0.15\textwidth} }

		\caption{Numero di attributi per classe} \\

	\coloredTableHead
	\textbf{\color{white} Modulo} &
	\textbf{\color{white} RR} &
	\textbf{\color{white} RP} &
	\textbf{\color{white} RQ} &
	\textbf{\color{white}RA}
	\tabularnewline
	\endhead

	% contenuto tabella
	% esempio: Modulo & Min & Max \\
	Etherless-cli & 0 & 1-2 & 1-4 & - \\
	Etherless-smart & 0 & 5 & 1-7 & - \\
	Etherless-server & 0 & 0 & 1-4 & - \\

\end{longtable}
 
		
		\subsubsection{Processi di Supporto}
			\subsubsubsection{Documentazione}
				\paragraph{Indice di Gulpease}
	Nel seguente grafico vengono riportati i valori di Gulpease, calcolati per ogni documento, in differenti momenti di maturazione del progetto.
	\begin{center}
		\begin{tikzpicture} [scale = 0.9]
			\begin{axis}[
					xlabel={\textbf{Tempo}},
					ylabel={\textbf{Indice di Gulpease}},
					date coordinates in=x,
					ymin=38,
					ymax=100,
					xtick=data,
					xticklabel style={
						rotate=90,
						anchor=near xticklabel,
					},
					xticklabel=\year-\month-\day,
				]
				\addplot table [col sep=comma,x=date,y=value,blue] {res/sections/B/gulpease/AdR.csv};
				\addplot table [col sep=comma,x=date,y=value,blue] {res/sections/B/gulpease/Glossario.csv};
				\addplot table [col sep=comma,x=date,y=value,blue] {res/sections/B/gulpease/NdP.csv};
				\addplot table [col sep=comma,x=date,y=value,blue] {res/sections/B/gulpease/PdP.csv};
				\addplot table [col sep=comma,x=date,y=value,blue] {res/sections/B/gulpease/PdQ.csv};
				\addplot table [col sep=comma,x=date,y=value,blue] {res/sections/B/gulpease/SdF.csv};		
				\addplot table [col sep=comma,x=date,y=value,blue] {res/sections/B/gulpease/Verbali.csv};
				\draw [line width=0.1, red](2020-4-1, 40)--(2020-4-20, 40);
				\legend{$Analisi dei Requisiti$ ,$Glossario$ ,$Norme di Progetto$, $Piano di Progetto$, $Piano di Qualifica$, $Studio di Fattibilita'$, $Media dei Verbali$};
			\end{axis}
		\end{tikzpicture} \\
	\end{center}

\paragraph{Correttezza ortografica}
	\hyperlink{val_correttezza ortografica}{\textit{(Vedi definizione)}}
	\hypertarget{def_correttezza ortografica}{}
	Questa metrica è relativa alla correttezza ortografica all'interno di un documento. Prima di ogni consegna vengono calcolati i risultati esposti nella seguente tabella.
	\rowcolors{2}{lightRowColor}{darkRowColor}
	\begin{longtable}{
			>{\centering}p{0.4\textwidth}
			>{\centering}p{0.1\textwidth}
			>{\centering}p{0.1\textwidth}
			>{\centering}p{0.1\textwidth}
			>{}p{0.1\textwidth} }
		
		\coloredTableHead
		\textbf{\color{white}Documento} &
		\textbf{\color{white}RR} &
		\textbf{\color{white}RP} &
		\textbf{\color{white}RQ} &
		\textbf{\color{white}RA}
		\tabularnewline
		\endhead
		
		% contenuto tabella
		% esempio: Documento & ValoreRR & ValoreRP & ValoreRQ & ValoreRA. \\
		Analisi dei Requisiti & 0 & - & - & - \\
		Glossario & 0 & - & - & - \\
		Norme di Progetto & 0 & - & - & - \\
		Piano di Progetto & 0 & - & - & - \\
		Piano di Qualifica & 0 & - & - & - \\
		Studio di Fattibilità & 0 & - & - & - \\
		Verbali & 0 & - & - & - \\
		
	\end{longtable}
	
	
	
\paragraph{Formula di Flesch}
	Questa metrica verrà sviluppata a seguito della stesura di documentazione in lingua inglese.
			\subsubsubsection{Gestione della Qualità}
				\subsection{Gestione della qualità}
    \subsubsection{Descrizione}
      Un'adeguata implementazione del processo di gestione della qualità è fondamentale per il corretto svolgimento del progetto. Questo processo intende garantire software e documenti di buona qualità, sviluppati attraverso processi chiari e ordinati. Il processo di gestione di qualità viene descritto nel dettaglio all'interno del \textit{\PdQ{} 2.0.0}. In particolare per ogni processo ed ogni prodotto\ped{\textit{G}} vengono descritti gli obiettivi e le metriche per la valutazione del raggiungimento degli stessi.
      
    \subsubsection{Attività}
    \subsubsubsection{Pianificazione}
       Il presupposto per l'implementazione del processo di gestione della qualità è la presenza di un \PdQ{}, a cui sia possibile fare riferimento per coordinare tutte le attività periodiche ed operazioni di controllo qualità. Tale piano resta valido per tutta la durata del progetto e viene costantemente aggiornato con le nuove metriche e procedure integrate.

    \subsubsubsection{Garanzia di qualità del prodotto}
      La qualità del prodotto\ped{\textit{G}} che si va a sviluppare viene garantita coordinatamente dai processi di Verifica e Validazione (descritti rispettivamente nelle sezioni \textsection3.4 e \textsection3.5), i quali puntano ad un miglioramento continuo e a verificare il rispetto delle metriche di qualità stabilite. Importante è anche il confronto costante con il Proponente\ped{\textit{G}}, per garantire che il software prodotto sia in grado di soddisfare i requisiti concordati.

    \subsubsubsection{Garanzia di qualità dei processi}
      La qualità dei processi che compongono il ciclo di vita del software deriva dallo svolgimento corretto e normato delle attività che compongono tali processi. Ottenendo quindi risultati in grado di soddisfare i requisiti previsti da contratto, rispettando norme e standard scelti come riferimento. È compito degli Amministratori di progetto monitorare lo svolgimento delle attività, ed intervenire per garantire il rispetto dei piani e delle procedure di gestione della qualità.
      L'obiettivo è il miglioramento continuo dei processi, perseguendo i principi di efficacia ed efficienza del prodotto\ped{\textit{G}} durante tutto il suo ciclo di vita.   
	
	\subsubsection{Metriche}
	\subsubsubsection{Metriche di processo e di prodotto}
		\paragraph{Percentuale di metriche soddisfatte (PMS)}
			\begin{itemize}
				\item \textbf{Descrizione}: è un valore percentuale che punta a rappresentare la qualità del processo/prodotto\ped{\textit{G}} sotto analisi. È basato sul numero di metriche che hanno raggiunto un valore considerato accettabile, rapportato con il numero totale di metriche applicate nella valutazione del processo/prodotto\ped{\textit{G}};
				\item \textbf{unità di misura}: la metrica è espressa tramite un numero percentuale;
				\item \textbf{formula}: PMS = $\displaystyle\frac{\#metriche\_soddisfatte}{\#totale\_di\_metriche}\times100$;
				\item \textbf{risultato}: 
				\begin{itemize}
					\item {se il risultato è minore di 60, la qualità del processo/prodotto\ped{\textit{G}} è considerata non accettabile. Vanno considerate procedure di risoluzione quali:	ricalcolo del PMS, revisione del processo/prodotto\ped{\textit{G}} sotto analisi, revisione delle metriche applicate;}
					\item {se il risultato è minore di 90, la qualità del processo/prodotto\ped{\textit{G}} è considerata accettabile, ma ancora migliorabile;}
					\item {se il risultato è maggiore di 90, la qualità del processo/prodotto\ped{\textit{G}} è considerata ideale.}
				\end{itemize}
			\end{itemize}	
    \subsubsection{Strumenti}
      Gli strumenti di riferimento per la qualità sono:
      \begin{itemize}
      	\item{parte dei processi forniti dallo standard ISO 12207;}
      	\item{standard ISO/IEC 15504;}
      	\item{ciclo di Deming;}
      	\item{standard ISO/IEC 9126;}
      	\item{metriche di qualità stabilite.}
      \end{itemize}
  	  Un'accurata descrizione di tali stumenti e modelli è presente nell'appendice A di questo stesso documento. \\
  	  Gli Amministratori del progetto sfruttano inoltre l'apparato Issue\ped{\textit{G}} Tracking System di \textbf{\mbox{GitHub}}\ped{\textit{G}}, per monitorare costantemente lo svolgimento delle attività del progetto.


			%\subsubsubsection{Verifica}
				%Questa metrica viene valutata nelle fasi successive del ciclo di vita del software.
				
		\subsubsection{Processi Organizzativi}
			\subsubsubsection{Gestione Organizzativa}
				I valori registrati per ogni periodo, relativi alle metriche per la gestione organizzativa, sono i seguenti:

\rowcolors{2}{lightRowColor}{darkRowColor}
\begin{longtable}{
		>{\centering}p{0.2\textwidth}
		>{\centering}p{0.2\textwidth}
		>{\centering}p{0.2\textwidth}
		>{\centering}p{0.2\textwidth}
		>{}p{0.2\textwidth} }

	\coloredTableHead
	\textbf{\color{white}Metrica} &
	\textbf{\color{white}Analisi e consolidamento dei requisiti} &
	\textbf{\color{white}Progettazione architetturale} &
	\textbf{\color{white}Progettazione di dettaglio e codifica} &
	\textbf{\color{white}Validazione e collaudo}
	\tabularnewline
	\endhead

	% contenuto tabella
	% esempio: ProvaMetrica & ProvaRR & ProvaRP & ProvaRQ & ProvaRA. \\
	BAC & 15.036,00 & 15.036,00 & 15.036,00 & 15.036,00 \\
	EAC & 15.036,00 & - & - & - \\
	ETC & 15.036,00 & 10.182,00 & 3.181,00 & 0,00 \\
	PV & 0,00 & 4.854,00 & 7.001,00 & 3.181,00 \\
	AC & 0,00 & - & - & - \\
	EV & 0,00 & 4.854,00 & 7.001,00 & 3.181,00 \\
	CV & 0 & - & - & - \\
	SV & 0 & - & - & - \\
\end{longtable}
			
	%\subsection{Qualità di Prodotto}
		%\subsubsection{Funzionalità}
			%\subsubsubsection{Completezza dell'implementazione}
				%Questa metrica viene valutata nelle fasi successive del ciclo di vita del software.

		%\subsubsection{Affidabilità}
			%\subsubsubsection{Densità errori}				% Grafico
				%Questa metrica viene valutata nelle fasi successive del ciclo di vita del software.
		%\subsubsection{Usabilità}
			%\subsubsubsection{Facilità di utilizzo}			% Vedere con gli altri
				%Questa metrica viene valutata nelle fasi successive del ciclo di vita del software.
			%\subsubsubsection{Facilità di apprendimento}
				%Questa metrica viene valutata nelle fasi successive del ciclo di vita del software.
		%\subsubsection{Manutenibilità}
			%\subsubsubsection{Facilità di comprensione} 	% Grafico
				%Questa metrica viene valutata nelle fasi successive del ciclo di vita del software.
			%\subsubsubsection{Semplicità delle classi}
				%Questa metrica viene valutata nelle fasi successive del ciclo di vita del software.
	
	
	%\subsection{Test di Verifica}
		%\subsubsection{Test di Unità}
			%Questa metrica viene valutata nelle fasi successive del ciclo di vita del software.
		%\subsubsection{Test di Integrazione}
			%Questa metrica viene valutata nelle fasi successive del ciclo di vita del software.
		%\subsubsection{Test di Sistema}
			%Questa metrica viene valutata nelle fasi successive del ciclo di vita del software.
		%\subsubsection{Test di Regressione}
			%Questa metrica viene valutata nelle fasi successive del ciclo di vita del software.
		
	
	
	\subsection{Test di Validazione}
		\subsubsection{Test di Accettazione}	
			\subsubsubsection{Test Funzionali}
				\subsubsection{Test funzionali}

\def\arraystretch{1.75}
\rowcolors{2}{lightRowColor}{darkRowColor}
\begin{longtable}{ 
		>{\centering}p{0.1\textwidth} 
		>{}p{0.5\textwidth} 
		>{\centering}p{0.17\textwidth}
		>{\centering}p{0.12\textwidth} }
	
	\caption{Tabella dei test funzionali} \\
	\coloredTableHead
	\textbf{\color{white}Test} & 
	\centering\textbf{\color{white}Requisito e Descrizione} & 
	\centering\textbf{\color{white}Implementato} &
	\textbf{\color{white}Superato} 
	\endfirsthead
	
	\rowcolor{white}\caption[]{(continua)}\\
	\coloredTableHead 
	\textbf{\color{white}Test} &
	\centering\textbf{\color{white}Requisito e Descrizione} &
	\centering\textbf{\color{white}Implementato} &
	\textbf{\color{white}Superato}
	\endhead

	% init
	TA2F1 & \textbf{R2F1}: L'utente può leggere una breve guida iniziale riguardante l'applicativo e i comandi per effettuare l'accesso & No & No \tabularnewline

	% help
	TA2F2 & \textbf{R2F2}: L'utente può richiedere di visualizzare una descrizione più approfondita
		 per ogni comando messo a disposizione da \textit{Etherless-cli}			& No & No \tabularnewline
	TA2F2.1 & \textbf{R2F2.1}: Per ottenere informazioni specifiche su un comando, l'utente deve
		inserire il comando \textit{help} seguito dal nome del comando di suo interesse	& No & No \tabularnewline
	TA2F2.2 & \textbf{R2F2.2}: Se il comando di cui si vogliono avere maggiori informazioni non
		è tra quelli messi a disposizione da \textit{Etherless-cli} deve essere
		mostrato un messaggio di errore												& No & No \tabularnewline

	% signup
	TA1F3 & \textbf{R1F3}: Un utente non registrato può richiedere la creazione di un nuovo account
			 all'interno della rete Ethereum\ped{\textit{G}}									& No & No \tabularnewline
	TA1F3.1 & \textbf{R1F3.1}: Una volta creato il nuovo account, il sistema deve mostrare nella 
			CLI\ped{\textit{G}} le credenziali a esso relative										& No & No \tabularnewline
	TA1F3.1.1 & \textbf{R1F3.1.1}: A seguito del completamento della procedura di registrazione viene
			mostrato l'address associato al nuovo account creato 					& No & No \tabularnewline
	TA1F3.1.2 & \textbf{R1F3.1.2}: A seguito del completamento della procedura di registrazione viene
			mostrata la private key associata al nuovo account creato 				& No & No \tabularnewline
	TA2F3.1.3 & \textbf{R2F3.1.3}: A seguito del completamento della procedura di registrazione viene
			mostrata la mnemonic phrase associata al nuovo account creato 			& No & No \tabularnewline
	TA2F3.2 & \textbf{R2F3.2}: L'utente può richiedere il salvataggio su file delle credenziali dell'account creato durante la procedura di registrazione					& No & No \tabularnewline

	%login
	TA1F4 & \textbf{R1F4}: Un utente può effettuare il login 										& No & No \tabularnewline
	TA1F4.1 & \textbf{R1F4.1}: Un utente si può autenticare manualmente tramite l'utilizzo 
			del comando \textit{login} 													& No & No \tabularnewline
	TA1F4.1.1 & \textbf{R1F4.1.1}: Per completare la procedura di login manuale l'utente deve inserire
			 il proprio address														& No & No \tabularnewline
	TA1F4.1.2 & \textbf{R1F4.1.2}: Per completare la proceduta di login manuale l'utente deve inserire
			 la propria private key 												& No & No \tabularnewline
	TA2F4.1.3 & \textbf{R2F4.1.3}: L'utente può decidere di completare la procedura di login manuale 
			 utilizzando la propria mnemonic phrase al posto della private key		& No & No \tabularnewline
	TA2F4.2 & \textbf{R2F4.2}: Durante la procedura di login manuale l'utente può richiedere che
			 le proprie credenziali siano memorizzate per accessi futuri 			& No & No \tabularnewline
	TA2F4.3 & \textbf{R2F4.3}: L'utente si può autenticare tramite login automatico 					& No & No \tabularnewline
	
	% logout
	TA1F5 & \textbf{R1F5}: L'utente può effettuare il logout 										& No & No \tabularnewline
	
	% whoami
	TA2F6 & \textbf{R2F6}: L'utente può richiedere di visualizzare l'address 
			associato alla sessione corrente 										& No & No \tabularnewline
	
	% info
	TA1F7 & \textbf{R1F7}: L'utente può richiedere di visualizzare la descrizione dettagliata di una funzione
		tramite il comando \textit{info}													& No & No \tabularnewline
	TA1F7.1 & \textbf{R1F7.1}: Per visualizzare la descrizione di una funzione l'utente deve inserire 
		il nome della funzione di interesse											& No & No \tabularnewline
	TA1F7.2 & \textbf{R1F7.2}: Nel caso in cui l'utente richieda di visualizzare la descrizione di una 
		funzione non presente nel sistema, deve essere mostrato un messaggio di
		errore															 			& No & No \tabularnewline
	
	% search 
	TA2F8 & \textbf{R2F8}: Il sistema deve permettere all'utente di cercare una funzione 
		attraverso una keyword 														& No & No \tabularnewline
	TA2F8.1 & \textbf{R2F8.1}: Per effettuare la ricerca è necessario che l'utente inserisca 
		una keyword 																& No & No \tabularnewline
	TA2F8.2 & \textbf{R2F8.2}: A seguito di una ricerca il sistema deve mostrare la lista di
	 tutte le funzioni che presentano la keyword indicata 
	 all'interno del proprio nome													& No & No \tabularnewline
	TA2F8.2.1 & \textbf{R2F8.2.1}: La visualizzazione di un risultato di ricerca include
		 la firma della funzione													& No & No \tabularnewline
  	TA2F8.2.2 & \textbf{R2F8.2.2}: La visualizzazione di un risultato di ricerca include
		  il costo di esecuzione della funzione										& No & No \tabularnewline
  	TA2F8.2.3 & \textbf{R2F8.2.3}: La visualizzazione di un risultato di ricerca include
		  l'address del creatore della funzione									& No & No \tabularnewline
	TA2F8.3 & \textbf{R2F8.3}: Se una ricerca non porta a nessun risultato deve essere mostrato un 
		messaggio di errore 														& No & No \tabularnewline	
	
	% run 
	TA1F9 & \textbf{R1F9}: L'utente deve essere in grado di eseguire le funzioni messe a 
		disposizione da \textit{Etherless} attraverso il comando \textit{run} 				& No & No \tabularnewline
	TA1F9.1 & \textbf{R1F9.1}: Per eseguire una funzione è necessario inserire il relativo nome 		& No & No \tabularnewline
	TA1F9.1.1 & \textbf{R1F9.1.1}: Nel caso in cui il nome inserito a seguito del comando \textit{run} non 
		corrisponda ad alcuna funzione presente nel sistema, deve essere 
		visualizzato un messaggio di errore											& No & No \tabularnewline 
	TA1F9.2 & \textbf{R1F9.2}: L'esecuzione di una funzione necessita dell'inserimento dei parametri necessari per la sua esecuzione 															& No & No \tabularnewline
	TA1F9.2.1 & \textbf{R1F9.2.1}: Se l'utente tenta di eseguire una funzione inserendo un numero 
		di parametri che non coincide con quanto richiesto, deve essere 
		visualizzato un messaggio di errore 										& No & No \tabularnewline
	TA1F9.2.2 & \textbf{R1F9.2.2}: Se l'utente tenta di eseguire una funzione inserendo almeno un parametro	con tipo differente da 
		quanto indicato nella firma della funzione, deve essere visualizzato
		un messaggio di errore 										& No & No \tabularnewline
	TA1F9.3 & \textbf{R1F9.3}: A seguito dell'esecuzione di una funzione il sistema deve mostrare 
		all'utente i relativi risultati 											& No & No \tabularnewline
	TA1F9.4 & \textbf{R1F9.4}: Nel caso in cui l'utente richieda di eseguire una funzione senza 
		avere credito sufficiente, deve essere mostrato un messaggio di errore		& No & No \tabularnewline
	
	% list  
	TA1F10 & \textbf{R1F10}: L'utente deve essere in grado di visualizzare tutte le funzioni 
		disponibili in \textit{Etherless} tramite il comando \textit{list} 				& No & No \tabularnewline
	TA2F10.1 & \textbf{R2F10.1}: L'utente può richiede di visualizzare solo le funzioni da 
		lui caricate tramite l'utilizzo di un apposito flag 						& No & No \tabularnewline
	TA1F10.2 & \textbf{R1F10.2}: La visualizzazione di un elemento della lista ottenuta a seguito 
		del comando \textit{list} include la firma della funzione 							& No & No \tabularnewline
	TA1F10.3 & \textbf{R1F10.3}: La visualizzazione di un elemento della lista ottenuta a seguito 
		del comando \textit{list} include il costo di esecuzione della funzione 			& No & No \tabularnewline
	TA1F10.4 & \textbf{R1F10.4}: La visualizzazione di un elemento della lista ottenuta a seguito 
		del comando \textit{list} include il creatore della funzione 						& No & No \tabularnewline
	TA1F10.5 & \textbf{R1F10.5}: Nel caso in cui il risultato del comando \textit{list} sia vuoto, deve 
		essere visualizzato un apposito messaggio 									& No & No \tabularnewline
	
	%deploy
	TA1F11 & \textbf{R1F11}: L'utente deve essere in grado di eseguire il deploy\ped{\textit{G}} di una propria
		funzione all'interno della piattaforma \textit{Etherless} 					& No & No \tabularnewline
	TA1F11.1 & \textbf{R1F11.1}: Per eseguire il deploy\ped{\textit{G}} l'utente deve inserire il percorso del file 
		contenente il codice della funzione 										& No & No \tabularnewline
	TA2F11.1.1 & \textbf{R2F11.1.1}: Se il formato del file indicato durante la procedura di deploy\ped{\textit{G}} 
	non è supportato dall'applicativo deve essere mostrato un messaggio di errore																		& No & No \tabularnewline
	TA1F11.1.2 & \textbf{R1F11.1.2}: Se il file indicato durante la procedura di deploy\ped{\textit{G}} non esiste, 
		deve essere visualizzato un messaggio di errore								& No & No \tabularnewline
	TA1F11.2 & \textbf{R1F11.2}: Per eseguire il deploy\ped{\textit{G}} l'utente deve inserire il nome della 
		funzione considerata 														& No & No \tabularnewline
	TA1F11.2.1 & \textbf{R1F11.2.1}: Nel caso in cui il nome della funzione di cui si tenta di fare 
		il deploy\ped{\textit{G}} sia troppo lungo, deve essere visualizzato un messaggio di errore & No & No \tabularnewline
	TA1F11.2.2 & \textbf{R1F11.2.2}: Nel caso in cui il nome della funzione di cui si tenta di fare 
		il deploy\ped{\textit{G}} sia già usato nel sistema, deve essere visualizzato un messaggio 
		di errore																	& No & No \tabularnewline
	TA2F11.3 & \textbf{R2F11.3}: Per eseguire il deploy\ped{\textit{G}} l'utente deve inserire una descrizione 
		della funzione 																& No & No \tabularnewline
	TA2F11.3.1 & \textbf{R2F11.3.1}: Se la descrizione inserita durante la procedura di deploy\ped{\textit{G}} supera la 
		lunghezza massima, deve essere mostrato un messaggio di errore 				& No & No \tabularnewline
	TA1F11.4 & \textbf{R1F11.4}: Nel caso in cui l'utente tenti di eseguire il deploy\ped{\textit{G}} di una funzione
		senza avere il credito necessario, deve essere visualizzato un messaggio 
		di errore & No & No \tabularnewline

	% modify 
	TA1F12 & \textbf{R1F12}: L'utente deve essere in grado di modificare le informazioni relative 
		ad una funzione da lui caricata 											& No & No \tabularnewline
	TA1F12.1 & \textbf{R1F12.1}: Per eseguire la procedura di modifica è necessario che l'utente 
		indichi il nome della funzione che vuole modificare 						& No & No \tabularnewline
	TA1F12.1.1 & \textbf{R1F12.1.1}: Nel caso in cui, durante la procedura di modifica, l'utente 
		inserisca il nome di una funzione non presente all'interno della piattaforma
		\textit{Etherless}, deve essere mostrato un messaggio di errore				& No & No \tabularnewline
	TA1F12.1.2 & \textbf{R1F12.1.2}: Nel caso in cui, durante la procedura di modifica, l'utente 
		inserisca il nome di una funzione che non è di sua proprietà, deve essere 
		mostrato un messaggio di errore												& No & No \tabularnewline
	TA1F12.2 & \textbf{R1F12.2}: Il sistema deve permettere all'utente di modificare la descrizione 
		associata ad una propria funzione 											& No & No \tabularnewline
	TA1F12.2.1 & \textbf{R1F12.2.1}: L'utente deve visualizzare un errore nel caso in cui, durante 
		la procedura di modifica, venga inserita una descrizione di lunghezza
		superiore a quella massima consentita 										& No & No \tabularnewline
	TA1F12.3 & \textbf{R1F12.3}: Il sistema deve permettere all'utente di aggiornare il codice di 
		una propria funzione 														& No & No \tabularnewline	
	TA1F12.3.1 & \textbf{R1F12.3.1}: Se il file indicato durante la procedura di aggiornamento del 
		codice di una funzione non esiste, deve essere mostrato un messaggio di 
		errore																		& No & No \tabularnewline
	TA1F12.3.2 & \textbf{R1F12.3.2}: Se il file indicato durante la procedura di aggiornamento del codice di una funzione presenta un formato
	non supportato, deve essere	mostrato un messaggio di errore						& No & No \tabularnewline
		
	% history
	TA2F13 & \textbf{R2F13}: L'utente deve essere in grado di visualizzare la propria cronologia 
		di richieste di esecuzione 													& No & No \tabularnewline
	TA2F13.1 & \textbf{R2F13.1}: L'utente deve poter essere in grado di richiedere di visualizzare
		solo una porzione della propria cronologia di esecuzione 					& No & No \tabularnewline
	TA2F13.2 & \textbf{R2F13.2}: La visualizzazione di un elemento della cronologia include 
		l'identificativo della richiesta di esecuzione 								& No & No \tabularnewline
	TA2F13.3 & \textbf{R2F13.3}: La visualizzazione di un elemento della cronologia include 
		il nome della funzione richiesta 											& No & No \tabularnewline				
	TA2F13.4 & \textbf{R2F13.4}: La visualizzazione di un elemento della cronologia include 
		il valore dei parametri indicati nella chiamata alla funzione				& No & No \tabularnewline
	TA2F13.5 & \textbf{R2F13.5}: La visualizzazione di un elemento della cronologia include 
		il risultato della richiesta di esecuzione									& No & No \tabularnewline
	TA2F13.6 & \textbf{R2F13.6}: La visualizzazione di un elemento della cronologia include 
		la data e orario della richiesta 											& No & No \tabularnewline
	
	% delete
	TA1F14 & \textbf{R1F14}: L'utente deve essere in grado di eliminare una funzione da lui caricata & No & No \tabularnewline
	TA1F14.1 & \textbf{R1F14.1}: Per eseguire l'operazione di eliminazione l'utente deve inserire 
		il nome della funzione da eliminare 										& No & No \tabularnewline
	TA1F14.1.1 & \textbf{R1F14.1.1}: Nel caso in cui il nome inserito durante la procedura di eliminazione
		non si riferisca ad alcuna funzione presente all'interno del sistema, deve 
		essere mostrato un messaggio di errore										& No & No \tabularnewline
	TA1F14.1.2 & \textbf{R1F14.1.2}: Nel caso in cui la funzione considerata nella procedura di eliminazione
		non sia di proprietà dell'utente, deve essere visualizzato un messaggio 
		di errore																	& No & No \tabularnewline

\end{longtable}
				\subsubsection{Test di qualità}

\def\arraystretch{1.75}
\rowcolors{2}{lightRowColor}{darkRowColor}
\begin{longtable}{ 
		>{\centering}p{0.1\textwidth} 
		>{}p{0.5\textwidth} 
		>{\centering}p{0.17\textwidth}
		>{\centering}p{0.12\textwidth} }
	
	\caption{Tabella dei test di qualità} \\ 
	\coloredTableHead
	\textbf{\color{white}Test} & 
	\centering\textbf{\color{white}Requisito e Descrizione} & 
	\centering\textbf{\color{white}Implementato} &
	\textbf{\color{white}Superato} 
	\endfirsthead
	 
 	\rowcolor{white}\caption[]{(continua)}\\
	\coloredTableHead 
	\textbf{\color{white}Test} &
	\centering\textbf{\color{white}Requisito e Descrizione} &
	\centering\textbf{\color{white}Implementato} &
	\textbf{\color{white}Superato} 
	\endhead
	
	% contenuto tabella
	TA1Q1 & \textbf{R1Q1}: La progettazione e la codifica devono rispettare le norme e 
			le metriche definite nei documenti 
			\textit{Norme di Progetto v1.0.0} 
			e \textit{Piano di Qualifica v1.0.0} 							& No & No \tabularnewline
	TA1Q2 & \textbf{R1Q2}: Il sistema deve essere pubblicato con licenza MIT 				& No & No \tabularnewline
	TA1Q3 & \textbf{R1Q3}: Il codice sorgente di \textit{Etherless} deve essere pubblicato
			e versionato usando Github o GitLab 							& No & No \tabularnewline
	TA1Q4 & \textbf{R1Q4}: Deve essere redatto un manuale sviluppatore 						& No & No \tabularnewline
	TA1Q4.1 & \textbf{R1Q4.1}: Il manuale sviluppatore deve contenere le informazioni per
				eseguire e fare il deploy dei moduli						& No & No \tabularnewline
	TA1Q5 & \textbf{R1Q5}: Deve essere redatto un manuale utente 							& No & No \tabularnewline
	TA1Q5.1 & \textbf{R1Q5.1}: Il manuale utente deve contenere tutte le informazioni
				necessarie all'utente finale per utilizzare correttamente 
				il sistema 													& No & No \tabularnewline
	TA1Q6 & \textbf{R1Q6}: La documentazione per l'utilizzo del software deve essere 
		 	scritta in lingua inglese. 										& No & No \tabularnewline
	TA1Q7 & \textbf{R1Q7}: Nella scrittura del codice JavaScript deve essere seguita 
			la guida sullo stile di programmazione Airbnb JavaScript 
			style guide 													& No & No \tabularnewline
	TA1Q8 & \textbf{R1Q8}: Lo sviluppo del codice JavaScript deve essere supportato 
			dal software di analisi statica del codice ESLint 				& No & No \tabularnewline

\end{longtable}


				\pagebreak
				\subsubsection{Test di vincolo}

\def\arraystretch{1.75}
\rowcolors{2}{lightRowColor}{darkRowColor}
\begin{longtable}{ 
		>{\centering}p{0.1\textwidth} 
		>{}p{0.5\textwidth} 
		>{\centering}p{0.17\textwidth}
		>{\centering}p{0.12\textwidth} }
	
	\caption{Tabella dei test di qualità} \\ 
	\coloredTableHead
	\textbf{\color{white}Test} & 
	\centering\textbf{\color{white}Requisito e Descrizione} & 
	\centering\textbf{\color{white}Implementato} &
	\textbf{\color{white}Superato} 
	\endfirsthead
	
	\rowcolor{white}\caption[]{(continua)}\\
	\coloredTableHead 
	\textbf{\color{white}Test} &
	\centering\textbf{\color{white}Requisito e Descrizione} &
	\centering\textbf{\color{white}Implementato} &
	\textbf{\color{white}Superato} 
	\endhead
	
	% contenuto tabella 
	TA1V1 & \textbf{R1V1}: Gli smart contract devono essere scritti in Solidity 			& No & No \tabularnewline
	TA1V2 & \textbf{R1V2}: Gli smart contract devono poter essere aggiornati 				& No & No \tabularnewline
	TA1V3 & \textbf{R1V3}: L'applicativo deve essere sviluppato utilizzando TypeScript 3.6 	& No & No \tabularnewline
	TA1V3.1 & \textbf{R1V3.1}: Deve essere utilizzato il meccanismo delle promise/async-await 
			come approccio principale 										& No & No \tabularnewline
	TA1V4 & \textbf{R1V4}: Il modulo \textit{Etherless-server} deve essere implementato 
			utilizzando il Framework Serverless 							& No & No \tabularnewline
	TA1V5 & \textbf{R1V5}: Il progetto deve utilizzare i seguenti ambienti di sviluppo: 
			ambiente di sviluppo locale, ambiente di testing e ambiente 
			di staging 														& No & No \tabularnewline
	TA2V5.1 & \textbf{R2V5.1}: Gli ambienti per la fase di sviluppo locale e testing possono 
			fare utilizzo della rete testrpc fornita dal framework Truffle  & No & No \tabularnewline
	TA2V5.2 & \textbf{R2V5.2}: Per la fase di staging è desiderabile l'utilizzo della rete 
			Ethereum Ropsten												& No & No \tabularnewline
	TA1V5.3 & \textbf{R1V5.3}: Durante la fase di staging l'applicativo deve essere 
			pubblicamente accessibile 										& No & No \tabularnewline
	TA1V5.4 & \textbf{R1V5.4}: Al termine del progetto il prodotto deve essere pronto 
			per la produzione 												& No & No \tabularnewline
	TA3V5.4.1 & \textbf{R3V5.4.1}: L'ambiente di produzione deve fare utilizzo dell'Ethereum
			main network 													& No & No \tabularnewline
	TA3V6 & \textbf{R3V6}: Il pagamento deve essere gestito tramite un meccanismo di escrow	& No & No \tabularnewline
		
\end{longtable}