% pacakges 
\usepackage{geometry}
\usepackage{hyperref} %  link
\usepackage{graphicx} %immagini
\usepackage[shortlabels]{enumitem} % per elenchi personalizzati

% package per la lingua / caratteri 
\usepackage[italian]{babel} 
\usepackage[utf8]{inputenc}
\usepackage[T1]{fontenc}

\usepackage{fancyhdr} % per header e footer 
\usepackage{lastpage} % per avere l'indice dell'ultima pagina 

\usepackage{tabularx} % tabelle
\usepackage[table]{xcolor} % definizione colore di sfondo per tabelle
\usepackage{longtable} % permette di estendere le tabelle su più pagine  

\usepackage{chngcntr} % per numerazione immagini e tabelle 

% configurazione
% link
\hypersetup{
	colorlinks=true,
	linkcolor=black,
	filecolor=magenta,      
	urlcolor=blue,
}

% intestazione e piè di pagina 
\pagestyle{fancy}
% intestazione 
\setlength{\headheight}{25pt}
\lhead{ \includegraphics[scale=0.05]{./res/img/cropped_logo.png} } 
\rhead{ \docTitle } 
% piè di pagina \\
\renewcommand{\footrulewidth}{0.4pt} % per avere una linea nel footer
\cfoot{}
\rfoot{Pagina \thepage{} di \pageref{LastPage}}

% tabelle 
\def\arraystretch{1.5} % padding 
% comandi e colori tabelle
\definecolor{lightRowColor}{HTML}{fafafa}
\definecolor{darkRowColor}{HTML}{ffcccb}

\newcommand{\coloredTableHead}{\rowcolor[HTML]{b61827}}
\newcommand{\lightTableRow}{\rowcolor{lightRowColor}}
\newcommand{\darkTableRow}{\rowcolor{darkRowColor}}

% per numerazione immagini e tabelle 
% --> la numerazione dipende dalla subsection in cui ci si trova 
\counterwithin{table}{subsection}
\counterwithin{figure}{subsection}


