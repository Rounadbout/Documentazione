\subsection{Capitolato C3 - NaturalAPI}

\subsubsection{Informazioni generali}
	\begin{itemize}
		\item \textbf{Nome:} NaturalAPI; 
		\item \textbf{Proponente\ped{\textit{G}}:} teal.blue;
		\item \textbf{Committente\ped{\textit{G}}:} \TV{} e \RC{}. 
	\end{itemize}

\subsubsection{Descrizione}
Creare un toolkit\ped{\emph{G}} per restringere il divario presente tra le specifiche di progetto e le API\ped{\emph{G}}, permettendo così ai programmatori di dedicarsi principalmente allo sviluppo di nuove funzionalità piuttosto che provare a replicare il modello aziendale. 

\subsubsection{Obiettivo finale}
Il toolkit\ped{\emph{G}} richiesto si dovrà comporre di tre parti: 
\begin{itemize}
	\item \textbf{NaturalAPI Discover:} si occupa di identificare ed estrarre da documenti di testo potenziali entità di business (oggetti/nomi), processi (azioni/verbi) e predicati. In questo modo viene creato il Business Domain Language\ped{\emph{G}} (BDL\ped{\emph{G}});  
	\item \textbf{NaturalAPI Design:} si occupa di creare un Business Application Language\ped{\emph{G}} (BAL\ped{\emph{G}}) a partire dai documenti in Gherkin\ped{\emph{G}}; 
	\item \textbf{NaturalAPI Develop:} converte il BAL\ped{\emph{G}} in API\ped{\emph{G}} scritte in uno dei linguaggi di programmazione e framework\ped{\emph{G}} supportati; il tutto gestendo sia la creazione di nuove repository\ped{\textit{G}} per il codice, sia la modifica di quelle già esistenti. 
\end{itemize}

\subsubsection{Tecnologie coinvolte}
	\begin{itemize}
		\item \textbf{Natural language processing:} il toolkit\ped{\emph{G}} utilizza tecniche di elaborazione del linguaggio naturale\ped{\emph{G}}; 
		\item \textbf{Gherkin}\ped{\emph{G}}\textbf{:} lo standard principale per scrivere in linguaggio naturale\ped{\emph{G}} specifiche di funzionalità; 
		\item \textbf{Cucumber}\ped{\emph{G}}\textbf{:} framework\ped{\emph{G}} per eseguire test automatici per BDD\ped{\emph{G}} 
		(Behavior-driven development\ped{\emph{G}}); 
		\item \textbf{Swagger}\ped{\emph{G}}\textbf{:} tool di generazione di codice che si basa su OpenAPI\ped{\textit{G}}. 
	\end{itemize}

\subsubsection{Aspetti positivi}
	\begin{itemize}
		\item Non viene richiesta un'interfaccia grafica, lasciando quindi più tempo al team di dedicarsi allo sviluppo delle funzionalità richieste; 
		\item il capitolato\ped{\emph{G}} è molto dettagliato e descrive in maniera precisa il comportamento richiesto dal toolkit\ped{\emph{G}}.
	\end{itemize}

\subsubsection{Criticità}
\begin{itemize}
	\item Il capitolato\ped{\emph{G}} non ha suscitato molto interesse nel gruppo; 
	\item gli argomenti trattati sono stati considerati complessi e di difficile comprensione. 
\end{itemize}

\subsubsection{Esito}
Lo scopo del capitolato\ped{\emph{G}} non è risultato molto stimolante. Si è quindi deciso di non considerare questa proposta. 
