\subsection{Capitolato C3 - NaturalAPI}

\subsubsection{Informazioni generali}
	\begin{itemize}
		\item \textbf{Nome:} NaturalAPI; 
		\item \textbf{Proponente:} teal.blue;
		\item \textbf{Committente:} \TV{} e \RC{}. 
	\end{itemize}

\subsubsection{Descrizione}
Creare un toolkit per restringere il divario presente tra le specifiche di progetto e le API, permettendo così ai programmatori di dedicarsi principalmente allo sviluppo di nuove funzionalità 
piuttosto che provare a replicare il modello aziendale. 

\subsubsection{Obiettivo finale}
Il toolkit richiesto si dovrà comporre di tre parti: 
\begin{itemize}
	\item \textbf{NaturalAPI Discover:} si occupa di identificare ed estrarre da documenti di testo potenziali entità di business (oggetti/nomi), processi (azioni/verbi)
	e predicati. In questo modo viene creato il Business Domain Language (BDL);  
	\item \textbf{NaturalAPI Design:} si occupa di creare un Business Application Language (BAL) a partire dai documenti in Gherkin; 
	\item \textbf{NaturalAPI Develop:} converte il BAL in API scritte in uno dei linguaggi di programmazione e framework supportati; il tutto gestendo sia la creazione di nuove repository per il codice, sia la modifica di quelle già esistenti. 
\end{itemize}

\subsubsection{Tecnologie coinvolte}
	\begin{itemize}
		\item \textbf{Natual language processing:} il toolkit utilizza tecniche di elaborazione del linguaggio naturale; 
		\item \textbf{Gherkin:} lo standard principale per scrivere in linguaggio naturale specifiche di funzionalità; 
		\item \textbf{Cucumber:} framework per eseguire test automatici per BDD 
		(Behavior-driven development); 
		\item \textbf{Swagger:} tool di generazione di codice che si basa su OpenAPI. 
	\end{itemize}

\subsubsection{Aspetti positivi}
	\begin{itemize}
		\item Non viene richiesta un'interfaccia grafica, lasciando quindi più tempo al team di dedicarsi allo sviluppo delle funzionalità richieste; 
		\item il capitolato è molto dettagliato e descrive in maniera precisa il comportamento richiesto dal toolkit.
	\end{itemize}

\subsubsection{Criticità}
\begin{itemize}
	\item Il capitolato non ha suscitato molto interesse nel gruppo; 
	\item gli argomenti trattati sono stati considerati complessi e di difficile comprensione. 
\end{itemize}

\subsubsection{Esito}
Lo scopo del capitolato non è risultato molto stimolante. Si è quindi deciso di non considerare questa proposta. 
