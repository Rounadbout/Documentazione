\subsection{Capitolato C1 - Autonomous Highlights Platform}

\subsubsection{Informazioni generali}
	\begin{itemize}
		\item \textbf{Nome:} Autonomous Highlights Platform; 
		\item \textbf{Proponente:} Zero12;
		\item \textbf{Committente:} \TV{} e \RC{}. 
	\end{itemize}

\subsubsection{Descrizione}
Creare una piattaforma web che riceve in input dei video di eventi sportivi come una partita di calcio, F1, Moto GP e crea in autonomia un video di massimo 5 minuti contenente i suoi momenti salienti. 

\subsubsection{Obiettivo finale}
Viene richiesto che la piattaforma sia dotata di un modello di machine learning in grado di identificare ogni momento importante dell’evento.
Il flusso di generazione di tale video deve essere così strutturato:
	\begin{itemize}
		\item caricamento del video;
		\item identificazione dei momenti salienti;
		\item estrazione delle corrispondenti parti di video;
		\item generazione del video di sintesi. 
	\end{itemize}

\subsubsection{Tecnologie coinvolte}
	\begin{itemize}
		\item \textbf{Elastic Container Service o Elastic Kubernetes Service:} è un servizio di orchestrazione di contenitori altamente dimensionabile ad elevate
		prestazioni; 
		\item \textbf{DynamoDB:} database NoSQL dalle alte performance ideale per la conservazione di tag o altre informazioni a supporto dell'applicativo; 
		\item \textbf{AWS Transcode:} servizio gestito per la conversione ed elaborazione di diversi formati video; 
		\item \textbf{Sage Maker:} servizio completamente gestito che copre l'intero flusso di lavoro dell'apprendimento automatico per etichettare e preparare i
		dati, scegliere un algoritmo, formare il modello, ottimizzarlo per la distribuzione, effettuare previsioni e intraprendere azioni; 
		\item \textbf{AWS Rekognition video:} servizio di analisi video basato su apprendimento approfondito; è in grado di riconoscere i movimenti delle persone
		in un fotogramma e di riconoscere soggetti, volti, oggetti, celebrità e contenuti inappropriati. 
	\end{itemize}

\subsubsection{Aspetti positivi}
	\begin{itemize}
		\item Tutte le tecnologie coinvolte sono ben documentate; 
		\item il proponente zero12 fornisce attività di formazione sulle principali tecnologie AWS e wireframe dell’interfaccia della console web di analisi e controllo dello stato di elaborazione dei video per la generazione del video di highlights.
	\end{itemize}

\subsubsection{Criticità}
	\begin{itemize}
		\item Il proponente non fornisce alcun data-set per effettuare il training del modello di machine learning; 
		\item l'identificazione manuale di tutti i momenti principali di un evento sportivo è un'azione ripetitiva e costosa in termini di tempo. 
	\end{itemize}

\subsubsection{Esito}
Il capitolato non è risultato molto stimolante, in quanto lo sviluppo di alcune componenti sembra caratterizzato da attività ripetitive. Si è quindi deciso di non considerare questa proposta. 
