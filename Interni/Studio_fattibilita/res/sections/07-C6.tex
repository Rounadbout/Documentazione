\subsection{Capitolato d'appalto C6: ThiReMa - Things Relationship Management}
	\subsubsection{Introduzione al capitolato}
	\begin{itemize}
		\item \textbf{Nome:} ThiReMa - Things Relationship Management
		\item \textbf{Proponente:} Sanmarco Informatica
		\item \textbf{Committenti:} \TV, \RC
	\end{itemize}
	\subsubsection{Descrizione}
	Si vule applicare l'analisi predittiva su un insieme di dati eterogenei provenienti da un complesso di dispositivi IoT. Tali dispositivi, potenzialmente di natura distinta, comunicano con un database centralizzato dove verranno memorizzati ed elaborati i valori raccolti allo scopo di fornire un servizio di manutenzione preventiva.  
	\subsubsection{Obiettivo Finale}
	Il prodotto finale rappresenta una web-app che permetta di valutare la correlazione tra dati operativi, ovvero le misure, e i fattori influenzanti.\\
	Tale applicazione deve permettere di memorizzare ed elaborare una grande quantità di misure eterogenee raccolte da numerosi dispositivi collegati alla rete.
	\begin{itemize}
	\item La memorizzazione deve essere veloce ed economica in termini di performance, vista la numerosità di dispositivi che si vogliono gestire.
	\item L'analisi dei dati deve poter fornire semplici previsioni sull'andamento dei dati.
	Viene richiesto inoltre un sistema automatizzato che segnali in anticipo la necessità di manutenzione di un certo dispositivo.	
	\end{itemize}
	È richiesta la suddivisione della web-app in 3 macro-sezioni:
	\begin{itemize}
		\item Censimento dei sensori e dei relativi dati
		\item Modulo di analisi di correlazione
		\item Modulo di monitoraggio per ente
	\end{itemize}
	L'interfaccia utente è suddivisa in 2 parti:
	\begin{itemize}
	\item \textbf{L’interfaccia di gestione}: consente agli utenti autorizzati il monitoraggio remoto del cluster, inclusa la possibilità di attivare e disattivare singoli nodi, il censimento dei sensori, e la definizione degli utenti e delle loro autorizzazioni.
	\item \textbf{L’interfaccia di interrogazione} permette di seguire graficamente in tempo reale l’andamento di uno o più sensori, eventualmente aggregati, e visualizzare alcuni dati di interesse, ad es. numero di errori/giorno ecc.
	\end{itemize}
	\subsection{Tecnologie Coinvolte}
	\begin{itemize}
		\item \textbf{Java:} linguaggio di programmazione consgliato per lo sviluppo delle componenti Kafka.
		\item \textbf{Apache Kafka:} piattaforma a bassa latenza ed alta velocità per la gestione di feed dati in tempo reale.
		\item \textbf{Time-Series Database:} Database in cui l'ordine temporale della raccolta dei dati è fondamentale. I database consigliati sono:
			\begin{itemize}
			\item \textbf{PostgreSQL}
			\item \textbf{TimescaleDB5}
			\item \textbf{ClickHouse6}
			\end{itemize}
		\item \textbf{Bootstrap:} Sviluppo Front End
	\end{itemize}
	\subsubsection{Aspetti Positivi}
	Questo progetto rappressenta un'opportunità di imparare molto sul mondo dell'IoT, una tecnologia sempre più utilizzata. Offre inoltre l'opportunità di migliorare le proprie conoscenze su argomenti parzialmente conosciuti dal team, come lo sviluppo server-side.
	\subsubsection{Criticità}
	Lo scarso interesse da parte dei membri del team e la poca conoscenza del mondo dell'IoT e le relative tecnologie coinvolte.
	\subsubsection{Esito}
	Vista la non disponibilità del progetto non è stato possibile candidarsi.
		
		