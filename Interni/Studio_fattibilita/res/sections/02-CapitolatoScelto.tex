\section{Capitolato scelto C2 - \NomeProgetto}

\subsection{Informazioni generali}
	\begin{itemize}
		\item \textbf{Nome:} \NomeProgetto{}; 
		\item \textbf{Proponente}\ped{\textit{G}}\textbf{:} \Proponente{};
		\item \textbf{Committente}\ped{\textit{G}}\textbf{:} \TV{} e \RC{}. 
	\end{itemize}

\subsection{Descrizione}
Etherless è una piattaforma cloud\ped{\emph{G}} che permette agli sviluppatori di effettuare il deploy\ped{\emph{G}} di funzioni Javascript\ped{\emph{G}} al suo interno e far pagare gli utenti finali per la loro esecuzione da remoto, tramite l'utilizzo di smart-contract\ped{\emph{G}}. Etherless si basa sull'integrazione di due tecnologie: Serverless\ped{\emph{G}} e Ethereum\ped{\emph{G}}. Ethereum\ped{\emph{G}} viene utilizzato per la gestione dei pagamenti e delle richieste di esecuzione di funzioni. Serverless\ped{\emph{G}}, invece, per l'effettiva esecuzione di tali funzioni. La comunicazione tra queste due tecnologie è resa possibile tramite l'ascolto e l'emissione di eventi\ped{\emph{G}} Ethereum\ped{\emph{G}}.

\subsection{Obiettivo finale}
La struttura richiesta per Etherless si compone di 3 moduli\ped{\emph{G}} principali: 
\begin{itemize}
	\item \textbf{Etherless-cli}\ped{\textit{G}}\textbf{:} è il modulo\ped{\emph{G}} con cui gli sviluppatori interagiscono con Etherless. Supporta diversi comandi, tramite i quali lo sviluppatore può: 
		\begin{itemize}
			\item configurare il proprio account; 
			\item eseguire il deploy\ped{\textit{G}} di funzioni Javascript\ped{\emph{G}}; 
			\item visualizzare la lista delle funzioni di cui ha già svolto il deploy\ped{\textit{G}}; 
			\item richiedere l'esecuzione di una funzione e visualizzarne il risultato; 
			\item visualizzare i log\ped{\textit{G}} relativi all'esecuzione di una specifica funzione; 
		\end{itemize}  
	A seconda del comando, \textit{Etherless-cli\ped{\textit{G}}} si occuperà di emettere o rimanere in attesa di un determinato evento\ped{\emph{G}} Ethereum\ped{\emph{G}}. 
	
	\item \textbf{Etherless-smart:} un insieme di Ethereum\ped{\emph{G}} smart-contract\ped{\emph{G}} che gestiscono la comunicazione tra \textit{Etherless-cli\ped{\textit{G}}} e \textit{Etherless-server}, ed i pagamenti in ETH\ped{\emph{G}} necessari per l'esecuzione delle funzioni; 
	\item \textbf{Etherless-server:} si occupa di ascoltare gli eventi\ped{\emph{G}} trasmessi da \textit{Etherless-smart} e di avviare l'esecuzione delle funzioni così richieste. I risultati ottenuti vengono inviati tramite un ulteriore evento\ped{\emph{G}} nella blockchain\ped{\emph{G}} e mostrati all'utente attraverso \textit{Etherless-cli\ped{\textit{G}}};  
\end{itemize}

\subsection{Tecnologie coinvolte}
	\begin{itemize}
		\item \textbf{Ethereum}\ped{\emph{G}}\textbf{:} piattaforma che permette ai suoi utenti di scrivere applicazioni decentralizzate (dette \DJ Apps\ped{\emph{G}}) che usano la tecnologia blockchain\ped{\emph{G}};
		\item \textbf{Ethereum}\ped{\emph{G}} \textbf{Virtual Machine}\ped{\emph{G}} \textbf{(EVM}\ped{\emph{G}}\textbf{):} macchina virtuale distribuita sulla rete Ethereum\ped{\emph{G}}; 
		\item \textbf{Solidity}\ped{\textit{G}}\textbf{:} linguaggio usato per la codifica degli smart-contract\ped{\emph{G}}; 
		\item \textbf{Serverless}\ped{\emph{G}} \textbf{Framework}\ped{\emph{G}}\textbf{:} framework\ped{\emph{G}} per la realizzazione di applicazioni che vengono eseguite in architetture serverless\ped{\emph{G}} (come AWS\ped{\emph{G}} Lambda, Google Cloud\ped{\emph{G}} Functions, Microsoft Azure Functions); 
		\item \textbf{Typescript}\ped{\emph{G}} \textbf{3.6:} Linguaggio che supporta Javascript\ped{\emph{G}} ES6, introducendo alcune funzionalità, tra cui la tipizzazione;
		\item \textbf{ESLint}\ped{\emph{G}}\textbf{:} tool di analisi statica del codice e syntax checking; 
		\item \textbf{AWS}\ped{\emph{G}} \textbf{Lambda:} consente di eseguire codice in risposta a determinati eventi\ped{\emph{G}}, senza dover effettuare il provisioning né gestire server.
	\end{itemize}

\subsection{Aspetti positivi}
	\begin{itemize}
		\item La blockchain\ped{\emph{G}} e le tecnologie serverless\ped{\emph{G}} sono argomenti che hanno suscitato un notevole interesse generale negli ultimi anni; per questo la relativa conoscenza potrebbe essere positiva da un punto di vista curricolare; 
		\item la conoscenza di Typescript\ped{\emph{G}} e Javascript\ped{\emph{G}} è molto richiesta, oltre ad essere la base di svariate librerie e framework\ped{\emph{G}}; 
		\item il tema principale del progetto è stato accolto con molto interesse dal gruppo, che si è dimostrato disponibile ad approfondire l'argomento. 
	\end{itemize}

\subsection{Criticità}
	\begin{itemize}
		\item L'azienda ha sede all'estero, quindi la comunicazione con i referenti sarà meno agevole rispetto ad aziende che si trovano nel territorio nazionale; 
		\item il capitolato\ped{\emph{G}} richiede l'utilizzo di tecnologie recenti, la cui documentazione potrebbe non essere pienamente esaustiva e il cui apprendimento richiederà una mole di studio non indifferente. 
	\end{itemize}

\subsection{Esito}
  Il progetto è stato subito accolto caldamente dai componenti del gruppo, le sfide tecnologiche sono stimolanti e gli argomenti trattati sono di interesse. Con questo progetto tutti i componenti del team avranno l’opportunità di studiare un campo dell’informatica che all’università è poco trattato, potendo aggiungere al proprio bagaglio curricolare delle voci particolarmente interessanti.