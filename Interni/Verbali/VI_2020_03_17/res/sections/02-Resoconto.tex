\section{Verbale}
	
	\subsection{Riepilogo lavoro svolto}
		Ciascun membro del gruppo espone il lavoro svolto ed esplicita i problemi incontrati, chiedendo supporto per quelli da risolvere. Si è posta particolare attenzione alla stesura dei processi nelle \NdP{} e al rispetto degli standard nei vari documenti.\\
		I Verificatori dopo una prima visione degli elaborati di Analisti e Amministratori propongono modifiche volte a migliorare i documenti redatti.
		
		
	\subsection{Nomenclatura documenti}
		\'E stato discusso sulle regole di nomenclatura dei documenti.
		
		
	\subsection{Gestione Glossario}
		Il gruppo ha definito le modalità di redazione del \Glossario{}. Esso deve essere un documento che dovrà contenere una raccolta di vocaboli meno comuni con la relativa definizione.\\
		Ciascun vocabolo che comparirà nel \Glossario{} sarà identificato nei vari documenti con una ’G’ a pedice.


	\subsection{Specifiche tecniche del capitolato C2 - \textit{Etherless}}
		Come annunciato la scorsa riunione, \EG{} procede con un'esposizione approfondita delle specifiche tecniche del capitolato\ped{\textit{G}} 2.\\
		Viene quindi spiegato agli altri membri del gruppo in maniera più dettagliata il funzionamento richiesto per la piattaforma 
		\textit{Etherless}. 
		
	
	\subsection{Domande per incontro RedBabel}
		In vista della prossima riunione esterna è stata proposta la creazione di un file condiviso in Microsoft Teams\ped{\textit{G}} contenente le domande da porre al Proponente\ped{\textit{G}} del capitolato\ped{\textit{G}} C2, rappresentato da Alessandro Maccagnan.


	\subsection{Prossima riunione}
		La prossima riunione è stata fissata come segue: 
		\begin{itemize}
			\item \textbf{Luogo: } chiamata tramite Skype\ped{\textit{G}}; 
			\item \textbf{Data: } 2020-03-18; 
			\item \textbf{Ora di inizio: } 11:30;
			\item \textbf{Tipologia: } riunione esterna con Alessandro Maccagnan di RedBabel (capitolato\ped{\textit{G}} 2).
		\end{itemize}