\section{Verbale}
\subsection{Riepilogo post-meeting}
Il gruppo ha ragionato sui punti di interesse emersi durante il colloquio tramite piattaforma Zoom con il \TV{} e i membri dei gruppi 14 e 15. Particolare attenzione è stata data alla necessità di verificare, direttamente con i proponenti dei vari capitolati, eventuali variazioni della loro disponibilità. A tale scopo si è deciso di consultare gli altri gruppi interessati per organizzare in maniera ordinata il contatto con i proponenti.

\subsection{Pianificazione scadenze e responsabilità}
Data l'intenzione del gruppo di partecipare alla Revisione dei Requisiti (RR) prevista per il 2020-04-20 e con consegna documenti 2020-04-13 si è deciso di stabilire varie scadenze ideali per l'avanzamento dei lavori:
\begin{description}
	\item[2020-03-14:] verifica ed approvazione Studio di fattibilità;
	\item[2020-03-17:] avere iniziato la stesura della prima parte delle Norme di Progetto e per poi iniziare la loro prima verifica;
	\item[2020-04-06:] data prevista per il completamento dei lavori sulla Documentazione RR.
\end{description}
Sono state inoltre stabilite varie responsabilità immediate per le figure del gruppo:
\begin{description}
	\item[Responsabile di progetto:] informarsi sul Piano di Progetto con particolare attenzione alla rendicontazione delle ore di lavoro e dei costi;
	\item[Amministratore di progetto:] stesura Norme di Progetto;
	\item[Verificatori:] aggiornare costantemente il Glossario a seguito delle attività di verifica dei documenti.
\end{description}

\subsection{Specifiche tecniche del capitolato C2 - \NomeProgetto}
\EG{}, che ha provveduto ad informarsi ed analizzare il capitolato scelto, ha illustrato brevemente al gruppo le specifiche tecniche di base legate a tale progetto. Un'analisi più approfondita è prevista per la riunione successiva.

\subsection{Decisioni organizzative aggiuntive}
Allo scopo di garantire a tutti i membri del gruppo un'esperienza varia nella realizzazione del progetto, si è deciso di effettuare una rotazione dei ruoli ogni due settimane.
 

\subsection{Prossima riunione}
La prossima riunione è stata fissata come segue: 
\begin{itemize}
	\item \textbf{Luogo: } videochiamata tramite Microsoft Teams; 
	\item \textbf{Data: } 2020-03-17; 
	\item \textbf{Ora di inizio: } 10.00
\end{itemize}