\section{Verbale}

	\subsection{Discussione relativa a materiale e scadenze RP}
	È stato fatto il punto della situazione per quanto riguarda il materiale che sarà necessario approntare in vista della Revisione di Progettazione (RP), identificando come elementi: \textit{Technology Baseline}\ped{\textit{G}} e documentazione associata, in aggiunta ad un'eventuale revisione della documentazione di progetto già presentata, secondo il giudizio del Committente\ped{\textit{G}}.
	Sono state valutate le date ufficiali per la consegna dei documenti (2020-05-18) e colloqui per la \textit{Technology Baseline}\ped{\textit{G}} (circa dal 2020-05-06 al 2020-05-08) in relazione al lavoro da svolgere.

	\subsection{Aggiornamento sulla rotazione dei ruoli del gruppo}
	In conformità con quanto previsto dal \PdP{} i componenti del gruppo si apprestano al cambio di ruolo di progetto, in vista dell'inizio del nuovo periodo.

	\subsection{Accordo sulla gestione del repository}
	Al fine di garantire un'adeguata gestione del repository\ped{\textit{G}} Documentazione, che andrà espanso ed aggiornato per consentire l'archiviazione di componenti software, si è deciso di valutare l'impiego dei Git\ped{\textit{G}} Submodules. La decisione definitiva sulla loro implementazione è prevista per la prossima riunione, così da consentire ai membri del gruppo di informarsi su tali tools.

	\subsection{Definizione di un obiettivo e suddivisione del lavoro}
	Si è poi svolta una discussione volta a chiarire nel dettaglio le caratteristiche del \textit{Proof of Concept} (PoC), che sarà necessario presentare in sede di Revisione. L'obiettivo generico identificato è la realizzazione di un'architettura di base, contenente un'integrazione di tutte le tecnologie e framework\ped{\textit{G}} precedentemente individuati.
	Per poter raggiungere tale obiettivo il gruppo ha concordato una divisione in 3 sotto-gruppi, che verranno assegnati allo sviluppo (base) di ognuno dei 3 moduli\ped{\textit{G}} di Etherless, secondo questa modalità:
	\subsubsection*{Etherless-CLI}
	\begin{itemize}
		\item \EG{};
		\item \FJ{};
		\item \MP{}.
	\end{itemize}
	\subsubsection*{Etherless-Smart}
	\begin{itemize}
		\item \VB{};
		\item \NF{}.
	\end{itemize}
	\subsubsection*{Etherless-Server}
	\begin{itemize}
		\item \LB{};
		\item \AS{};
		\item \AZ{}.
	\end{itemize}
	\pagebreak %Evita che l'elenco porti l'ultimo elemento nella pagina successiva
	
	\noindent Considerata anche la necessità di operare sulla documentazione da produrre durante questo periodo, è stato stabilito che due componenti dei sotto-gruppi più popolosi portino a termine tale mansione in parallelo al compito principale. È naturalmente previsto un supporto da parte degli altri componenti del sotto-gruppo, così da alleggerire il loro carico di lavoro.
	\subsection{Prossima riunione}
		La prossima riunione è stata fissata come segue:
		\begin{itemize}
			\item \textbf{Luogo: } chiamata tramite Microsoft Teams\ped{\textit{G}};
			\item \textbf{Data: } 2020-04-22;
			\item \textbf{Ora di inizio: } 9.30;
			\item \textbf{Tipologia: } riunione interna.
		\end{itemize}
