\section{Verbale}

	\subsection{Aggiornamento sull'apprendimento delle tecnologie}
	Ogni membro del gruppo ha aggiornato gli altri componenti sul proprio apprendimento delle tecnologie e su eventuali sviluppi da lui compiuti in uno dei tre moduli che compone \textit{Etherless}. 
	
	\subsection{Decisione della libreria da usare per l'interazione con la blockchain Ethereum}
	A seguito di diverse discussioni e confronti è stato deciso di usare la libreria \textit{Ethers.js} per l'interazione con la blockchain Ethereum; tale decisione è dovuta a: 
		\begin{itemize}
			\item buona documentazione; 
			\item supporto nativo a TypeScript; 
			\item frequente manutenzione. 
		\end{itemize}
	\noindent In associazione ad \textit{Ethers.js} si è deciso di usare il framework \textit{Waffle} per la creazione di smart-contract.
	
	\subsection{Identificazione dei requisiti da soddisfare nel PoC}
	Sono stati identificati i requisiti che il gruppo si impegna a soddisfare per il Proof Of Concept, 
	in particolare: 
		\begin{itemize}
			\item procedura di login (R1F4.1); 
			\item signup (R1F3); 
			\item esecuzione di una funzione (R1F9). 
		\end{itemize}
	Nel caso in cui il soddisfacimento di tali requisiti richieda meno tempo del previsto, si è deciso di valutarne altri, in maniera da permettere al gruppo di approfondire ed applicare maggiormente le tecnologie considerate. 
	
	\subsection{Discussione su come strutturare gli smart-contract utilizzati da \textit{Etherless}}
	A seguito di un confronto si è deciso di memorizzare all'interno del modulo \textit{Etherless-smart} alcune informazioni relative ad ogni funzione, come: nome, proprietario e costo. In questo modo \textit{Etherless-smart} è in grado di effettuare appositi controlli durante procedure quali: esecuzione e rimozione di funzioni. 
	Per essere sicuri che tale soluzione non sia troppo onerosa, saranno eseguiti alcuni test nella testnet Ethereum Ropsten. 
	
	\subsection{Analisi della funzionalità di deploy}
	Sono state prese in considerazione diverse proposte su come gestire la funzionalità di deploy messa a disposizione dal prodotto. Pur non avendo finalizzato nel dettaglio l'approccio da usare, una prima idea prevede che  \textit{Etherless-cli} comunichi ad \textit{Etherless-smart} un hash crittografico dei file in cui è codificata la funzione. In questo modo \textit{Etherless-server} è in grado di identificare in maniera univoca i file da accettare. \\
	Tale argomento sarà trattato più nel dettaglio nei prossimi incontri. 
	
	\subsection{Prossima riunione}
		La prossima riunione è stata fissata come segue:
		\begin{itemize}
			\item \textbf{Luogo: } chiamata tramite Microsoft Teams; 
			\item \textbf{Data: } 2020-04-26;
			\item \textbf{Ora di inizio: } 15.30;
			\item \textbf{Tipologia: } riunione interna.
		\end{itemize}
