\section{Verbale}
\subsection{Presentazione dei membri del gruppo}
Ogni membro si è presentato e ha esposto la propria situazione universitaria agli altri. 

\subsection{Decisione del nome e logo del gruppo}
A seguito di un breve brainstorming è stato deciso all'unanimità il nome del gruppo: \Gruppo.  Successivamente \AZ{} si è occupato della creazione del relativo logo. 

\subsection{Creazione indirizzo e-mail del gruppo}
\AZ{} è stato incaricato di creare un indirizzo e-mail del gruppo per comunicare con il docente e con i proponenti dei capitolati\ped{\textit{G}}. 
L'indirizzo e-mail scelto è: \Mail.

\subsection{Scelta del capitolato}
Ogni componente ha esposto quelli che, secondo il proprio giudizio, erano gli aspetti positivi e negativi di ogni progetto candidato. Questo confronto ha permesso al gruppo di stilare una lista dei tre capitolati\ped{\textit{G}} più apprezzati; in ordine di importanza decrescente: 
\begin{enumerate}
	\item Etherless (capitolato\ped{\textit{G}} 2)
	\item Predire in Grafana\ped{\textit{G}} (capitolato\ped{\textit{G}} 4)
	\item Autonomous Highlights\ped{\textit{G}} Platform (capitolato\ped{\textit{G}} 1)
\end{enumerate}
Il gruppo ha poi deciso che il progetto riguarderà il capitolato\ped{\textit{G}} C2, ovvero \NomeProgetto. 

\subsection{Strumenti per la documentazione} 
Il gruppo ha deciso di usare: 
\begin{itemize}
	\item \textbf{Git}\ped{\textit{G}}\textbf{:} come strumento di controllo del versionamento\ped{\textit{G}}; 
	\item \textbf{Github}\ped{\textit{G}}\textbf{:} come piattaforma di hosting; 
	\item \textbf{\LaTeX{}}\ped{\textit{G}}\textbf{:} come strumento per realizzare in modo collaborativo la documentazione. 
\end{itemize}

\subsection{Decisioni organizzative}
Il gruppo ha convenuto di utilizzare Microsoft Teams\ped{\textit{G}} per la gestione della comunicazione, in quanto esso permette di: 
\begin{itemize}
	\item organizzare videochiamate tra i vari membri del gruppo; 
	\item trasferire file; 
	\item dividere le conversazioni in base all'argomento di interesse.  
\end{itemize}
È stato inoltre deciso di svolgere almeno due riunioni settimanali: il martedì e il venerdì. Eventuali impegni che si sovrappongono alle riunioni già programmate devono essere comunicati al gruppo con anticipo. 

\subsection{Prossima riunione}
La prossima riunione è stata fissata come segue: 
\begin{itemize}
	\item \textbf{Luogo: } videochiamata tramite Microsoft Teams\ped{\textit{G}}; 
	\item \textbf{Data: } 2020-03-13; 
	\item \textbf{Ora di inizio: } 15.45 .
\end{itemize}