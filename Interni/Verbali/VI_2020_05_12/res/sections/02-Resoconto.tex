\section{Verbale}

	\subsection{Informazioni da inserire negli smart contract}
	Per ogni funzione di cui è stato fatto il deploy\ped{\textit{G}} il gruppo ha deciso di mantenere nello smart contract\ped{\textit{G}} informazioni quali il nome della funzione, il suo prezzo e l'indirizzo del suo sviluppatore. Per il momento è stata rimandata la discussione sull'inclusione, lato smart contract\ped{\textit{G}} oppure lato server, della descrizione della funzione e della segnatura della stessa.
	
	\subsection{Metodo di pagamento per l'esecuzione delle funzioni}
	E' stato deciso di implementare il metodo di pagamento escrow\ped{\textit{G}} alla richiesta di esecuzione di una funzione, in modo da poter restituire all'utente la somma pagata in caso tale esecuzione non vada a buon fine.
	
	\subsection{Procedura di deploy}
	La procedura di deploy\ped{\textit{G}} avviene come segue:
		\begin{itemize}
			\item lato CLI\ped{\textit{G}}, l'utente carica il file contenente la funzione oppure uno zip contenente diversi file (da decidere);
			\item il modulo cli calcola l'hash di tale file/zip e lo trasmette ad Etherless-smart assieme alla quantità di Ether\ped{\textit{G}} necessaria al deploy\ped{\textit{G}};
			\item il modulo\ped{\textit{G}} Etherless-smart trasmette al modulo\ped{\textit{G}} Etherless-server tale hash;
			\item nello stesso momento, anche il modulo Etherless-CLI trasmette al modulo Etherless-server il file contenente la funzione/lo zip contenente i vari file;
			\item il modulo\ped{\textit{G}} Etherless-server calcola l'hash sui file ricevuti da Etherless-CLI e li confronta con l'hash ricevuto da Etherless-smart. Se questo corrisponde, viene fatto il deploy\ped{\textit{G}} della funzione su AWS\ped{\textit{G}} Lambda\ped{\textit{G}}.	
		\end{itemize}
	Resta tuttavia ancora da decidere quale canale scegliere per effettuare il deploy (DynamoDB\ped{\textit{G}} o il servizio AWS\ped{\textit{G}} Serverless\ped{\textit{G}} Application Repository\ped{\textit{G}}).
	
	\subsection{Presentazione per la RP}
	I vari componenti del gruppo hanno esposto le slide che hanno pensato di includere per la presentazione della RP che avverrà in data 2020-05-18.
		
		
	\subsection{Prossima riunione}
		La prossima riunione è stata fissata come segue:
		\begin{itemize}
			\item \textbf{Luogo: } chiamata tramite Microsoft Teams\ped{\textit{G}}; 
			\item \textbf{Data: } 2020-05-15;
			\item \textbf{Ora di inizio: } 15.00;
			\item \textbf{Tipologia: } riunione interna.
		\end{itemize}