\section{Verbale}

	\subsection{Aggiornamento sull'apprendimento delle tecnologie}
		Ciascun membro del team ha esposto agli altri il punto in cui si trova riguardo l'apprendimento delle tecnologie e sono stati chiariti alcuni dubbi espressi riguardo gli sviluppi apportati ai tre moduli \textit{Etherless-cli}, \textit{Etherless-smart} ed \textit{Etherless-server}.
		
	\subsection{Preparazione dell'incontro con il Proponente\ped{\textit{G}}}
		In vista dell'incontro con il Proponente\ped{\textit{G}} sono stati individuati i dubbi da chiarire e si è deciso di riportare tutti i quesiti da porre in un foglio di lavoro condiviso in Microsoft Teams\ped{\textit{G}}.
		
	\subsection{Discussione esito della RR} 
		Si è discusso sull'esito della RR, analizzando tutte le segnalazioni fatte e cercando di capire come risolvere e correggere gli errori.
		Poiché sono state individuate alcune segnalazioni non molto chiare ai componenti del gruppo si è deciso di richiedere un colloquio al Committente\ped{\textit{G}}, in modo da poter chiarire tutti i dubbi riscontrati e da poter correggere al meglio gli errori segnalati.
	
	\subsection{Organizzazione della repository\ped{\textit{G}} per la correzione dei documenti}
		Per la correzione dei documenti si è deciso di aprire dei branch\ped{\textit{G}} appositi denominati [siglaDelNomeDelDocumento]\_[X], dove [X] potrebbe essere il nome della modifica oppure il codice relativo ad un'eventuale issue\ped{\textit{G}} nella quale viene spiegata più nel dettaglio la modifica che si vuole apportare.
		Al termine di ciascuna modifica, chi si è occupato di apportarla pubblicherà su GitHub\ped{\textit{G}} il lavoro svolto e aprirà una pull-request, assegnando la verifica della pull-request ad un Verificatore; quest'ultimo avrà il compito di verificarla oppure di segnalare eventuali errori riscontrati, in modo che il membro del gruppo che ha apportato la modifica possa correggerli.
	
	\subsection{Organizzazione dei ruoli}
		Si è discusso sulla distribuzione dei ruoli all'interno del team, in particolare sui ruoli che ciascun componente andrà a ricoprire a seguito della prossima rotazione dei ruoli prevista per il 2020-05-04.
		
	\subsection{Prossima riunione}
		La prossima riunione è stata fissata come segue:
		\begin{itemize}
			\item \textbf{Luogo: } chiamata tramite Zoom\ped{\textit{G}}; 
			\item \textbf{Data: } 2020-04-30;
			\item \textbf{Ora di inizio: } 18.00;
			\item \textbf{Tipologia: } riunione esterna con il Proponente\ped{\textit{G}}.
		\end{itemize}

