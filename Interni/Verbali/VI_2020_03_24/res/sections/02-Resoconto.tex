\section{Verbale}
	
	\subsection{Riepilogo lavoro svolto}
		Ciascun membro del gruppo ha esposto al team il punto in cui si trova il documento di propria competenza. \\
		Riguardo le metriche di qualità abbiamo deciso di ricercare per poter adottare in seguito un'approccio che ci permette di essere il più tecnici e precisi possibile.
		
		
	\subsection{Registro delle modifiche}
		Ci siamo resi conto che il nome per la sezione "Diario delle modifiche" non è corretta e che va modificata con "Registro delle modifiche" anche per i documenti precedentemente redatti.

	\subsection{Comunicazioni interne}
		Per quanto riguarda le comunicazioni interne tra componenti che stanno lavorando ad uno stesso documento abbiamo deciso di continuare ad utilizzare i canali messi a disposizione da Microsoft Teams, utilizzando un canale per ciascun documento ancora da completare.
		
	\subsection{Prossimo incontro con RedBabel}
		Abbiamo deciso di richiedere un incontro con i proponenti per chiarire alcuni dubbi riguardo alcune tecnologie che potrebbero essere utilizzate per lo svilupo del progetto. \\
		I dubbi espressi ed eventuali altri dubbi che potrebbero sorgere fino al momento dell'incontro vanno scritti in un foglio di lavoro condiviso creato nella sezione apposita di Miscrosoft Teams, in modo da evitare di fare più volte le stesse domande. \\
		L'incontro è previsto per il 2020-03-27 alle ore 11.30 utilizzando una piattaforma per videochiamate ancora da concordare con i Proponenti.

	\subsection{Rotazione dei ruoli}
		Dopo aver analizzato i documenti che devono essere ancora ultimati abbiamo creato un foglio Excel condiviso in Microsoft Teams per decidere come assegnare i ruoli a seguito della rotazione prevista per il 2020-03-26. \\
		È risultato necessario prevedere l'assegnazione del ruolo di Progettista per la redazione del \PdQ.

	\subsection{Prossima riunione}
		La prossima riunione è stata fissata come segue: 
		\begin{itemize}
			\item \textbf{Luogo: } Chiamata tramite Microsoft Teams; 
			\item \textbf{Data: } 2020-03-26; 
			\item \textbf{Ora di inizio: } 10.00
			\item \textbf{Tipologia: } Riunione interna
		\end{itemize}