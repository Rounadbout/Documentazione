\section{Verbale}

	\subsection{Aggiornamento relativo al lavoro svolto entro la data attuale}
	Il gruppo ha discusso del lavoro finora svolto, seguendo la lista degli obiettivi decisi per la data attuale. Nello specifico:
	\begin{itemize}
		\item \textbf{Funzionalità implementate con successo}: sono state implementate con successo le funzionalità di: \texttt{edit}, \texttt{init} e parzialmente \texttt{history}; 
		\item \textbf{Installazione dell'applicativo}: l'installazione dell'applicativo, tramite il comando \texttt{npm install TeamRoundabout/etherless -g} è oramai prossima al completamento. La finalizzazione è stata stabilita come obiettivo per la prossima milestone;
		\item \textbf{Implementazione dei messaggi di errore}: l'implementazione dei messaggi di errore relativi alle varie tipologie di errore riscontrabili nell'utilizzo dell'applicativo richiede ancora del lavoro. Il gruppo ha quindi valutato la possibilità di fornire più forza lavoro per raggiungere questo obiettivo, e fissato il suo avanzamento o completamento per la prossima milestone;
		\item \textbf{Funzionalità di deploy con dipendenze}: a seguito di uno studio e varie prove svolte da alcuni membri del gruppo, questi hanno potuto esporre delle proposte per la realizzazione della funzionalità di deploy con dipendenze. È stato quindi stabilito di iniziare a codificare o effettuare test più concreti per questo scopo, entro la prossima milestone.
		\item \textbf{Test di Sistema e Test di Accettazione}: iniziata una prima stesura di proposte per i Test di Sistema e Test di Accettazione, il gruppo ne ha discusso e valutato i contenuti.
	\end{itemize}
	\subsection{Correzioni ed incrementi ai documenti}
		In luce delle segnalazioni rilevate nella valutazione RQ, il gruppo ha stabilito corrispondenti Issue, volte a sanare i difetti riscontrati. Entro la prossima milestone si è deciso di:
		\begin{itemize}
			\item Procedere con correzioni strutturali allo \textit{User Manual};
			\item Ultimare gli incrementi preliminari al \textit{Developer Manual};
			\item Ultimare l'aggiornamento dei consuntivi nel \PdP{};
			\item Procedere con correzioni ed incrementi al \PdQ{}.
		\end{itemize}
	\subsection{Prossimo incontro con il Proponente}
	Il gruppo ha deciso di contattare il Proponente\ped{\textit{G}} il prima possibile, per fissare un incontro di aggiornamento all'inizio della prossima settimana.
	\pagebreak
	\subsection{Prossima riunione}
		La prossima riunione è stata fissata come segue:
		\begin{itemize}
			\item \textbf{Luogo:} chiamata tramite Microsoft Teams\ped{\textit{G}}; 
			\item \textbf{Data:} 2020-07-06;
			\item \textbf{Ora di inizio:} 10.00;
			\item \textbf{Tipologia:} riunione interna.
		\end{itemize}