\section{Verbale}
	\subsection{Aggiornamento dei tre moduli}
		Ogni sotto-gruppo ha esposto lo stato attuale del modulo\textit{\ped{G}} di propria competenza.	In particolare ci siamo concentrati sull'esposizione di quanto sviluppato e sulla rappresentazione grafica delle scelte architetturali effettuate.
		
		\subsubsection{Argomenti trattati}
			Durante l'esposizione sopra citata, sono emersi alcuni dubbi relativamente i seguenti argomenti:
			\begin{itemize}
				\item per la funzionalità di 'deploy', si è evidenziata la necessità di fornire un prezzo, da far pagare allo sviluppatore, per il caricamento della funzione all'interno del server AWS Lambda\ped{\textit{G}}. Dopo una discussione del gruppo, valutando attentamente le indicazioni del Proponente, si è convenuto far pagare allo sviluppatore un prezzo fisso per una qualsiasi funzione. Il prezzo differenziato riferito a diverse tipologie di funzioni è stato deciso di considerarlo di futura estensione del prodotto;
				\item si è discusso relativamente alle risposte da ritornare nel caso di differenti tipologie di richieste. Abbiamo scelto di ritornare solamente il nome delle funzioni che hanno subito gli effetti delle funzionalità 'delete' e 'deploy';
				\item nel caso di errori all'interno del server, è emersa la necessità di restituire la somma pagata per un servizio che effettivamente non è stato fornito. Abbiamo discusso sull'argomento e la scelta è stata quella di restituire il valore corrisposto nel caso siano sorti degli errori, mentre non viene previsto il rimborso nel caso in cui l'intero processo ha successo senza la generazione di alcun tipo di errore.
			\end{itemize}
		
	\subsection{Preparazione alla Product Baseline}
		In vista della presentazione della \textit{Product Baseline} è stato creato lo scheletro della presentazione. Inoltre si è proceduto a definire una struttura solida, efficace ed efficiente, considerato anche che gli argomenti da trattare sono consistenti e il rischio di dispersione è elevato.\\
		Abbiamo quindi scelto di suddividere la presentazione in 4 parti:
		\begin{itemize}
			\item Introduzione generale e visione d'insieme dell'architettura;
			\item Etherless-cli;
			\item Etherless-smart;
			\item Etherless-server.
		\end{itemize}
		Sono state definite delle scadenze interne per la stesura delle parti e la conseguente prova della presentazione.
	
	
	\subsection{Prossima riunione}
		La prossima riunione è stata fissata come segue:
		\begin{itemize}
			\item \textbf{Luogo:} chiamata tramite Microsoft Teams\ped{\textit{G}}; 
			\item \textbf{Data:} 2020-06-03;
			\item \textbf{Ora di inizio:} 21.00;
			\item \textbf{Tipologia:} riunione interna.
		\end{itemize}