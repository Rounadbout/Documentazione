\section{Verbale}
	\subsection{Aggiornamento dei tre moduli}
		Ogni sotto-gruppo ha esposto lo stato attuale del modulo\textit{\ped{G}} di propria competenza.	In particolare ci siamo concentrati sull'esposizione di quanto sviluppato e sulla rappresentazione grafica delle scelte architetturali effettuate.
		
		\subsubsection{Argomenti trattati}
			Durante l'esposizione sopra citata, sono emersi alcuni dubbi riguardo ai seguenti argomenti:
			\begin{itemize}
				\item per il deployment\ped{\textit{G}} di funzioni si è evidenziata la necessità di decidere un prezzo da far pagare allo sviluppatore per il caricamento della propria funzione all'interno del servizio AWS Lambda\ped{\textit{G}}. Dopo una discussione il gruppo ha convenuto di far pagare allo sviluppatore un prezzo fisso per il deployment\ped{\textit{G}} di una qualsiasi funzione;
				\item per l'esecuzione di una funzione si è deciso di far pagare un prezzo fisso all'utente. L'introduzione di un prezzo variabile a seconda delle risorse effettivamente usate dalla funzione stessa viene considerata come futura estensione del prodotto\ped{\textit{G}}; 
				\item si è discusso relativamente alle risposte che il modulo Etherless-smart deve creare nel caso di differenti tipologie di richieste. Si è deciso di utilizzare un unico evento Ethereum\ped{\textit{G}} di risposta, caratterizzato da un identificativo univoco e una stringa di risposta;
				\item nel caso di errori all'interno del server, è emersa la necessità di restituire all'utente la somma pagata per un servizio che effettivamente non è stato fornito. Si è scelto di restituire tale somma all'utente finale nel caso in cui sorgano problemi a livello server durante l'esecuzione di una funzione.
			\end{itemize}
		
	\subsection{Preparazione alla Product Baseline}
		In vista della presentazione della \textit{Product Baseline} è stato creato lo scheletro della presentazione. Inoltre si è proceduto a definire una struttura solida ed efficace, considerato anche che gli argomenti da trattare sono consistenti e il rischio di dispersione è elevato.\\
		Abbiamo quindi scelto di suddividere la presentazione in 4 parti:
		\begin{itemize}
			\item Introduzione generale e visione d'insieme dell'architettura;
			\item Etherless-cli;
			\item Etherless-smart;
			\item Etherless-server.
		\end{itemize}
		Sono state definite delle scadenze interne per la stesura delle parti e la conseguente prova della presentazione.
	
	\subsection{Prossima riunione}
		La prossima riunione è stata fissata come segue:
		\begin{itemize}
			\item \textbf{Luogo:} chiamata tramite Microsoft Teams\ped{\textit{G}}; 
			\item \textbf{Data:} 2020-06-05;
			\item \textbf{Ora di inizio:} 15.30;
			\item \textbf{Tipologia:} riunione interna.
		\end{itemize}