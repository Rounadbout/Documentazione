\section{Verbale}
	
	\subsection{Riepilogo lavoro svolto}
		Ciascun membro del gruppo espone il lavoro svolto ed espone i problemi incontrati, chiedendo supporto per quelli da risolvere.
		\begin{itemize}
			\item I Verificatori chiariscono le correzioni da apportare trovate durante la visione dei documenti.
			\item Gli Amministratori evidenziano le modifiche apportate al documento \textit{Norme di Progetto} a seguito dell'operato dei Verificatori.
			\item Il Responsabile di Progetto espone la situazione attuale riguardo la stesura del documento \textit{Piano di Progetto}
			\item Gli Analisti espongono la situazione attuale riguardo la stesura del documento \textit{Analisi dei Requisiti}
		\end{itemize}
		
		
	\subsection{Scelte sulla stesura dei documenti}
		\'E stato discusso su dettagli decisionali per la stesura dei documenti, in particolare sulla nomenclatura del testo, alternanza tra corsivo, grassetto e variazioni tra maiuscole e minuscole.\\
		Lo scopo di questa discussione è quella di avere i documenti redatti il più uniforme possibile, nonostante vengano scritti da diverse persone.
		
		
	\subsection{Proposta per i futuri incontri RedBabel}
		Premesso che il primo incontro con i Proponenti svolto mercoledì 18 marzo è stato effettuato tramite una videochiamata Skype.\\
		Viste le difficoltà di connessione avute da diversi membri del gruppo, abbiamo deciso di suggerire ai Proponenti di trovare un nuovo mezzo di comunicazione. Le scelte potrebbero vertere sull'applicativo \textbf{Zoom} o \textbf{Hangouts}.


	\subsection{Prossima riunione}
		La prossima riunione è stata fissata come segue: 
		\begin{itemize}
			\item \textbf{Luogo: } Chiamata tramite Microsoft Teams; 
			\item \textbf{Data: } 2020-03-24; 
			\item \textbf{Ora di inizio: } 10:00
			\item \textbf{Tipologia: } Riunione interna
		\end{itemize}