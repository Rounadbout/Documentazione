\section{Verbale}
	
	\subsection{Riepilogo lavoro svolto}
		Ciascun membro del gruppo ha esposto il lavoro svolto ed i problemi incontrati, chiedendo supporto per quelli di complicata risoluzione:
		\begin{itemize}
			\item I Verificatori hanno chiarito le problematiche rilevate durante la revisione dei documenti ed eventuali correzioni da apportare.
			\item Gli Amministratori hanno evidenziato le modifiche apportate al documento \textit{Norme di Progetto}, a seguito dell'operato dei Verificatori.
			\item Il Responsabile di Progetto ha esposto l'attuale stato della stesura del documento \textit{Piano di Progetto}.
			\item Gli Analisti hanno esposto l'attuale stato della stesura del documento \textit{Analisi dei Requisiti}.
		\end{itemize}
		
		
	\subsection{Scelte sulla stesura dei documenti}
		Abbiamo poi discusso alcuni dettagli legati agli standard di stesura dei documenti, in particolare: la gestione della nomenclatura all'interno testo, l'alternanza tra corsivo, grassetto e le variazioni tra maiuscole e minuscole.\\
		Lo scopo di questa discussione è stato quello di ottenere, in futuro, dei documenti redatti nel modo più uniforme possibile, nonostante vengano scritti da diversi componenti del gruppo.
		
		
	\subsection{Proposta per i futuri incontri RedBabel}
		Premesso che il primo incontro con i Proponenti svolto mercoledì 18 marzo è stato effettuato tramite una videochiamata Skype.\\
		Viste le difficoltà di connessione riscontrate da diversi membri del gruppo, abbiamo deciso di suggerire ai Proponenti di trovare un nuovo mezzo di comunicazione. Le scelte potrebbero vertere sull'applicativo \textbf{Zoom} o \textbf{Google Hangouts}.


	\subsection{Prossima riunione}
		La prossima riunione è stata fissata come segue: 
		\begin{itemize}
			\item \textbf{Luogo: } Chiamata tramite Microsoft Teams; 
			\item \textbf{Data: } 2020-03-24; 
			\item \textbf{Ora di inizio: } 10:00
			\item \textbf{Tipologia: } Riunione interna
		\end{itemize}