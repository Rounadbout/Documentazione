\section{Verbale}
	
	\subsection{Riepilogo lavoro svolto}
		Ciascun membro del gruppo ha esposto il lavoro svolto ed i problemi incontrati, chiedendo supporto per quelli di complicata risoluzione:
		\begin{itemize}
			\item i Verificatori hanno chiarito le problematiche rilevate durante la revisione dei documenti ed eventuali correzioni da apportare;
			\item gli Amministratori hanno evidenziato le modifiche apportate al documento \NdP{}, a seguito dell'operato dei Verificatori;
			\item il Responsabile di Progetto ha esposto l'attuale stato della stesura del documento \PdP{};
			\item gli Analisti hanno esposto l'attuale stato della stesura del documento \AdR{}.
		\end{itemize}
		
		
	\subsection{Scelte sulla stesura dei documenti}
		Abbiamo poi discusso alcuni dettagli legati agli standard di stesura dei documenti, in particolare: la gestione della nomenclatura all'interno testo, l'alternanza tra corsivo, grassetto e le variazioni tra maiuscole e minuscole.\\
		Lo scopo di questa discussione è stato quello di ottenere, in futuro, dei documenti redatti nel modo più uniforme possibile, nonostante vengano scritti da diversi componenti del gruppo.
		
		
	\subsection{Proposta per i futuri incontri RedBabel}
		Premesso che il primo incontro con il Proponente\ped{\textit{G}} svolto mercoledì 18 marzo è stato effettuato tramite una videochiamata Skype\ped{\textit{G}}.\\
		Viste le difficoltà di connessione riscontrate da diversi membri del gruppo, abbiamo deciso di suggerire al Proponente\ped{\textit{G}} di trovare un nuovo mezzo di comunicazione. Le scelte potrebbero vertere sull'applicativo Zoom\ped{\textit{G}} o Google Hangouts.


	\subsection{Prossima riunione}
		La prossima riunione è stata fissata come segue: 
		\begin{itemize}
			\item \textbf{Luogo: } chiamata tramite Microsoft Teams\ped{\textit{G}}; 
			\item \textbf{Data: } 2020-03-24; 
			\item \textbf{Ora di inizio: } 10:00;
			\item \textbf{Tipologia: } riunione interna.
		\end{itemize}