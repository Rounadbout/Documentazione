\section{Verbale}
	
	\subsection{Struttura della presentazione per la Technology Baseline}
	Si è discusso sul contenuto della presentazione per la \textit{Technology Baseline}\ped{\textit{G}}, che si terrà come concordato con il \RC{} il 2020-05-07 alle ore 13.10. La decisione finale è quella di aprire la presentazione con un discorso generale sulla struttura di \textit{Etherless}\ped{\textit{G}} e sui requisiti che la \textit{PoC} soddisfa, per poi inoltrarsi in un discorso più approfondito che riguarda le tecnologie scelte e come queste vengano usate all'interno del progetto (includendo anche ad esempio snippet di codice).
	
	\subsection{Struttura e contenuto dei ReadMe}
	Come richiesto dal Proponente\ped{\textit{G}} per ogni modulo vengono inclusi dei \textit{ReadMe} in lingua inglese su GitHub\ped{\textit{G}} per guidare l'utente nell'uso dell'applicativo. La struttura di questi \textit{ReadMe} dovrà consistere di una prima parte denominata \textit{Instructions} (un semplice elenco puntato dei passi necessari all'installazione di quel modulo) e poi di una parte descrittiva in cui vengono aggiunte ulteriori informazioni.
	
		\subsection{Stato delle modifiche dei documenti}
	Ogni componente del gruppo ha esposto agli altri a che punto si trova con il lavoro. È stato inoltre fatto il punto della situazione per quanto rigurda i documenti che contengono informazioni utili ad altri membri del gruppo.
			
	\subsection{Prossima riunione}
		La prossima riunione è stata fissata come segue:
		\begin{itemize}
			\item{\textbf{Luogo:} chiamata tramite Zoom\ped{\textit{G}}; }
			\item{\textbf{Data:} 2020-05-06;}
			\item{\textbf{Ora di inizio:} 14.30;}
			\item{\textbf{Tipologia:} riunione interna.}
		\end{itemize}

