\section{Verbale}

	\subsection{Discussione problemi legati al Piano di Progetto consegnato}
		È stato rilevato da alcuni membri del gruppo il mancato aggiornamento dell'appendice "Riscontro rischi" del documento \PdP{} \textit{2.0.0} consegnato in funzione della RP.
		La discussione all'interno del gruppo ha permesso di identificare la fonte di tale mancanza, legata ad un'incorretta gestione del meccanismo di pull request di GitHub\ped{\textit{G}} da parte di chi si è occupato della verifica di tale documento.
		Nello specifico, la suddivisione delle modifiche da apportare al documento in due feature branch paralleli ha generato gravi conflitti tra i file modificati, irrisolvibili tramite semplice risoluzione dei conflitti fornita da GitHub\ped{\textit{G}}.
		Il Verificatore ha quindi gestito manualmente il merge dei file, ed una disattenzione introdotta da quest'ultimo ha causato il mancato aggiornamento del file relativo all'appendice "Riscontro rischi". Tale difetto non è stato poi rilevato in sede di approvazione, e data l'immediata scadenza di consegna quest'ultimo è stato riportato anche nel documento infine consegnato. \\
		Sono stati quindi identificati tre miglioramenti da apportare al metodo di lavoro, per minimizzare la possibilità di ripresentarsi di tale problematica:
		\begin{itemize}
			\item è necessario notificare prontamente tutti i membri del gruppo qualora sussistano problematiche legate alla gestione della repository\ped{\textit{G}};
			\item è necessario ridurre la quantità di modifiche effettuate all'interno di ogni feature branch e, dove possibile, agire in maniera sequenziale invece che parallela. In modo da garantire l'assenza di conflitti derivanti dalle modifiche;
			\item è necessario stabilire scadenze di completamento dei prodotti\ped{\textit{G}} che precedano di molto la data di consegna ufficiale, così da consentire una revisione efficacie di ciò che viene poi consegnato.
		\end{itemize}

	\subsection{Prove generali di presentazione RP e revisione slide}
		Il gruppo ha poi proseguito con due prove generali della presentazione da esporre in sede di RP, intervallate da una discussione circa la struttura delle slide e dei discorsi preparati dai singoli membri.
