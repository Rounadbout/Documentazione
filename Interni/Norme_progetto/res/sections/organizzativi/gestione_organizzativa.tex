\subsection{Gestione Organizzativa}

		\subsubsection{Descrizione}
			La gestione organizzativa è il processo che descrive le scelte sottostanti la suddivisione e il coordinamento del lavoro all'interno del progetto. Lo scopo principale di questo processo è fornire ai membri del gruppo un \PdP{} per l'organizzazione del lavoro, con focus sull'efficacia e l'efficienza. Nello specifico, le attività che costituiscono questo processo sono:
			\begin{itemize}
				\item {inizializzazione e definizione degli obiettivi;}
				\item {pianificazione e gestione di scadenze, rischi e risorse;}
				\item {esecuzione e controllo dei processi;}
				\item {revisione e valutazione;}
				\item {Definition of Done\ped{\textit{G}}.}
			\end{itemize}
			Le operazioni svolte durante queste attività vengono in seguito descritte indirettamente, attraverso l'esplicitazione delle mansioni dei vari ruoli di progetto.
			\subsubsection{Attività}
			\subsubsubsection{Gestione ruoli di progetto}
			Ciascun componente del gruppo ricopre un ruolo di progetto a rotazione, facendo sì che ogni membro possa assumere almeno una volta ciascuno di essi nel corso del progetto. Nel documento \PdP{} \textit{3.0.0} vengono organizzate e pianificate le attività assegnate ai specifici ruoli previsti nell'attività di progetto. Ciascun ruolo viene descritto di seguito.

			\subsubsubsection*{Responsabile di Progetto}
			Il Responsabile di Progetto è una figura essenziale, che partecipa al progetto dall'inizio fino alla fine. È responsabile delle decisioni e scelte che vengono intraprese, coordinando l'intero progetto. Rappresenta il gruppo di fronte a soggetti esterni, come Committente\ped{\textit{G}} e Proponente\ped{\textit{G}}.
			Riassunto delle mansioni:
			\begin{itemize}
				\item coordina, pianifica e controlla le attività;
				\item gestisce le risorse umane;
				\item approva la documentazione;
				\item approva l'offerta economica;
				\item preventiva l'analisi dei rischi e la loro eventuale gestione;
				\item determina e verifica le condizioni di completezza.
			\end{itemize}

			\subsubsubsection*{Amministratore}
			L'Amministratore è la figura che ha come compito principale il controllo e l'amministrazione dell'ecosistema lavorativo. Inoltre ha diretta responsabilità sull'efficienza e sulla capacità operativa dell'ambiente di lavoro.
			Riassunto delle mansioni:
			\begin{itemize}
				\item studia e ricerca strumenti che riducano il più possibile l'impiego di risorse umane e che automatizzino tutto ciò che è possibile fare attraverso l'utilizzo di software;
				\item ricerca soluzioni ai problemi legati alla difficoltà di gestione dei processi e risorse, attraverso la realizzazione o la ricerca di strumenti adatti a tale scopo;
				\item controlla le versioni e le configurazioni del prodotto\ped{\textit{G}};
				\item gestisce il versionamento della documentazione del progetto e della sua archiviazione;
				\item fornisce procedure e strumenti di monitoraggio/segnalazione, in modo da garantire un corretto controllo di qualità.
			\end{itemize}

			\subsubsubsection*{Analista}
			L'Analista è il responsabile delle attività di analisi. Deve effettuare studi e ricerche in maniera molto approfondita per conoscere bene il dominio\ped{\textit{G}} del problema. Non è necessario che partecipi al progetto fino al termine, ma il suo operato è fondamentale fin dalla fase iniziale, in quanto le sue scelte decisionali hanno un grande impatto sul successo dell'intero progetto.
			Riassunto delle mansioni:
			\begin{itemize}
				\item studia e definisce il problema da risolvere, rilevandone la complessità;
				\item compie l'analisi del dominio\ped{\textit{G}} delle richieste tramite lo studio dei bisogni, espliciti ed impliciti;
				\item compie l'analisi del dominio\ped{\textit{G}} applicativo determinando gli utilizzatori e l'ambiente di utilizzo;
				\item redige i documenti \AdR{} e \SdF{}.
			\end{itemize}

			\subsubsubsection*{Progettista}
			Il Progettista è la figura che si occupa delle scelte architetturali del progetto e ne influenza gli aspetti tecnici e tecnologici. Utilizzando le attività svolte dall'Analista, il Progettista ha il compito di trovare una soluzione attuabile, comprensibile e motivata.
			Riassunto delle mansioni:
			\begin{itemize}
				\item produce una soluzione attuabile, comprensibile e motivata;
				\item effettua scelte su aspetti progettuali, applicando al prodotto\ped{\textit{G}} soluzioni note ed ottimizzate;
				\item effettua scelte che portino ad avere un prodotto\ped{\textit{G}} facilmente manutenibile.
			\end{itemize}

			\subsubsubsection*{Programmatore}
			Il Programmatore è la figura responsabile della codifica del codice e della creazione delle componenti di supporto, indispensabili per poter effettuare le prove di verifica e di validazione.
			Riassunto delle mansioni:
			\begin{itemize}
				\item implementa in maniera precisa e scrupolosa le soluzioni generate dal Progettista;
				\item scrive codice sorgente altamente e facilmente manutenibile;
				\item si occupa del versionamento e della documentazione del codice prodotto\ped{\textit{G}};
				\item realizza strumenti per la verifica e la validazione del software.
			\end{itemize}

			\subsubsubsection*{Verificatore}
			Il Verificatore è la figura che ha il compito di effettuare una attenta verifica del prodotto\ped{\textit{G}}, dando particolare attenzione al rispetto delle normative di progetto. Oltre ad una grande conoscenza di tali norme, il verificatore deve avere buone capacità di giudizio.
			Riassunto delle mansioni:
			\begin{itemize}
				\item controlla che le attività svolte siano conformi alle normative stabilite;
				\item vigila sull'integrità del prodotto\ped{\textit{G}} ad ogni stadio del suo ciclo di vita.
				\item comunica eventuali errori identificati al responsabile dell'oggetto preso in esame.
			\end{itemize}

			\subsubsubsection{Gestione delle comunicazioni}
			\subsubsubsection*{Comunicazioni interne}
			Le comunicazioni interne sono gestite principalmente attraverso tre canali:
			\begin{itemize}
				\item \textbf{gruppo Telegram}\ped{\textit{G}}: dove è possibile scambiare rapidamente comunicazioni brevi ed informali, coinvolgendo l'intero team.
				\item \textbf{Microsoft Teams}\ped{\textit{G}}: piattaforma dotata di funzioni di messaggistica istantanea, con possibilità di creare canali specifici divisi per argomento, e videochiamata. Viene utilizzata per lo scambio di comunicazioni più consistenti e tecniche, come aggiornamenti sulle attività di Verifica o discussioni legate a specifiche attività.
				\item \textbf{GitHub}\ped{\textit{G}}: anche se meno focalizzato sulla comunicazione, il sistema di gestione Issue\ped{\textit{G}} di GitHub\ped{\textit{G}} consente di aggiornare rapidamente l'intero gruppo sull'avanzamento dei lavori.
			\end{itemize}
			\subsubsubsection*{Comunicazioni esterne}
			Le comunicazioni esterne, gestite principalmente dal Responsabile di progetto, avvengono attraverso due canali:
			\begin{itemize}
				\item \textbf{Gmail}\ped{\textit{G}}: per comunicare con il Committente\ped{\textit{G}} e, almeno inizialmente, con il Proponente\ped{\textit{G}} viene utilizzata la casella di posta elettronica creata appositamente per il team \Gruppo{}.
				\item \textbf{canale Slack}\ped{\textit{G}}: è stato concordato con il Proponente\ped{\textit{G}} \Proponente{} l'utilizzo della piattaforma Slack\ped{\textit{G}}, con un canale dedicato al gruppo \Gruppo{}. Viene utilizzata per rapidi scambi di informazioni o la pianificazione di eventuali incontri.
			\end{itemize}
			\subsubsubsection{Gestione delle riunioni}
			Le riunioni sono indette dal Responsabile di Progetto, il quale ha il compito di:
			\begin{itemize}
				\item definire la data e l'orario delle riunioni, sia interne che esterne, considerando la disponibilità dei partecipanti;
				\item stabilire l'oggetto della riunione;
				\item valutare le richieste relative alle riunioni da parte dei componenti del Team \Gruppo{} e dai soggetti esterni;
				\item verificare ed approvare il verbale\ped{\textit{G}} redatto dal Segretario della riunione, il quale verrà nominato ad inizio incontro;
			\end{itemize}
			I partecipanti devono presentarsi puntuali alle riunioni e, in caso di imprevisti, comunicarli con congruo preavviso.
			Le decisioni da intraprendere durante le riunioni, sono ritenute approvate nel caso di maggioranza da parte dei partecipanti.\\
			Al termine di ciascuna riunione viene redatto un \Verbale{}\ped{\textit{G}} da parte del Segretario, che deve contenere tutte le informazioni sulla riunione.
			\subsubsubsection*{Riunioni interne}
				Le riunioni interne sono aperte esclusivamente agli 8 membri del gruppo \Gruppo{}. Le occasioni di incontro sono ritenute valide esclusivamente in due modalità:
				\begin{itemize}
					\item \textbf{incontri in video-conferenza:} Effettuati tramite l'applicativo Microsoft Teams\ped{\textit{G}};
					\item \textbf{incontri di persona:} Effettuati trovandosi fisicamente in uno stesso luogo.
				\end{itemize}
				Considerata la situazione extra-progettuale relativa all'emergenza COVID-19\ped{\textit{G}}, le riunioni iniziali saranno effettuate esclusivamente a distanza tramite video-conferenza.
				Affinché le riunioni siano ritenute valide, all'incontro dovranno essere presenti almeno 6 componenti del team.

			\subsubsubsection*{Riunioni esterne}
				Le riunioni esterne comprendono tutti gli incontri che coinvolgono, oltre ai membri del team \Gruppo{}, anche altri soggetti esterni.\\
				Questi incontri sono tenuti telematicamente, fino al termine dell'emergenza COVID-19\ped{\textit{G}} e, successivamente, potranno tenersi attraverso riunioni fisiche. In caso di video-conferenza è da privilegiare lo strumento di comunicazione proposto dai soggetti esterni, similmente per le riunioni fisiche, in luoghi proposti dai soggetti esterni.\\
				Nel caso si scelga di usufruire dei locali di Torre Archimede è necessario chiedere il permesso al \TV{} all'indirizzo mail \href{mailto:tullio.vardanega@math.unipd.it}{tullio.vardanega@math.unipd.it}.
			\subsubsubsection{Gestione degli strumenti di coordinamento e versionamento}
			\subsubsubsection*{Issue Tracking System}
			Come già menzionato, il gruppo ha deciso di utilizzare come strumento per l'organizzazione del carico di lavoro, la componente Issue\ped{\textit{G}} Tracking System (ITS) della piattaforma GitHub\ped{\textit{G}}. Grazie a tale strumento, è possibile rappresentare ogni singolo compito tramite Issue\ped{\textit{G}}, che verranno assegnate ai rispettivi responsabili per essere svolte entro scadenze prestabilite. È possibile inoltre avere una visione d'insieme delle Issue\ped{\textit{G}} (aperte o chiuse), così che il Responsabile di Progetto o eventuali Amministratori possano facilmente gestire e monitorare l'andamento del progetto.
			La procedura seguita per l'assegnazione di una Issue\ped{\textit{G}} è la seguente:
			\begin{itemize}
				\item vengono decisi uno o più responsabili per la nuova Issue\ped{\textit{G}};
				\item viene inserito un titolo alla Issue\ped{\textit{G}}, che renda chiaro il suo obiettivo;
				\item viene impostata una scadenza per la Issue\ped{\textit{G}}, se precedentemente concordata;
				\item la Issue\ped{\textit{G}} viene concretamente assegnata ai responsabili scelti;
				\item viene associata un'etichetta alla Issue\ped{\textit{G}}, utile ad identificare il contesto in cui questa opera o a classificarne la tipologia;
				\item qualora il titolo non sia sufficientemente esplicativo, o sia necessario aggiungere ulteriori informazioni, viene aggiunto un breve riassunto dell'obiettivo della Issue\ped{\textit{G}};
			\end{itemize}
			Il processo viene completato con una conferma, che provoca un'immediata notifica a tutti i componenti del gruppo.

			\subsubsubsection*{Repository}
			Per la gestione del versionamento e l'archiviazione dei file di progetto, viene sfruttata la componente VCS\ped{\textit{G}} della piattaforma GitHub\ped{\textit{G}}. Gli Amministratori si sono occupati della creazione della repository\ped{\textit{G}} \textbf{etherless}, e dei Git\ped{\textit{G}} submodule (repository\ped{\textit{G}} annidate): \textbf{Documentazione}, \textbf{etherless-cli}, \textbf{etherless-smart}, \textbf{etherless-server},  ed è loro compito garantire l'ordine e la pulizia di quest'ultimi. Il contenuto dei repository\ped{\textit{G}} relativi alle componenti software è quindi soggetto a repentini cambiamenti, con però una più stabile struttura di base.\\
			La struttura del repository\ped{\textit{G}} \textbf{etherless}, è la seguente:
			\begin{itemize}
				\item \textbf{Submodule Documentazione}: riferimento all'ultimo commit\ped{\textit{G}} stabile della repository\ped{\textit{G}} \textit{Documentazione};
				\item \textbf{Submodule etherless-cli}: riferimento all'ultimo commit\ped{\textit{G}} stabile della repository\ped{\textit{G}} \textit{etherless-cli};
				\item \textbf{Submodule etherless-smart}: riferimento all'ultimo commit\ped{\textit{G}} stabile della repository\ped{\textit{G}} \textit{etherless-smart};
				\item \textbf{Submodule etherless-server}: riferimento all'ultimo commit\ped{\textit{G}} stabile della repository\ped{\textit{G}} \textit{etherless-server};
				\item \textbf{File .gitmodules}: file utilizzato per la configurazione dei Git\ped{\textit{G}} modules;
				\item \textbf{File README.md}: file descrittivo della repository\ped{\textit{G}} di progetto, contenente istruzioni per la sua esecuzione. Scritto in lingua inglese;
				\item \textbf{File package.json}: file utilizzato per effettuare l'installazione dei moduli\ped{\textit{G}} necessari all'esecuzione del prodotto\ped{\textit{G}};
				\item \textbf{File tsconfig.json}: file utilizzato per la configurazione di Typescript\ped{\textit{G}}.
			\end{itemize}
			La struttura della repository\ped{\textit{G}} \textbf{Documentazione}, è la seguente:
			\begin{itemize}
				\item \textbf{Cartella Esterni}: contiene i file e sotto-cartelle relativi ai documenti esterni del progetto;
				\item \textbf{Cartella Interni}: contiene i file e sotto-cartelle relativi ai documenti interni del progetto;
				\item \textbf{Cartella Latex}\ped{\textit{G}}: contiene i file template\ped{\textit{G}} e configurazioni per il layout dei documenti \LaTeX{}\ped{\textit{G}};
				\item \textbf{File .gitignore}: specifica i tipi di file che devono essere ignorati e non presenti all'interno della repository\ped{\textit{G}}. In particolare si desidera conservare solo file di tipo .tex, .pdf, .jpg e .png.
			\end{itemize}
			La struttura della repository\ped{\textit{G}} \textbf{etherless-cli}, è la seguente:
			\begin{itemize}
				\item \textbf{Cartella contracts}: contiene i file relativi agli smart contract\ped{\textit{G}};
				\item \textbf{Cartella src}: contiene i file e sotto-cartelle relativi al codice eseguibile del modulo\ped{\textit{G}} Etherless-cli;
				\item \textbf{File .editorconfig}: file di supporto agli sviluppatori, per mantenere uno stile di codifica consistente tra editor e IDE\ped{\textit{G}} diversi;
				\item \textbf{File .eslintrc.js}: file utilizzato per la configurazione di ESLint\ped{\textit{G}};
				\item \textbf{File .gitignore}: specifica i tipi di file che devono essere ignorati e non presenti all'interno della repository\ped{\textit{G}};
				\item \textbf{File README.md}: file descrittivo della repository\ped{\textit{G}} del modulo, contenente istruzioni per la sua esecuzione. Scritto in lingua inglese;
				\item \textbf{File package.json}: file utilizzato per effettuare l'installazione dei moduli\ped{\textit{G}} necessari all'esecuzione del prodotto\ped{\textit{G}};
				\item \textbf{File package-lock.json}: file utilizzato per l'installazione dei moduli\ped{\textit{G}} necessari all'esecuzione del prodotto\ped{\textit{G}};
				\item \textbf{File tsconfig.json}: file utilizzato per la configurazione di Typescript\ped{\textit{G}}.
			\end{itemize}
			La struttura della repository\ped{\textit{G}} \textbf{etherless-smart}, è la seguente:
			\begin{itemize}
				\item \textbf{Cartella contracts}: contiene i file relativi agli smart contract\ped{\textit{G}};
				\item \textbf{Cartella migrazioni}: contiene i file relativi alle migrazioni;
				\item \textbf{File .gitignore}: specifica i tipi di file che devono essere ignorati e non presenti all'interno della repository\ped{\textit{G}};
				\item \textbf{File README.md}: file descrittivo della repository\ped{\textit{G}} del modulo, contenente istruzioni per la sua esecuzione. Scritto in lingua inglese;
				\item \textbf{File package.json}: file utilizzato per effettuare l'installazione dei moduli\ped{\textit{G}} necessari all'esecuzione del prodotto\ped{\textit{G}};
				\item \textbf{File package-lock.json}: file utilizzato per l'installazione dei moduli\ped{\textit{G}} necessari all'esecuzione del prodotto\ped{\textit{G}};
				\item \textbf{File truffle-config.json}: file utilizzato per la configurazione di Truffle\ped{\textit{G}}.
			\end{itemize}
			La struttura della repository\ped{\textit{G}} \textbf{etherless-server}, è la seguente:
			\begin{itemize}
				\item \textbf{Cartella configs}: contiene i file relativi alle configurazioni generali;
				\item \textbf{Cartella contracts}: contiene i file relativi agli smart contract\ped{\textit{G}};
				\item \textbf{Cartella serverless}: contiene i file e sotto-cartelle relativi al framework\ped{\textit{G}} Serverless\ped{\textit{G}};
				\item \textbf{Cartella src}: contiene i file e sotto-cartelle relativi al codice eseguibile del modulo\ped{\textit{G}} Etherless-server;
				\item \textbf{Cartella test}: contiene i file e sotto-cartelle relativi ai test;
				\item \textbf{File .eslintrc.js}: file utilizzato per la configurazione di ESLint\ped{\textit{G}}.
				\item \textbf{File .gitignore}: specifica i tipi di file che devono essere ignorati e non presenti all'interno della repository\ped{\textit{G}};
				\item \textbf{File README.md}: file descrittivo della repository\ped{\textit{G}} del modulo, contenente istruzioni per la sua esecuzione. Scritto in lingua inglese;
				\item \textbf{File package.json}: file utilizzato per effettuare l'installazione dei moduli\ped{\textit{G}} necessari all'esecuzione del prodotto\ped{\textit{G}};
				\item \textbf{File package-lock.json}: file utilizzato per l'installazione dei moduli\ped{\textit{G}} necessari all'esecuzione del prodotto\ped{\textit{G}};
				\item \textbf{File tsconfig.json}: file utilizzato per la configurazione di Typescript\ped{\textit{G}}.
			\end{itemize}
			Al fine di tracciare in maniera più efficace le modifiche effettuate nel repository\ped{\textit{G}}, è richiesto di specificare alla fine di ogni commit\ped{\textit{G}} le Issue\ped{\textit{G}} che quest'ultimo va a chiudere tramite il commento: \texttt{"close \#"} seguito dall'ID della Issue\ped{\textit{G}}. Si è convenuto di utilizzare nelle repository\ped{\textit{G}} \textbf{etherless-cli}, \textbf{etherless-smart} ed \textbf{etherless-server} esclusivamente la lingua inglese, mentre per la documentazione, ad eccezione dei manuali sviluppatore e utente, si è scelto di utilizzare la lingua italiana.

			\subsubsubsection{Gestione dei rischi}
			Tutte le problematiche che potrebbero ostacolare il corretto proseguimento del progetto devono essere opportunamente analizzate e gestite. È compito del Responsabile di Progetto svolgere tale mansione e documentare i risultati all'interno del \PdP{}{3.0.0}. La procedura da seguire per la gestione dei rischi comprende:
			\begin{itemize}
				\item individuazione dei potenziali fattori di rischio;
				\item analisi dei fattori di rischio, con documentazione nel \PdP{}{3.0.0};
				\item pianificazione di controllo rischi e mitigazione effetti;
				\item monitoraggio costante, per individuare nuovi rischi e gestire i conosciuti.
			\end{itemize}
			Per la classificazione dei fattori di rischio, si è deciso di dividerli in:
			\begin{itemize}
				\item \textbf{RT}: Rischi Tecnologici;
				\item \textbf{RO}: Rischi Organizzativi;
				\item \textbf{RI}: Rischi Interpersonali;
				\item \textbf{RR}: Rischi legati ai Requisiti.
			\end{itemize}
		\subsubsection{Metriche}
			\subsubsubsection{Metriche per la gestione dei rischi}
				\paragraph{Budget at Completion (BAC)}
					\begin{itemize}
						\item \textbf{Descrizione}: valore che indica il budget inizialmente allocato per la realizzazione del progetto;
						\item \textbf{unità di misura}: la metrica è espressa tramite un numero intero;
						\item \textbf{risultato}: l'obiettivo è ottenere un valore pari al preventivato, accettando un errore massimo del $\pm{}5\%$. Un errore maggiore di tale valore indica un'inadeguatezza del preventivo stilato o della gestione delle risorse, si rende quindi necessaria una revisione di quest'ultimi.
					\end{itemize}
				\paragraph{Estimated at Completion (EAC)}
					\begin{itemize}
						\item \textbf{Descrizione}: rappresenta il budget stimato per la realizzazione del progetto, aggiornato allo stato attuale, con quindi conoscenza dei costi già sostenuti.
						\item \textbf{unità di misura}: la metrica è espressa tramite un numero intero;
						\item \textbf{formula}: $EAC = AC + ETC$;
						\item \textbf{risultato}: l'obiettivo è ottenere un valore pari al preventivato, accettando un errore massimo di $\pm{}5\%$. Un errore maggiore di tale valore indica un'inadeguatezza del preventivo stilato o della gestione delle risorse, si rende quindi necessaria una revisione di quest'ultimi.
					\end{itemize}
				\paragraph{Estimated to Complete (ETC)}
					\begin{itemize}
						\item \textbf{Descrizione}: rappresenta il budget stimato per la realizzazione delle rimanenti attività necessarie al completamento del progetto;
						\item \textbf{unità di misura}: la metrica è espressa tramite un numero intero;
						\item \textbf{risultato}: l'obiettivo è ottenere un valore pari o inferiore al preventivato. Nel caso sia stato speso un budget maggiore, è necessario fornire una o più valide motivazioni a giustificare il fatto.
					\end{itemize}
				\paragraph{Planned Value (PV)}
					\begin{itemize}
						\item \textbf{Descrizione}: costo pianificato per la realizzazione delle attività di progetto fino a quel momento;
						\item \textbf{unità di misura}: la metrica è espressa tramite un numero di giorni o di \euro;
						\item \textbf{formula}: $PV = \%lavoro\_pianificato \times BAC$;
						\item \textbf{risultato}: si attende un valore che sia maggiore o uguale a 0. In caso contrario si presume sia presente un errore nel calcolo ed è necessaria una sua reiterazione.
					\end{itemize}
				\paragraph{Actual Cost (AC)}
					\begin{itemize}
						\item \textbf{Descrizione}: costo effettivamente sostenuto fino al momento del calcolo;
						\item \textbf{unità di misura}: la metrica è espressa tramite un numero di giorni o di \euro;
						\item \textbf{risultato}: si attende un valore maggiore o uguale di zero e minore di BAC. In caso contrario, qualora il valore sia negativo si presume sia presente un errore di calcolo, nel caso sia maggiore di BAC è necessario fornire valide motivazioni per il superamento del budget previsto.
					\end{itemize}
				\paragraph{Earned Value (EV)}
					\begin{itemize}
						\item \textbf{Descrizione}: valore totale delle attività portate a termine al momento del calcolo;
						\item \textbf{unità di misura}: la metrica è espressa tramite un numero di giorni o di \euro;
						\item \textbf{formula}: $EV = \%lavoro\_completato \times BAC$;
						\item \textbf{risultato}: ci si aspetta di ottenere un valore maggiore o uguale di 0. In caso contrario si presume sia presente un errore nel calcolo ed è necessaria una sua reiterazione.
					\end{itemize}
				\paragraph{Cost Variance (CV)}
					\begin{itemize}
						\item \textbf{Descrizione}: è un indicatore di produttività o efficienza nella gestione del progetto. Stabilisce se il budget effettivamente speso è maggiore, uguale o minore, rispetto a quanto si era previsto di spendere;
						\item \textbf{unità di misura}: la metrica è espressa tramite un numero di giorni o di \euro;
						\item \textbf{formula}: $CV = EV - AC$;
						\item \textbf{risultato}: si attende un valore uguale o maggiore di zero, ad indicare che il progetto produce con efficienza pari o maggiore rispetto a quanto pianificato. In caso questo valore sia negativo, significa che ci si trova in una situazione \textit{over budget}, ed è necessario rivedere la gestione delle risorse nel progetto.
					\end{itemize}
				\paragraph{Schedule Variance (SV)}
					\begin{itemize}
						\item \textbf{Descrizione}: è un indicatore di efficacia nei confronti del Proponente\ped{\textit{G}} e della capacità del gruppo di mantenersi in linea con la pianificazione temporale delle attività del progetto.
						\item \textbf{unità di misura}: la metrica è espressa tramite un numero di giorni o di \euro;
						\item \textbf{formula}: $SV = EV - PV$;
						\item \textbf{risultato}: l'obiettivo è ottenere un valore uguale o maggiore di zero, ad indicare una produzione in pari o in anticipo rispetto alla pianificazione temporale. Un valore negativo rappresenta un ritardo nella tabella di marcia e la necessità di ottimizzare eventuali attività successive, per poter tornare a lavorare coerentemente al piano stabilito.
					\end{itemize}
          \subsubsubsection*{Correlazione tra CV e SV}
		        Lo stato di un progetto è esprimibile dalla correlazione tra \textit{Cost Variance} e \textit{Schedule Variance}, in particolare:
		        \begin{enumerate}
			          \item{\textbf{SV e CV positive}: il progetto è in anticipo rispetto alla pianificazione e rientra nel budget previsto;}
			          \item{\textbf{SV positiva, CV negativa}: il progetto è in anticipo rispetto alla pianificazione ma ha superato il budget allocato;}
			          \item{\textbf{SV negativa, CV positiva}: il progetto è in ritardo rispetto alla pianificazione ma rientra nel budget previsto;}
			          \item{\textbf{SV e CV negative}: il progetto è in ritardo rispetto alla pianificazione e ha superato il budget previsto.}
		        \end{enumerate}
			\subsubsection{Strumenti}
				I membri del gruppo \Gruppo{} possono lavorare indifferentemente su Windows, Mac OS X o Linux in quanto i principali strumenti necessari ai fini del progetto sono disponibili per tutti i sistemi operativi citati. Gli applicativi organizzativi utilizzati sono di seguito descritti.

				\subsubsubsection{Microsoft Teams}
					Si è individuato Microsoft Teams\ped{\textit{G}} come strumento di condivisione e comunicazione interno. La decisione è motivata dalla notevole multifunzionalità dell'applicativo, in quanto consente video-chiamate, condivisione di file e chat divise per argomento.

				\subsubsubsection{Git e GitHub}
					Come software di controllo versione\ped{\textit{G}} si è deciso di impiegare Git\ped{\textit{G}}, che rappresenta uno dei migliori strumenti attualmente esistenti per quanto riguarda performance e facilità di utilizzo. Per lo sviluppo collaborativo abbiamo deciso di appoggiarci al servizio GitHub\ped{\textit{G}} il quale fornisce non solo un repository\ped{\textit{G}} Git\ped{\textit{G}}, ma anche funzionalità utili per la cooperazione fra più persone.

				\subsubsubsection{Gmail}
					Per la gestione della corrispondenza si è scelto di creare una casella mail su dominio\ped{\textit{G}} Gmail\ped{\textit{G}}.

				\subsubsubsection{Slack}
					In accordo con il Proponente\ped{\textit{G}}, si è concordato l'utilizzo della piattaforma Slack\ped{\textit{G}} per avere un canale di comunicazione più diretto e veloce rispetto alle mail.

				\subsubsubsection{Zoom}
					Per la gestione delle video-chiamate con il Proponente\ped{\textit{G}} ed il Committente\ped{\textit{G}}, si è scelto di utilizzare l'applicazione Zoom\ped{\textit{G}}.

				\subsubsubsection{Telegram}
					Utilizzato come mezzo di comunicazione interno per scambiare informazioni velocemente ed in maniera informale.
