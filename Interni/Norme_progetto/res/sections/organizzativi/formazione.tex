\subsection{Formazione}
		\subsubsection{Descrizione}	
		Per avere i membri del gruppo \Gruppo{} allineati relativamente alle conoscenze degli strumenti applicativi utilizzati, si rende necessario procedere ad un adeguata formazione dei singoli attraverso lo studio autonomo. Lo scopo di questo processo è appunto quello di fornire ai diversi membri del gruppo una sufficiente formazione.\\
		In particolare, oltre al materiale indicato nella sotto sezione \textit{Riferimenti Informativi}, si segnala anche la seguente documentazione:
		\begin{itemize}
			\item \LaTeX{}\ped{\textit{G}}: \href{latex-project.org}{latex-project.org}
			\item Solidity\ped{\textit{G}}: \href{solidity.readthedocs.io}{solidity.readthedocs.io}
			\item TypeScript\ped{\textit{G}}: \href{www.typescriptlang.org}{www.typescriptlang.org}
		\end{itemize}
		Oltre allo studio autonomo, un altro metodo di formazione è quello dell'apprendimento dall'operato altrui, di modo che le proprie conoscenze possano essere arricchite da quelle di altre persone maggiormente preparate in specifici argomenti.
		
		\subsubsection{Condivisione documentazione}
		Oltre ai riferimenti già citati è prevista una condivisione fra i membri del gruppo di informazioni utili alla formazione personale, il tutto gestito attraverso la piattaforma Microsoft Teams\ped{\textit{G}}. È stato infatti creato un documento all'interno dello spazio di archiviazione condiviso dal team, che può venire costantemente aggiornato da un qualsiasi membro con eventuali nuove informazioni interessanti o link ipertestuali a siti di formazione. Anche lo stesso spazio di archiviazione può divenire luogo di condivisione di documentazione tecnica o utili video-tutorial.
