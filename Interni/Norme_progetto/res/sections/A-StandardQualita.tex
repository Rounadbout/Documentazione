\section{Standard di qualità}

\subsection{ISO/IEC 15504}
ISO/IEC 15504, conosciuto anche come SPICE (\textit{Software Process Improvement and Capability Determination}), è lo standard internazionale usato per valutare la qualità dei processi software e perseguirne il miglioramento continuo. La qualità di ogni processo viene valutata oggettivamente mediante la misurazione della capacità dello stesso tramite specifici attributi. \\
Lo standard ISO/IEC 15504 definisce un modello di riferimento che si suddivide in:
\begin{itemize}
	\item{dimensione del processo;}
	\item{dimensione della capacità.}
\end{itemize}

\subsubsection{Dimensione del processo}
La \textit{process dimension} divide i processi in cinque categorie:
	\begin{itemize}
		\item{Customer/Supplier;}
		\item{Engineering;}
		\item{Supporting;}
		\item{Management;}
		\item{Organization.}
	\end{itemize}

\subsubsection{Dimensione della capacità}
Per ogni processo, ISO/IEC 15504 definisce un livello di maturità basato sulla scala rappresentata di seguito:

	\rowcolors{2}{lightRowColor}{darkRowColor}
	\begin{longtable}{ 
		>{\centering}p{0.4\textwidth} 
		>{\centering\arraybackslash}p{0.5\textwidth}}
		\caption {Scala di maturità dello standard ISO/IEC 15504}		\\
		\coloredTableHead
		\textbf{\color{white}Livello} &
		\textbf{\color{white}Nome}
		\tabularnewline  
		\endhead
		% Contenuto della tabella
		5 & Optimizing process \\
		4 & Predictable process \\
		3 & Established process \\
		2 & Managed process \\
		1 & Performed process \\
		0 & Incomplete process
	\end{longtable}	
	
La capacità (o maturità) di ciascun processo viene misurata tramite gli attributi descritti di seguito:
	
	\rowcolors{2}{lightRowColor}{darkRowColor}
	\begin{longtable}{ 
		>{\centering}p{0.4\textwidth} 
		>{\centering\arraybackslash}p{0.5\textwidth}}
		\caption {Attributi per la misurazione della capacità dello standard ISO/IEC 15504}		\\
		\coloredTableHead
		\textbf{\color{white}Appartenenza al livello ed identificativo} &
		\textbf{\color{white}Nome}
		\tabularnewline  
		\endhead
		% Contenuto della tabella
		1.1 & Process Performance \\
		2.1 & Performance Management \\
		2.2 & Work Product Management \\
		3.1 & Process Definition \\
		3.2 & Process Deployment \\
		4.1 & Process Measurement \\
		4.2 & Process Control \\
		5.1 & Process Innovation \\
		5.2 & Process Optimization 
	\end{longtable}	
	
Ciascun attributo consiste di una o più pratiche generiche che aiutano nella fase di valutazione. Inoltre, ciascun attributo è valutato secondo una scala a quattro valori (N-P-L-F):

	\rowcolors{2}{lightRowColor}{darkRowColor}
	\begin{longtable}{ 
		>{\centering}p{0.4\textwidth} 
		>{\centering\arraybackslash}p{0.5\textwidth}}
		\caption {Scala di valutazione degli attributi dello standard ISO/IEC 15504}		\\
		\coloredTableHead
		\textbf{\color{white}Stato} &
		\textbf{\color{white}Range di valori corrispondenti}
		\tabularnewline  
		\endhead
		% Contenuto della tabella
		Not achieved & 0 - 15\% \\
		Partially achieved & >15 - 50\%  \\
		Largely achieved & >50 - 85\% \\
		Fully achieved & >85 - 100\% \\
	\end{longtable}	

La valutazione viene fatta sulla base di evidenze oggettive acquisite durante la fase di assessment. La figura di seguito mostra la relazione tra livello di maturità, attributi di misurazione e relativa scala di valutazione nell'attività di valutazione della qualità di processo.

\begin{figure}[H]
		\centering
		\includegraphics[scale=0.5]{./res/img/ISO_IEC_15504.png}
		\caption[Modello ISO/IEC 15504]{SPICE Capability (fonte: Janos Ivanyos, \texttt{researchgate.net})}
\end{figure}
\pagebreak

\subsection{Ciclo di Deming}
Il ciclo di Deming\ped{\textit{G}} (o ciclo di PDCA, acronimo dall'inglese di \textit{Plan-Do-Check-Act}) è un metodo di gestione iterativo utilizzato per il controllo ed il miglioramento continuo dei processi e, di riflesso, anche dei prodotti da questi risultanti. 
Il ciclo di Deming\ped{\textit{G}} è strutturato in quattro fasi, come illustrate di seguito:
	\begin{figure}[H]
		\centering
		\includegraphics[scale=0.1]{./res/img/PDCA.png}
		\caption[Ciclo PDCA]{Ciclo PDCA per il miglioramento continuo (fonte: Wikipedia)}
	\end{figure}

\begin{enumerate}
	\item{\textbf{Plan}: è la fase di pianificazione in cui vengono stabiliti gli obiettivi ed i processi necessari a raggiungerli. In questa fase vengono definite tutte le attività da svolgere, le risorse da assegnarvi e le scadenze da rispettare;}
	\item{\textbf{Do}: è la fase di esecuzione del programma precedentemente stilato, dapprima -se possibile- in contesti circoscritti. E' possibile ed auspicabile in questa fase raccogliere dati utili alle fasi di \textit{Check} e \textit{Act};}
	\item{\textbf{Check}: è la fase di controllo per accertarsi che la fase \textit{Do} sia stata eseguita in accordo con ciò che era stato deciso nella fase \textit{Plan}. In questa fase vengono anche studiati i risultati confrontandoli con quelli attesi;}
	\item{\textbf{Act}: è la fase di attuazione, che permette di rendere definitivi i processi i cui esiti sono stati positivi o apportare modifiche migliorative in caso contrario.}
\end{enumerate}
I quattro punti sopra indicati costituiscono una sequenza logica che viene ripetuta finché l'obiettivo finale non è raggiunto.
\pagebreak

\subsection{ISO/IEC 9126}
ISO/IEC 9126 è lo standard internazionale usato per valutare la qualità del software. Esso è articolato in quattro parti:
\begin{enumerate}
	\item{modello per la qualità del software, che a sua volta è suddiviso in:}
	\begin{itemize}
		\item{modello per la qualità esterna ed interna;}
		\item{modello per la qualità in uso;}
	\end{itemize}
	\item{metriche per la qualità interna;}
	\item{metriche per la qualità esterna;}
	\item{metriche per la qualità in uso.}
\end{enumerate}

	\subsubsection{Modello per la qualità del software}
	\begin{figure}[H]
		\centering
		\includegraphics[scale=0.5]{./res/img/ISO_IEC_9126.png}
		\caption[Modello ISO/IEC 9126]{Modello ISO/IEC 9126 (fonte: Wikipedia)}
	\end{figure}
	
	\subsubsubsection{Modello per la qualità esterna ed interna}
	Il modello per la definizione della qualità interna ed esterna è composto da sei caratteristiche generali e varie sotto caratteristiche misurabili attraverso delle metriche. Tali caratteristiche sono:
	
	\subsubsubsection*{Funzionalità}
	E' la capacità del software di fornire funzioni che soddisfano esigenze stabilite nell'\textit{\AdR{}} e che permettono di operare in condizioni specifiche. Questa capacità si traduce nelle seguenti sotto caratteristiche:
	\begin{itemize}
		\item{\textbf{appropriatezza}: capacità di fornire funzioni appropriate per attività specifiche che permettano di raggiungere gli obiettivi prefissati;}
		\item{\textbf{accuratezza}: capacità di fornire risultati corretti e con la precisione richiesta;}
		\item{\textbf{interoperabilità}: capacità di interagire con uno o più sistemi specificati;}
		\item{\textbf{conformità}: capacità di aderire a standard rilevanti al settore in esame;}
		\item{\textbf{sicurezza}: capacità di proteggere informazioni e dati.}
	\end{itemize}
	
	\subsubsubsection*{Affidabilità}
	E' la capacità del software di mantenere uno dato livello di prestazioni quando usato in specifiche condizioni. Questa capacità si traduce nelle seguenti sotto caratteristiche:
	\begin{itemize}
		\item{\textbf{maturità}: capacità di evitare il verificarsi di errori o malfunzionamenti in fase di esecuzione;}
		\item{\textbf{tolleranza agli errori}: capacità di mantenere livelli predeterminati di prestazioni anche in presenza di malfunzionamenti o errori;}
		\item{\textbf{recuperabilità}: capacità di ripristinare il livello di prestazioni e di recupero delle informazioni rilevanti in seguito ad un malfunzionamento;}
		\item{\textbf{aderenza}: capacità di aderire a standard e regole inerenti all'affidabilità.}
	\end{itemize}
	
	\subsubsubsection*{Efficenza}
	E' la capacità del software di eseguire le funzioni prefissate minimizzando il tempo necessario e sfruttando al meglio le risorse disponibili. Questa capacità si traduce nelle seguenti sotto caratteristiche:
	\begin{itemize}
		\item{\textbf{nel tempo}: capacità di fornire appropriati tempi di risposta;}
		\item{\textbf{nello spazio}: capacità di utilizzare un appropriato numero di risorse.}
	\end{itemize}
	
	\subsubsubsection*{Usabilità}
	E' la capacità del software di essere capito, appreso, usato ed accettato positivamente dall'utente. Questa capacità si traduce nelle seguenti sotto caratteristiche:
	\begin{itemize}
		\item{\textbf{comprensibilità}: capacità di essere chiaro riguardo le proprie funzionalità ed il proprio utilizzo;}
		\item{\textbf{apprendibilità}: capacità di essere facilmente apprendibile;}
		\item{\textbf{operabilità}: capacità di permettere all'utente di raggiungere i suoi scopi e controllarne l'uso;}
		\item{\textbf{attrattività}: capacità di essere piacevole per l'utente che ne fa uso;}
		\item{\textbf{conformità}: capacità di aderire a standard o convenzioni relativi all'usabilità.}
	\end{itemize}

	\subsubsubsection*{Manutenibilità}
	E' la capacità del software di essere modificato includendo correzioni, miglioramenti od adattamenti. Questa capacità si traduce nelle seguenti sotto caratteristiche:
	\begin{itemize}
		\item{\textbf{analizzabilità}: capacità di essere facilmente analizzato al fine di individuare un errore;}
		\item{\textbf{modificabilità}: capacità di essere agevolmente modificato nel codice, nella progettazione o nella documentazione;}
		\item{\textbf{stabilità}: capacità di evitare effetti indesiderati a seguito di una modifica;}
		\item{\textbf{testabilità}: capacità di essere facilmente testato al fine di validare le modifiche apportate.}
	\end{itemize}

	\subsubsubsection*{Portabilità}
	E' la capacità del software di essere trasportato da un ambiente hardware/software ad un altro seguendo le evoluzioni tecnologiche. Questa capacità si traduce nelle seguenti sotto caratteristiche:
	\begin{itemize}
		\item{\textbf{adattabilità}: capacità di essere facilmente adattato a differenti ambienti operativi senza applicare modifiche;}
		\item{\textbf{installabilità}: capacità di essere installato in uno specifico ambiente;}
		\item{\textbf{conformità}: capacità di aderire a standard e convenzioni relative alla portabilità;}
		\item{\textbf{sostituibilità}: capacità di essere utilizzato al posto di un altro software per svolgere gli stessi compiti nello stesso ambiente.}
	\end{itemize}
	
	\subsubsubsection{Modello per la qualità in uso}
	Il modello per la definizione della qualità in uso elenca quattro caratteristiche generali che permettono agli utenti di ottenere specifici obiettivi. Tali caratteristiche sono:
	\begin{itemize}
		\item{\textbf{efficacia}: capacità del software di permettere agli utenti di raggiungere gli obiettivi specificati con accuratezza e completezza;}
		\item{\textbf{produttività}: capacità del software di essere efficiente rispetto alle risorse necessarie;}
		\item{\textbf{soddisfazione}: capacità del software di soddisfare gli utenti;}
		\item{\textbf{sicurezza}: capacità del software di avere dei livelli di rischio accettabili rispetto a danni nei confronti di persone, apparecchiature e ambiente operativo.}
	\end{itemize}
	
	\subsubsection{Metriche per la qualità interna}
	La qualità interna viene rilevata tramite analisi statica, ovvero le metriche scelte vengono applicate a software non eseguibile e permettono di individuare eventuali problemi che potrebbero influire sulla qualità finale del prodotto\ped{\textit{G}}. Le misure effettuate permettono di prevedere il livello di qualità esterna ed in uso del prodotto\ped{\textit{G}} finale.
	
	\subsubsection{Metriche per la qualità esterna}
	La qualità esterna viene rilevata tramite analisi dinamica. Le metriche sono applicate al software in esecuzione e ne misurano il comportamento attraverso attività di test in funzione degli obbiettivi prefissati.
	Idealmente la qualità esterna determina la qualità in uso.
	
	\subsubsection{Metrica per la qualità in uso}
	Si tratta di metriche applicabili solo al prodotto\ped{\textit{G}} finito ed in uso in condizioni reali. La qualità in uso viene raggiunta solo se è stato raggiunto sia il livello di qualità interna che di qualità esterna.