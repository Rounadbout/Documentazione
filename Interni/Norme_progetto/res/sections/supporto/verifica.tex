\subsection{Verifica}
    \subsubsection{Descrizione}
	Il processo di Verifica determina se il prodotto\ped{\textit{G}} di un'attività sia conforme o meno alle specifiche e alle aspettative già confermate, individuando quindi possibili difetti nel software e nella documentazione, e garantendo prodotti\ped{\textit{G}} corretti e completi. \\
	Il gruppo si impegna inoltre a garantire una verifica costante dei requisiti mantenendosi in stretto contatto con il proponente. 
		
    \subsubsection{Attività}
      \subsubsubsection{Analisi}
        \subsubsubsection*{Analisi statica}
          L'Analisi statica riguarda sia la documentazione che il codice, e valuta la conformità alle norme stabilite e la sua correttezza formale, senza l'esecuzione del prodotto\ped{\textit{G}} software. Questo tipo di analisi, si serve di metodi formali (implementati tramite macchine) e metodi manuali (implementati solo per prodotti\ped{\textit{G}} semplici).\\ I metodi manuali sono:
          \begin{itemize}
            \item \textbf{walkthrough:} analisi dei documenti (o diversi file) nella loro interezza, per ricercare eventuali difetti identificati da un Verificatore, che andranno eventualmente corretti dallo sviluppatore (o redattore\ped{\textit{G}} del documento);
            \item \textbf{inspection:} analisi mirata dell'oggetto di verifica da parte di un Verificatore, che utilizza delle liste di controllo per cercare difetti specifici in sezioni specifiche. Alcuni errori comuni sono elencati di seguito:
              \begin{itemize}
                \item formato di date (YYYY-MM-DD);
                \item formato di elenchi puntati (punteggiatura alla fine di ogni elemento ";" o ".");
                \item stile del testo per termini particolari (es. Termini in corsivo);
                \item tempo verbale (il presente è sempre preferibile).
              \end{itemize}
          \end{itemize}

        \subsubsubsection*{Analisi dinamica}
          L'Analisi dinamica richiede l'esecuzione del prodotto\ped{\textit{G}} software e viene effettuata tramite una suite di test, che garantisce nel complesso la verifica del prodotto\ped{\textit{G}} stesso.

      \subsubsubsection{Test}
        Sono parte costituente dell'Analisi dinamica e hanno lo scopo di verificare il corretto funzionamento del codice. I test, per poter essere considerati ben scritti, devono essere:
        \begin{itemize}
          \item \textbf{Ripetibili:} quindi devono specificare:
            \begin{itemize}
              \item l'ambiente di esecuzione\ped{\textit{G}};
              \item l'input e l'output atteso;
              \item metodi di interpretazione dei risultati.
            \end{itemize}
          \item \textbf{Automatizzati:} quindi devono basarsi su strumenti che ne permettano l'esecuzione automatica, ne registrino i risultati e provvedano a notificare i soggetti ad essi interessati.
        \end{itemize}
        Esistono diversi tipi di test:
        \begin{itemize}
          \item \textbf{Test di unità:} riguardano il funzionamento di singole unità di software;
          \item \textbf{Test di integrazione:} test eseguiti per verificare la corretta integrazione di multiple unità in un unica componente. I componenti vengono verificati in maniera incrementale, con l'aggiunta di altre unità o gruppi di unità corretti, fino ad arrivare al sistema completo;
          \item \textbf{Test di sistema:} test del sistema nella sua interezza e confronto dei risultati con quanto richiesto dall'\AdR{}, così da accertare la copertura dei requisiti funzionali. Introduce il processo di Validazione;
          \item \textbf{Test di regressione:} utili a controllare che eventuali modifiche al sistema non compromettano funzionalità già testate in precedenza. Consiste nella reiterazione di test già esistenti, per l'unità modificata ed altre unità connesse a quest'ultima.
        \end{itemize}

	\subsubsection{Metriche}
		\subsubsection*{Code coverage (CC)}
			\begin{itemize}
				\item{\textbf{Descrizione}}: la metrica CC indica la percentuale di righe di codice percorse durante il test rispetto alle linee di codice totali. A granularità più fine, la code coverage misura quattro aspetti: 
					\begin{itemize}
						\item{\textbf{statement coverage}}: percentuale di statement percorsi dai test; 
						\item{\textbf{branch coverage}}: percentuale di diramazioni percorse dai test; 
						\item{\textbf{function converage}}: percentuale di funzioni percorse dai test; 
						\item{\textbf{line coverage}}: percentuale di righe di codice percorse dai test.  
					\end{itemize} 
				\item{\textbf{unità di misura}}: la metrica viene espressa tramite una percentuale;
				\item{\textbf{formula: }} $ CC = \frac{\#linee\_di\_codice\_eseguite\_dal\_test}{\#linee\_di\_codice\_totali} $
				\item{\textbf{risultato}}: il risultato è compreso tra 0.0\% e 100.0\%, in particolare: 
					\begin{itemize}
						\item se il risultato è pari a 0.0\%, allora non sono presenti test all'interno dell'applicativo; 
						\item se il risultato è pari a 100.0\%, allora i test coprono in maniera completa il codice che costituisce l'applicativo. 
					\end{itemize} 
			\end{itemize}

    \subsubsection{Strumenti}
    \subsubsubsection{Verifica ortografica}
    Per verifica ortografica durane la redazione dei documenti è stato utilizzato lo strumento integrato in TexStudio. Esso si occupa di sottolineare in verde le ripetizioni a breve distanza e in rosso le parole errate.
    