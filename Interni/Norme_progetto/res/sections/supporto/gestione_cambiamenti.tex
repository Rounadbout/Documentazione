\subsection{Gestione dei cambiamenti}
	
	\subsubsection{Descrizione}
		La Gestione dei cambiamenti risulta fondamentale a seguito della rilevazione di uno o più errori. Tramite questo processo è possibile definire ed attuare attività volte a correggere le problematiche riscontrate.
	
	
	\subsubsection{Attività}
		\subsubsubsection{Gestione delle Issue}
			Durante il progetto il team ha previsto di dover risolvere uno o più errori attivando un piano di mitigazione degli stessi. Si è individuato il sistema Issue\ped{\textit{G}} Tracking System (ITS) attraverso il quale è possibile trasformare in Issue quanto rilevato erroneo. Grazie a tale strumento è possibile risolvere tali problematiche assegnandole ai rispettivi responsabili per essere svolte entro scadenze prestabilite.
	
	
	\subsubsection{Metriche}
		\subsubsubsection{Numero di Issue aperte}
			Si è convenuto che il numero di Issue\ped{\textit{G}} aperte debba essere pari a 0. Nel caso in cui siano presenti pendenze, sarà necessario attuare quanto possibile per risolverle, sempre rimanendo entro la scadenza prefissata delle stesse.
			
			
	\subsubsection{Strumenti}
		\subsubsubsection{GitHub}
			Per comprendere questo processo il team ha scelto di utilizzare la componente Issue\ped{\textit{G}} Tracking System (ITS) di GitHub\ped{\textit{G}}. Principalmente è stato scelto di utilizzare questa piattaforma in quanto già utilizzata per il sistema di versionamento\ped{\textit{G}}.