  \subsection{Gestione della configurazione}
    \subsubsection{Descrizione}
      Questo processo, definisce le norme utili alla predisposizione di un workspace ordinato e accessibile, andando a controllare e automatizzare lo stato dei documenti e delle componenti software. Si compone di una serie di attività, di seguito descritte.
      %da aggiungere
    \subsubsection{Attività}
    \subsubsubsection{Versionamento}
      \subsubsubsection*{Codice di versione}
        Tutti i documenti prodotti vengono conservati in una repository\ped{\textit{G}} e devono essere versionati tramite un sistema identificativo. I numeri di versione rispettano la struttura seguente:
        \begin{center}
          \textbf{X.Y.Z}
        \end{center}
        Dove:
        \begin{itemize}
          \item \textbf{X:} rappresenta una versione completa del documento pronta al rilascio esterno:
            \begin{itemize}
              \item parte da 0 (puó soltanto aumentare);
              \item viene incrementato soltanto dopo l'approvazione del documento da parte del Responsabile di progetto.
            \end{itemize}
          \item \textbf{Y:} rappresenta una versione del documento che è stata oggetto di verifica, da parte di un Verificatore:
            \begin{itemize}
              \item parte da 0;
              \item viene incrementato ad ogni verifica dell'intero documento;
              \item si resetta ad ogni incremento di X.
            \end{itemize}
          \item \textbf{Z:} rappresenta una versione del documento in fase di stesura:
            \begin{itemize}
              \item parte da 0;
              \item incrementato dal redattore\ped{\textit{G}} ad ogni modifica validata da un Verificatore;
              \item si resetta ad ogni incremento di Y.
            \end{itemize}
        \end{itemize}

      \subsubsubsection*{Repository}
        La versione ufficiale del progetto viene mantenuta in una repository\ped{\textit{G}} remota nel sito GitHub\ped{\textit{G}}, appartenete alla \textit{GitHub\ped{\textit{G}} organization} del gruppo \Gruppo{}, al link:
        \begin{center}
          \url{https://github.com/RoundaboutTeam/etherless}
        \end{center}
    	Tale repository\ped{\textit{G}} contiene a sua volta quattro Git\ped{\textit{G}} submodule, che fanno riferimento alle ultime versioni stabili dei tre moduli di Etherless e della documentazione ufficiale:
    	\begin{center}
    		\url{https://github.com/RoundaboutTeam/etherless-cli}
    		\url{https://github.com/RoundaboutTeam/etherless-smart}
    		\url{https://github.com/RoundaboutTeam/etherless-server}
    		\url{https://github.com/RoundaboutTeam/Documentazione}
    	\end{center}
        Ogni membro del gruppo lavora su una copia locale della repository\ped{\textit{G}} sul proprio computer, interagendo con il VCS\ped{\textit{G}} e la repository\ped{\textit{G}} remota sia attraverso linea di comando, sia tramite software come GitHub\ped{\textit{G}} Desktop e GitKraken.

      \subsubsubsection*{Utilizzo di Git}
        Per sfruttare in modo efficace le funzionalità offerte da Git\ped{\textit{G}} e promuovere il lavoro collaborativo, la repository\ped{\textit{G}} è strutturata in vari branch\ped{\textit{G}}, ognuno relativo ad una particolare feature o componente del progetto (es. Un branch\ped{\textit{G}} per la stesura delle \NdP{} chiamato feature/norme\_di\_progetto). Dunque ogni membro del gruppo che necessita di operare su una certa componente del progetto deve:
        \begin{itemize}
          \item scegliere il branch\ped{\textit{G}} adatto su cui lavorare;
          \item spostarsi su tale branch\ped{\textit{G}};
          \item effettuare un pull\ped{\textit{G}} dal repository\ped{\textit{G}} remoto per trasferire una copia aggiornata nella propria repository\ped{\textit{G}} locale;
          \item svolgere il lavoro;
          \item eseguire un commit\ped{\textit{G}} del lavoro svolto, allegando una descrizione;
          \item eseguire il push\ped{\textit{G}} delle modifiche sulla repository\ped{\textit{G}} remota.
        \end{itemize}
    	Per gestire eventuali operazioni di verifica da eseguire sulle modifiche apportate è stato introdotto il meccanismo delle pull request. Quindi ogni membro del gruppo, una volta terminata la modifica di una certa componente del progetto, deve:
    	\begin{itemize}
    		\item aprire una pull request sul branch\ped{\textit{G}} dove sono state apportate le modifiche;
    		\item assegnare tale pull request ai Verificatori per una revisione e, nel caso di prodotti completi, al Responsabile per l'approvazione;
    		\item i Verificatori effettueranno la revisione ed eventuali correzioni minori;
    		\item nel caso non siano rilevati difetti bloccanti, la pull request viene accettata dai revisori e viene effettuato il merge di tale branch\ped{\textit{G}} con il branch\ped{\textit{G}} di \textit{develop}.
    	\end{itemize}

      \subsubsubsection{Gestione delle modifiche}
        Tutte le modifiche effettuate nei documenti vengono memorizzate all'interno del "Registro delle modifiche" situato in ogni documento nella pagina successiva alla copertina, come descritto nella sezione \textsection3.1.2.2.\\
        Ad ogni modifica corrisponde un commit\ped{\textit{G}} Git\ped{\textit{G}}, al quale è opportuno allegare un commento in modo da facilitare il tracciamento delle modifiche e del lavoro svolto, tramite Issue\ped{\textit{G}}.
        Ogni modifica apportata ai documenti deve essere discussa con i collaboratori che si stanno occupando del medesimo documento; nel caso in cui la modifica sia di rilevante importanza, è opportuno discuterne con tutto il gruppo durante gli appositi incontri settimanali.
      \subsubsection{Metriche}
      Non sono state attualmente individuate metriche specifiche al processo di gestione della configurazione.
      \subsubsection{Strumenti}
      Vengono utilizzati Git\ped{\textit{G}} come VCS\ped{\textit{G}} distribuito e GitHub\ped{\textit{G}} per ospitare la repository\ped{\textit{G}} di progetto.
