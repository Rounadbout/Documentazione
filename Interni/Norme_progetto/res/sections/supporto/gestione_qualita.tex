\subsection{Gestione della qualità}
    \subsubsection{Descrizione}
      Un'adeguata implementazione del processo di gestione della qualità è fondamentale per il corretto svolgimento del progetto. Questo processo intende garantire software e documenti di buona qualità, sviluppati attraverso processi chiari e ordinati. Il processo di gestione di qualità viene descritto nel dettaglio all'interno del \textit{\PdQ{} v1.0.0}. In particolare per ogni processo ed ogni prodotto\ped{\textit{G}} vengono descritti gli obiettivi e le metriche per la valutazione del raggiungimento degli stessi.
      
    \subsubsection{Attività}
    \subsubsubsection{Pianificazione}
       Il presupposto per l'implementazione del processo di gestione della qualità è la presenza di un \PdQ{}, a cui sia possibile fare riferimento per coordinare tutte le attività periodiche ed operazioni di controllo qualità. Tale piano resta valido per tutta la durata del progetto e viene costantemente aggiornato con le nuove metriche e procedure integrate.

    \subsubsubsection{Garanzia di qualità del prodotto}
      La qualità del prodotto\ped{\textit{G}} che si va a sviluppare viene garantita coordinatamente dai processi di Verifica e Validazione (descritti rispettivamente nelle sezioni \textsection3.4 e \textsection3.5), i quali puntano ad un miglioramento continuo e a verificare il rispetto delle metriche di qualità stabilite. Importante è anche il confronto costante con il Proponente\ped{\textit{G}}, per garantire che il software prodotto sia in grado di soddisfare i requisiti concordati.

    \subsubsubsection{Garanzia di qualità dei processi}
      La qualità dei processi che compongono il ciclo di vita del software deriva dallo svolgimento corretto e normato delle attività che compongono tali processi. Ottenendo quindi risultati in grado di soddisfare i requisiti previsti da contratto, rispettando norme e standard scelti come riferimento. È compito degli Amministratori di progetto monitorare lo svolgimento delle attività, ed intervenire per garantire il rispetto dei piani e delle procedure di gestione della qualità.
      L'obiettivo è il miglioramento continuo dei processi, perseguendo i principi di efficacia ed efficienza del prodotto\ped{\textit{G}} durante tutto il suo ciclo di vita.   
	
	\subsubsection{Metriche}
	\subsubsubsection*{Percentuale di metriche soddisfatte (PMS)}
	\begin{itemize}
		\item \textbf{Descrizione}: è un valore percentuale che punta a rappresentare la qualità del processo/prodotto\ped{\textit{G}} sotto analisi. È basato sul numero di metriche che hanno raggiunto un valore considerato accettabile, rapportato con il numero totale di metriche applicate nella valutazione del processo/prodotto\ped{\textit{G}};
		\item \textbf{unità di misura}: la metrica è espressa tramite un numero percentuale;
		\item \textbf{formula}: PMS = $\displaystyle\frac{\#metriche\_soddisfatte}{\#totale\_di\_metriche}\times100$;
		\item \textbf{risultato}: 
		\begin{itemize}
			\item {se il risultato è minore di 60, la qualità del processo/prodotto\ped{\textit{G}} è considerata non accettabile. Vanno considerate procedure di risoluzione quali:	ricalcolo del PMS, revisione del processo/prodotto\ped{\textit{G}} sotto analisi, revisione delle metriche applicate;}
			\item {se il risultato è minore di 90, la qualità del processo/prodotto\ped{\textit{G}} è considerata accettabile, ma ancora migliorabile;}
			\item {se il risultato è maggiore di 90, la qualità del processo/prodotto\ped{\textit{G}} è considerata ideale.}
		\end{itemize}
	\end{itemize}	
    \subsubsection{Strumenti}
      Gli strumenti di riferimento per la qualità sono:
      \begin{itemize}
      	\item{parte dei processi forniti dallo standard ISO 12207;}
      	\item{metriche di qualità stabilite.}
      \end{itemize}
  	  Gli Amministratori del progetto sfruttano inoltre l'apparato Issue\ped{\textit{G}} Tracking System di \textbf{\mbox{GitHub}}\ped{\textit{G}}, per monitorare costantemente lo svolgimento delle attività del progetto.

