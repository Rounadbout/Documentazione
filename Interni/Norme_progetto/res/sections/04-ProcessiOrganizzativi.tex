\section{Processi Organizzativi}
\subsection{Scopo}
	 Fornire ai membri del gruppo un piano organizzativo efficace ed efficiente.
	 
	 
\subsection{Aspettative}
	Le aspettative di questo Processo possono essere riassunte in:
	\begin{itemize}
		\item Definizione dei ruoli all'interno del progetto;
		\item Definizione delle procedure da adottate per:
		\begin{itemize}
			\item Effettuare riunioni;
			\item Definire un piano per l'esecuzione dei compiti assegnati attraverso la formazione;
			\item Comunicare efficacemente tra le diverse parti attraverso la descrizione degli strumenti utilizzati.
		\end{itemize}
	\end{itemize}

	
\subsection{Descrizione}
	I Processi Organizzativi sono tutti i processi propedeutici o ausiliari, in alcuni casi essenziali affinché i Processi Principali funzionino in modo adeguato. In questa sezione verranno illustrate tutte le norme da rispettare per l'organizzazione tra le parti operanti.
	



\subsection{Ruoli di progetto}
	Tutti i ruoli saranno ricoperti da ciascun componente del gruppo in rotazione, facendo sì che ogni membro possa assumere almeno una volta ciascuno di essi. Nel documento {\it Piano di Progetto} vengono organizzate e pianificate le attività assegnate ai specifici ruoli previsti nell’attività di progetto. Nello specifico ciascun ruolo viene esposto di seguito:
	
	\subsubsection{Responsabile di Progetto}
	Il Responsabile di Progetto ricopre una figura essenziale partecipando al progetto dall'inizio fino alla fine. Ha la responsabilità delle decisioni e scelte che vengono intraprese, coordinando l'intero progetto. Rappresenta il gruppo nei confronti dei Committenti e del Proponente.
	Riassunto delle mansioni:
	\begin{itemize}
		\item Coordinamento, pianificazione e controllo delle attività;
		\item Gestione delle risorse umane;
		\item Approvazione della documentazione;
		\item Approvazione dell'offerta economica;
		\item Preventiva l'analisi dei rischi e loro eventuale gestione.
	\end{itemize}
	
	\subsubsection{Amministratore}
	L'Amministratore è la figura che ha come compito principale quello di controllare ed amministrare l'ecosistema lavorativo. Inoltre ha la diretta responsabilità sull'efficienza e sulla capacità operativa dell'ambiente di lavoro.
	Riassunto delle mansioni:
	\begin{itemize}
		\item Studio e ricerca di strumenti che riducano il più possibile l'impiego di risorse umane e che automatizzino tutto ciò che è possibile fare attraverso l'utilizzo di software;
		\item Trovare soluzioni ai problemi legati alla difficoltà di gestione dei processi e risorse, attraverso la realizzazione o ricerca di strumenti adatti a tale scopo;
		\item Controllo delle versioni e configurazioni del prodotto;
		\item Gestione del versionamento della documentazione del progetto e della sua archiviazione;
		\item Fornisce procedure e strumenti di monitoraggio/segnalazione, in modo da garantire un corretto controllo di qualità.
	\end{itemize}

	\subsubsection{Analista}
	L'Analista è la figura che ha come suo compito principale quello di essere il responsabile delle attività di analisi. Deve effettuare studi e ricerche in maniera molto approfondita per conoscere bene il dominio del problema. Non è necessario che partecipi al progetto fino al termine, ma il suo operato è fondamentale fin dalla fase iniziale, in quanto le sue scelte decisionali hanno un grande impatto sul successo dell'intero progetto.
	Riassunto delle mansioni:
	\begin{itemize}
		\item Studia e definisce il problema da risolvere, rilevandone la complessità;
		\item Analizza il dominio delle richieste tramite lo studio dei bisogni, espliciti ed impliciti;
		\item Analizza il dominio applicativo: gli utilizzatori e l'ambiente di utilizzo;
		\item Ha il compito di stesura dei documenti {\it Analisi dei Requisiti} e dello {\it Studio di Fattibilità}.
	\end{itemize}

	\subsubsection{Progettista}
	Il Progettista è la figura che si occupa delle scelte architetturali del progetto e ne influenza gli aspetti tecnici e tecnologici. Utilizzando le attività svolte dall'analista, il progettista ha il compito di trovare una soluzione attuabile, comprensibile e motivata.
	Riassunto delle mansioni:
	\begin{itemize}
		\item Produrre una soluzione attuabile, comprensibile e motivata;
		\item Effettuare scelte su aspetti progettuali applicando al prodotto soluzioni note ed ottimizzate;
		\item Effettuare scelte che portino ad avere il prodotto facilmente manutenibile.
	\end{itemize}

	\subsubsection{Programmatore}
	Il Programmatore è la figura che ha la responsabilità di codifica del codice e della creazione delle componenti di supporto, indispensabili per poter effettuare le prove di verifica e di validazione.
	Riassunto delle mansioni:
	\begin{itemize}
		\item Implementazione precisa e scrupolosa delle soluzioni generate dal Progettista;
		\item Scrittura di codice sorgente altamente e facilmente mantenibile;
		\item Versionamento e documentazione del codice prodotto;
		\item Realizzazione degli strumenti per la verifica e la validazione del software.
	\end{itemize}

	\subsubsection{Verificatore}
	Il Verificatore è la figura che ha il compito di effettuare una attenta verifica del prodotto, dando particolare attenzione che rispecchi le normative di progetto. Oltre alle grandi conoscenze di tali norme, il verificatore deve avere buone capacità di giudizio.
	Riassunto delle mansioni:
	\begin{itemize}
		\item Controllare che le attività svolte siano conformi alle normative stabilite;
		\item Vigilare sulla conformità di ogni stadio del ciclo di vita del prodotto.
	\end{itemize}


\subsection{Procedure}
	\subsubsection{Riunioni interne}
		Le riunioni interne sono aperte esclusivamente agli 8 membri del gruppo Roundabout. Modalità di incontro sono ritenute valide esclusivamente con due modalità:
		\begin{itemize}
			\item \textbf{Incontri in video-conferenza:} Effettuati tramite l'applicativo \textit{Microsoft Teams};
			\item \textbf{Incontri di persona:} Effettuati trovandosi fisicamente in uno stesso luogo.
		\end{itemize}
		Considerata la situazione extra-progettuale relativa all'emergenza COVID-19, le riunioni iniziali saranno effettuate esclusivamente a distanza tramite Video-Conferenza.
		Affinché le riunioni siano ritenute valide, all'incontro dovranno essere presenti almeno 6 componenti del team.
	
	\subsubsection{Riunioni esterne}
		Le riunioni esterne comprendono tutti gli incontri che coinvolgono, oltre ai membri del team Roundabout, anche altri soggetti esterni.\\
		Questi incontri saranno tenuti telematicamente, fino al termine dell'emergenza COVID-19 e, successivamente potranno tenersi in riunioni fisiche. In caso di video-conferenza sarà da privilegiare lo strumento di comunicazione proposto dai soggetti esterni, similmente per le riunioni fisiche, in luoghi proposti dai soggetti esterni.\\
		Nel caso si scelga di usufruire dei locali di Torre Archimede occorre chiedere il permesso al Prof. Tullio Vardanega all'indirizzo mail \href{mailto:tullio.vardanega@math.unipd.it}{tullio.vardanega@math.unipd.it}.
	
	\subsubsection{Gestione delle riunioni}
		Le riunioni sono indette dal \textit{Responsabile di Progetto}, il quale ha il compito di:
		\begin{itemize}
			\item Definire la data e l'orario delle riunioni, sia interne che esterne, considerando la disponibilità dei partecipanti;
			\item Stabilire l'oggetto della riunione;
			\item Valutare le richieste relative alle riunioni da parte dei componenti del Team Roundabout e dai soggetti esterni;
			\item Verificare ed approvare il verbale redatto dal Segretario della riunione, il quale verrà nominato ad inizio incontro;
		\end{itemize}
		I partecipanti delle riunioni dovranno presentarsi puntuali alle riunioni e, in caso di imprevisti comunicarli con congruo preavviso.
		Le decisioni da intraprendere durante le riunini, saranno ritenute approvate nel caso di maggioranza da parte dei partecipanti.\\
		Al termine di ciascuna riunione sarà redatto un \textbf{Verbale}, da parte del Segretario, che dovrà contenere tutte le informazioni sulla riunione. In particolare dovrà contenere:
		\begin{itemize}
			\item Luogo;
			\item Data;
			\item Ora di inizio;
			\item Ora di fine;
			\item Elenco dei partecipanti e, nel caso di abbandono di un partecipante per esigenze di qualsiasi motivazione, dovrà essere annotato;
			\item Segretario;
			\item Ordine del giorno;
			\item Decisioni e discussioni nel dettaglio suddivise per punti;
			\item Riepilogo.
		\end{itemize}
	
	\subsubsection{Formazione}
		Per avere i membri del gruppo Roundabout allineati relativamente alle conoscenze degli strumenti applicativi utilizzati, si rende necessario procedere ad un adeguata formazione dei singoli attraverso lo studio autonomo. In particolare, oltre al materiale indicato nella sotto sezione \textit{Riferimenti Informativi}, anche la seguente documentazione:
		\begin{itemize}
			\item \LaTeX{}: \href{latex-project.org}{latex-project.org}
			\item Solidity: \href{solidity.readthedocs.io}{solidity.readthedocs.io}
			\item TypeScript: \href{www.typescriptlang.org}{www.typescriptlang.org}
		\end{itemize}
		Oltre allo studio autonomo, un altra forma di formazione sarà quella di apprendere dall'operato altrui, facendo si che le proprie conoscenze vengano arrichite da quelle di altre persone maggiormente preparate in specifici argomenti.
		
	

\subsection{Strumenti}
	I membri del gruppo Roundabout potranno lavorare indifferentemente su Windows, Mac OS X o Linux in quanto i principali strumenti necessari ai fini del progetto, sono disponibili per tutti i sistemi operativi utilizzati. Gli applicativi organizzativi utilizzati saranno:
	
	\subsubsection{Microsoft Teams}
		Si è individuato Microsoft Teams come strumento di condivisione e comunicazione interno. La decisione è motivata dalla notevole multifunzionalità dell'applicativo, in quanto consente video-chiamate, condivisione di file e chat divise per argomento.
		
	\subsubsection{Git e GitHub}
		Come strumento di versionamento si è deciso di utilizzare \textbf{Git}, uno strumento di versionamento veloce e facile da utilizzare, che rappresenta uno dei migliori strumenti attualmente esistenti. Per lo sviluppo collaborativo abbiamo deciso di appoggiarci al servizio \textbf{GitHub} il quale fornisce non solo un repository Git, ma anche strumenti utili alla collaborazione fra più persone.
		
	\subsubsection{\LaTeX{}}
		Per gestire la documentazione e quindi la scrittura dei documenti si utilizzerà \textbf{\LaTeX}. Nonostante non sia di immediata comprensione, si è scelto questo linguaggio di markup per le sue potenzialità:
		\begin{itemize}
			\item Creazione di documenti formali e divisi in sezioni molto velocemente;
			\item Separazione del contenuto dalla formattazione;
			\item Impaginazione perfetta;
		\end{itemize}
	
	\subsubsection{Gmail}
		Per la gestione della corrispondenza si è scelto di creare una casella mail su dominio \textbf{Gmail}.
		
	\subsubsection{Telegram}
		Utilizzato come mezzo di comunicazione interno per scambiare informazioni veloci ed informali.
		
	%Da inserire i programmi di sviluppo per Solidity e TypeScript


