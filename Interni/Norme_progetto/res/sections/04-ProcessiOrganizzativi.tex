\section{Processi Organizzativi}
	Questa sezione comprende tutti i processi propedeutici o ausiliari, utili affinché i Processi Principali funzionino in modo adeguato. Di seguito vengono illustrate tutte le norme da rispettare per un'efficace organizzazione tra le parti operanti.

	\subsection{Gestione Organizzativa}
	
		\subsubsection{Descrizione}
			La Gestione Organizzativa è il processo che descrive le scelte sottostanti la suddivisione e coordinazione del lavoro, all'interno del progetto. Lo scopo principale di questo processo è quello di fornire ai membri del gruppo un piano organizzativo efficace ed efficiente.
			
		\subsubsection{Ruoli di progetto}
			Ciascun componente del gruppo ricopre un ruolo di progetto a rotazione, facendo sì che ogni membro possa assumere almeno una volta ciascuno di essi nel corso del progetto. Nel documento \PdP{} vengono organizzate e pianificate le attività assegnate ai specifici ruoli previsti nell'attività di progetto. Ciascun ruolo viene descritto di seguito:
			
			\subsubsubsection{Responsabile di Progetto}
			Il Responsabile di Progetto è una figura essenziale, che partecipa al progetto dall'inizio fino alla fine. È responsabile delle decisioni e scelte che vengono intraprese, coordinando l'intero progetto. Rappresenta il gruppo di fronte a soggetti esterni, come Committente\ped{\textit{G}} e Proponente\ped{\textit{G}}.
			Riassunto delle mansioni:
			\begin{itemize}
				\item coordinamento, pianificazione e controllo delle attività;
				\item gestione delle risorse umane;
				\item approvazione della documentazione;
				\item approvazione dell'offerta economica;
				\item preventiva l'analisi dei rischi e loro eventuale gestione.
			\end{itemize}
			
			\subsubsubsection{Amministratore}
			L'Amministratore è la figura che ha come compito principale quello di controllare ed amministrare l'ecosistema lavorativo. Inoltre ha diretta responsabilità sull'efficienza e sulla capacità operativa dell'ambiente di lavoro.
			Riassunto delle mansioni:
			\begin{itemize}
				\item studio e ricerca di strumenti che riducano il più possibile l'impiego di risorse umane e che automatizzino tutto ciò che è possibile fare attraverso l'utilizzo di software;
				\item trovare soluzioni ai problemi legati alla difficoltà di gestione dei processi e risorse, attraverso la realizzazione o la ricerca di strumenti adatti a tale scopo;
				\item controllo delle versioni e configurazioni del prodotto\ped{\textit{G}};
				\item gestione del versionamento della documentazione del progetto e della sua archiviazione;
				\item fornisce procedure e strumenti di monitoraggio/segnalazione, in modo da garantire un corretto controllo di qualità.
			\end{itemize}
		
			\subsubsubsection{Analista}
			L'Analista è il responsabile delle attività di analisi. Deve effettuare studi e ricerche in maniera molto approfondita per conoscere bene il dominio\ped{\textit{G}} del problema. Non è necessario che partecipi al progetto fino al termine, ma il suo operato è fondamentale fin dalla fase iniziale, in quanto le sue scelte decisionali hanno un grande impatto sul successo dell'intero progetto.
			Riassunto delle mansioni:
			\begin{itemize}
				\item studia e definisce il problema da risolvere, rilevandone la complessità;
				\item analizza il dominio\ped{\textit{G}} delle richieste tramite lo studio dei bisogni, espliciti ed impliciti;
				\item analizza il dominio\ped{\textit{G}} applicativo: gli utilizzatori e l'ambiente di utilizzo;
				\item ha il compito di stesura dei documenti: \AdR{} e \SdF{}.
			\end{itemize}
		
			\subsubsubsection{Progettista}
			Il Progettista è la figura che si occupa delle scelte architetturali del progetto e ne influenza gli aspetti tecnici e tecnologici. Utilizzando le attività svolte dall'Analista, il Progettista ha il compito di trovare una soluzione attuabile, comprensibile e motivata.
			Riassunto delle mansioni:
			\begin{itemize}
				\item produrre una soluzione attuabile, comprensibile e motivata;
				\item effettuare scelte su aspetti progettuali, applicando al prodotto\ped{\textit{G}} soluzioni note ed ottimizzate;
				\item effettuare scelte che portino ad avere un prodotto\ped{\textit{G}} facilmente manutenibile.
			\end{itemize}
		
			\subsubsubsection{Programmatore}
			Il Programmatore è la figura responsabile della codifica del codice e della creazione delle componenti di supporto, indispensabili per poter effettuare le prove di verifica e di validazione.
			Riassunto delle mansioni:
			\begin{itemize}
				\item implementazione precisa e scrupolosa delle soluzioni generate dal Progettista;
				\item scrittura di codice sorgente altamente e facilmente manutenibile;
				\item versionamento e documentazione del codice prodotto\ped{\textit{G}};
				\item realizzazione di strumenti per la verifica e la validazione del software.
			\end{itemize}
		
			\subsubsubsection{Verificatore}
			Il Verificatore è la figura che ha il compito di effettuare una attenta verifica del prodotto\ped{\textit{G}}, dando particolare attenzione al rispetto delle normative di progetto. Oltre ad una grande conoscenza di tali norme, il verificatore deve avere buone capacità di giudizio.
			Riassunto delle mansioni:
			\begin{itemize}
				\item controllare che le attività svolte siano conformi alle normative stabilite;
				\item vigilare sull'integrità del prodotto\ped{\textit{G}} ad ogni stadio del suo ciclo di vita.
				\item comunicare eventuali errori identificati al responsabile dell'oggetto preso in esame.
			\end{itemize}
		
		
		\subsubsection{Procedure}
			\subsubsubsection{Riunioni interne}
				Le riunioni interne sono aperte esclusivamente agli 8 membri del gruppo \Gruppo{}. Le occasioni di incontro sono ritenute valide esclusivamente in due modalità:
				\begin{itemize}
					\item \textbf{incontri in video-conferenza:} Effettuati tramite l'applicativo Microsoft Teams\ped{\textit{G}};
					\item \textbf{incontri di persona:} Effettuati trovandosi fisicamente in uno stesso luogo.
				\end{itemize}
				Considerata la situazione extra-progettuale relativa all'emergenza COVID-19\ped{\textit{G}}, le riunioni iniziali saranno effettuate esclusivamente a distanza tramite Video-Conferenza.
				Affinché le riunioni siano ritenute valide, all'incontro dovranno essere presenti almeno 6 componenti del team.
			
			\subsubsubsection{Riunioni esterne}
				Le riunioni esterne comprendono tutti gli incontri che coinvolgono, oltre ai membri del team \Gruppo{}, anche altri soggetti esterni.\\
				Questi incontri sono tenuti telematicamente, fino al termine dell'emergenza COVID-19\ped{\textit{G}} e, successivamente potranno tenersi attraverso riunioni fisiche. In caso di video-conferenza è da privilegiare lo strumento di comunicazione proposto dai soggetti esterni, similmente per le riunioni fisiche, in luoghi proposti dai soggetti esterni.\\
				Nel caso si scelga di usufruire dei locali di Torre Archimede è necessario chiedere il permesso al \TV{} all'indirizzo mail \href{mailto:tullio.vardanega@math.unipd.it}{tullio.vardanega@math.unipd.it}.
			
			\subsubsubsection{Gestione delle riunioni}
				Le riunioni sono indette dal Responsabile di Progetto, il quale ha il compito di:
				\begin{itemize}
					\item definire la data e l'orario delle riunioni, sia interne che esterne, considerando la disponibilità dei partecipanti;
					\item stabilire l'oggetto della riunione;
					\item valutare le richieste relative alle riunioni da parte dei componenti del Team \Gruppo{} e dai soggetti esterni;
					\item verificare ed approvare il verbale redatto dal Segretario della riunione, il quale verrà nominato ad inizio incontro;
				\end{itemize}
				I partecipanti devono presentarsi puntuali alle riunioni e, in caso di imprevisti comunicarli con congruo preavviso.
				Le decisioni da intraprendere durante le riunioni, sono ritenute approvate nel caso di maggioranza da parte dei partecipanti.\\
				Al termine di ciascuna riunione viene redatto un \Verbale{}\ped{\textit{G}}, da parte del Segretario, che deve contenere tutte le informazioni sulla riunione.
		
		
		\subsubsection{Strumenti}
			I membri del gruppo \Gruppo{} possono lavorare indifferentemente su Windows, Mac OS X o Linux in quanto i principali strumenti necessari ai fini del progetto, sono disponibili per tutti i sistemi operativi citati. Gli applicativi organizzativi utilizzati sono:
			
			\subsubsubsection{Microsoft Teams}
				Si è individuato Microsoft Teams\ped{\textit{G}} come strumento di condivisione e comunicazione interno. La decisione è motivata dalla notevole multifunzionalità dell'applicativo, in quanto consente video-chiamate, condivisione di file e chat divise per argomento.
				
			\subsubsubsection{Git e GitHub}
				Come software di controllo versione\ped{\textit{G}} si è deciso di impiegare \textbf{Git}\ped{\textit{G}}, che rappresenta uno dei migliori strumenti attualmente esistenti, per quanto riguarda performance e facilità di utilizzo. Per lo sviluppo collaborativo abbiamo deciso di appoggiarci al servizio \textbf{GitHub}\ped{\textit{G}} il quale fornisce non solo un repository\ped{\textit{G}} Git\ped{\textit{G}}, ma anche funzionalità utili per la cooperazione fra più persone.
							
			\subsubsubsection{Gmail}
				Per la gestione della corrispondenza si è scelto di creare una casella mail su dominio \textbf{Gmail}\ped{\textit{G}}.
				
			\subsubsubsection{Slack}
				In accordo con i Proponenti\ped{\textit{G}}, si è concordato l'utilizzo della piattaforma \textbf{Slack}\ped{\textit{G}} per avere un canale di comunicazione più diretto e veloce rispetto alle mail.
			
			\subsubsubsection{Zoom}
				Per la gestione delle video-chiamate con il Proponente\ped{\textit{G}} ed il Committente, si è scelto di utilizzare l'applicazione \textbf{Zoom}\ped{\textit{G}}.
				
			\subsubsubsection{Telegram}
				Utilizzato come mezzo di comunicazione interno per scambiare informazioni velocemente ed in maniera informale.
				

	\subsection{Formazione}
		
		\subsubsection{Descrizione}	
		Per avere i membri del gruppo \Gruppo{} allineati relativamente alle conoscenze degli strumenti applicativi utilizzati, si rende necessario procedere ad un adeguata formazione dei singoli attraverso lo studio autonomo. Lo scopo di questo processo è appunto quello di fornire ai diversi membri del gruppo una sufficiente formazione.\\
		In particolare, oltre al materiale indicato nella sotto sezione \textit{Riferimenti Informativi}, anche la seguente documentazione:
		\begin{itemize}
			\item \LaTeX{}\ped{\textit{G}}: \href{latex-project.org}{latex-project.org}
			\item Solidity\ped{\textit{G}}: \href{solidity.readthedocs.io}{solidity.readthedocs.io}
			\item TypeScript\ped{\textit{G}}: \href{www.typescriptlang.org}{www.typescriptlang.org}
		\end{itemize}
		Oltre allo studio autonomo, un altro metodo di formazione è quello dell'apprendimento dall'operato altrui, facendo si che le proprie conoscenze possano essere arricchite da quelle di altre persone maggiormente preparate in specifici argomenti.
