\section{Processi di Supporto}
  \subsection{Documentazione}
  \subsubsection{Scopo}
  Lo scopo di questa sezione é definire gli standard per la stesura e l'approvazione dei documenti prodotti durante il ciclo di vita del software.

  \subsubsection{Aspettative}
  Ci si aspetta di:
  \begin{itemize}
    \item Fornire un riferimento unico per la struttura  della documentazione da produrre;
    \item Indicare una serie di regole per la stesura di documenti coerenti e validi rispetto agli standard indicati.
  \end{itemize}

  \subsubsection{Descrizione}
  Questa sezione fornisce le norme per la stesura dei documenti. Queste regole devono essere seguite per la stesura di tutti i documenti ufficiali.

  \subsubsection{Attivitá}
    \subsubsubsection{Implementazione del processo}
      \paragraph{Ciclo di vita dei documenti\\}
      Ogni documento attraversa diversi stadi:
      \begin{itemize}
        \item \textbf{Creazione del documento:} il documento viene creato nella sua directory di appartenenza (secondo le indicazioni nella sezione "Documenti interni ed esterni"). Viene utilizzato un template \LaTeX contenuto nella cartella "Latex" della repository;
        \item \textbf{Stesura:} scrittura effettiva dei contenuti del documento da parte di un redattore;
        \item \textbf{Verifica:} attivitá eseguita dai Verificatori, i quali si occupano di controllare che il documento sia conforme alle \textit{\NdP{}} sia sintatticamente che semanticamente. Alla fine di ogni controllo, il resoconto della verifica verrá consegnato al Responsabile di Progetto che provvederá a notificare il redattore in caso di errori, riportando il documento al passo di "Stesura". Quando l'attivitá di verifica non rileverá ulteriori errori, il responsabile passerá il documento alla fase di "Approvazione";
        \item \textbf{Approvazione:} in questa fase il verificatore ha terminato i controlli con esito positivo, comunicandoli al responsabile il quale si occuperá di approvare il documento e rilasciarlo.
      \end{itemize}
      %Aggiungere o correggere stati del documento

      \paragraph{Documenti interni ed esterni\\}
      Ogni documento deve essere classificato come interno o esterno:
      \begin{itemize}
        \item \textbf{Interno:} il documento viene utilizzato all' interno al team;
        \item \textbf{Esterno:} il documento viene condiviso con i Committenti e la Proponente.
      \end{itemize}

      \paragraph{\NdP\\}
      (Questo stesso documento) Lo scopo delle \textit{\NdP{}} é descritto nella sezione \textsection1.1 "Scopo del Documento"\\
      Il documento é ad uso Interno.

      \paragraph{\SdF\\}
      Lo \textit{\SdF{}} ha l'obiettivo di esporre (brevemente) ogni capitolato\ped{\emph{G}} e di elencare per ognuno gli aspetti positivi e le criticitá che il team ha individuato.\\
      Il documento é ad uso Interno.
      \subparagraph*{Destinatari:}
      \begin{itemize}
        \item Membri del team \Gruppo{}.
      \end{itemize}

      \paragraph{\AdR\\}
      Lo  scopo  dell' \textit{\AdR{}} é di esporre dettagliatamente i requisiti individuati per lo sviluppo del capitolato\ped{\emph{G}} scelto C2 - \NomeProgetto{}.\\
      Il documento é ad uso Esterno.
      \subparagraph*{Destinatari:}
      \begin{itemize}
        \item Membri del team \Gruppo{};
        \item Committente;
        \item Proponente.
      \end{itemize}

      \paragraph{\PdP\\}
      Lo scopo del \textit{\PdP{}} é di organizzare le attivitá in modo da gestire le risorse disponibili in termini di tempo e "forza lavoro".\\
      Il documento é ad uso Esterno.
      \subparagraph*{Destinatari:}
      \begin{itemize}
        \item Membri del team \Gruppo{};
        \item Committente;
        \item Proponente.
      \end{itemize}

      \paragraph{\PdQ\\}
      Lo scopo del \textit{\PdQ{}} é di presentare i metodi di verifica e validazione implementati dal gruppo per garantire la qualità del prodotto e dei processi adottati.
      Il documento é ad uso Esterno.
      \subparagraph*{Destinatari:}
      \begin{itemize}
        \item Membri del team \Gruppo{};
        \item Committente;
        \item Proponente.
      \end{itemize}

      \paragraph{\Glossario\\}
      Il \Glossario{}  ha lo scopo di disambiguare alcuni termini che compaiono all'interno dei documenti e vengono utilizzati nelle comunicazioni interne ed esterne.
      \subparagraph*{Destinatari:}
      \begin{itemize}
        \item Membri del team \Gruppo{};
        \item Committente;
        \item Proponente.
      \end{itemize}

      \paragraph{Verbali\\}
      I Verbali hanno lo scopo di riassumere gli argomenti di discussione e le decisioni prese durante una riunione sia interna che esterna.
      \subparagraph*{Destinatari:}
      \begin{itemize}
        \item Membri del team \Gruppo{} (Verbali Interni);
        \item Membri del team \Gruppo{} e Proponente / Committente (Verbali Esterni).
      \end{itemize}

    \subsubsubsection{Design e sviluppo}
      \paragraph{Template \LaTeX\\}
      Per uniformare la struttura dei documenti il gruppo ha deciso di creare un template \LaTeX riutilizzabile in tutti i documenti ufficiali.

      \paragraph{Copertina\\}
      La copertina é la prima pagina di ogni documento e contiene alcune informazioni generali:
      \begin{itemize}
        \item Logo del gruppo \Gruppo{};
        \item Nome del gruppo e nome del capitolato\ped{\emph{G}} scelto: \Gruppo{} - \NomeProgetto{};
        \item Nome del documento (nel caso dei Verbali accompagnato dalla data della riunione);
        \item \textbf{Versione:} versione attuale del documento del documento (vedi sezione \textsection3.2);%aggiungere sezione norme di Versionamento
        \item \textbf{Approvazione:} nome e cognome del \textit{Responsabile di progetto};
        \item \textbf{Redazione:} nome e cognome dei Redattori del documento;
        \item \textbf{Verifica:} nome e cognome del \textit{Verificatore};
        \item \textbf{Stato:} stato del ciclo di vita in cui si trova il documento (vedi sezione \textsection3.1.4.1 paragrafo "Ciclo di vita dei documenti");
        \item \textbf{Uso:} indica se il documento é ad uso Interno o Esterno;
        \item \textbf{Destinato a:} elenco dei destinatari del documento (vedi sezione \textsection3.1.4.1 paragrafo del singolo documento);
        \item \textbf{Descrizione:} descrizione breve del documento;
        \item \textbf{Email:} indirizzo email del gruppo \Gruppo{}.
      \end{itemize}

      \paragraph{Diario delle modifiche\\}
      Inizia nella seconda pagina del documento e contiene un registro delle modifiche apportate al documento, strutturate in forma tabellare.\\
      Questa sezione é presente anche nei Verbali di riunione, nei quali si limita alla stesura, la verifica e l'approvazione del documento in questo ordine, per evitare modifiche retroattive ai verbali.\\
      Ogni riga rappresenta una modifica ed é suddivisa in quattro colonne:
      \begin{itemize}
        \item \textbf{Versione:} aggiornamento progressivo della versione:
        \item \textbf{Data:} data della modifica;
        \item \textbf{Nominativo:} nome e cognome del membro del gruppo che ha effettuato la modifica
        \item \textbf{Ruolo:} ruolo ricoperto in quel momento dal membro del team che ha effettuato la modifica;
        \item \textbf{Descrizione:} descrizione breve della modifica effettuata.
      \end{itemize}

      \paragraph{Indice\\}
      L'indice si trova nella pagina successiva e ha lo scopo di aiutare nella navigazione del documento e riassumerne la struttura in maniera visuale.\\
      Nell'indice sono elencati i numeri delle sezioni, seguiti dal titolo e dal numero di pagina. Ogni riga dell'indice é un link che porta alla sezione specificata del documento.\\
      L'indice dei contenuti puó essere seguito da un indice delle immagini e un indice delle tabelle.

      \paragraph{Contenuto\\}
      Tutte le pagine successive sono occupate dal contenuto e sono strutturate come segue:
      \begin{itemize}
        \item In alto a destra il nome del documento;
        \item In alto a sinistra il logo del gruppo \Gruppo{};
        \item Il contenuto della pagina diviso da intestazione e pié di pagina con una riga orizzontale;
        \item In basso a destra il numero di pagina nel formato: \textit{Pagina [numero di pagina] di [numero totale di pagine]}.
      \end{itemize}

      \paragraph{Verbali\\}
      Seguono la stessa struttura generale degli altri documenti ma hanno un'organizzazione specifica del contenuto, in particolare sono suddivisi in:
      \begin{itemize}
        \item \textbf{Informazioni generali:} che contengono le Informazioni dell'incontro e l'ordine del giorno;
        \item \textbf{Verbale:} che contiene la descrizione e un riassunto degli argomenti elencati nell' Ordine del giorno;
        \item \textbf{Riepilogo:} contiene il riepilogo delle decisioni prese durante la riunione, in forma tabellare. Ogni decisione e identificata da un codice scritto in forma \textit{V[T]\_1.[numero progressivo]} dove [T] indica il tipo di verbale che puó essere Interno (I) o Esterno (E).
      \end{itemize}

\newpage

      \paragraph{Norme tipografiche}
      \paragraph{Nomi dei file}
      I nomi dei file seguono le segueni regole:
      \begin{itemize}
        \item Sono composti di diverse parole,con la prima lettera minuscola, fatta eccezione per i verbali che, come descritto in seguito, iniziano per VI oppure VE;
        \item Ogni parola é separata da un underscore;
        \item Il nome del file corrisponde al nome del documento senza escludere le preposizioni, fatta eccezione per i verbali che seguono un tipo diverso di nominazione, come descritto in seguito.
      \end{itemize}
      I Verbali sono nominati seguendo la struttura \textit{V[T]\_[data]} : [T] indica il tipo di verbale che puó essere Interno (I) o Esterno (E), e [data] indica la data della riunione in formato "YYYY\_MM\_DD", ovvero anno(YYYY), mese(MM) e giorno(DD) separati da underscore.\\
      Esempi di nomi corretti:
      \begin{itemize}
        \item Per un documento qualunque: nome\_di\_file.ext;
        \item Per un verbale: VI\_2020\_03\_16.tex.
      \end{itemize}

      \paragraph{Stile del testo}
      Le seguenti norme vanno seguite per l'utilizzo di particolari stili di testo:
      \begin{itemize}
        \item \textbf{Grassetto:} usato se necessario all'inizio delle voci di un elenco puntato, a titoli o a termini significativi;
        \item \textbf{Corsivo:} usato per evidenziare proposizioni particolari all' interno del testo, in particolare il nome del progetto \textit{\NomeProgetto}, il nome del gruppo \textit{\Gruppo{}}, il nome dei vari documenti ufficiali e dei ruoli di progetto;
        \item \textbf{Maiuscoletto:} usato per la "G" a pedice che indica le parole presenti nel \Glossario{}.
      \end{itemize}

      \paragraph{Date}
       Sono scritte usando la struttura "YYYY-MM-DD" ovvero anno(YYYY), mese(MM) e giorno(DD).
      \paragraph{Orari}
      Gli orari sono scritti usando la struttura "HH.MM".
      \paragraph{Elenchi puntati:}
      Ogni elemento di un elenco puntato deve essere seguito da un ";", fatta eccezione per l'ultimo elemento che deve essere seguito da un punto.
      \paragraph{Termini del \textit{\Glossario}:}
      Ogni istanza di un termine presente nel \Glossario va contrassegnata con una G maiuscola al pedice;
      \paragraph{Tabelle e Immagini}
      Le tabelle hanno sempre un titolo e un numero, e vengono indicizzate separatamente dal normale contenuto.\\
      Le immagini sono accompagnate da una didascalia e da un numero e anch'esse sono indicizzate separatamente.

  \subsubsection{Strumenti}
    \subsubsubsection{\LaTeX}
    Per la stesura della documentazione si é scelto il linguaggio di markup \LaTeX. Nonostante non sia di immediata comprensione, si è scelto questo linguaggio in quanto permette di scrivere documenti in modo collaborativo, modulare e scalabile.
    \subsubsubsection{File condivisi di Microsoft Teams}
    Si utilizza la condivisione di file interna a Microsoft Teams per file utili non ufficiali. Questa piattaforma infatti offre un'integrazione per il lavoro simultaneo sugli stessi file, che si presenta utile per l'organizzazione interna del lavoro di gruppo.
    \subsubsubsection{Editor di testo}
    Si ha deliberato la scelta dell'editor di testo ai singoli membri del gruppo per permettere un lavoro piú efficiente, in quanto ognuno presenta familiaritá con diversi editor e ambienti di lavoro.



  \subsection{Gestione della configurazione}
  \subsection{Gestione della qualità}
  \subsection{Verifica}
  \subsection{Validazione}
