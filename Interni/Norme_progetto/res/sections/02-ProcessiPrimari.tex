\section{Processi Primari}
    \subsection{Fornitura}
        \subsubsection{Descrizione}
        Il processo di fornitura ha lo scopo di determinare l'insieme delle attività necessarie allo svolgimento del progetto. %Tale processo è composto dalle seguenti attività:
       % \begin{itemize}
           % \item{avvio;}
            %\item{preparazione di risposta alle richieste;}
           % \item{contrattazione;}
            %\item{pianificazione;}
            %\item{esecuzione e controllo;}
           % \item{revisione e valutazione;}
            %\item{consegna e completamento;}
        %\end{itemize}
         In questa sezione vengono esposte le regole che i membri del team \Gruppo{} si impegnano a rispettare nel corso delle fasi di progettazione, sviluppo e consegna della piattaforma \NomeProgetto{}, per diventare fornitori nei confronti del Proponente \Proponente{} e dei Committenti \TV{} e \RC{}. \\
         L'obiettivo del gruppo è quello di mantenere un costante dialogo con il Proponente\ped{\textit{G}} al fine di:
         \begin{itemize}
		\item{instaurare un rapporto di collaborazione;} 
		\item{comprenderne a fondo le richieste;}
		\item{determinare vincoli sui processi e sui requisiti;}
		\item{stimare i costi;}
		\item{promuovere una verifica continua;}
		\item{avere un riscontro efficace sul lavoro svolto.}
         \end{itemize}
        
        \subsubsection{Attività}
        \subsubsubsection{Studio di Fattibilità}
        Lo \SdF{} viene redatto dagli Analisti ed indica il risultato dell'attività di valutazione di ogni capitolato\ped{\textit{G}} proposto. Tra le informazioni da esso fornite, sono presenti:
        \begin{itemize}
            \item{\textbf{Informazioni generali}: presentano il nome del progetto, del Proponente\ped{\textit{G}} e del Committente\ped{\textit{G}};}
            \item{\textbf{Descrizione}: descrive sinteticamente il capitolato\ped{\textit{G}} sotto analisi;}
            \item{\textbf{Obiettivo finale}: rappresenta il prodotto\ped{\textit{G}} risultante dal completamento del progetto, con soddisfacimento dei requisiti;}
            \item{\textbf{Tecnologie coinvolte}: descrive le tecnologie necessarie al raggiungimento dell'obiettivo finale;}
            \item{\textbf{Aspetti positivi}: espongono gli elementi del capitolato\ped{\textit{G}} che hanno suscitato entusiasmo nel team;}
            \item{\textbf{Criticità}: analizza i punti critici ed i fattori di rischio relativi alla realizzazione del progetto;}
            \item{\textbf{Esito}: illustra la posizione finale del team nei confronti del capitolato\ped{\textit{G}}.}
        \end{itemize}
        
        \subsubsubsection{Piano di Progetto}
        Il Responsabile di Progetto, con l'aiuto degli Amministratori, redige un \PdP{} volto a pianificare le attività del team. Tale documento contiene:
        \begin{itemize}
        		\item{\textit{Mancano riferimenti}}
        \end{itemize}
        
        \subsubsubsection{Piano di Qualifica}
        I Progettisti ed i Verificatori redigono il \PdQ{}, il cui scopo è raccogliere le metriche e le strategie impiegate per garantire la qualità sia dei processi attuati che del prodotto\ped{\textit{G}}, nel tempo. Tale documento contiene:
        \begin{itemize}
        		\item{\textit{Mancano riferimenti}}
        \end{itemize}
        
        \subsubsection{Strumenti}
        Di seguito vengono elencati gli strumenti usati durante il processo di fornitura:
        
    \subsection{Sviluppo}
        \subsubsection{Descrizione}
        Il processo di sviluppo contiene le attività ed i compiti che devono essere svolti per produrre il software richiesto. Esso viene svolto in conformità allo standard ISO/IEC 12207-1995, comprendendo perciò le seguenti attività:
        \begin{itemize}
        \item{Analisi dei Requisiti;}
        \item{Progettazione;}
        \item{Codifica.}
        \end{itemize}
	In particolare, l'obiettivo del team è quello di sviluppare un prodotto\ped{\textit{G}} software che superi i test, soddisfi le richieste ed i requisiti, fissati dal Proponente\ped{\textit{G}}.
        
            \subsubsection{Attività}
            \subsubsubsection{Analisi dei Requisiti}
           Nell'\AdR{} gli Analisti individuano ed elencano tutti i requisiti richiesti dal Proponente\ped{\textit{G}} per il progetto in questione. Tali requisiti possono derivare da:
           \begin{itemize}
           		\item{capitolato\ped{\textit{G}} d'appalto;}
				\item{verbali di riunioni interne o esterne;}
				\item{casi d'uso.}
           \end{itemize}
           E sono fondamentali per:
            \begin{itemize}
        		\item{descrivere lo scopo del lavoro;}
				\item{fornire ai Progettisti riferimenti specifici ed affidabili;}
				\item{fissare funzionalità concordate con il cliente;}
				\item{fornire una base per raffinamenti successivi al fine di garantire un miglioramento continuo;}
				\item{facilitare le revisioni del codice;}
				\item{stimare i costi in base alla quantità di lavoro prevista.}
        	\end{itemize}                 
                 \subsubsubsection*{Requisiti}
                 
 				    Ogni requisito emerso durante l'attività di analisi dev'essere descritto dalla seguente struttura:
                    \begin{itemize}
                    	\item{codice identificativo;}
                     	\item{fonti;} 
                     	\item{relazioni di dipendenza con altri requisiti;}
                      	\item{descrizione;}
                      	\item{importanza.}
                     \end{itemize}
                     
                	\noindent Ogni requisito dev'essere corredato da un codice identificativo, che segue la dicitura: 
                    \begin{center}
                    \textbf{R[Importanza][Tipologia][Codice]}
                    \end{center}
                    Dove:
                    \begin{itemize}
                        	\item{\textbf{Importanza}: indica il grado di importanza del requisito ai fini del progetto. Può assumere i valori:}
                        	\begin{itemize}
                                	\item{\textbf{1}: requisito obbligatorio ai fini del progetto, irrinunciabile per gli stakeholders;}
                                	\item{\textbf{2}: requisito desiderabile. Non strettamente necessario ai fini del progetto ma che porta valore aggiunto;}
                                	\item{\textbf{3}: requisito opzionale, contrattabile più avanti nel progetto.}
                        	\end{itemize}
                        	\item{\textbf{Tipologia}: classe a cui appartiene il requisito in questione. Può assumere i valori:}
                        	\begin{itemize}
                        		\item{\textbf{F}: funzionale;}
                        		\item{\textbf{P}: prestazionale;}
                        		\item{\textbf{Q}: qualitativo;}
                        		\item{\textbf{V}: vincolo.}
                        	\end{itemize}
                        	\item{\textbf{Codice}: identificatore univoco del requisito}.
                    \end{itemize}
                    
                    \noindent Il codice stabilito secondo la convenzione precedente, una volta associato ad un requisito, non può più essere modificato.
                    
                \subsubsubsection*{Casi d'uso}
                                Dopo aver identificato i casi d'uso\ped{\textit{G}}, compito degli Analisti è quello di elencarli con un grado di precisione che va dal generale al particolare, usando la struttura:
                \begin{itemize}
                \item{\textit{Mancano riferimenti}}
                \end{itemize}

                \noindent Ogni caso d'uso dev'essere corredato da un codice identificativo, che segue la dicitura: 
                \begin{center}
                \textbf{UC[codice\_padre].[codice\_figlio]}
                \end{center}
                
                \noindent Dove:
                \begin{itemize}
                 	\item{\textbf{Codice padre}: numero che identifica univocamente i casi d'uso generici;}
			\item{\textbf{Codice figlio}: numero progressivo che identifica i sottocasi. Può a sua volta includere altri livelli.}
                \end{itemize}
                                
                \subsubsection*{UML}
                I diagrammi UML\ped{\textit{G}} devono essere realizzati utilizzando la versione del linguaggio \textit{v2.0}.
                
            \subsubsubsection{Progettazione}
            L'attività di progettazione definisce una soluzione del problema presentato, soddisfacente per tutti gli stakeholders, in relazione ai requisiti specificati nel documento \AdR{}. Questo garantisce che il prodotto\ped{\textit{G}} sviluppato soddisfi le proprietà e i bisogni specificati dal Proponente\ped{\textit{G}}. 
            La progettazione ha quindi come obiettivo l'elaborazione di una struttura adeguata per il sistema, poi descritta nei seguenti documenti:
            \begin{itemize}
            	\item{\textbf{\TB{}}\ped{\textit{G}}: contiene le specifiche della progettazione ad alto livello, i diagrammi UML\ped{\textit{G}} utilizzati per la realizzazione dell'architettura ed i test di verifica;}
		\item{\textbf{\PB{}}\ped{\textit{G}}: approfondisce l'attività di progettazione precedentemente trattata nella Technology Baseline\ped{\textit{G}}, specifica le definizioni delle classi e definisce i test necessari alla verifica.}
	     \end{itemize}
	     
	     \subsubsubsection*{Technology Baseline}
	     Tale documento viene redatto dal Progettista e contiene:
	     \begin{itemize}
	     	\item{\textbf{Diagrammi UML\ped{\textit{G}}}: usati per rendere più chiare le scelte progettuali adottate e ridurre le ambiguità. Possono essere:
                    \begin{itemize}
                    	\item{diagrammi delle attività;}
    			\item{diagrammi delle classi;}
    			\item{diagrammi dei package;}    			
			\item{diagrammi di sequenza.}                    
		\end{itemize}
}
		\item{\textbf{Design pattern}\ped{\textit{G}}: vengono esplicitati chiaramente i design pattern\ped{\textit{G}} utilizzati per l'architettura, accompagnandoli con una descrizione ed un diagramma, così da esporne il significato e la struttura;}
		\item{\textbf{Tracciamento delle componenti}: viene rappresentata la relazione tra ogni requisito ed il componente che lo soddisfa. Dimostrando quindi concretamente la loro completa esaustione;}
		\item{\textbf{Test di integrazione}: vengono definite delle classi di verifica, per accertarsi che ogni componente del sistema funzioni come previsto.}
	     \end{itemize}
	     
	     \subsubsubsection*{Product Baseline}
	     Tale documento viene redatto dal Progettista e contiene:
	     \begin{itemize}
	     	\item{\textbf{Diagrammi UML}\ped{\textit{G}}:
			\begin{itemize}
				\item{diagrammi delle attività;}
    				\item{diagrammi delle classi;}
				\item{diagrammi di sequenza.}
			\end{itemize}
			}
		\item{\textbf{Definizione delle classi}: ogni classe viene descritta illustrandone lo scopo e le funzionalità;}
		\item{\textbf{Tracciamento delle classi}: ogni requisito viene tracciato, in modo da garantire che ogni classe ne soddisfi almeno uno;}
		\item{\textbf{Test di unità}: vengono definiti dei test di unità per verificare che le componenti del sistema funzionino come previsto.}
	     \end{itemize}
	     
             \subsubsubsection{Codifica}
             È l'attività che permette di concretizzare la soluzione architetturale elaborata durante la fase di progettazione, mediante la programmazione vera e propria.
             Di seguito vengono fissate le norme e le convenzioni da seguire, al fine di ottenere codice leggibile ed uniforme, agevolare le fasi di manutenzione, verifica e validazione, e migliorare la qualità del prodotto\ped{\textit{G}}.
             \begin{itemize}
             	\item{\textbf{Indentazione}: i blocchi innestati devono essere correttamente indentati, usando per ciascun livello di indentazione quattro (4) spazi, fatta eccezione per i commenti. Si consiglia di configurare adeguatamente il proprio IDE\ped{\textit{G}} al fine di rispettare tale norma;}
		\item{\textbf{Parentesizzazione}: le parentesi di delimitazione dei costrutti vanno inserite in linea e non al di sotto di essi;}
		\item{\textbf{Lunghezza dei metodi}: ove possibile, preferire metodi brevi (poche righe di codice) e che assolvano il minor numero di compiti possibile;}
		\item{\textbf{Univocità dei nomi}: classi, variabili e metodi devono avere un nome univoco ed esplicativo (nome parlante), al fine di evitare ambiguità e consentire ad eventuali lettori di comprendere lo scopo di quel preciso elemento, anche senza conoscenze informatiche pregresse;}
		\item{\textbf{Classi}: le parole componenti i nomi delle classi devono iniziare con la lettera maiuscola (e.g. NomeClasse);}
		\item{\textbf{Costanti}: le costanti devono essere scritte usando solo lettere maiuscole (e.g. COSTANTE). Nel caso siano composte da più parole, queste devono essere separate dal carattere speciale underscore (e.g. NOME\_COSTANTE);}
		\item{\textbf{Metodi}: i nomi dei metodi devono iniziare con una lettera minuscola e, nel caso siano composti da più parole, quelle successive devono iniziare con una lettera maiuscola (e.g. nomeMetodo);}
		\item{\textbf{Lingua}: le parti testuali di codice ed i commenti ad esso riferiti devono essere scritti in lingua inglese;}
                	\item{\textbf{Ricorsione}: l'uso della ricorsione va evitato quanto più possibile.}
		\end{itemize}
                
        \subsubsection{Strumenti}
        Di seguito vengono elencati gli strumenti usati durante il processo di sviluppo: