\section{Processi Primari}
    \subsection{Fornitura}
        \subsubsection{Descrizione}
        Il processo di fornitura ha lo scopo di determinare l'insieme delle attività necessarie allo svolgimento del progetto. Tale processo è composto dalle seguenti fasi:
        \begin{itemize}
            \item{avvio;}
            \item{preparazione di risposta alle richieste;}
            \item{contrattazione;}
            \item{pianificazione;}
            \item{esecuzione e controllo;}
            \item{revisione e valutazione;}
            \item{consegna e completamento;}
        \end{itemize}
         In questa sezione vengono esposte le regole che i membri del team \Gruppo{} si impegnano a rispettare nel corso delle fasi di progettazione, sviluppo e consegna della piattaforma \NomeProgetto{}, per diventare fornitori nei confronti del Proponente \Proponente{} e dei Committenti \TV{} e \RC{}. \\
         L'obiettivo del gruppo è quello di mantenere un costante dialogo con il Proponente al fine di comprenderne a fondo le richieste, instaurare un rapporto di collaborazione ed avere un riscontro efficace sul lavoro svolto.
        
        \subsubsection{Studio di Fattibilità}
        Lo \SdF{} viene redatto dagli analisti ed indica per ogni capitolato:
        \begin{itemize}
            \item{\textbf{Informazioni generali}: sezione che include il nome del progetto, del Proponente e dei Committenti;}
            \item{\textbf{Descrizione}: breve descrizione del capitolato sotto analisi;}
            \item{\textbf{Obiettivo finale}: rappresenta il dominio applicativo del prodotto software da sviluppare, ovvero il suo ambito di utilizzo;}
            \item{\textbf{Tecnologie coinvolte}: descrive le tecnologie necessarie al raggiungimento dell'obiettivo finale, illustrandone il dominio tecnologico;}
            \item{\textbf{Aspetti positivi}: elementi del capitolato che hanno entusiasmato il team;}
            \item{\textbf{Criticità}: elementi del capitolato che hanno messo il team in difficoltà;}
            \item{\textbf{Esito}: posizione finale del team nei confronti del capitolato;}
        \end{itemize}
        
        \subsubsection{Piano di Progetto}
        Il Responsabile di Progetto, con l'aiuto degli Amministratori, redige un \PdP{} volto a pianificare le attività del team. Tale documento contiene:
        \begin{itemize}
        		\item{\textit{Mancano riferimenti}}
        \end{itemize}
        
        \subsubsection{Piano di Qualifica}
        I Progettisti ed i Verificatori redigono il \PdQ{}, volto a raccogliere le metriche e le strategie per garantire la qualità sia dei processi attuati che del prodotto nel tempo. Tale documento contiene:
        \begin{itemize}
        		\item{\textit{Mancano riferimenti}}
        \end{itemize}
        
        \subsubsection{Strumenti}
        Di seguito vengono elencati gli strumenti usati durante il processo di fornitura:
        
    \subsection{Sviluppo}
        \subsubsection{Descrizione}
        Il processo di sviluppo racchiude le attività ed i compiti che devono essere effettivamente svolti per realizzare il prodotto richiesto.
       Lo standard ISO/IEC 12207-1995 identifica come componenti del processo di sviluppo le seguenti attività:
        \begin{itemize}
            \item{implementazione di processo;}
            \item{analisi dei requisiti (di sistema e del software);}
            \item{design architetturale (di sistema e del software);}
            \item{design dettagliato del software;}
            \item{codifica e testing del software;}
            \item{integrazione (di sistema e del software);}
            \item{qualification testing (di sistema e del software);}
            \item{installazione del software;}
            \item{acceptance test del software;}
        \end{itemize}
        In particolare, l'obbiettivo del team è quello di sviluppare un prodotto software che supera i test e soddisfa le richieste e i requisiti fissati dal Proponente.
            
            \subsubsection{Analisi dei Requisiti}
            Compito degli Analisti è redigere l'\AdR{}, che individua appunto i requisiti del progetto che si va a realizzare. Questi vengono definiti al fine di:
            \begin{itemize}
        		\item{descrivere lo scopo del lavoro;}
		\item{fornire ai Progettisti riferimenti specifici ed affidabili;}
		\item{fissare funzionalità e requisiti concordati con il cliente;}
		\item{fornire una base per raffinamenti successivi al fine di garantire un miglioramento continuo del prodotto e del processo di sviluppo;}
		\item{facilitare le revisioni del codice;}
		\item{fornire ai Verificatori riferimenti circa i casi d'uso principali ed alternativi;}
		\item{in base alla quantità di lavoro prevista, stimare i costi.}
        	   \end{itemize}
                 
                 Il documento redatto deve seguire le specifiche riportate:
                 
                \subsubsubsection{Casi d'uso}
                Dopo aver identificato i casi d'uso, compito degli Analisti è quello di elencarli con un grado di precisione che va dal generico verso il dettaglio.
                Il codice di ogni caso d'uso segue la dicitura: 
                \begin{center}
                \textbf{UC[codice\_padre].[codice\_identificativo]}
                \end{center}
                
                \begin{itemize}
                 	\item{\textbf{Codice padre}: numero che identifica univocamente i casi d'uso generici;}
			\item{\textbf{Codice identificativo}: numero progressivo che identifica i sottocasi specifici.}
                \end{itemize}
                
                Ogni caso d'uso viene inoltre definito mediante la seguente struttura:
                \begin{itemize}
                \item{\textit{Mancano riferimenti}}
                \end{itemize}
                
                \subsubsubsection{Requisiti}
                	Durante l'attività di analisi dei requisiti, è compito degli Analisti stilare una lista di requisiti emersi, seguendo per ognuno la dicitura:
                    \begin{center}
                    \textbf{R[Importanza][Tipologia][Codice]}
                    \end{center}
                    
                    Dove:
                    \begin{itemize}
                        	\item{\textbf{Importanza}: indica il grado di importanza del requisito ai fini del progetto. Può assumere i valori:}
                        	\begin{itemize}
                                	\item{\textbf{1}: requisito obbligatorio ai fini del progetto, irrinunciabile per gli stakeholders;}
                                	\item{\textbf{2}: requisito desiderabile. Non strettamente necessario ai fini del progetto ma che porta valore aggiunto;}
                                	\item{\textbf{3}: requisito opzionale, contrattabile più avanti nel progetto.}
                        	\end{itemize}
                        	\item{\textbf{Tipologia}: classe a cui appartiene il requisito in questione. Può assumere i valori:}
                        	\begin{itemize}
                        		\item{\textbf{F}: funzionale;}
                        		\item{\textbf{P}: prestazionale;}
                        		\item{\textbf{Q}: qualitativo;}
                        		\item{\textbf{V}: vincolo.}
                        	\end{itemize}
                        	\item{\textbf{Codice}: identificatore univoco del requisito}.
                    \end{itemize}
                    
                    Il codice stabilito secondo la convenzione precedente, una volta associato ad un requisito non può più essere cambiato.
                    
                    Ciascun requisito deve inoltre essere accompagnato dalle sue fonti, dalle sue relazioni di dipendenza con altri requisiti, da una descrizione che ne specifichi a parole lo scopo e dalla sua importanza.
                    
                \subsubsubsection{UML}
                \textit{Mancano riferimenti}
                
            \subsubsection{Progettazione}
            L'attività di progettazione segue il procedimento inverso rispetto all'analisi, in quanto rimette insieme le parti specificando le funzionalità dei sottoinsiemi. Il suo scopo è quello di definire una soluzione del problema soddisfacente per tutti gli stakeholders in relazione ai requisiti specificati nel documento \AdR{}, garantendo cioè che il prodotto sviluppato soddisfi le proprietà e i bisogni specificati. Il risultato di tale processo è la realizzazione dell'architettura di sistema.
            E' compito dei Progettisti svolgere tale attività, che è suddivisibile in due parti:
            \begin{itemize}
            	\item{\textbf{Technology baseline}: contiene le specifiche della progettazione ad alto livello, l'elenco dei diagrammi UML utilizzati per la realizzazione dell'architettura ed i test di verifica (in particolare i test di integrazione);}
		\item{\textbf{Product baseline}: approfondisce l'attività di progettazione della Technology baseline e definisce i test necessari alla verifica (in particolare quelli di unità).}
	     \end{itemize}
	     
                \subsubsubsection{Design Patterns}
                	Durante l'attività di progettazione si ha la necessità di adottare opportune soluzioni progettuali a problemi ricorrenti. Tali soluzioni devono essere flessibili e consentire una certa libertà d'uso ai Programmatori. Per questo i Progettisti devono esplicitare chiaramente i design pattern utilizzati per l'architettura, accompagnandoli con una descrizione ed un diagramma, così da esporne il significato e la struttura.
	
                \subsubsubsection{Diagrammi UML}
                Al fine di rendere più chiare le scelte progettuali adottate e ridurre le ambiguità vengono usati vari tipi di diagrammi UML 2.0, tra cui:
                \begin{itemize}
                		\item{\textbf{Diagrammi delle attività}: permettono di descrivere il flusso di operazioni di un'attività. Possono descrivere procedure che esulano dal software;}
			\item{\textbf{Diagrammi delle classi}: descrivono relazioni, metodi e attributi di classe e tipi;}
			\item{\textbf{Diagrammi dei package}: illustrano raggruppamenti di classi;}
			\item{\textbf{Diagrammi di sequenza}: rappresentano sequenze di azioni sotto forma di scelte ben definite.}
                \end{itemize}
                
             \subsubsection{Codifica}
             E' l'attività che permette di concretizzare la soluzione architetturale trovata durante la progettazione mediante la programmazione vera e propria.
             Questa sezione vuole fissare norme e convenzioni riguardanti la stesura di codice al fine di ottenere codice leggibile ed uniforme, agevolare le fasi di manutenzione, verifica e validazione e migliorare la qualità del prodotto.
             	\subsubsubsection{Indentazione}
		I blocchi innestati devono essere correttamente indentati, usando per ciascun livello di indentazione quattro (4) spazi, fatta eccezione per i commenti. Si consiglia di configurare adeguatamente il proprio IDE al fine di rispettare tale norma.
		\subsubsubsection{Parentesizzazione}
		Le parentesi di delimitazione dei costrutti vanno inserite in linea e non al di sotto di essi.
		\subsubsubsection{Lunghezza dei metodi}
		Ove possibile, preferire metodi brevi (poche righe di codice) e che assolvono al minor numero di compiti possibile.
		\subsubsubsection{Univocità dei nomi}
		Classi, variabili e metodi devono avere un nome univoco ed esplicativo (nome parlante) al fine di evitare ambiguità e di far capire ad eventuali lettori del codice che non ne siano anche i programmatori lo scopo di quel preciso elemento.
		\subsubsubsection{Classi}
		I nomi delle classi devono iniziare con la lettera maiuscola (ad esempio: NomeClasse).
		\subsubsubsection{Costanti}
		Le costanti devono essere scritte usando solo lettere maiuscole (ad esempio: COSTANTE). Nel caso siano composte da più parole, queste devono essere separate dal carattere speciale underscore (ad esempio: NOME\_COSTANTE).
		\subsubsubsection{Metodi}
		I nomi dei metodi devono iniziare con una lettera minuscola e, nel caso siano composti da più parole, quelle successive devono iniziare con una lettera maiuscola (e.g: nomeMetodo).
		\subsubsubsection{Lingua}
		Il codice, così come i commenti a questo riferiti, deve essere scritto in lingua inglese.
                	\subsubsubsection{Ricorsione}
		L'uso della ricorsione va evitato quanto più possibile.
                
        \subsubsection{Strumenti}
        Di seguito vengono elencati gli strumenti usati durante il processo di sviluppo: