\section{Introduzione}

\subsection{Scopo del documento}
Questo documento ha lo scopo di fissare le regole, le convezioni e le tecnologie che i membri del gruppo \textit{Roundabout} devono adottare durante tutto il corso del progetto, al fine di garantire uniformità nello svolgimento del lavoro e collaborazione tra tutti i membri del team. 
Verrà utilizzato un approccio incrementale, in modo da normare ogni decisione discussa
e applicata dal team. Le attività presenti all'interno di questo documento sono state prese da processi appartenenti allo standard ISO/IEC 12207:1995. \\
Ciascun componente ha il dovere di prendere visione di tale documento e a rispettare le norme in esso descritte.

\subsection{Scopo del prodotto}
L'applicativo che si vuole sviluppare è \textit{Etherless}, una piattaforma cloud\ped{\textit{G}} che sfrutta la tecnologia degli smart contract\ped{\textit{G}} caratteristica della rete Ethereum\ped{\textit{G}}. Lo scopo di \textit{Etherless} è duplice: da una parte permettere agli \textit{sviluppatori software} di rilasciare funzioni Javascript\ped{\textit{G}} nel cloud\ped{\textit{G}}, dall'altra permettere agli \textit{utenti} di beneficiare di queste funzioni in seguito ad un pagamento per il loro uso. 
\textit{Etherless} è gestita e mantenuta dai suoi \textit{amministratori}.

\subsection{Glossario}
Al fine di evitare possibili ambiguità, i termini tecnici utilizzati nei documenti formali vengono chiariti ed approfonditi nel \textit{Glossario 4.0.0}. Per facilitare la lettura, i termini presenti in tale documento sono contrassegnati in tutto il resto della documentazione da una 'G' a pedice.

\subsection{Riferimenti}

	\subsubsection{Riferimenti normativi}
	\begin{itemize}
		\item \textbf{Capitolato}\ped{\textit{G}}\textbf{ d'appalto C2 - Etherless}: \\
			\url{https://www.math.unipd.it/~tullio/IS-1/2019/Progetto/C2.pdf};
		\item \textbf{Standard ISO 8601}: \\
			\url{https://it.wikipedia.org/wiki/ISO_8601}.	
	\end{itemize}
		
	\subsubsection{Riferimenti informativi}
	\begin{itemize}
		\item \textbf{Standard ISO/IEC 12207:1995}: \\
			\url{https://www.math.unipd.it/~tullio/IS-1/2009/Approfondimenti/ISO_12207-1995.pdf};
			
		\item \textbf{Standard ISO 15504}: \\
		\url{https://en.wikipedia.org/wiki/ISO/IEC_15504};
		
		\item \textbf{Standard ISO 9126}: \\
		\url{https://it.wikipedia.org/wiki/ISO/IEC_9126};
		
		\item \textbf{Slide del corso Ingegneria del Software - Diagrammi delle classi}: \\
			\url{https://www.math.unipd.it/~tullio/IS-1/2019/Dispense/E01b.pdf};
		
		\item \textbf{Slide del corso Ingegneria del Software - Diagrammi dei package}:\\ 
			\url{https://www.math.unipd.it/~tullio/IS-1/2019/Dispense/E01c.pdf};
		
		\item \textbf{Slide del corso Ingegneria del Software - Diagrammi di sequenza}: \\
			\url{https://www.math.unipd.it/~tullio/IS-1/2019/Dispense/E02a.pdf};
		
		\item \textbf{Slide del corso Ingegneria del Software - Diagrammi di attività}: \\
			\url{https://www.math.unipd.it/~tullio/IS-1/2019/Dispense/E02b.pdf};
			
		\item \textbf{Slide del corso Ingegneria del Software - Analisi dei Requisiti}: \\
			\url{https://www.math.unipd.it/~tullio/IS-1/2019/Dispense/L08.pdf};
		
		\item \textbf{Documentazione \LaTeX{}}\ped{\textit{G}}: \\
			\url{https://www.latex-project.org/help/documentation/};
	
		\item \textbf{Airbnb}\ped{\textit{G}} \textbf{JavaScript}\ped{\textit{G}} \textbf{Style Guide}: \\ 
			\url{https://github.com/airbnb/javascript/blob/master/README.md};
		
		\item \textbf{Solidity}\ped{\textit{G}} \textbf{Style Guide}: \\
			\url{https://solidity.readthedocs.io/en/v0.5.7/style-guide.html}
	
		\item \textbf{Sito ufficiale Git}\ped{\textit{G}}: \\
			\url{https://git-scm.com/};
		
		\item \textbf{Ciclo di Deming}\ped{\textit{G}}: \\
			\url{https://it.wikipedia.org/wiki/Ciclo_di_Deming};
		
		\item \textbf{Documentazione Ethers.js}\ped{\textit{G}}: \\
			\url{https://docs.ethers.io/ethers.js/html};
			
		\item \textbf{Documentazione Typescript}\ped{\textit{G}}: \\
			\url{https://www.typescriptlang.org/docs/home.html};
			
		\item \textbf{Documentazione Serverless}\ped{\textit{G}}: \\
			\url{https://aws.amazon.com/it/serverless};
		
		\item \textbf{Documentazione AWS}\ped{\textit{G}}: \\
			\url{https://docs.aws.amazon.com/}.
			
		
	\end{itemize}