\section{Introduzione}

\subsection{Scopo del documento}
Questo documento ha lo scopo di fissare le regole, le convezioni e le tecnologie che i membri del gruppo \textit{Roundabout} si impegnano ad adottare durante tutto il corso del progetto, al fine di garantire uniformità nello svolgimento del lavoro e collaborazione tra tutti i membri del team.

\subsection{Scopo del prodotto}
L'applicativo che si vuole sviluppare è \textit{Etherless}, una piattaforma cloud che sfrutta la tecnologia degli smart contract caratteristica del network Ethereum. Lo scopo di \textit{Etherless} è duplice: da una parte permettere agli \textit{sviluppatori software} di rilasciare funzioni Javascript nel cloud, dall'altra permettere agli \textit{utenti} di beneficiare di queste funzioni in seguito ad un pagamento per il loro uso. 
\textit{Etherless} è gestita e mantenuta dai suoi \textit{amministratori}.

\subsection{Glossario}
Al fine di evitare possibili ambiguità, i termini tecnici utilizzati nei documenti formali vengono chiariti ed approfonditi nel \textit{Glossario Interno 1.0.0}. Per facilitare la lettura, i termini presenti in tale documento sono contrassegnati in tutto il resto della documentazione da una 'G' a pedice.

\subsection{Riferimenti}

	\subsubsection{Riferimenti normativi}
	\begin{itemize}
		\item \textbf{Capitolato d'appalto C2 - Etherless}: \\
		\url{https://www.math.unipd.it/~tullio/IS-1/2019/Progetto/C2.pdf}.
	\end{itemize}
	
	
	\subsubsection{Riferimenti informativi}
	\begin{itemize}
		\item \textbf{Standard ISO/IEC 12207:1995}: \\
		\url{https://www.math.unipd.it/~tullio/IS-1/2009/Approfondimenti/ISO_12207-1995.pdf};
		\item \textbf{Piano di Progetto}: \textit{Piano di Progetto v1.0.0};
		\item \textbf{Piano di Qualifica}: \textit{Piano di Qualifica v1.0.0};
		\item \textbf{Guide to the Software Engineering Body of Knowledge (SWEBOK), V3.0}:\\
		\url{https://www.computer.org/education/bodies-of-knowledge/software-engineering};
		\item \textbf{Documentazione LaTeX}: \\
		\url{https://www.latex-project.org/help/documentation/};
		\item \textbf{Sito ufficiale Git}: \\
		\url{https://git-scm.com/};
	\end{itemize}