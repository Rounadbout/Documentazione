\section{Introduzione}

\subsection{Scopo del documento}
Questo documento ha lo scopo di fissare le regole, le convezioni e le tecnologie che i membri del gruppo \textit{Roundabout} si impegnano ad adottare durante tutto il corso del progetto, al fine di garantire uniformità nello svolgimento del lavoro e collaborazione tra tutti i membri del team.
%Il documento che segue verrà aggiornato incrementalmente ogni qualvolta si incorrerà nella necessità di redigere nuove norme. Per questo motivo alcune parti (e.g. la parte relativa alla codifica) non verranno incluse fino a quando non saranno effettivamente necessarie allo svolgimento del progetto.

\subsection{Scopo del prodotto}
L'applicativo che si vuole sviluppare è \textit{Etherless}, una piattaforma cloud che sfrutta il modello degli smart contract caratteristico del network Ethereum. Lo scopo di \textit{Etherless} è duplice: da una parte permette agli \textit{sviluppatori software} di rilasciare funzioni Javascript nel cloud, dall'altra permette agli \textit{utenti} di beneficiare di queste funzioni in seguito ad un pagamento per il loro uso. %Le quote pagate dagli utenti vengono parzialmente trattenute dalla piattaforma stessa per far fronte alle spese risultanti dall'esecuzione delle funzioni.
\textit{Etherless} è gestita e mantenuta dai suoi \textit{amministratori}.

\subsection{Glossario}
Al fine di evitare possibili ambiguità, i termini tecnici utilizzati nei documenti formali vengono chiariti ed approfonditi nel \textit{Glossario Interno 1.0.0}. Per facilitare la lettura, i termini presenti in tale documento sono contrassegnati in tutto il resto della documentazione da una 'G' a pedice.

\subsection{Riferimenti}

	\subsubsection{Riferimenti normativi}
	\begin{itemize}
		\item \textbf{Standard ISO/IEC 12207:1995}: \\
		\url{https://www.math.unipd.it/~tullio/IS-1/2009/Approfondimenti/ISO_12207-1995.pdf};
		\item \textbf{Capitolato d'appalto C2 - Etherless}: \\
		\url{https://www.math.unipd.it/~tullio/IS-1/2019/Progetto/C2.pdf}.
	\end{itemize}
	
	
	\subsubsection{Riferimenti informativi}
	\begin{itemize}
		\item \textbf{Piano di Progetto}: \textit{Piano di Progetto v0.0.1};
		\item \textbf{Piano di Qualifica}: \textit{Piano di Qualifica v0.0.1};
		\item \textbf{Guide to the Software Engineering Body of Knowledge (SWEBOK), V3.0}:\\
		\url{https://www.computer.org/education/bodies-of-knowledge/software-engineering};
		\item \textbf{Documentazione LaTeX}: \\
		\url{https://www.latex-project.org/help/documentation/};
		\item \textbf{Sito ufficiale Git}: \\
		\url{https://git-scm.com/};
	\end{itemize}